\documentclass{beamer}
\usepackage{braket}
\usepackage{xcolor}
\usepackage{amsmath}
\usepackage{amsfonts}
\usepackage{tikz}
\usepackage{adjustbox}
\usepackage{subcaption}
\usepackage{svg}
\usepackage{graphicx}
\usepackage{media9}
\usepackage{float}
\usetikzlibrary{calc}
\usepackage{array}
\usetheme{EastLansing}
\title[Crisis] % (optional, only for long titles)
{Building A Computer Without A Computer.}
\subtitle{ ( Intrucdction To Error Crorrection And Fault Tolarnce Computation. ) }
\author[D.~Ponarovsky] % (optional, for multiple authors)
	{D.~Ponarovsky\inst{1}}

\institute[Hebrew University of Jerusalem] % (optional)
	{ \inst{1} Faculty of Computer Science\newline
	  Hebrew University of Jerusalem
	}
\date[2022-23] % (optional)
{Qubit meeting 2022-23, Israel Quntum Tech Community.}
\subject{Quantum Error Correction}

\begin{document}
     \maketitle
     \section{About}
	\begin{frame}
	  \frametitle{ About this Presention.  }
	   \begin{columns}[T] % contents are top vertically aligned
     \begin{column}{.23\textwidth} % each column can also be its own environment
     Contents of first column \newline split into two lines
     \end{column}
     \begin{column}{.23\textwidth} % each column can also be its own environment
     Contents of first column \newline split into two lines
     \end{column}
     \begin{column}{.23\textwidth} % each column can also be its own environment
     Contents of first column \newline split into two lines
     \end{column}
     \begin{column}{.23\textwidth} % alternative top-align that's better for graphics
          %\includegraphics[height=3cm]{graphic}
     \end{column}
     \end{columns}
	\end{frame}
	\section{Motivation}
	\begin{frame}
	  \frametitle{ Motivation. }
		here you can put any text/equation etc. 
		$a^2 + b^2 = c^2$.		
	\end{frame}\begin{frame}
	  \frametitle{ Sounds Grate, Whats is the catch? }
		here you can put any text/equation etc. 
		$a^2 + b^2 = c^2$.		
	\end{frame}\begin{frame}
	  \frametitle{ Wait a minute. }
		here you can put any text/equation etc. 
		$a^2 + b^2 = c^2$.		
	\end{frame}
	\begin{frame}
		\frametitle{This is the second slide}
		\framesubtitle{A bit more information about this}
		Some random text.		
	\end{frame}
\end{document}
