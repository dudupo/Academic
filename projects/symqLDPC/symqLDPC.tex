

%\newcommand*{\ACM}{}%

\ifdefined\ACM

%\documentclass[sigplan,screen]{acmart}
  \documentclass[manuscript,screen,review]{acmart}

\else
  \documentclass{article}
  \usepackage[utf8]{inputenc}
\usepackage[a4paper, total={6in, 9in}]{geometry}
\usepackage{braket}
\usepackage{xcolor}
\usepackage{amsmath}
\usepackage{amsfonts}
\usepackage{amsthm}
\usepackage{amssymb}
%\usepackage[ocgcolorlinks]{hyperref}
\usepackage{hyperref}
%\usepackage{hyperref,xcolor}
%\usepackage[ocgcolorlinks]{ocgx2}
\usepackage{cleveref}
\usepackage{graphicx}
\usepackage{svg}
\usepackage{float}
\usepackage{tikz}
\usetikzlibrary{patterns, shapes.arrows}
\usepackage{adjustbox}
%\usepackage{tikz-network}
\usepackage{tkz-graph}
\usepackage{tkz-berge}
\usepackage[linesnumbered]{algorithm2e}
\usepackage{multicol}
\usepackage[backend=biber,style=alphabetic,sorting=ynt]{biblatex}
%\usepackage{xcolor}
%\usepackage{tkz-berge}
%\usepackage{tkz-graph}
\usepackage{pgfplots}
\usepackage{sagetex}
\usepackage{setspace}
\usepackage{etoc}
%\usepackage{wrapfig}
\usepackage{pgfgantt}
\DeclareUnicodeCharacter{2212}{−}
\usepgfplotslibrary{groupplots,dateplot}
\pgfplotsset{compat=newest}

\newtheorem{theorem}{Theorem}
\newtheorem{definition}{Definition}
\newtheorem{example}{Example}
\newtheorem{claim}{Claim}
\newtheorem{fact}{Fact}
\newtheorem{remark}{Remark}
\newtheorem*{theorem*}{Theorem}
\newtheorem{lemma}{Lemma}
\crefname{lemma}{Lemma}{Lemmas}
\hypersetup{colorlinks=true}
% , allcolors=blue,allbordercolors=blue,pdfborderstyle={0 0 1}}
%\hypersetup{pdfborder={2 2 2}}
% pdfpagemode=FullScreen,
% backref 

\newtheorem{problem}{Problem}
\crefname{problem}{Problem}{Problems}

\DeclareMathOperator{\Ima}{Im}


  \addbibresource{./sample.bib} 

\fi

\begin{document}

\newcommand{\commentt}[1]{\textcolor{blue}{ \textbf{[COMMENT]} #1}}
\newcommand{\ctt}[1]{\commentt{#1}}
\newcommand{\prb}[1]{ \mathbf{Pr} \left[ #1 \right]}
\newcommand{\prbm}[2]{ \mathbf{Pr}_{ #2 }\left[ #1 \right]}
\newcommand{\prbc}[3]{ \mathbf{Pr}_{ #2 }\left[ #1 \right | #3]}
\newcommand{\prbcprb}[3]{ \prbc{#2}{#1}{#3} \cdot \prb{#3} } 
\newcommand{\expp}[1]{ \mathbf{E} \left[ {#1} \right]}
\newcommand{\onotation}[1]{\(\mathcal{O} \left( {#1}  \right) \)}
\newcommand{\ona}[1]{\onotation{#1}}
\newcommand{\PSI}{{\ket{\psi}}}
\newcommand{\xij} { X_{ij} } 
\DeclareMathOperator{\Ima}{Im}
%\newcommand{\LESn}{\ket{\psi_n}}
%\newcommand{\LESa}{\ket{\phi_n}}
%\newcommand{\LESs}{\frac{1}{\sqrt{n}}\sum_{i}{\ket{\left(0^{i}10^{n-i}\right)^{n}}}}
%\newcommand{\Hn}{\mathcal{H}_{n}}
%\newcommand{\Ep}{\frac{1}{\sqrt{2^n}}\sum^{2^n}_{x}{ \ket{xx}}}
%\newcommand{\HON}{\ket{\psi_{\text{honest}}}}
%\newcommand{\Lemma}{\paragraph{Lemma.}}
\newcommand{\Cpa}{[n, \rho n, \delta n]}
%\setlength{\columnsep}{0.6cm}
\newcommand{\Jvv}{ \bar{J_{v}} } 
\newcommand{\Cvv}{ \tilde{C_{v}} } 

\newcommand{\Gz}{ G_{z}^{\delta} } 
\newcommand{ \Tann } {  \mathcal{T}\left( G, C_0 \right) }
\newcommand{\ireducable}{ireducable \hyperref[ire]{[\ref{ire}]} }
\newcommand{\cutUU}{E(U_{-1} \bigcup U_{+1} ,U)} 
\newcommand{\wcutUU}{w\left( E(U_{-1} \bigcup U_{+1} ,U)  \right)}
\newcommand{\testgo}{  \mathcal{T}\left(J, q , C_{0}\right) } 

\newcommand{\duC}{\left( C_{A}^{\perp}\otimes C_{B}^{\perp} \right)^{\perp}}
\newcommand{\duduC}{\left( C_{A}\otimes C_{B}\right)^{\perp}}
  





\title{Another reason that makes finding good qLDPC an hard task.} 
\author{David Ponarovsky}
  \maketitle
  
  \begin{claim}
    Let $C_{X}/C_{Z}^{\perp}$ be CSS a qLDPC code with non constant distance. Denote by $H_{X},H_{Z}$ their parity check matrices and by $C_{Z}^{\prime},H_{Z}^{\prime}$ the code and the parity check matrix obtaind by removing one arbitrary check form $H_{Z}$. Then $C_{X}/C_{Z}^{\perp \prime}$ is a CSS pair with constant distance.    
  \end{claim}
  \begin{proof}
    First notice that any of the rows of $H_{Z}^{\prime}$ commute with the rows of $H_{X}$, so we defently obtain a CSS code with higher rate. Seconed any codeword of the quantum code $C_{X}/C_{Z}^{\perp, \prime}$ has the form  
    \begin{equation*}
      \begin{split}
        \ket{ \mathbf{x} } = \sum_{z \in C_{Z}^{\perp, \prime}}{\ket{ x + z}}
      \end{split}
    \end{equation*}
    Using the fact that the generator matrix of the dual of any binary code is the transposed parity check matrix of it, the above become:
\begin{equation*}
      \begin{split}
        \ket{ \mathbf{x} } = \sum_{z \in \mathbb{F}_{2}^{s}}{\ket{ x + H_{Z}^{\perp, prime} z}}
      \end{split}
    \end{equation*}
Observe that because $C_{X}/C_{Z}^{\perp} \subset  C_{X}/C_{Z}^{\prime, \perp}$ we have also that the following state is in $C_{X}/C_{Z}^{\perp, \prime}$:
\begin{equation*}
  \begin{split} 
    \ket{ \mathbf{x^{\prime}} } = & \sum_{z \in \mathbb{F}_{2}^{s+1}}{\ket{ x + H_{Z}^{\perp} z}} \\
      =&  \sim_{w \in \mathbb{F}_{2}}\sum_{z \in \mathbb{F}_{2}^{s}}{\ket{ x + H_{Z}^{\perp, prime} z + h^{\prime} w  }}
  \end{split}
\end{equation*}
Where $h^{\prime}$ is the check that removed form $H_{Z}$ to obtain $C_{Z}^{\prime}$. Now let us give an quantum circuit act non-trivaliy on no more than constant qubits and with probability $\frac{1}{2}$ transform $\ket{\mathbf{x}}$ to $ \ket{\mathbf{x}^{\prime}}$. So first we prapre ancila in the $\ket{+}$ state, then controlled on it's value we add $h^{\prime}$. After that we rotate back the ancila by applaing $H$ (Hadamard) again and measuring, with probability $\frac{1}{2}$ we measure $\ket{0}$ and the remaining qubits hold the state $\ket{\mathbf{x}^{\prime}}$. As $h^{\prime}$ is also a check of the LDPC code $C_{Z}$ it has a constant weight and thus all the circuit toach a constant number of qubits. Therfore the operator which transform $\ket{\mathbf{x}}$ into $\ket{\mathbf{x}^{\prime}}$ is supported only on paulis with constanst degree.     
  \end{proof}


  \begin{claim}
Let $C_X/C_Z^{\perp}$ be a CSS qLDPC code with non-constant distance. Denote by $H_X$, $H_Z$ their parity check matrices and by $C_Z^{\prime}$, $H_Z^{\prime}$ the code and the parity check matrix obtained by removing one arbitrary check from $H_Z$. Then $C_X/C_Z^{\perp \prime}$ is a CSS pair with constant distance.
\end{claim}

\begin{proof}
First, notice that any of the rows of $H_Z^{\prime}$ commute with the rows of $H_X$, so we definitely obtain a CSS code with higher rate. Second, any codeword of the quantum code $C_X/C_Z^{\perp \prime}$ has the form  
\begin{equation*}
  \begin{split}
    \ket{\mathbf{x}} = \sum_{z \in C_Z^{\perp \prime}}{\ket{x + z}}
  \end{split}
\end{equation*}
Using the fact that the generator matrix of the dual of any binary code is the transposed parity check matrix of it, the above becomes:
\begin{equation*}
  \begin{split}
    \ket{\mathbf{x}} = \sum_{z \in \mathbb{F}_2^s}{\ket{x + H_Z^{\perp \prime}z}}
  \end{split}
\end{equation*}
Observe that because $C_X/C_Z^{\perp} \subset C_X/C_Z^{\prime \perp}$, we have also that the following state is in $C_X/C_Z^{\perp \prime}$:
\begin{equation*}
  \begin{split} 
    \ket{\mathbf{x'}} = & \sum_{z \in \mathbb{F}_2^{s+1}}{\ket{x + H_Z^{\perp}z}} \\
    =& \sum_{w \in \mathbb{F}_2}\sum_{z \in \mathbb{F}_2^s}{\ket{x + H_Z^{\perp \prime}z + h'w  }} = \frac{1}{\sqrt{2}}\left( \ket{\mathbf{x}} + \ket{\mathbf{x} + h^{\prime}} \right)
  \end{split}
\end{equation*}
Where $h'$ is the check that was removed from $H_Z$ to obtain $C_Z^{\prime}$. Now let us give a quantum circuit that acts non-trivially on no more than a constant number of qubits and with probability $\frac{1}{2}$ transforms $\ket{\mathbf{x}}$ to $\ket{\mathbf{x'}}$. So first we prepare an ancilla in the $\ket{+}$ state, then controlled on its value we add $h'$. After that, we rotate back the ancilla by applying $H$ (Hadamard) again and measuring, with probability $\frac{1}{2}$ we measure $\ket{0}$ and the remaining qubits hold the state $\ket{\mathbf{x'}}$. As $h'$ is also a check of the LDPC code $C_Z$, it has a constant weight and thus all the circuit touches a constant number of qubits. Therefore, the operator which transforms $\ket{\mathbf{x}}$ into $\ket{\mathbf{x'}}$ is supported only on Paulis with constant degree.     
\end{proof}

%\begin{multicols*}{2}
% \section{Preambles}

Localy Testable Codes, or LTC, are error correction codes such that verfining a uinformly random cchoosen check, would be enough to detect any error with probability proportional to it's size. Bisdes the clear computional adventage they offer, they took roles at the eriler PCP proofs.  
  In this work, we propose a new construction for good LTC codes, which also have a good testability parameter. In the sense   Our proof also indirectly answers the following question. Why most of the good LDPC codes are known to be bad in terms of detecting errors? In other words, It seems that for most of them, there exist strings that are very far from being in the code and, meanwhile, fail to satisfy only a small number of restrictions.
  While the previous LDPC constructions focused on ensuring that the yielded code would have a good rate and distance parameters, our construction enforces the restrictions collection to have a nontrivial fraction of degeneration. That is, removing a single restriction will not change the code, as any restriction is linearly dependent on the others.




\printbibliography
\end{document}





