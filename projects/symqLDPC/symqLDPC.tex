

%\newcommand*{\ACM}{}%

\ifdefined\ACM

%\documentclass[sigplan,screen]{acmart}
  \documentclass[manuscript,screen,review]{acmart}

\else
  \documentclass{article}
  \usepackage[utf8]{inputenc}
\usepackage[a4paper, total={6.5in, 10in} ]{geometry}
\usepackage{braket}
\usepackage{xcolor}
\usepackage{amsmath}
\usepackage{amssymb}
\usepackage{amsfonts}
\usepackage{graphicx}
\usepackage{svg}
\usepackage{float}
\usepackage{tikz}
\usetikzlibrary{patterns, shapes.arrows}
\usepackage{adjustbox}
\usepackage{tikz-network}
\usepackage[ruled,lined,linesnumbered]{algorithm2e}
\usepackage{multicol}
\usepackage[backend=biber,style=alphabetic,sorting=ynt]{biblatex}
\usepackage{xcolor}
\usepackage{pgfplots}
\DeclareUnicodeCharacter{2212}{−}
\usepgfplotslibrary{groupplots,dateplot}
\pgfplotsset{compat=newest}



  \addbibresource{./sample.bib} 

\fi

\begin{document}

\input{newcommands}

\title{Another reason that makes finding good qLDPC an hard task.} 
\author{David Ponarovsky}
  \maketitle
  
  \begin{claim}
    Let $C_{X}/C_{Z}^{\perp}$ be CSS a qLDPC code with non constant distance. Denote by $H_{X},H_{Z}$ their parity check matrices and by $C_{Z}^{\prime},H_{Z}^{\prime}$ the code and the parity check matrix obtaind by removing one arbitrary check form $H_{Z}$. Then $C_{X}/C_{Z}^{\perp \prime}$ is a CSS pair with constant distance.    
  \end{claim}
  \begin{proof}
    First notice that any of the rows of $H_{Z}^{\prime}$ commute with the rows of $H_{X}$, so we defently obtain a CSS code with higher rate. Seconed any codeword of the quantum code $C_{X}/C_{Z}^{\perp, \prime}$ has the form  
    \begin{equation*}
      \begin{split}
        \ket{ \mathbf{x} } = \sum_{z \in C_{Z}^{\perp, \prime}}{\ket{ x + z}}
      \end{split}
    \end{equation*}
    Using the fact that the generator matrix of the dual of any binary code is the transposed parity check matrix of it, the above become:
\begin{equation*}
      \begin{split}
        \ket{ \mathbf{x} } = \sum_{z \in \mathbb{F}_{2}^{s}}{\ket{ x + H_{Z}^{\perp, prime} z}}
      \end{split}
    \end{equation*}
Observe that because $C_{X}/C_{Z}^{\perp} \subset  C_{X}/C_{Z}^{\prime, \perp}$ we have also that the following state is in $C_{X}/C_{Z}^{\perp, \prime}$:
\begin{equation*}
  \begin{split} 
    \ket{ \mathbf{x^{\prime}} } = & \sum_{z \in \mathbb{F}_{2}^{s+1}}{\ket{ x + H_{Z}^{\perp} z}} \\
      =&  \sim_{w \in \mathbb{F}_{2}}\sum_{z \in \mathbb{F}_{2}^{s}}{\ket{ x + H_{Z}^{\perp, prime} z + h^{\prime} w  }}
  \end{split}
\end{equation*}
Where $h^{\prime}$ is the check that removed form $H_{Z}$ to obtain $C_{Z}^{\prime}$. Now let us give an quantum circuit act non-trivaliy on no more than constant qubits and with probability $\frac{1}{2}$ transform $\ket{\mathbf{x}}$ to $ \ket{\mathbf{x}^{\prime}}$. So first we prapre ancila in the $\ket{+}$ state, then controlled on it's value we add $h^{\prime}$. After that we rotate back the ancila by applaing $H$ (Hadamard) again and measuring, with probability $\frac{1}{2}$ we measure $\ket{0}$ and the remaining qubits hold the state $\ket{\mathbf{x}^{\prime}}$. As $h^{\prime}$ is also a check of the LDPC code $C_{Z}$ it has a constant weight and thus all the circuit toach a constant number of qubits. Therfore the operator which transform $\ket{\mathbf{x}}$ into $\ket{\mathbf{x}^{\prime}}$ is supported only on paulis with constanst degree.     
  \end{proof}


%\begin{multicols*}{2}
% \section{Preambles}
  In this work, we propose a new construction for good LDPC codes, which also have a good testability parameter. In the sense that verfining a constant number of random checks, would be enough to detect any error with probability proportional to the error size. In contrast to previews, constructions made by \cite{Dinur}, \cite{leverrier2022quantum} and \cite{Pavel}, our construction doesn't require spicel properties of the small codes, such as $w$-robustness and $p$-resistance for puncturing. 
  
  Our proof also indirectly answers the following question. Why most of the good LDPC codes are known to be bad in terms of detecting errors? In other words, It seems that for most of them, there exist strings that are very far from being in the code and, meanwhile, fail to satisfy only a small number of restrictions.
  While the previous LDPC constructions focused on ensuring that the yielded code would have a good rate and distance parameters, our construction enforces the restrictions collection to have a nontrivial fraction of degeneration. That is, removing a single restriction will not change the code, as any restriction is linearly dependent on the others.




\printbibliography
\end{document}





