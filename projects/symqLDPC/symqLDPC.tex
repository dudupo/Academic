

%\newcommand*{\ACM}{}%

\ifdefined\ACM

%\documentclass[sigplan,screen]{acmart}
  \documentclass[manuscript,screen,review]{acmart}

\else
  \documentclass{article}
  \usepackage[utf8]{inputenc}
\usepackage[a4paper, total={6.5in, 10in} ]{geometry}
\usepackage{braket}
\usepackage{xcolor}
\usepackage{amsmath}
\usepackage{amssymb}
\usepackage{amsfonts}
\usepackage{graphicx}
\usepackage{svg}
\usepackage{float}
\usepackage{tikz}
\usetikzlibrary{patterns, shapes.arrows}
\usepackage{adjustbox}
\usepackage{tikz-network}
\usepackage[ruled,lined,linesnumbered]{algorithm2e}
\usepackage{multicol}
\usepackage[backend=biber,style=alphabetic,sorting=ynt]{biblatex}
\usepackage{xcolor}
\usepackage{pgfplots}
\DeclareUnicodeCharacter{2212}{−}
\usepgfplotslibrary{groupplots,dateplot}
\pgfplotsset{compat=newest}



  \addbibresource{./sample.bib} 

\fi

\begin{document}

\input{newcommands}

\title{Another reason that makes finding good quantum LDPCs a difficult task.} 
\author{David Ponarovsky}
  \maketitle
  
  \begin{claim}
Let $C_X/C_Z^{\perp}$ be a CSS qLDPC code with non-constant distance. Denote by $H_X$, $H_Z$ their parity check matrices and by $C_Z^{\prime}$, $H_Z^{\prime}$ the code and the parity check matrix obtained by removing one arbitrary check from $H_Z$. Then $C_X/C_Z^{\perp \prime}$ is a CSS pair with constant distance.
\end{claim}

\begin{proof}
First, notice that any of the rows of $H_Z^{\prime}$ commute with the rows of $H_X$, so we definitely obtain a CSS code with higher rate. Second, any codeword of the quantum code $C_X/C_Z^{\perp \prime}$ has the form  
\begin{equation*}
  \begin{split}
    \ket{\mathbf{x}} = \sum_{z \in C_Z^{\perp \prime}}{\ket{x + z}}
  \end{split}
\end{equation*}
Using the fact that the generator matrix of the dual of any binary code is the transposed parity check matrix of it, the above becomes:
\begin{equation*}
  \begin{split}
    \ket{\mathbf{x}} = \sum_{z \in \mathbb{F}_2^s}{\ket{x + H_Z^{\top \prime}z}}
  \end{split}
\end{equation*}
Observe that because $C_X/C_Z^{\perp} \subset C_X/C_Z^{\prime \perp}$, we have also that the following state is in $C_X/C_Z^{\perp \prime}$:
\begin{equation*}
  \begin{split} 
    \ket{\mathbf{x'}} = & \sum_{z \in \mathbb{F}_2^{s+1}}{\ket{x + H_Z^{\top}z}} \\
    =& \sum_{w \in \mathbb{F}_2}\sum_{z \in \mathbb{F}_2^s}{\ket{x + H_Z^{\top \prime}z + h'w  }} \\ 
    =& \frac{1}{\sqrt{2}}\left( \ket{\mathbf{x}} + \ket{\mathbf{x} + h^{\prime}} \right)
  \end{split}
\end{equation*}
Where $h'$ is the check that was removed from $H_Z$ to obtain $C_Z^{\prime}$. Now let us give a quantum circuit that acts non-trivially on no more than a constant number of qubits and with probability $\frac{1}{2}$ transforms $\ket{\mathbf{x}}$ to $\ket{\mathbf{x'}}$. So first we prepare an ancilla in the $\ket{+}$ state, then controlled on its value we add $h'$. After that, we rotate back the ancilla by applying $H$ (Hadamard) again and measuring.

With probability $\frac{1}{2}$ we measure $\ket{0}$ and the remaining qubits hold the state $\ket{\mathbf{x'}}$. As $h'$ is also a check of the LDPC code $C_Z$, it has a constant weight and thus all the circuit touches a constant number of qubits. Therefore, the operator which transforms $\ket{\mathbf{x}}$ into $\ket{\mathbf{x'}}$ is supported only on Paulis with constant degree.     
\end{proof}  

%\begin{multicols*}{2}
% \section{Preambles}
  In this work, we propose a new construction for good LDPC codes, which also have a good testability parameter. In the sense that verfining a constant number of random checks, would be enough to detect any error with probability proportional to the error size. In contrast to previews, constructions made by \cite{Dinur}, \cite{leverrier2022quantum} and \cite{Pavel}, our construction doesn't require spicel properties of the small codes, such as $w$-robustness and $p$-resistance for puncturing. 
  
  Our proof also indirectly answers the following question. Why most of the good LDPC codes are known to be bad in terms of detecting errors? In other words, It seems that for most of them, there exist strings that are very far from being in the code and, meanwhile, fail to satisfy only a small number of restrictions.
  While the previous LDPC constructions focused on ensuring that the yielded code would have a good rate and distance parameters, our construction enforces the restrictions collection to have a nontrivial fraction of degeneration. That is, removing a single restriction will not change the code, as any restriction is linearly dependent on the others.




\printbibliography
\end{document}




