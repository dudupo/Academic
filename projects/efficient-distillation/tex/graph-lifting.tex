

%\newcommand*{\ACM}{}%

\ifdefined\ACM

%\documentclass[sigplan,screen]{acmart}
  \documentclass[manuscript,screen,review]{acmart}

\else
  \documentclass{article}
  \usepackage{libertine}
  \usepackage[utf8]{inputenc}
\usepackage[a4paper, total={6in, 9in}]{geometry}
\usepackage{braket}
\usepackage{xcolor}
\usepackage{amsmath}
\usepackage{amsfonts}
\usepackage{amsthm}
\usepackage{amssymb}
%\usepackage[ocgcolorlinks]{hyperref}
\usepackage{hyperref}
%\usepackage{hyperref,xcolor}
%\usepackage[ocgcolorlinks]{ocgx2}
\usepackage{cleveref}
\usepackage{graphicx}
\usepackage{svg}
\usepackage{float}
\usepackage{tikz}
\usetikzlibrary{patterns, shapes.arrows}
\usepackage{adjustbox}
%\usepackage{tikz-network}
\usepackage{tkz-graph}
\usepackage{tkz-berge}
\usepackage[linesnumbered]{algorithm2e}
\usepackage{multicol}
\usepackage[backend=biber,style=alphabetic,sorting=ynt]{biblatex}
%\usepackage{xcolor}
%\usepackage{tkz-berge}
%\usepackage{tkz-graph}
\usepackage{pgfplots}
\usepackage{sagetex}
\usepackage{setspace}
\usepackage{etoc}
%\usepackage{wrapfig}
\usepackage{pgfgantt}
\DeclareUnicodeCharacter{2212}{−}
\usepgfplotslibrary{groupplots,dateplot}
\pgfplotsset{compat=newest}

\newtheorem{theorem}{Theorem}
\newtheorem{definition}{Definition}
\newtheorem{example}{Example}
\newtheorem{claim}{Claim}
\newtheorem{fact}{Fact}
\newtheorem{remark}{Remark}
\newtheorem*{theorem*}{Theorem}
\newtheorem{lemma}{Lemma}
\crefname{lemma}{Lemma}{Lemmas}
\hypersetup{colorlinks=true}
% , allcolors=blue,allbordercolors=blue,pdfborderstyle={0 0 1}}
%\hypersetup{pdfborder={2 2 2}}
% pdfpagemode=FullScreen,
% backref 

\newtheorem{problem}{Problem}
\crefname{problem}{Problem}{Problems}

\DeclareMathOperator{\Ima}{Im}


  \usepackage{cancel}
  \usepackage{subcaption}
  \addbibresource{./sample.bib} 

\fi

\begin{document}
\newcommand{\commentt}[1]{\textcolor{blue}{ \textbf{[COMMENT]} #1}}
\newcommand{\ctt}[1]{\commentt{#1}}
\newcommand{\prb}[1]{ \mathbf{Pr} \left[ #1 \right]}
\newcommand{\prbm}[2]{ \mathbf{Pr}_{ #2 }\left[ #1 \right]}
\newcommand{\prbc}[3]{ \mathbf{Pr}_{ #2 }\left[ #1 \right | #3]}
\newcommand{\prbcprb}[3]{ \prbc{#2}{#1}{#3} \cdot \prb{#3} } 
\newcommand{\expp}[1]{ \mathbf{E} \left[ {#1} \right]}
\newcommand{\onotation}[1]{\(\mathcal{O} \left( {#1}  \right) \)}
\newcommand{\ona}[1]{\onotation{#1}}
\newcommand{\PSI}{{\ket{\psi}}}
\newcommand{\xij} { X_{ij} } 
\DeclareMathOperator{\Ima}{Im}
%\newcommand{\LESn}{\ket{\psi_n}}
%\newcommand{\LESa}{\ket{\phi_n}}
%\newcommand{\LESs}{\frac{1}{\sqrt{n}}\sum_{i}{\ket{\left(0^{i}10^{n-i}\right)^{n}}}}
%\newcommand{\Hn}{\mathcal{H}_{n}}
%\newcommand{\Ep}{\frac{1}{\sqrt{2^n}}\sum^{2^n}_{x}{ \ket{xx}}}
%\newcommand{\HON}{\ket{\psi_{\text{honest}}}}
%\newcommand{\Lemma}{\paragraph{Lemma.}}
\newcommand{\Cpa}{[n, \rho n, \delta n]}
%\setlength{\columnsep}{0.6cm}
\newcommand{\Jvv}{ \bar{J_{v}} } 
\newcommand{\Cvv}{ \tilde{C_{v}} } 

\newcommand{\Gz}{ G_{z}^{\delta} } 
\newcommand{ \Tann } {  \mathcal{T}\left( G, C_0 \right) }
\newcommand{\ireducable}{ireducable \hyperref[ire]{[\ref{ire}]} }
\newcommand{\cutUU}{E(U_{-1} \bigcup U_{+1} ,U)} 
\newcommand{\wcutUU}{w\left( E(U_{-1} \bigcup U_{+1} ,U)  \right)}
\newcommand{\testgo}{  \mathcal{T}\left(J, q , C_{0}\right) } 

\newcommand{\duC}{\left( C_{A}^{\perp}\otimes C_{B}^{\perp} \right)^{\perp}}
\newcommand{\duduC}{\left( C_{A}\otimes C_{B}\right)^{\perp}}
  




\title{ Amplifying the expansion while preserving a low portion of noncommuting checks. }
\author{David Ponarovsky}
\maketitle

\section{Notations and Definitions.}
Let $G=(L,R,E)$ be an undirected bipartite graph, where $L$ and $R$ stand for the left and right vertices, and $E$ is the set of edges connecting them. We will think of $G$ as the Tanner graph of a linear code, constitutes the set of all possible bits assignments over the left vertices such that any right vertex sees an even number of turned-on bits (left vertices assigned to $1$). The local view of the right vertex $v \in R$ is the assignment of bits to its neighbors. The bipartite graph obtained by setting $L$ as the right vertices and $R$ as the left vertices will be called the transposed graph, denoted by $G^{\top} = (R,L,E)$. $n$ and $m$ will be used to denote $|L|$ and $|R|$, and will often be referred to as the number of bits = code length and the number of checks.
The parity check matrix of the code $H \in \mathbb{F}_2^{m \times n}$ is the adjacency matrix defined by $E$. This means that $H_{u,v} = 1$ only if there is an edge ${u,v} \in E$, and zero otherwise. It is easy to see that if the assignment $c \in \mathbb{F}_{2}^{n}$ is a codeword, then $Hc = 0$, which is why $H$ got its name. For any $c \in \mathbb{F}_{2}^{n}$ that is not necessarily a codeword, we call $s = Hc \in \mathbb{F}_{2}^{m}$ the syndrome of $c$, and think of any non-zero entry of $s$ as a check that does not pass.

Let $x,y$ be two different rows of $H$, or in code language terminology, two different checks. We will say that $x$ and $y$ commute if $xy = 0$ and uncommute otherwise. The uncommuting rate will be defined as the probability of choosing two different uncommute checks, and will be denoted by $P(G)$.

\begin{equation*}
  \begin{split}
    P(G) =  \prbm{ xy \neq 0 }{x\neq y \in \text{ rows } H } 
  \end{split}
\end{equation*}
From now on, we will assume that $G$ has a fixed left and right degree, $\Delta_{l}$ and $\Delta_{r}$ respectively. This means that any left vertex is connected to exactly $\Delta_{l}$ right vertices, and similarly, any right vertex is connected to exactly $\Delta_{r}$ left vertices. We use $\Delta$ to denote the maximum of them, $\Delta = \max \{ \Delta_{l}, \Delta_{r} \}$. 

For any subset of vertices $S \subset L \cup R$, we will denote the vertices connected to $S$ by $\Gamma(S)$. $G$ will be said to be a $(\tau, \varepsilon)$ left-expander if for any $S \subset L$ of size at most $\tau n$, it holds that $|\Gamma(S)| > (1-\varepsilon)\Delta_{l}|S|$. In the same way, we define a right-expander.

We are interested in the following question: for fixed constants $\Delta, \varepsilon, \tau, \beta$, is there a family of bipartite graphs such that both $G$ and $G^{\top}$ are $(\tau,\varepsilon)$ left-expanders, and their uncommuting rate is bounded above by $\beta$?

\section{The Zig-Zag Product.}
The Zig-Zag product is a procedure to construct a fixed-degree graph from another given graph without reducing the second's expansion too much. We slightly change the original definition and process to support biparitie grpahs with inequal left and right degree. 
\begin{definition}
  Let $G$ be a bipartite graph as defined above, and let $H_{l}$ and $H_{r}$ be graphs with $\Delta_{l}$ and $\Delta_{r}$ vertices, respectively. We define $G^{\prime} = G \cdot_{z} [H_{l},H_{r}]$, the graph obtained by multiplying the graphs $G$, $H_{l}$, and $H_{r}$ using the Zig-Zag product, to be the graph obtained by the following steps:
  \begin{enumerate}
    \item For any vertex $v$ in $L$, mark $\Delta_{l}$ new vertices on its edge, one vertex for each edge. Connect them according to the edges of $H_{l}$. Then, remove $v$. Each of the new vertices will have a degree of $\deg H_{l} + 1$, where one edge is associated with an original edge in $G$ and the other matches the structure of $H_{l}$.
    \item Repeat the above process, but replace each right vertex with $H_{R}$.
    \item Now, we will color any edge associated with the smaller graphs $H_{l}$ and $H_{r}$ in blue, and any of the original edges of $G$ in red. We define an edge ${u,v}$ in $G^{\prime}$ if there is a path composed of a blue edge, a red edge, and another blue edge.
  \end{enumerate}
  Notice that $G^\prime$ remains bipartite and its left and right degrees are $\deg H_{l} \cdot \deg H_{r}$.
\end{definition}

\begin{claim}[Zig-Zag product preservs uncommuting-rate.]  
Let $G$ be a bipartite graph, and let $H_{l}$ and $H_{r}$ be the complete graphs with $\Delta_{l}$ and $\Delta_{r}$ vertices, respectively. Assume that $P(G) < \beta$. Then:
\begin{equation*}
  \begin{split}
      P(G \cdot_{z} [H_{l},H_{r}]) < \beta
  \end{split}
\end{equation*}
\end{claim}
\begin{proof}
  Consider two checks $x$ and $y$ in $G^{\prime}$. Let us abuse notation and refer to the vertices that define those checks as $x$ and $y$. We will say that a vertex $X$ of $G$, is the souce of $x$ if $x$ defiend by marking a vetex on $X$ edges. Let $X,Y$ be the sources of $x,y$ respectively. Now let's split to cases:
  \begin{itemize}
    \item If $X = Y$, namely $x$ and $y$ have the same source. In this case, any blue-red-blue path $x \rightarrow o \rightarrow o^{\prime} \rightarrow w$ from $x$ to $w \in H_{l}$ that passes through a vertex $o \in H_{r}/\{y\}$, and $\{o,y\} \in H_{r}$ matches to a blue-red-blue path $y \rightarrow o \rightarrow o^{\prime} \rightarrow w$ from $y$ to $w$, where the first blue edge $\{ x, o \}$ is replaced by $\{ y, o \}$ to set $y$ at the beginning of the path.
   
    So, conditioned on $X=Y$, the probability that $x,y$ do not commute is either $0$ if $\deg H_{l}$ is even, or $P(H_{r})$.
  \item \ctt{Assume $H_{r}, H_{l}$ are the complete graphs}. If $X\neq Y$, then with probability $1 - P(G)$, $X$ and $Y$ commute in $G$ and therefore there are even number of red edges at the form $\{o , o^{\prime}\}$ to connect paths from $x,y \rightarrow w$. Denote by $Q(G)$ the probability of choosing $X,Y$ right vertices in $G$ which both commute and share at least one common neighbor (bit).  
By double counting we get the bound: 
    \begin{equation*}
      \begin{split}
        & P(G){ m \choose 2} + Q(G){ m \choose 2} \le \frac{1}{2}m \Delta^{2} \\ 
        & Q(G) \le { m \choose 2 }^{-1}\cdot \frac{1}{2}m\Delta^{2} - P(G)
      \end{split}
    \end{equation*}
    Suppose that $X,Y$ uncommute in $G$, then  
    \begin{equation*}
      \begin{split}
        P(G^{\prime})  & \le \left( 1 - \frac{1}{m} \right) \left( P(G) + Q(G)  \frac{1}{\Delta} \right) = \left( 1 - \frac{1}{m} \right) \left(  \frac{\Delta}{m-1} + \left( 1 -  \frac{1}{\Delta}\right)P(G) \right) \\
       \rightarrow_{m \rightarrow \infty } & \le  \left(  1 - \frac{1}{\Delta} \right)P(G)
      \end{split}
    \end{equation*}

  \end{itemize}
\end{proof}






\section{Dihedral Group.}

%Let $\Gamma$ be a group, and denote by $A$ a colsoed to inverse generator set of $A$, i.e if $g \in A$ then also $g^{-1} \in A$ and in addition any elmeent of $\Gamma$ can obtained by chaining products of $A$'s elemnents. Recall that the Caylay graph $\text{Cay}(\Gamma, A )$ is the graph such it's vertices are the $\Gamma$'s elements and there is an edge between $u$ and $v$ only if there is $g\in A$ such $gu =v$. In this work will use the notation $H_{\Gamma}$ to denote the biparitie graph (or check matrix) obtained by:  
Let $\Gamma$ be a group, and denote by $A$ a closed inverse generator set of $\Gamma$, i.e. if $g \in A$, then also $g^{-1} \in A$ and any element of $\Gamma$ can be obtained by chaining products of $A$'s elements. Recall that the Cayley graph $\text{Caylay}(\Gamma, A)$ is the graph such that its vertices are the elements of $\Gamma$, and there is an edge between $u$ and $v$ only if there is $g \in A$ such that $gu = v$, for paritcular $g$ we call to such edges $g$-edges. In this work, we will use the notation $H_{\Gamma}$ to denote the bipartite graph (or check matrix) obtained by:
\begin{equation*}
  \begin{split}
    H_{\Gamma} = \begin{bmatrix}
      0 & 1 \\
      1 & 0
    \end{bmatrix} \otimes \text{Cayley} \left( \Gamma, A \right)
  \end{split}
\end{equation*}
Now, consider the code when the Dihedral group $D_n$ taken as $\Gamma$ and the $A = \{ \tau, \sigma^{i_{1}}, \sigma^{i_2} .. \}$ is taken as the generator set. 
\begin{claim}
  The uncommuting-rate of $H_{D_{n}}$ is at most $(|A|-1)/2n $
\end{claim}
\begin{proof}
Suppose that $v,u \in L$ share a neighbor $w\in R$. Let $g,h \in A$ such that $\{v,w\}$ is a $g$-edge and $\{u,w\}$ is an $h$-edge. Now, if both $g$ and $h$ are in $A/\{\tau\}$, then they commute, i.e., $gh=hg$. Hence:
  \begin{equation*}
    \begin{split}
      h^{-1}v = h^{-1}g^{-1}gv = h^{-1}g^{-1}hu =g^{-1}u
    \end{split}
  \end{equation*}
Since $\Gamma$ is a group, the map $(g,h) \mapsto (g^{-1},h^{-1})$ is a bijection, so overlaps of generator pairs from $A/\{\tau\}$ contribute only an even number to the intersction parity. Therefore, 
  \begin{equation*}
    \begin{split}
      P(G) & =  \prbm{ xy \neq 0 }{x\neq y \in \text{ rows } H } \le  \prbm{\tau x = gy }{x\neq y \in D_{n}, g\in A } \\ 
      & \le \sum_{g \in A / \{ \sigma \} }{ \prbm{\tau x = gy }{x\neq y \in D_{n} }}\le \frac{|A|-1}{2n} 
    \end{split}
  \end{equation*}
\end{proof}

\section{Rate Amplification.}

\begin{claim}
  Let $V$ be a finte set at size $n$. And let $ f : V \times V \rightarrow R$. Then  for $\beta \in (0,1)$, there exists $C$ indepandent on $n$, and a partition of $V$, $\mathcal{I} = S_{1},S_{2},..,S_{k}$, not necessarily disjoint, such that: 
  \begin{enumerate}
    \item $|S_{i}| < \beta n$ for each $i$
    \item For any $x,y \in S, x\neq y$ there exactly $C$ subsets in $\mathcal{I}$, such both $x$ and $y$ belongs to $S \in \mathcal{I}$
  \end{enumerate}
   and the distitubtaion over the values of $f$ when sampling from $V \times V$ uninformly, i.e $\prbm{ f(x,y) = q  }{x,y \sim V \times V}$  equals to $C \cdot \prbm{f(x,y) = q }{x,y \sim S_{i}, i\sim [k]}$ .
\end{claim}


%\section{Left-Rigth Cayley complex \textbf{varition}. \ctt{spalling were not activated, no internent connection 24/6.}}
%The left-right cayley complex is the complex obtained by taking the union of two Cayley graphs where in the first we generate the group by multiplying the group elements from the left while in the seconed we generate the group by multiplying from right. 
%
%Suppose for seconed that the complex is bipartite graph, and consider the code defiend by it, means that any chack is vertex associated with group element $g \in \Gamma$. observs that any check $agb$ that intersect with $g$ on $ag$ is also intersect with $g$ on $gb$. Thus $g$ and $agb$ commute. $a^\prime a g$ is not commute with $g$. they both intersect on $ag$.   
%
%\begin{claim}
%
%\end{claim}


\end{document}



