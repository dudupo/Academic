\documentclass{beamer}
%\usepackage{split}

\usepackage{amsmath}
\usepackage{amsfonts}
\usepackage{braket}
\usepackage{subcaption}
\usepackage{tikz}
\usepackage{quantikz}
\begin{document} 

\newcommand*{\Tr}{\textbf{Tr }}


\begin{frame}
  \title{Hardness of Computing Fault Tolerance.}
    \author{David Ponarovsky}
    \date{\today}
    \titlepage
\end{frame}


\begin{frame}

\frametitle{Introduction}
\begin{itemize}
    \item Brief overview of the topic
    \item Importance and relevance
    \item Objectives of the presentation
\end{itemize}
\end{frame}

\begin{frame}
\frametitle{Key Points}
\begin{itemize}
    \item Main point 1
    \item Main point 2
    \item Main point 3
\end{itemize}
\end{frame}


\begin{frame}
  \frametitle{Nosiy Circuit.}

\begin{quantikz}[row sep=0.3cm, column sep=0.7cm]
\lstick{$q_1$} & \gate[wires=8]{U_0} & \gate{\mathcal{N}} & \gate[wires=8]{U_1}   & \gate{\mathcal{N}} & \gate[wires=8]{U_2} & \gate{\mathcal{N}}& \qw \\
\lstick{$q_2$} &                      & \gate{\mathcal{N}} &                      & \gate{\mathcal{N}} &                     & \gate{\mathcal{N}} & \qw \\
\lstick{$q_3$} &                      & \gate{\mathcal{N}} &                      & \gate{\mathcal{N}} &                     & \gate{\mathcal{N}} & \qw \\
\lstick{$q_4$} &                      & \gate{\mathcal{N}} &                      & \gate{\mathcal{N}} &                     & \gate{\mathcal{N}} & \qw \\
\lstick{$q_5$} &                      & \gate{\mathcal{N}} &                      & \gate{\mathcal{N}} &                     & \gate{\mathcal{N}} & \qw \\
\lstick{$q_6$} &                      & \gate{\mathcal{N}} &                      & \gate{\mathcal{N}} &                     & \gate{\mathcal{N}} & \qw \\
\lstick{$q_7$} &                      & \gate{\mathcal{N}} &                      & \gate{\mathcal{N}} &                     & \gate{\mathcal{N}} & \qw \\
\lstick{$q_8$} &                      & \gate{\mathcal{N}} &                      & \gate{\mathcal{N}} &                     & \gate{\mathcal{N}} & \qw
\end{quantikz}
\end{frame}




\begin{frame}
  \frametitle{Threshoold Theorem.} 
\end{frame}



\begin{frame}
  \frametitle{Pippenger's Construction.} 
  
Encode each bit with the repetition code $0 \mapsto 0^{m}$, $1 \mapsto 1^{m}$. Now observe that any logical operation, without decoding, can be made in $O(1)$ depth.

For example, OR($\bar{x}, \bar{y}$) can be computed by applying in parallel OR($x_{i}, y_{i}$) for each $i$.

\end{frame}


\begin{frame}
  \frametitle{The 'Decoding' trick.} 

Instead of completely decoding, we would apply only a single step of partial decoding. We assume that in each code block the bits are partitioned into random disjoint triples, and we will apply a local correction to each of the triples by majority.



\begin{block}{Claim}
There are constants $\alpha, \eta \in (0,1)$ such that for any bit string $x$ at a distance $\le \alpha n$ from the code (Repetition Code), one cycle of local correction on $x$ yields $x^\prime$ such that:
  \begin{equation*}
    \begin{split}
      d(x^{\prime}, C) \le d(x, C)
    \end{split}
  \end{equation*}
\end{block}
\end{frame}


\begin{frame}
  \frametitle{The 'Decoding' trick.} 
  
  Suppose that a bit obserb a bit flip with probability $p$. So in expectation we expect that entire bolck at length $n$ will absorb $pn$ flips.  
  \begin{equation*}
    \begin{split}
      \eta \left( \beta + p  \right) n &\le \beta n \\ 
      \beta \ge \frac{p}{ 1 - \eta}
    \end{split}
  \end{equation*}






\end{frame}

\begin{frame}
  \frametitle{The Franch's Construction.}
\end{frame}


\begin{frame}
  \frametitle{  }


\begin{figure}[h]
    \centering
    \includegraphics[width=\textwidth]{Hypergraph_prod.png}
    \caption{Caption for the image}
    \label{fig:your-label}
\end{figure}

\end{frame}

\begin{frame}
  \frametitle{  }

\begin{figure}[h]
    \centering
    \includegraphics[width=0.8\textwidth]{toric_prod.png}
    \caption{Caption for the image}
    \label{fig:your-label}
\end{figure}

\end{frame}
\begin{frame}
  \frametitle{ } 
\begin{figure}[h]
    \centering
    \includegraphics[width=0.8\textwidth]{magic_prod.png}
    \caption{Caption for the image}
    \label{fig:your-label}
\end{figure}
\end{frame}



\end{document}
