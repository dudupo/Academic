

%\newcommand*{\ACM}{}%

\ifdefined\ACM

%\documentclass[sigplan,screen]{acmart}
  \documentclass[manuscript,screen,review]{acmart}

\else
  \documentclass{article}
  \usepackage[utf8]{inputenc}
\usepackage[a4paper, total={6in, 9in}]{geometry}
\usepackage{braket}
\usepackage{xcolor}
\usepackage{amsmath}
\usepackage{amsfonts}
\usepackage{amsthm}
\usepackage{amssymb}
%\usepackage[ocgcolorlinks]{hyperref}
\usepackage{hyperref}
%\usepackage{hyperref,xcolor}
%\usepackage[ocgcolorlinks]{ocgx2}
\usepackage{cleveref}
\usepackage{graphicx}
\usepackage{svg}
\usepackage{float}
\usepackage{tikz}
\usetikzlibrary{patterns, shapes.arrows}
\usepackage{adjustbox}
%\usepackage{tikz-network}
\usepackage{tkz-graph}
\usepackage{tkz-berge}
\usepackage[linesnumbered]{algorithm2e}
\usepackage{multicol}
\usepackage[backend=biber,style=alphabetic,sorting=ynt]{biblatex}
%\usepackage{xcolor}
%\usepackage{tkz-berge}
%\usepackage{tkz-graph}
\usepackage{pgfplots}
\usepackage{sagetex}
\usepackage{setspace}
\usepackage{etoc}
%\usepackage{wrapfig}
\usepackage{pgfgantt}
\DeclareUnicodeCharacter{2212}{−}
\usepgfplotslibrary{groupplots,dateplot}
\pgfplotsset{compat=newest}

\newtheorem{theorem}{Theorem}
\newtheorem{definition}{Definition}
\newtheorem{example}{Example}
\newtheorem{claim}{Claim}
\newtheorem{fact}{Fact}
\newtheorem{remark}{Remark}
\newtheorem*{theorem*}{Theorem}
\newtheorem{lemma}{Lemma}
\crefname{lemma}{Lemma}{Lemmas}
\hypersetup{colorlinks=true}
% , allcolors=blue,allbordercolors=blue,pdfborderstyle={0 0 1}}
%\hypersetup{pdfborder={2 2 2}}
% pdfpagemode=FullScreen,
% backref 

\newtheorem{problem}{Problem}
\crefname{problem}{Problem}{Problems}

\DeclareMathOperator{\Ima}{Im}


  \addbibresource{./sample.bib} 
  \addbibresource{./exactalgo-rs.bib}
\fi

\begin{document}

\newcommand{\commentt}[1]{\textcolor{blue}{ \textbf{[COMMENT]} #1}}
\newcommand{\ctt}[1]{\commentt{#1}}
\newcommand{\prb}[1]{ \mathbf{Pr} \left[ #1 \right]}
\newcommand{\prbm}[2]{ \mathbf{Pr}_{ #2 }\left[ #1 \right]}
\newcommand{\prbc}[3]{ \mathbf{Pr}_{ #2 }\left[ #1 \right | #3]}
\newcommand{\prbcprb}[3]{ \prbc{#2}{#1}{#3} \cdot \prb{#3} } 
\newcommand{\expp}[1]{ \mathbf{E} \left[ {#1} \right]}
\newcommand{\onotation}[1]{\(\mathcal{O} \left( {#1}  \right) \)}
\newcommand{\ona}[1]{\onotation{#1}}
\newcommand{\PSI}{{\ket{\psi}}}
\newcommand{\xij} { X_{ij} } 
\DeclareMathOperator{\Ima}{Im}
%\newcommand{\LESn}{\ket{\psi_n}}
%\newcommand{\LESa}{\ket{\phi_n}}
%\newcommand{\LESs}{\frac{1}{\sqrt{n}}\sum_{i}{\ket{\left(0^{i}10^{n-i}\right)^{n}}}}
%\newcommand{\Hn}{\mathcal{H}_{n}}
%\newcommand{\Ep}{\frac{1}{\sqrt{2^n}}\sum^{2^n}_{x}{ \ket{xx}}}
%\newcommand{\HON}{\ket{\psi_{\text{honest}}}}
%\newcommand{\Lemma}{\paragraph{Lemma.}}
\newcommand{\Cpa}{[n, \rho n, \delta n]}
%\setlength{\columnsep}{0.6cm}
\newcommand{\Jvv}{ \bar{J_{v}} } 
\newcommand{\Cvv}{ \tilde{C_{v}} } 

\newcommand{\Gz}{ G_{z}^{\delta} } 
\newcommand{ \Tann } {  \mathcal{T}\left( G, C_0 \right) }
\newcommand{\ireducable}{ireducable \hyperref[ire]{[\ref{ire}]} }
\newcommand{\cutUU}{E(U_{-1} \bigcup U_{+1} ,U)} 
\newcommand{\wcutUU}{w\left( E(U_{-1} \bigcup U_{+1} ,U)  \right)}
\newcommand{\testgo}{  \mathcal{T}\left(J, q , C_{0}\right) } 

\newcommand{\duC}{\left( C_{A}^{\perp}\otimes C_{B}^{\perp} \right)^{\perp}}
\newcommand{\duduC}{\left( C_{A}\otimes C_{B}\right)^{\perp}}
  





\title{Dining Philosophers (Short Note)} 
\author{David Ponarovsky}

\ifdefined\ACM
  \affiliation{%
    \institution{The Th{\o}rv{\"a}ld Group}
    \streetaddress{1 Th{\o}rv{\"a}ld Circle}
    \city{Hekla}
  \country{Iceland}}
  \email{larst@affiliation.org}
\else
  \maketitle
\fi

\begin{abstract} 
This paper presents an alternative proof for the Dining Philosophers problem's impossibility of solving by deterministic protocol. The proof offered in this article is not novel; instead, it is an alternate way of providing the understanding that the protocols that intend to resolve the problem deterministically are bound to fail. By providing a new perspective, the paper aims to help computer science researchers and students understand the problem and the impossibility of deterministic protocols. Overall, the proof highlights the familiar idea that deterministic protocols for resolving the Dining Philosophers problem violate one or more of the necessary properties.
\end{abstract}

%
\abstract{We propose an alternative simple construction of good LTC codes. In contrast to previews, constructions made by \cite{Dinur}, \cite{leverrier2022quantum} and \cite{Pavel}, our construction doesn't require spicel properties of the small codes, such as $w$-robustness and $p$-resistance for puncturing.  
} 


\ifdefined\ACM
  \maketitle
\fi

%\begin{multicols*}{2}
% \section{Preambles}

Localy Testable Codes, or LTC, are error correction codes such that verfining a uinformly random cchoosen check, would be enough to detect any error with probability proportional to it's size. Bisdes the clear computional adventage they offer, they took roles at the eriler PCP proofs.  
  In this work, we propose a new construction for good LTC codes, which also have a good testability parameter. In the sense   Our proof also indirectly answers the following question. Why most of the good LDPC codes are known to be bad in terms of detecting errors? In other words, It seems that for most of them, there exist strings that are very far from being in the code and, meanwhile, fail to satisfy only a small number of restrictions.
  While the previous LDPC constructions focused on ensuring that the yielded code would have a good rate and distance parameters, our construction enforces the restrictions collection to have a nontrivial fraction of degeneration. That is, removing a single restriction will not change the code, as any restriction is linearly dependent on the others.



%\section{Introduction.}
.. bla bla bla.. bla bla ..     
\definition[General Entanglement State]{ We say that $\PSI$ is general entanglement \label{def:gEnt} if .. }

\definition[Local-Measure-Circuit] { We say that a quantum circuit $C$ is a local measure circuit \label{def:lmc} if it's can be described as a decomposition of poly classical circuit and a constant depth quantum circuit which contains only 1-qubit gates and measurements. 

We would think about local measure circuits as chip circuits. }

\definition[$p_{0}-\Delta$ Fault Tolerance Circuit]{ We say that $\mathcal{C}$ is $p_{0}-\Delta$ fault tolerance \label{def:gft} presentation of abstract circuit $C$ if there exists a local measure circuit $C_{0}$ \ref{lmc} such it's grunted that for noise $p < p_{0}$ $\mathcal{C}$ compute $C$ w.h.p,
And in addition, if $p \in \left( p_{0}, p_{0} + \varepsilon \right)$ then by applying a $C_{0}$ on $\mathcal{C}$ output yields a general entanglement state \label{def:gEnt}}       

\ctt{We would like to add a complexity parameter for the above definition, for example, ``a general entanglement state over more than $\frac{1}{5}$ of the qubits.}  


% commands taken form the original paper. 

\newcommand{\Oh}{{\mathcal{O}}}
\newcommand{\bitsize}{N}
\newcommand{\longversion}[1]{#1}
\newcommand{\abpartization}{{\sc Vertex $(r,\ell)$-Partization}}


\section{Todo.}
\begin{enumerate}
  \item Write the table (sage script). \ctt{all most done, it's left to handle the parameterized lines.}
  \item Add definitions. Problem description.  
  \item Complete the 'proof'. 
  \item Prove lower bound. 
  \item Add figures of covering the space by balls and cones. 
\end{enumerate}

\section{Introduction.} 
The Dining Philosophers problem is a classic synchronization problem in computer science, where a group of philosophers sit at a round table and alternate between thinking and eating. The problem arises when they share common resources (forks) and a set of rules must be established to prevent deadlocks and starvation.

The impossibility result for the deterministic case states that it is impossible to design a solution to the Dining Philosophers problem if the philosophers behave deterministically and the resource allocation is symmetric. This is because each philosopher would require the same resources as their neighbor at the same time, which leads to deadlocks.Michael Rabin proposed a solution to the randomized case, where the philosophers behave randomly in choosing which fork to pick up first. This randomization breaks the symmetry and prevents deadlocks.

Attempts have also been made to solve the problem using quantumness, with  \ctt{Adi Shamir and Avi Wigderson} proposing a quantum analog to the Dining Philosophers problem. In \ctt{2003, Dorit Aharonov} proposed a quantum solution to the problem, which involves using entanglement to share the forks between philosophers.

An example of a real-world application of the Dining Philosophers problem is in resource allocation in computer networks, where multiple nodes may need access to a shared resource. The problem can also be used as a teaching tool in computer science to illustrate the importance of synchronization and avoiding deadlocks.

An incident related to the problem occurred in \ctt{2008}, when a bug in the synchronization code of the iPhone's email application caused it to hang, leading to a flurry of frustrated complaints from iPhone users dubbed the "Dining Philosopher's bug".

\begin{theorem} 
 It is impossible to solve the Dining Philosophers problem for any number of philosophers larger than one.
\end{theorem}

%\begin{proof}
  We prove the theorem by induction. First, we show that the base case of two philosophers is impossible to solve.

\begin{claim}
  It is impossible to solve the Dining Philosophers problem for a pair of philosophers.
\end{claim}
\begin{proof} 
  Suppose there are two philosophers sitting at a table, each with a fork. Each philosopher needs both forks to eat, but only one fork is available to each philosopher. Thus, they will forever wait for the other philosopher to release the fork, resulting in a deadlock.
\end{proof}
\begin{claim} If there exists an odd number $n > 1$ such that there is a valid protocol to solve the Dining Philosophers problem for $n$ philosophers, then there exists an even number $n' = n + 1$ such that the problem can be solved by a protocol.
\end{claim}
\begin{proof}
Let $P_1, P_2, ..., P_n$ denote the philosophers sitting around the table, and let $F_1, F_2, ..., F_n$ denote the forks placed between them, with fork $F_i$ between philosophers $P_i$ and $P_{i+1}$ (with $P_{n+1} = P_1$). We add an additional fork $F_{n+1}$ between philosophers $P_1$ and $P_{\frac{n}{2}+1}$. That is, for $1\leq i\leq \frac{n}{2}$, fork $F_{n+i}$ is placed between philosophers $P_i$ and $P_{i+\frac{n}{2}}$.

Define the rules of the protocol as follows: 

\begin{enumerate}
  
  \item Each philosopher can only pick up one fork at a time.
  
  \item In order to eat, a philosopher must have both the fork to their left and the fork to their right in their possession.
  
  \item Philosophers $P_1, P_2, ..., P_n$ take turns picking up one fork at a time, in a clockwise direction starting from $P_1$. 
  
  \item When a philosopher has $F_1$ and $F_2$ in their possession, they can also pick up $F_{n+1}$ and eat.
  
  \item When a philosopher has $F_{n+1}$ in their possession, they must put it down and wait for their next turn to attempt to pick it up again.
  
\end{enumerate}

It is easy to verify that these rules ensure that the philosophers can eat when their respective two forks are available, and that there is no deadlock. Since there is a protocol for an odd number of philosophers, there must exist a protocol for an even number of philosophers.
\end{proof}

\begin{claim} 
  The existence of such even number follows a valid solution for the base case of only two philosophers.
\end{claim}
\begin{proof}
Suppose there is an even number $n = 2m$. We show that a protocol exists in which two Philosophers simulate the $n$ Philosophers in the original protocol.

Let the original protocol have $n$ Philosophers and $n$ forks, labeled $F_0, F_1, ..., F_{n-1}$, where $F_i$ is between Philosopher $i$ and Philosopher $(i+1) \mod n$. In the new protocol, the two simulating Philosophers, $P1$ and $P2$, will each be responsible for simulating a continuous half of the original Philosophers, i.e., $P1$ will simulate Philosophers $0$ to $m-1$ and $P2$ will simulate Philosophers $m$ to $n-1$. 

To do this, $P1$ will define $m-1$ imaginary forks, labeled $I_1, I_2, ..., I_{m-1}$, where $I_i$ is between $P1$ and $P2$. Similarly, $P2$ will define $m-1$ imaginary forks, labeled $I'_1, I'_2, ..., I'_{m-1}$, where $I'_i$ is also between $P1$ and $P2$. Note that $I_1$ and $I'_1$ are shared forks, held by both $P1$ and $P2$.
\begin{enumerate}
\item $P1$ and $P2$ both pick up their left-hand fork and $I_1$.
\item $P1$ picks up their right-hand fork and $I_2$.
\item $P2$ picks up their right-hand fork and $I'_2$.
\item Both $P1$ and $P2$ now have two forks, and can eat.
\item After eating, $P1$ and $P2$ put down their forks and $I_1$.
\item $P1$ puts down their right-hand fork and $I_2$.
\item $P2$ puts down their right-hand fork and $I'_2$.
\item $P1$ and $P2$ are now ready to repeat the process.
\end{enumerate}
It can be seen that this protocol never allows for a deadlock to occur, as both $P1$ and $P2$ are only ever holding two forks and either $P1$ or $P2$ will always be able to obtain another fork when they need it. Thus, the claim is proved.

\end{proof}



%f"{Groverize_complixity(c_values[i]):.3}^{{k}}"
%\begin{sagesilent}
%\end{sagesilent}

% \input{sagelocal.py}
% \begin{sagesilent}
%   c = 8 
%   d = 4
%   f(x) = (2 - (1/c))^x
%   g(x) = (2 - (1/(d*c))^2)^(x/2)
% \end{sagesilent}
% %\begin{figure}{H}
% \scalebox{0.8}{
%   \sageplot{plot(f, 0, 7, color = 'red')+ plot(g, 0, 7)}
% }
\printbibliography
\end{document}





