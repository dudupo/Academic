

%\newcommand*{\ACM}{}%

\ifdefined\ACM

%\documentclass[sigplan,screen]{acmart}
  \documentclass[manuscript,screen,review]{acmart}

\else
  \documentclass{article}
  \usepackage[utf8]{inputenc}
\usepackage[a4paper, total={6.5in, 10in} ]{geometry}
\usepackage{braket}
\usepackage{xcolor}
\usepackage{amsmath}
\usepackage{amssymb}
\usepackage{amsfonts}
\usepackage{graphicx}
\usepackage{svg}
\usepackage{float}
\usepackage{tikz}
\usetikzlibrary{patterns, shapes.arrows}
\usepackage{adjustbox}
\usepackage{tikz-network}
\usepackage[ruled,lined,linesnumbered]{algorithm2e}
\usepackage{multicol}
\usepackage[backend=biber,style=alphabetic,sorting=ynt]{biblatex}
\usepackage{xcolor}
\usepackage{pgfplots}
\DeclareUnicodeCharacter{2212}{−}
\usepgfplotslibrary{groupplots,dateplot}
\pgfplotsset{compat=newest}



  \addbibresource{./sample.bib} 
  \addbibresource{./exactalgo-rs.bib}
\fi

\begin{document}

\input{newcommands}

\title{Dining Philosophers (Short Note)} 
\author{David Ponarovsky}

\ifdefined\ACM
  \affiliation{%
    \institution{The Th{\o}rv{\"a}ld Group}
    \streetaddress{1 Th{\o}rv{\"a}ld Circle}
    \city{Hekla}
  \country{Iceland}}
  \email{larst@affiliation.org}
\else
  \maketitle
\fi

\begin{abstract} 
This paper presents an alternative proof for the Dining Philosophers problem's impossibility of solving by deterministic protocol. The proof offered in this article is not novel; instead, it is an alternate way of providing the understanding that the protocols that intend to resolve the problem deterministically are bound to fail. By providing a new perspective, the paper aims to help computer science researchers and students understand the problem and the impossibility of deterministic protocols. Overall, the proof highlights the familiar idea that deterministic protocols for resolving the Dining Philosophers problem violate one or more of the necessary properties.
\end{abstract}

%\begin{abstract}
  Quantum feasibility hinges on the assumption that the basic gate's noise rate is below a certain threshold. Here we study the behavior of computation models when the noise is slightly greater than that threshold. In particular, We ask if one can design a fault tolerance schema such that if the noise is above the threshold, it is still grunted that the final generated state would have a value. 
\end{abstract}

\ifdefined\ACM
  \maketitle
\fi

%\begin{multicols*}{2}
% \section{Preambles}
  In this work, we propose a new construction for good LDPC codes, which also have a good testability parameter. In the sense that verfining a constant number of random checks, would be enough to detect any error with probability proportional to the error size. In contrast to previews, constructions made by \cite{Dinur}, \cite{leverrier2022quantum} and \cite{Pavel}, our construction doesn't require spicel properties of the small codes, such as $w$-robustness and $p$-resistance for puncturing. 
  
  Our proof also indirectly answers the following question. Why most of the good LDPC codes are known to be bad in terms of detecting errors? In other words, It seems that for most of them, there exist strings that are very far from being in the code and, meanwhile, fail to satisfy only a small number of restrictions.
  While the previous LDPC constructions focused on ensuring that the yielded code would have a good rate and distance parameters, our construction enforces the restrictions collection to have a nontrivial fraction of degeneration. That is, removing a single restriction will not change the code, as any restriction is linearly dependent on the others.



%%Coding theory has emerged by the need to transfer information in noisy communication channels. By embedding a message in higher dimension space, one can guarantee robustness against possible faults. The ratio of the original content length to the passed message \emph{length} is the \emph{rate} of the code, and it measures how consuming our communication protocol is. Furthermore, the \emph{distance} of the code quantifies how many faults the scheme can absorb such that the receiver can recover the original message. We could consider the code as all the strings that satisfy a specified restrictions collection.
%  
%
%  Non-formally, code is good if its distance and rate are scaled linearly in the encoded message length. In practice, one is also interested in implementing those checks efficiently. We say that a code is an LDPC if any bit is involved in a constant number of restrictions, each of which is a linear equation, and if any restriction contains a fixed number of variables.
%
%  Furthermore, finally, another characteristic of the code is its testability, which is the complexity of the number of random checks one should do to negate that a given candidate is in the code. Besides good codes being considered efficient in terms of robustness and overhead, they are also vital components in establishing secure multiparty computation \cite{MultiParty} and have a deep connection to probabilistic proofs.
%
%  First, we state the notations, definitions, and formal theorem in section 2. Then in sections 3 and 4, we review past results and provide their proofs to make this paper self-contained. Readers familiar with the basic concepts of LDPC, Tanner, and Expanders codes construction should consider skipping directly to section 5, in which we provide our proof. 
%

Coding theory has emerged due to the need to transfer information in noisy communication channels. By embedding a message in a higher-dimensional space, one can guarantee robustness against possible faults. The ratio of the original content length to the transmitted message \emph{length} is the \emph{rate} of the code, and it measures how consuming our communication protocol is. Additionally, the \emph{distance} of the code quantifies how many faults the scheme can absorb such that the receiver can recover the original message. We can consider the code as a collection of all strings that satisfy specified restrictions.

Non-formally, a code is good if its distance and rate scale linearly with the encoded message length. In practice, one is also interested in implementing these checks efficiently. We say that a code is an LDPC if any bit is involved in a constant number of restrictions, each of which is a linear equation, and if any restriction contains a fixed number of variables.

Moreover, another characteristic of the code is its testability, which is the complexity of the number of random checks one must do to verify that a given candidate is in the code. Besides being considered efficient in terms of robustness and overhead, good codes are also vital components in establishing secure multiparty computation \cite{MultiParty} and in the proof of the PCP theorem~\cite{PCPoriginal}.

%In Section 2, we state the notations, definitions, and formal theorem. Then, in Sections 3 and 4, we review past results and provide their proofs to make this paper self-contained. Readers familiar with the basic concepts of LDPC, Tanner, and Expanders codes construction may consider skipping directly to Section 5, in which we provide our proof.
%Readers familiar with the basic concepts of LDPC, Tanner, and Expanders codes construction may skip Sections 2, 3, and 4 and proceed directly to Section 5, where we provide our proof.


% commands taken form the original paper. 

\newcommand{\Oh}{{\mathcal{O}}}
\newcommand{\bitsize}{N}
\newcommand{\longversion}[1]{#1}
\newcommand{\abpartization}{{\sc Vertex $(r,\ell)$-Partization}}


\section{Todo.}
\begin{enumerate}
  \item Check the events suggested by GPT.  
  \item Draw a nice Tikz figure of Dining Philosophers.   
  \item Complete the base case for $n=2$.  
\end{enumerate}

\section{Introduction.} 
The Dining Philosophers problem is a classic synchronization problem in computer science, where a group of philosophers sit at a round table and alternate between thinking and eating. The problem arises when they share common resources (forks) and a set of rules must be established to prevent deadlocks and starvation.

The impossibility result for the deterministic case states that it is impossible to design a solution to the Dining Philosophers problem if the philosophers behave deterministically and the resource allocation is symmetric. This is because each philosopher would require the same resources as their neighbor at the same time, which leads to deadlocks.Michael Rabin proposed a solution to the randomized case, where the philosophers behave randomly in choosing which fork to pick up first. This randomization breaks the symmetry and prevents deadlocks.

Attempts have also been made to solve the problem using quantumness, with  \ctt{Adi Shamir and Avi Wigderson} proposing a quantum analog to the Dining Philosophers problem. In \ctt{2003, Dorit Aharonov} proposed a quantum solution to the problem, which involves using entanglement to share the forks between philosophers.

An example of a real-world application of the Dining Philosophers problem is in resource allocation in computer networks, where multiple nodes may need access to a shared resource. The problem can also be used as a teaching tool in computer science to illustrate the importance of synchronization and avoiding deadlocks.

An incident related to the problem occurred in \ctt{2008}, when a bug in the synchronization code of the iPhone's email application caused it to hang, leading to a flurry of frustrated complaints from iPhone users dubbed the "Dining Philosopher's bug".

\begin{theorem} 
 It is impossible to solve the Dining Philosophers problem for any number of philosophers larger than one.
\end{theorem}

\section{The Proof.}

%\begin{proof}
  We prove the theorem by induction. First, we show that the base case of two philosophers is impossible to solve.

\begin{claim}
  It is impossible to solve the Dining Philosophers problem for a pair of philosophers.
\end{claim}
\begin{proof} 
  Suppose there are two philosophers sitting at a table, each with a fork. Each philosopher needs both forks to eat, but only one fork is available to each philosopher. Thus, they will forever wait for the other philosopher to release the fork, resulting in a deadlock.
\end{proof}

\subsection{Validity for Odds Number Yields Validity for Even Ones.}  

\begin{lemma} If there exists an odd number $n > 1$ such that there is a valid protocol to solve the Dining Philosophers problem for $n$ philosophers, then there exists an even number $n' = n+1$ such that the problem can be solved by a protocol.
\end{lemma}
\begin{proof}
  Let $P_1, P_2, ..., P_{n+1}$ denote the $n+1$ philosophers sitting around the table, and let $F_1, F_2, ..., F_{n+1}$ denote the forks placed between them. Consider the running of the $n$-philosophers protocol over $n'$-philosophers. That it, each philosopher execute exactly the same code as it would execute in the $n$-philosophers protocol. 

  \begin{claim} \label{claim:lr} 
Suppose there is a deadlock and all the forks are held, then either all the philosophers hold the fork to their right or all of them hold the fork to their left.
  \end{claim} 
  \begin{proof}
    Assume through contradiction otherwise, then there are must to be two adjoined philosophers $P_{i}$ and $P_{i+1}$ such one of them hold the fork to his left while the other hold the fork to his right. So either $F_{i+1}$ is held by both of them, which by definition cannot be true, or $F_{i}$ is free, which contradict the fact the second restriction of the claim. 
  \end{proof}
  \begin{claim} \label{claim:up} 
    We can assume when deadlock occur then all the forks are held by the philosophers. 
  \end{claim} 
  The proof of \cref{claim:up} is given in the Appendix. Combining \cref{claim:lr} we can assume w.l.g that each $P_{i}$ holds $F_{i+1}$. Now denote by $j$ the philosopher who is the last one to pick it's fork. We will show, that the protocol has a running on $n$ philosophers in which for any $i\neq j$, the $i$th philosopher sees the same local view as $P_{i}$ in the $n'$-protocol. Formally: 

  \begin{claim}
    For any running over $P_1, P_2 ... P_{n+1}$ that stuck in deadlock, there is a running over $n$ philosophers denoted by $P'_{i}$ where $i \in [n+1]/\{j\}$. Such that for any $i\neq j$ the view of $P_{i}$ and $P'_{i}$ are the same.     
  \end{claim}
 
  \begin{proof}
    For convenient we will use the notation $\Pi$ and $\pi$ to refer the protocols over $n'$ and $n$ philosophers. Choose $j$ to be the last philosopher to pick it's fork, and denote the time in which $j$ left's sibling start to act at $t_{0}$. Clearly, The running till time $t_{0}$ in $\Pi$  independent on $P_{j}$ and therefore $\Pi$ and $\pi$ identical till time $t_{0}$. Hence at time $t_{0}$ $P'_{j-1}$ pick his right fork, that because he sees the same view as $P_{j-1}$ see in $\Pi$ at time $t_{0}$. Sperate for the following tow cases: 
    \begin{enumerate}
      \item $P'_{j+1}$ act in $\pi$ the same as $P_{j+1}$ in $\Pi$ namely pick his right fork.  
      \item $P'_{j+1}$ act in $\pi$ differently, choose not pick at all. 
    \end{enumerate}
    not adjoint  And By our assumption that all of the philosophers hold their right fork, it follow that their is a running in which 
  \end{proof}

\end{proof}

\subsection{ Back From $2m$ Protocol to $2$-philosophers Protocol.}

  \begin{lemma} 
  The existence of such even number follows a valid solution for the base case of only two philosophers.
\end{lemma}
\begin{proof}
Suppose there is an even number $n = 2m$ for which the there is a valid protocol for $n$ philosophers. We show that a protocol exists in which two philosophers simulate the $n$ philosophers in the original protocol.

Let the original protocol have $n$ philosophers and $n$ forks, labeled $F_0, F_1, ..., F_{n-1}$, where $F_i$ is between philosopher $i$ and philosopher $(i+1) \mod n$. In the new protocol, the two simulating Philosophers, $P_1$ and $P_2$, will each be responsible for simulating a continuous half of the original Philosophers, i.e., $P_1$ will simulate Philosophers $0$ to $m-1$ and $P_2$ will simulate Philosophers $m$ to $n-1$. 

To do this, $P_1$ will define $m-1$ imaginary forks, labeled $I_1, I_2, \dots, I_{m-1}$, where $I_i$ matches the original fork $F_{i}$ for $i \le m-1$. Similarly, $P_2$ will define $m-1$ imaginary forks, labeled $I'_1, I'_2, \dots, I'_{m-1}$. Denote by $J_{0}$ and $J'_{0}$ the real shared forks between $P_1$ and $P_2$. Now each of the philosophers follows the $n$-protocol, such that when the $i$th philosopher picks a fork, then both $P_1$ and $P_2$ pick $I_i$ and $I'_i$ accordingly (or $J_0$, $J'_0$ if $i$ is zero). Notice that the protocol is symmetric.

Now, observe that any running of the $2$-philosophers protocol is also a valid instance of the $n$-philosophers protocol. Therefore any scenario in which deadlock or starvation occur either on real philosophers or imaginary ones matches a possible occurrence of the same incident while executing the $n$-philosophers protocol. We can conclude that any protocols that guarantee correctness for even numbers of philosophers yield a correct protocol for the base case.
\end{proof}



%f"{Groverize_complixity(c_values[i]):.3}^{{k}}"
%\begin{sagesilent}
%\end{sagesilent}

% \input{sagelocal.py}
% \begin{sagesilent}
%   c = 8 
%   d = 4
%   f(x) = (2 - (1/c))^x
%   g(x) = (2 - (1/(d*c))^2)^(x/2)
% \end{sagesilent}
% %\begin{figure}{H}
% \scalebox{0.8}{
%   \sageplot{plot(f, 0, 7, color = 'red')+ plot(g, 0, 7)}
% }
\printbibliography

\section{Appendix}
proof of \cref{claim:up}. 
\begin{proof}
  Let $\mathcal{A}$ be a protocol, and consider the protocol $\mathcal{A}^{\prime}$ in which each time that one of the philosopher is get stack then he pass a number with initial value $0$ through the net by sending it to the philosopher sit to his right. Then any philosopher upon receiving, increase the number by one, records it and resend the bit to his right sibling and so on. In the end the original writer of the message receive a number with value $n$. On recording $n$ numbers $1 ,2 ... n$ the philosopher knows that every one on the circle gets stuck (and it easy to see that is the only scenario in which a philosopher record $n$ different values). 

  At this point, $\mathcal{A}^{\prime}$ define the philosopher to pick any of the forks next to it if he can do so. Notice that if $\mathcal{A}$ is a deadlock free protocol so is $\mathcal{A}^{\prime}$.         
  \end{proof}

\end{document}





