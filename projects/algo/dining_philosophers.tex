

%\newcommand*{\ACM}{}%

\ifdefined\ACM

%\documentclass[sigplan,screen]{acmart}
  \documentclass[manuscript,screen,review]{acmart}

\else
  \documentclass{article}
  \usepackage[utf8]{inputenc}
\usepackage[a4paper, total={6.5in, 10in} ]{geometry}
\usepackage{braket}
\usepackage{xcolor}
\usepackage{amsmath}
\usepackage{amssymb}
\usepackage{amsfonts}
\usepackage{graphicx}
\usepackage{svg}
\usepackage{float}
\usepackage{tikz}
\usetikzlibrary{patterns, shapes.arrows}
\usepackage{adjustbox}
\usepackage{tikz-network}
\usepackage[ruled,lined,linesnumbered]{algorithm2e}
\usepackage{multicol}
\usepackage[backend=biber,style=alphabetic,sorting=ynt]{biblatex}
\usepackage{xcolor}
\usepackage{pgfplots}
\DeclareUnicodeCharacter{2212}{−}
\usepgfplotslibrary{groupplots,dateplot}
\pgfplotsset{compat=newest}



  \addbibresource{./sample.bib} 
  \addbibresource{./exactalgo-rs.bib}
\fi

\begin{document}

\input{newcommands}

\title{Dining Philosophers (Short Note)} 
\author{David Ponarovsky}

\ifdefined\ACM
  \affiliation{%
    \institution{The Th{\o}rv{\"a}ld Group}
    \streetaddress{1 Th{\o}rv{\"a}ld Circle}
    \city{Hekla}
  \country{Iceland}}
  \email{larst@affiliation.org}
\else
  \maketitle
\fi

\begin{abstract} 
In this paper, we improve a wide range of upper bounds for a variety of \textbf{NP} problems, by plugging Grover into the work made by \cite{fomin2015exact}. We emphasize that this work has only a technical value and does not present any new idea. Nevertheless, we think it is worth sharing how easy and straightforwardly integrating quantum might be. We coin the term Groverize, harvesting a quantum improvement by doing almost nothing.
\end{abstract}

%\begin{abstract}
  Quantum feasibility hinges on the assumption that the basic gate's noise rate is below a certain threshold. Here we study the behavior of computation models when the noise is slightly greater than that threshold. In particular, We ask if one can design a fault tolerance schema such that if the noise is above the threshold, it is still grunted that the final generated state would have a value. 
\end{abstract}

\ifdefined\ACM
  \maketitle
\fi

%\begin{multicols*}{2}
% \section{Preambles}
  In this work, we propose a new construction for good LDPC codes, which also have a good testability parameter. In the sense that verfining a constant number of random checks, would be enough to detect any error with probability proportional to the error size. In contrast to previews, constructions made by \cite{Dinur}, \cite{leverrier2022quantum} and \cite{Pavel}, our construction doesn't require spicel properties of the small codes, such as $w$-robustness and $p$-resistance for puncturing. 
  
  Our proof also indirectly answers the following question. Why most of the good LDPC codes are known to be bad in terms of detecting errors? In other words, It seems that for most of them, there exist strings that are very far from being in the code and, meanwhile, fail to satisfy only a small number of restrictions.
  While the previous LDPC constructions focused on ensuring that the yielded code would have a good rate and distance parameters, our construction enforces the restrictions collection to have a nontrivial fraction of degeneration. That is, removing a single restriction will not change the code, as any restriction is linearly dependent on the others.



%%Coding theory has emerged by the need to transfer information in noisy communication channels. By embedding a message in higher dimension space, one can guarantee robustness against possible faults. The ratio of the original content length to the passed message \emph{length} is the \emph{rate} of the code, and it measures how consuming our communication protocol is. Furthermore, the \emph{distance} of the code quantifies how many faults the scheme can absorb such that the receiver can recover the original message. We could consider the code as all the strings that satisfy a specified restrictions collection.
%  
%
%  Non-formally, code is good if its distance and rate are scaled linearly in the encoded message length. In practice, one is also interested in implementing those checks efficiently. We say that a code is an LDPC if any bit is involved in a constant number of restrictions, each of which is a linear equation, and if any restriction contains a fixed number of variables.
%
%  Furthermore, finally, another characteristic of the code is its testability, which is the complexity of the number of random checks one should do to negate that a given candidate is in the code. Besides good codes being considered efficient in terms of robustness and overhead, they are also vital components in establishing secure multiparty computation \cite{MultiParty} and have a deep connection to probabilistic proofs.
%
%  First, we state the notations, definitions, and formal theorem in section 2. Then in sections 3 and 4, we review past results and provide their proofs to make this paper self-contained. Readers familiar with the basic concepts of LDPC, Tanner, and Expanders codes construction should consider skipping directly to section 5, in which we provide our proof. 
%

Coding theory has emerged due to the need to transfer information in noisy communication channels. By embedding a message in a higher-dimensional space, one can guarantee robustness against possible faults. The ratio of the original content length to the transmitted message \emph{length} is the \emph{rate} of the code, and it measures how consuming our communication protocol is. Additionally, the \emph{distance} of the code quantifies how many faults the scheme can absorb such that the receiver can recover the original message. We can consider the code as a collection of all strings that satisfy specified restrictions.

Non-formally, a code is good if its distance and rate scale linearly with the encoded message length. In practice, one is also interested in implementing these checks efficiently. We say that a code is an LDPC if any bit is involved in a constant number of restrictions, each of which is a linear equation, and if any restriction contains a fixed number of variables.

Moreover, another characteristic of the code is its testability, which is the complexity of the number of random checks one must do to verify that a given candidate is in the code. Besides being considered efficient in terms of robustness and overhead, good codes are also vital components in establishing secure multiparty computation \cite{MultiParty} and in the proof of the PCP theorem~\cite{PCPoriginal}.

%In Section 2, we state the notations, definitions, and formal theorem. Then, in Sections 3 and 4, we review past results and provide their proofs to make this paper self-contained. Readers familiar with the basic concepts of LDPC, Tanner, and Expanders codes construction may consider skipping directly to Section 5, in which we provide our proof.
%Readers familiar with the basic concepts of LDPC, Tanner, and Expanders codes construction may skip Sections 2, 3, and 4 and proceed directly to Section 5, where we provide our proof.


% commands taken form the original paper. 

\newcommand{\Oh}{{\mathcal{O}}}
\newcommand{\bitsize}{N}
\newcommand{\longversion}[1]{#1}
\newcommand{\abpartization}{{\sc Vertex $(r,\ell)$-Partization}}


\section{Todo.}
\begin{enumerate}
  \item Write the table (sage script). \ctt{all most done, it's left to handle the parameterized lines.}
  \item Add definitions. Problem description.  
  \item Complete the 'proof'. 
  \item Prove lower bound. 
  \item Add figures of covering the space by balls and cones. 
\end{enumerate}

\section{Introduction.} 
The Dining Philosophers problem is a classic synchronization problem in computer science, where a group of philosophers sit at a round table and alternate between thinking and eating. The problem arises when they share common resources (forks) and a set of rules must be established to prevent deadlocks and starvation.

The impossibility result for the deterministic case states that it is impossible to design a solution to the Dining Philosophers problem if the philosophers behave deterministically and the resource allocation is symmetric. This is because each philosopher would require the same resources as their neighbor at the same time, which leads to deadlocks.Michael Rabin proposed a solution to the randomized case, where the philosophers behave randomly in choosing which fork to pick up first. This randomization breaks the symmetry and prevents deadlocks.

Attempts have also been made to solve the problem using quantumness, with  \ctt{Adi Shamir and Avi Wigderson} proposing a quantum analog to the Dining Philosophers problem. In \ctt{2003, Dorit Aharonov} proposed a quantum solution to the problem, which involves using entanglement to share the forks between philosophers.

An example of a real-world application of the Dining Philosophers problem is in resource allocation in computer networks, where multiple nodes may need access to a shared resource. The problem can also be used as a teaching tool in computer science to illustrate the importance of synchronization and avoiding deadlocks.

An incident related to the problem occurred in \ctt{2008}, when a bug in the synchronization code of the iPhone's email application caused it to hang, leading to a flurry of frustrated complaints from iPhone users dubbed the "Dining Philosopher's bug".

\begin{definition}[Parameterized Computable] 
Let $L$ be a problem family, and use the notation $P \in L$ to indicate that $P$ is an instance of $L$ (e.g., if $L$ is a $3$-SAT problem), and by $|P|$ to denote the length of the binary string encoding $P$. $L$ is said to be parameterized computable if there exists a mapping $\phi : L \rightarrow \mathbb{N}$ and an algorithm $\mathcal{A}$ such that:

  \begin{enumerate}
    \item $\phi(P) \in [|P|]$ for any $P\in L$
    \item $\mathcal{A}$ returns on $P$ at running time $\mathcal{O}\left(c^{\phi(P)} \cdot Poly(|P|) \right)$ for a fixed $c$ which does not depend on $P$.
  \end{enumerate}
\end{definition}
 



%f"{Groverize_complixity(c_values[i]):.3}^{{k}}"
%\begin{sagesilent}
%\end{sagesilent}

% \input{sagelocal.py}
% \begin{sagesilent}
%   c = 8 
%   d = 4
%   f(x) = (2 - (1/c))^x
%   g(x) = (2 - (1/(d*c))^2)^(x/2)
% \end{sagesilent}
% %\begin{figure}{H}
% \scalebox{0.8}{
%   \sageplot{plot(f, 0, 7, color = 'red')+ plot(g, 0, 7)}
% }
\printbibliography
\end{document}





