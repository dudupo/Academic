\documentclass{article}


\usepackage[utf8]{inputenc}
\usepackage[a4paper, total={6.5in, 10in} ]{geometry}
\usepackage{braket}
\usepackage{xcolor}
\usepackage{amsmath}
\usepackage{amssymb}
\usepackage{amsfonts}
\usepackage{graphicx}
\usepackage{svg}
\usepackage{float}
\usepackage{tikz}
\usetikzlibrary{patterns, shapes.arrows}
\usepackage{adjustbox}
\usepackage{tikz-network}
\usepackage[ruled,lined,linesnumbered]{algorithm2e}
\usepackage{multicol}
\usepackage[backend=biber,style=alphabetic,sorting=ynt]{biblatex}
\usepackage{xcolor}
\usepackage{pgfplots}
\DeclareUnicodeCharacter{2212}{−}
\usepgfplotslibrary{groupplots,dateplot}
\pgfplotsset{compat=newest}



\addbibresource{../general-tex/sample.bib} %Import the bibliography file
\input{../general-tex/newcommands.tex}
\begin{document}


\title{Recursion Code.} 
\author{David Ponarovsky}
\maketitle
\begin{abstract}None 
\end{abstract}

\section{Construction.}
\begin{definition} 
  Let $\Delta$ be an integer greater than $2$ and consider an algorithm $\mathcal{A}$ that for any $n$ that is power of $3$ construct a $\Delta$-regular graph over $n$ vertices. Now, let $G$ be $\Delta$-regular graph over $n$ vertices generated by $\mathcal{A}$. Define the \textbf{third graph obtained by $G$ } , labeled by $G^{\sim}$ to be the graph which $\mathcal{A}$ returns over $\frac{1}{3} n$ such that any of the edges could be associate by puncturing a $\frac{2}{3}$ fraction of the edges of each vertex.  
\end{definition}

\begin{definition} Let $C_{0}=\Delta[1,\rho_{0}, \delta{0}]$ be a binary linear code.  We will define the recursion code in recursive manner. First for a sufficiently large integer $n_{0}$, which is also power of $3$, $C\left( n_{0} \right)$ defined to be the Tanner code defined by the $C_{0}$ and graph $\mathcal{A}\left( n_{0} \right)$. Then let $n$ be any power of $3$, such that $n > n_{0}$, denote by $G$ the graph that constructed by the running of $\mathcal{A}\left( n \right)$.  Then let $C\left( n \right)$ be the code obtained by the joining the parity check matrix of the Tanner code $\Tann$ and by the checks of the $C\left( n / 3 \right)$ over the bits associated with the$G^{\sim}$.     
\end{definition}
\printbibliography 
\end{document}



