\documentclass{article}


\usepackage[utf8]{inputenc}
\usepackage[a4paper, total={6in, 9in}]{geometry}
\usepackage{braket}
\usepackage{xcolor}
\usepackage{amsmath}
\usepackage{amsfonts}
\usepackage{amsthm}
\usepackage{amssymb}
%\usepackage[ocgcolorlinks]{hyperref}
\usepackage{hyperref}
%\usepackage{hyperref,xcolor}
%\usepackage[ocgcolorlinks]{ocgx2}
\usepackage{cleveref}
\usepackage{graphicx}
\usepackage{svg}
\usepackage{float}
\usepackage{tikz}
\usetikzlibrary{patterns, shapes.arrows}
\usepackage{adjustbox}
%\usepackage{tikz-network}
\usepackage{tkz-graph}
\usepackage{tkz-berge}
\usepackage[linesnumbered]{algorithm2e}
\usepackage{multicol}
\usepackage[backend=biber,style=alphabetic,sorting=ynt]{biblatex}
%\usepackage{xcolor}
%\usepackage{tkz-berge}
%\usepackage{tkz-graph}
\usepackage{pgfplots}
\usepackage{sagetex}
\usepackage{setspace}
\usepackage{etoc}
%\usepackage{wrapfig}
\usepackage{pgfgantt}
\DeclareUnicodeCharacter{2212}{−}
\usepgfplotslibrary{groupplots,dateplot}
\pgfplotsset{compat=newest}

\newtheorem{theorem}{Theorem}
\newtheorem{definition}{Definition}
\newtheorem{example}{Example}
\newtheorem{claim}{Claim}
\newtheorem{fact}{Fact}
\newtheorem{remark}{Remark}
\newtheorem*{theorem*}{Theorem}
\newtheorem{lemma}{Lemma}
\crefname{lemma}{Lemma}{Lemmas}
\hypersetup{colorlinks=true}
% , allcolors=blue,allbordercolors=blue,pdfborderstyle={0 0 1}}
%\hypersetup{pdfborder={2 2 2}}
% pdfpagemode=FullScreen,
% backref 

\newtheorem{problem}{Problem}
\crefname{problem}{Problem}{Problems}

\DeclareMathOperator{\Ima}{Im}


\addbibresource{../general-tex/sample.bib} %Import the bibliography file
\newcommand{\commentt}[1]{\textcolor{blue}{ \textbf{[COMMENT]} #1}}
\newcommand{\ctt}[1]{\commentt{#1}}
\newcommand{\prb}[1]{ \mathbf{Pr} \left[ #1 \right]}
\newcommand{\prbm}[2]{ \mathbf{Pr}_{ #2 }\left[ #1 \right]}
\newcommand{\prbc}[3]{ \mathbf{Pr}_{ #2 }\left[ #1 \right | #3]}
\newcommand{\prbcprb}[3]{ \prbc{#2}{#1}{#3} \cdot \prb{#3} } 
\newcommand{\expp}[1]{ \mathbf{E} \left[ {#1} \right]}
\newcommand{\onotation}[1]{\(\mathcal{O} \left( {#1}  \right) \)}
\newcommand{\ona}[1]{\onotation{#1}}
\newcommand{\PSI}{{\ket{\psi}}}
\newcommand{\xij} { X_{ij} } 
\DeclareMathOperator{\Ima}{Im}
%\newcommand{\LESn}{\ket{\psi_n}}
%\newcommand{\LESa}{\ket{\phi_n}}
%\newcommand{\LESs}{\frac{1}{\sqrt{n}}\sum_{i}{\ket{\left(0^{i}10^{n-i}\right)^{n}}}}
%\newcommand{\Hn}{\mathcal{H}_{n}}
%\newcommand{\Ep}{\frac{1}{\sqrt{2^n}}\sum^{2^n}_{x}{ \ket{xx}}}
%\newcommand{\HON}{\ket{\psi_{\text{honest}}}}
%\newcommand{\Lemma}{\paragraph{Lemma.}}
\newcommand{\Cpa}{[n, \rho n, \delta n]}
%\setlength{\columnsep}{0.6cm}
\newcommand{\Jvv}{ \bar{J_{v}} } 
\newcommand{\Cvv}{ \tilde{C_{v}} } 

\newcommand{\Gz}{ G_{z}^{\delta} } 
\newcommand{ \Tann } {  \mathcal{T}\left( G, C_0 \right) }
\newcommand{\ireducable}{ireducable \hyperref[ire]{[\ref{ire}]} }
\newcommand{\cutUU}{E(U_{-1} \bigcup U_{+1} ,U)} 
\newcommand{\wcutUU}{w\left( E(U_{-1} \bigcup U_{+1} ,U)  \right)}
\newcommand{\testgo}{  \mathcal{T}\left(J, q , C_{0}\right) } 

\newcommand{\duC}{\left( C_{A}^{\perp}\otimes C_{B}^{\perp} \right)^{\perp}}
\newcommand{\duduC}{\left( C_{A}\otimes C_{B}\right)^{\perp}}
  




\begin{document}


\title{Recursion Code.} 
\author{David Ponarovsky}
\maketitle
\begin{abstract}None 
\end{abstract}

\section{Construction.}
\begin{definition} 
  Let $\Delta$ be an integer greater than $2$ and consider an algorithm $\mathcal{A}$ that for any $n$ that is power of $3$ construct a $\Delta$-regular graph over $n$ vertices. Now, let $G$ be $\Delta$-regular graph over $n$ vertices generated by $\mathcal{A}$. Define the \textbf{third graph obtained by $G$ } , labeled by $G^{\sim}$ to be the graph which $\mathcal{A}$ returns over $\frac{1}{3} n$ such that any of the edges could be associate by puncturing a $\frac{2}{3}$ fraction of the edges of each vertex.  
\end{definition}

\begin{definition}
  Let $G$ be a $\Delta$-regular graph, such that each edge is associated with integer in $[\Delta]$ and no vertex adjoints to two different indexed edges. For example consider a Cayley graph defined by $\frac{1}{2}\Delta$ generators, then undirected graph is $\Delta$-regular and any edge could be labeled by the corresponding number. Define the \textbf{$[a,b]$-fraction graph obtained by $G$ }, labeled by $G^{[a,b]}$ to be the graph which taken 
\end{definition}

\begin{definition}[Recursion Code] Let $C_{0}=\Delta[1,\rho_{0}, \delta_{0}]$ be a binary linear code.  We will define the recursion code in recursive manner. First for a sufficiently large integer $n_{0}$, which is also power of $3$, $C\left( n_{0} \right)$ defined to be the Tanner code defined by the $C_{0}$ and graph $\mathcal{A}\left( n_{0} \right)$. Then let $n$ be any power of $3$, such that $n > n_{0}$, denote by $G$ the graph that constructed by the running of $\mathcal{A}\left( n \right)$.  Then let $C\left( n \right)$ be the code obtained by the joining the parity check matrix of the Tanner code $\Tann$ and by the checks of the $C\left( n / 3 \right)$ over the bits associated with the $G^{\sim}$. We will call to that code family the \textbf{recursion code}.
\end{definition}
\begin{lemma}
  If $\rho_{0} > \frac{2}{3}$, then the recursion code has a positive rate.
\end{lemma}
\begin{proof}
By counting the restrictions we have that:  
\begin{equation*}
  \begin{split}
    H\left( n \right) = \Delta n \left( 1-\rho_{0} \right) + H\left( n/3 \right) \le \frac{3}{2}\Delta\left( 1 - \rho_{0} \right)\Delta n 
  \end{split}
\end{equation*}
So we dimension of the code is at least  $ \frac{1}{2}\Delta n - H\left( n \right) $ which is 
\begin{equation*}
  \begin{split}
    \frac{1}{2}n\Delta - \frac{3}{2}\Delta\left( 1 - \rho_{0} \right)\Delta n = \frac{1}{2}\Delta n\left(  3 \rho_{0} - 2  \right) 
  \end{split}
\end{equation*}
So for any $\rho_{0} > \frac{2}{3}$ we have that the rate of the $C_{n}$ is grater than constant. 
\end{proof}

\paragraph{Recursion Decoder.}  balabla

\begin{algorithm}[H]
  Decode $G^{\frac{2}{3}+}$ \\
  Decode $G^{\frac{2}{3}-}$ \\
  Decode $G^{\frac{1}{3}}$ \\
\end{algorithm}

\printbibliography 
\end{document}



