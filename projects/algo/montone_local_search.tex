

%\newcommand*{\ACM}{}%

\ifdefined\ACM

%\documentclass[sigplan,screen]{acmart}
  \documentclass[manuscript,screen,review]{acmart}

\else
  \documentclass{article}
  \usepackage[utf8]{inputenc}
\usepackage[a4paper, total={6in, 9in}]{geometry}
\usepackage{braket}
\usepackage{xcolor}
\usepackage{amsmath}
\usepackage{amsfonts}
\usepackage{amsthm}
\usepackage{amssymb}
%\usepackage[ocgcolorlinks]{hyperref}
\usepackage{hyperref}
%\usepackage{hyperref,xcolor}
%\usepackage[ocgcolorlinks]{ocgx2}
\usepackage{cleveref}
\usepackage{graphicx}
\usepackage{svg}
\usepackage{float}
\usepackage{tikz}
\usetikzlibrary{patterns, shapes.arrows}
\usepackage{adjustbox}
%\usepackage{tikz-network}
\usepackage{tkz-graph}
\usepackage{tkz-berge}
\usepackage[linesnumbered]{algorithm2e}
\usepackage{multicol}
\usepackage[backend=biber,style=alphabetic,sorting=ynt]{biblatex}
%\usepackage{xcolor}
%\usepackage{tkz-berge}
%\usepackage{tkz-graph}
\usepackage{pgfplots}
\usepackage{sagetex}
\usepackage{setspace}
\usepackage{etoc}
%\usepackage{wrapfig}
\usepackage{pgfgantt}
\DeclareUnicodeCharacter{2212}{−}
\usepgfplotslibrary{groupplots,dateplot}
\pgfplotsset{compat=newest}

\newtheorem{theorem}{Theorem}
\newtheorem{definition}{Definition}
\newtheorem{example}{Example}
\newtheorem{claim}{Claim}
\newtheorem{fact}{Fact}
\newtheorem{remark}{Remark}
\newtheorem*{theorem*}{Theorem}
\newtheorem{lemma}{Lemma}
\crefname{lemma}{Lemma}{Lemmas}
\hypersetup{colorlinks=true}
% , allcolors=blue,allbordercolors=blue,pdfborderstyle={0 0 1}}
%\hypersetup{pdfborder={2 2 2}}
% pdfpagemode=FullScreen,
% backref 

\newtheorem{problem}{Problem}
\crefname{problem}{Problem}{Problems}

\DeclareMathOperator{\Ima}{Im}


  \addbibresource{./sample.bib} 
  \addbibresource{./exactalgo-rs.bib}
\fi

\begin{document}

\newcommand{\commentt}[1]{\textcolor{blue}{ \textbf{[COMMENT]} #1}}
\newcommand{\ctt}[1]{\commentt{#1}}
\newcommand{\prb}[1]{ \mathbf{Pr} \left[ #1 \right]}
\newcommand{\prbm}[2]{ \mathbf{Pr}_{ #2 }\left[ #1 \right]}
\newcommand{\prbc}[3]{ \mathbf{Pr}_{ #2 }\left[ #1 \right | #3]}
\newcommand{\prbcprb}[3]{ \prbc{#2}{#1}{#3} \cdot \prb{#3} } 
\newcommand{\expp}[1]{ \mathbf{E} \left[ {#1} \right]}
\newcommand{\onotation}[1]{\(\mathcal{O} \left( {#1}  \right) \)}
\newcommand{\ona}[1]{\onotation{#1}}
\newcommand{\PSI}{{\ket{\psi}}}
\newcommand{\xij} { X_{ij} } 
\DeclareMathOperator{\Ima}{Im}
%\newcommand{\LESn}{\ket{\psi_n}}
%\newcommand{\LESa}{\ket{\phi_n}}
%\newcommand{\LESs}{\frac{1}{\sqrt{n}}\sum_{i}{\ket{\left(0^{i}10^{n-i}\right)^{n}}}}
%\newcommand{\Hn}{\mathcal{H}_{n}}
%\newcommand{\Ep}{\frac{1}{\sqrt{2^n}}\sum^{2^n}_{x}{ \ket{xx}}}
%\newcommand{\HON}{\ket{\psi_{\text{honest}}}}
%\newcommand{\Lemma}{\paragraph{Lemma.}}
\newcommand{\Cpa}{[n, \rho n, \delta n]}
%\setlength{\columnsep}{0.6cm}
\newcommand{\Jvv}{ \bar{J_{v}} } 
\newcommand{\Cvv}{ \tilde{C_{v}} } 

\newcommand{\Gz}{ G_{z}^{\delta} } 
\newcommand{ \Tann } {  \mathcal{T}\left( G, C_0 \right) }
\newcommand{\ireducable}{ireducable \hyperref[ire]{[\ref{ire}]} }
\newcommand{\cutUU}{E(U_{-1} \bigcup U_{+1} ,U)} 
\newcommand{\wcutUU}{w\left( E(U_{-1} \bigcup U_{+1} ,U)  \right)}
\newcommand{\testgo}{  \mathcal{T}\left(J, q , C_{0}\right) } 

\newcommand{\duC}{\left( C_{A}^{\perp}\otimes C_{B}^{\perp} \right)^{\perp}}
\newcommand{\duduC}{\left( C_{A}\otimes C_{B}\right)^{\perp}}
  





\title{Groverize Monotone Local Search. (Short Note)} 
\author{David Ponarovsky}

\ifdefined\ACM
  \affiliation{%
    \institution{The Th{\o}rv{\"a}ld Group}
    \streetaddress{1 Th{\o}rv{\"a}ld Circle}
    \city{Hekla}
  \country{Iceland}}
  \email{larst@affiliation.org}
\else
  \maketitle
\fi

\begin{abstract} 
In this paper, we improve a wide range of upper bounds for a variety of \textbf{NP} problems, by plugging Grover into the work made by \cite{fomin2015exact}. We emphasize that this work has only a technical value and does not present any new idea. Nevertheless, we think it is worth sharing how easy and straightforwardly integrating quantum might be. We coin the term Groverize, harvesting a quantum improvement by doing almost nothing.
\end{abstract}

%
\abstract{We propose an alternative simple construction of good LTC codes. In contrast to previews, constructions made by \cite{Dinur}, \cite{leverrier2022quantum} and \cite{Pavel}, our construction doesn't require spicel properties of the small codes, such as $w$-robustness and $p$-resistance for puncturing.  
} 


\ifdefined\ACM
  \maketitle
\fi

%\begin{multicols*}{2}
% \section{Preambles}

Localy Testable Codes, or LTC, are error correction codes such that verfining a uinformly random cchoosen check, would be enough to detect any error with probability proportional to it's size. Bisdes the clear computional adventage they offer, they took roles at the eriler PCP proofs.  
  In this work, we propose a new construction for good LTC codes, which also have a good testability parameter. In the sense   Our proof also indirectly answers the following question. Why most of the good LDPC codes are known to be bad in terms of detecting errors? In other words, It seems that for most of them, there exist strings that are very far from being in the code and, meanwhile, fail to satisfy only a small number of restrictions.
  While the previous LDPC constructions focused on ensuring that the yielded code would have a good rate and distance parameters, our construction enforces the restrictions collection to have a nontrivial fraction of degeneration. That is, removing a single restriction will not change the code, as any restriction is linearly dependent on the others.



%\section{Introduction.}
.. bla bla bla.. bla bla ..     
\definition[General Entanglement State]{ We say that $\PSI$ is general entanglement \label{def:gEnt} if .. }

\definition[Local-Measure-Circuit] { We say that a quantum circuit $C$ is a local measure circuit \label{def:lmc} if it's can be described as a decomposition of poly classical circuit and a constant depth quantum circuit which contains only 1-qubit gates and measurements. 

We would think about local measure circuits as chip circuits. }

\definition[$p_{0}-\Delta$ Fault Tolerance Circuit]{ We say that $\mathcal{C}$ is $p_{0}-\Delta$ fault tolerance \label{def:gft} presentation of abstract circuit $C$ if there exists a local measure circuit $C_{0}$ \ref{lmc} such it's grunted that for noise $p < p_{0}$ $\mathcal{C}$ compute $C$ w.h.p,
And in addition, if $p \in \left( p_{0}, p_{0} + \varepsilon \right)$ then by applying a $C_{0}$ on $\mathcal{C}$ output yields a general entanglement state \label{def:gEnt}}       

\ctt{We would like to add a complexity parameter for the above definition, for example, ``a general entanglement state over more than $\frac{1}{5}$ of the qubits.}  


% commands taken form the original paper. 

\newcommand{\Oh}{{\mathcal{O}}}
\newcommand{\bitsize}{N}
\newcommand{\longversion}[1]{#1}
\newcommand{\abpartization}{{\sc Vertex $(r,\ell)$-Partization}}


\section{Todo.}
\begin{enumerate}
  \item Write the table (sage script).
  \item Add definitions. Problem description.  
  \item Complete the 'proof'. 
  \item Prove lower bound. 
\end{enumerate}

\section{Introduction.} We follow the study of \cite{fomin2015exact}, who relate between the parametrized complexity to the general average case complexity. Crudely put, they shown that for particular wide range of \textbf{NP} hard problems, a solution which run exponentially at some complexity parameter, for example the tree-width of a graph, can be used to derive a batter than bruteforce solution for the general problem. We continue their work by plugin the Grover search   \cite{grover1996fast} routine instead the original sampling process.  We will simplify the definitions given at \cite{fomin2015exact} and use the following definitions instead:
 
A decision problem is said to have a parameterized algorithm if there is a mapping between its instances and the natural number $k$ such that there exists an algorithm that solves the problem in running time that is exponential in $k$ and polynomial in $n$. 

We will say that a problem having a parametrized algorithms is an \emph{extension problem} if for any instance of the problem $P$, requiring any of the input bits to be $1$ can be reduced to another instance of the problem $P^{\prime}$ such that $\phi\left( P^{\prime} \right) = \phi\left( P \right) - 1$. For example, consider $3$-\textbf{SAT} with the restriction that the Hamming weight of the assignment would be at most $k$. Fixing an arbitrary bit $x_{i}$ to be $1$ can be reduced to another $3$-\textbf{SAT} formula by erasing any of the clauses containing $x_{i}$ and replacing any of the occurrences of $\bar{x_{i}}$ by another terminal on the same clause (i.e. $ \bar{x_{i}} \wedge  \bar{y} \wedge z  \mapsto  \bar{y} \wedge \bar{y} \wedge z$). Now, note that an assignment that satisfies the new formula at Hamming weight at most $k-1$ combined with $x_{i} \leftarrow 1$ is an assignment to the original formula at weight at most $k$. Given the fact that we have a brute-force algorithm which tries all the partitions in time roughly $\mathcal{O}\left( n^{k} \right)$, it follows that this problem is an extension problem. Let us state the above:

\begin{definition}[Parameterized Computable] 
Let $L$ be a problem family, and use the notation $P \in L$ to indicate that $P$ is an instance of $L$ (e.g., if $L$ is a $3$-SAT problem), and by $|P|$ to denote the length of the binary string encoding $P$. $L$ is said to be parameterized computable if there exists a mapping $\phi : L \rightarrow \mathbb{N}$ and an algorithm $\mathcal{A}$ such that:

  \begin{enumerate}
    \item $\phi(P) \in [|P|]$ for any $P\in L$
    \item $\mathcal{A}$ returns on $P$ at running time $\mathcal{O}\left(c^{\phi(P)} \cdot Poly(|P|) \right)$ for a fixed $c$ which does not depend on $P$.
  \end{enumerate}
\end{definition}

\begin{definition}[extension problem]
A problem family $L$ which is parametrized computable will be said an extension problem, if for any $P \in L$, the problem obtained from $P$ by setting one of the input bits to be $1$ can be converted by polynomial-time reduction to another instance of $L$, denoted as $P^{\prime} \in L$, such that $\phi\left( P^{\prime} \right) = \phi\left( P \right) -1$.
\end{definition}

\paragraph{The General Idea.} \ctt{add a description of their solution}. 

Consider a decision problem inside \textbf{NP}. In this paper, we will associate two verifiers $U$ and $V$ with each language. $U$ stands for input validation and conceptually it is used for checking that the solution is valid within the problem world. For example, for the $3$-\textbf{SAT}, $U$ checks that the input indeed encodes an assignment. Formally, the role of $U$ is to restrict the inputs to a certain form. $V$ is responsible for verifying that the word is indeed in the language, i.e. it checks that the assignment satisfies the formula.


Any instance of the problem can be represented as the bit-wise union of two strings which pass $U$ verification. For example, any assignment satisfying a  $3$-\textbf{SAT} instance can be written as an or-wise of two assignments.


\begin{equation*}
  \begin{split}    
    & \sum_{k' \leq k}   \frac{1}{ \sqrt{ p(k') } } \cdot c^{k'-t} \bitsize^{\Oh(1)} \leq  \max_{k' \leq k} \left( \frac{{n - |X| \choose t}}{{k' \choose t}} \right)^{\frac{1}{2}} \cdot c^{k'-t} \bitsize^{\Oh(1)} = \\ 
    & \left( \max_{k' \leq k} \frac{{n - |X| \choose t}}{{k' \choose t}}  \cdot c^{2 \left( k'-t \right) } \right)^{\frac{1}{2}} \bitsize^{\Oh(1) } =  \left( \max_{k \leq n-|X|} \frac{{n - |X| \choose t}}{{k \choose t}} \right)^{\frac{1}{2}} \cdot c^{ 2 \left(  k-t \right) } \bitsize^{\Oh(1)} \le \\ 
    & \Rightarrow  \left(2-\frac{1}{c^{2}}\right)^{ \frac{ n-|X|}{2} }\bitsize^{\Oh(1)}
  \end{split}
\end{equation*}


%f"{Groverize_complixity(c_values[i]):.3}^{{k}}"
%\begin{sagesilent}
%\end{sagesilent}

\input{sagelocal.py}
\begin{sagesilent}
  c = 8 
  d = 4
  f(x) = (2 - (1/c))^x
  g(x) = (2 - (1/(d*c))^2)^(x/2)
\end{sagesilent}
%\begin{figure}{H}
\scalebox{0.8}{
  \sageplot{plot(f, 0, 7, color = 'red')+ plot(g, 0, 7)}
}
%\end{figure}
\newcommand{\tcite}[1]{\hfill\cite{#1}}
%\scalebox{0.8}{
  \begin{table}[H]
    \centering
    \setlength{\tabcolsep}{4pt}
    {\footnotesize
      \begin{tabular}{l l l l l}
        \toprule
        Problem Name                                  & Parameterized                                 & Groverize                    & New bound                                                                  & Previous Bound       \\
        \midrule
        {\sc Feedback Vertex Set}                     & $3^k$ (r) \tcite{cut-and-count}               & $\sagestr{next(genr)} $ & $1.6667^n$   (r)                                                           & \\
        {\sc Feedback Vertex Set}                     & $3.592^k$            \tcite{KociumakaP13}     & $\sagestr{next(genr)} $                          & $1.7217^n$                                                                 & $1.7347^n$ \tcite{FominTV15}  \\
        {\sc Subset Feedback Vertex Set}              & $4^k$         \tcite{Wahlstrom14}             & $\sagestr{next(genr)} $                           & $1.7500^n$                                                                 & $1.8638^n$ \tcite{FominHKPV14}  \\
        {\sc Feedback Vertex Set in Tournaments}     & $1.6181^k$        \tcite{KumarL16}            & $\sagestr{next(genr)} $ & $1.3820^n$                   & $1.4656^n$  \tcite{KumarL16}  \\
        {\sc  Group Feedback Vertex Set}             & $4^k$           \tcite{Wahlstrom14}           & $\sagestr{next(genr)} $ & $1.7500^n$                   & NPR    \\
        \textsc{Node Unique Label Cover}             & $|\Sigma|^{2k}$           \tcite{Wahlstrom14} & $\sagestr{next(genr)} $ & $(2-\frac{1}{|\Sigma|^2})^n$ & NPR    \\
        {\abpartization} ($r,\ell \leq 2$)           & $3.3146^k$   \tcite{BasteFKS15,KolayP15}      & $\sagestr{next(genr)} $ & $1.6984^n$                   & NPR  \\
      %\textsc{Odd Cycle Transversal}              & $2.3146^k$  \cite{LokshtanovNRRS14}           & $\sagestr{next(genr)} $ & $1.5680^n$                   & $1.4391^n$  \\
        {\sc Interval Vertex Deletion}               & $8^k$       \tcite{Cao8kinterval}             & $\sagestr{next(genr)} $ & $1.8750^n$                   & $(2-\varepsilon)^n$ for $\varepsilon <10^{-20}$  \tcite{BliznetsFPV13} \\
        {\sc Proper Interval Vertex Deletion}        & $6^k$           \tcite{HofV13,Cao15}          & $\sagestr{next(genr)} $ & $1.8334^n$                   & $(2-\varepsilon)^n$ for $\varepsilon <10^{-20}$  \tcite{BliznetsFPV13} \\
        {\sc Block Graph Vertex Deletion}            & $4^k$   \tcite{AgrawalLKS16}                  & $\sagestr{next(genr)} $ & $1.7500^n$                   & $(2-\varepsilon)^n$ for $\varepsilon <10^{-20}$  \tcite{BliznetsFPV13}  \\
      %{\sc Cograph Deletion}                      & $??^k$                                        & $\sagestr{next(genr)} $ &                              & deterministic    \\
        {\sc   Cluster Vertex Deletion}              & $1.9102^k$      \tcite{BoralCKP14}            & $\sagestr{next(genr)} $ & $1.4765^n$                   & $1.6181^n$  \tcite{FominGKLS10}  \\
        {\sc   Thread Graph Vertex Deletion}         & $8^k$    \tcite{Kante0KP15}                   & $\sagestr{next(genr)} $ & $1.8750^n$                   & NPR    \\
        {\sc   Multicut on Trees}                    & $1.5538^k$  \tcite{KanjLLTXXYZZZ14}           & $\sagestr{next(genr)} $ & $1.3565^n$                   & NPR    \\
      %{\sc    Vertex Cover}                       & $1.2738^k$ \cite{ChenKX10}                    & $\sagestr{next(genr)} $ & $1.2150^n$  (DNI)            & $1.1996^n$  \cite{XiaoN13}  \\
        {\sc    $3$-Hitting Set}                     & $2.0755^k$    \tcite{MagnusPhD07}             & $\sagestr{next(genr)} $ & $1.5182^n$                   & $1.6278^n$    \tcite{MagnusPhD07}  \\
        {\sc   $4$-Hitting Set}                      & $3.0755^k$      \tcite{FominGKLS10}           & $\sagestr{next(genr)} $ & $1.6750^n$                   & $1.8704^n$ \tcite{FominGKLS10}     \\
        {\sc   $d$-Hitting Set} ($d\geq 3$)          & $(d-0.9245)^k$     \tcite{FominGKLS10}        & $\sagestr{next(genr)} $ & $(2-\frac{1}{(d-0.9245)})^n$ & \tcite{CochefertCGK16,FominGKLS10}   \\
      %{\sc    Weighted $2$-SAT}                   & $1.2738^k$     \cite{MisraNRS10}              & $\sagestr{next(genr)} $ & $1.2150^n$  (DNI)            & $1.1996^n$    \\ \hline
        {\sc    Min-Ones $3$-SAT}                    & $2.562^k$    \tcite{abs-1007-1166}            & $\sagestr{next(genr)} $ & $1.6097^n$                   & NPR   \\
        {\sc    Min-Ones $d$-SAT} ($d\geq 4$)        & $d^k$                                         & $\sagestr{next(genr)} $ & $(2-\frac{1}{d})^n$          & NPR    \\
        {\sc    Weighted $d$-SAT} ($d\geq 3$)        & $d^k$                                         & $\sagestr{next(genr)} $ & $(2-\frac{1}{d})^n$          & NPR    \\
        {\sc Weighted Feedback Vertex Set}           & $3.6181^k$   \tcite{AgrawalLKS16}             & $\sagestr{next(genr)} $ & $1.7237^n$                   & $1.8638^n$  \tcite{FominGPR08-On} \\
        \textsc{Weighted 3-Hitting Set}              & $2.168^k$ \tcite{ShachnaiZ15}                 & $\sagestr{next(genr)} $ & $1.5388^n$                   & $1.6755^n$ \tcite{CochefertCGK16}\\
        \textsc{Weighted $d$-Hitting Set} ($d\ge 4$) & $(d-0.832)^k$ \tcite{FominGKLS10,ShachnaiZ15} & $\sagestr{next(genr)} $ & $(2-\frac{1}{d-0.932})^n$    & \tcite{CochefertCGK16} \\
        \bottomrule
      %{\almsat}                                    & $2.3146^k$                                    & $\sage{next(genr)} $ & $1.5680^n$                   & NPR    \\ \hline
      \end{tabular}

    }
    \caption{\label{fig:vertexresults}Summary of known and new results for different 
      optimization problems.
    %New results are highlighted in green (last row).
    NPR means  that we are not aware of any previous algorithms faster than brute-force. All bounds suppress factors polynomial in the input size $N$.\longversion{ The algorithms in the first row are randomized (r).}}
  % and DNI means that our approach does not improve over fastest previous results.}
  \end{table}
%}

\printbibliography
\end{document}





