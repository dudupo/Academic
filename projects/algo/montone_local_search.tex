

%\newcommand*{\ACM}{}%

\ifdefined\ACM

%\documentclass[sigplan,screen]{acmart}
  \documentclass[manuscript,screen,review]{acmart}

\else
  \documentclass{article}
  \usepackage[utf8]{inputenc}
\usepackage[a4paper, total={6.5in, 10in} ]{geometry}
\usepackage{braket}
\usepackage{xcolor}
\usepackage{amsmath}
\usepackage{amssymb}
\usepackage{amsfonts}
\usepackage{graphicx}
\usepackage{svg}
\usepackage{float}
\usepackage{tikz}
\usetikzlibrary{patterns, shapes.arrows}
\usepackage{adjustbox}
\usepackage{tikz-network}
\usepackage[ruled,lined,linesnumbered]{algorithm2e}
\usepackage{multicol}
\usepackage[backend=biber,style=alphabetic,sorting=ynt]{biblatex}
\usepackage{xcolor}
\usepackage{pgfplots}
\DeclareUnicodeCharacter{2212}{−}
\usepgfplotslibrary{groupplots,dateplot}
\pgfplotsset{compat=newest}



  \addbibresource{./sample.bib} 
  \addbibresource{./exactalgo-rs.bib}
\fi

\begin{document}

\input{newcommands}

\title{Groverize Monotone Local Search. (Short Note)} 
\author{David Ponarovsky}

\ifdefined\ACM
  \affiliation{%
    \institution{The Th{\o}rv{\"a}ld Group}
    \streetaddress{1 Th{\o}rv{\"a}ld Circle}
    \city{Hekla}
  \country{Iceland}}
  \email{larst@affiliation.org}
\else
  \maketitle
\fi
%\begin{abstract}
  Quantum feasibility hinges on the assumption that the basic gate's noise rate is below a certain threshold. Here we study the behavior of computation models when the noise is slightly greater than that threshold. In particular, We ask if one can design a fault tolerance schema such that if the noise is above the threshold, it is still grunted that the final generated state would have a value. 
\end{abstract}

\ifdefined\ACM
  \maketitle
\fi

%\begin{multicols*}{2}
% \section{Preambles}
  In this work, we propose a new construction for good LDPC codes, which also have a good testability parameter. In the sense that verfining a constant number of random checks, would be enough to detect any error with probability proportional to the error size. In contrast to previews, constructions made by \cite{Dinur}, \cite{leverrier2022quantum} and \cite{Pavel}, our construction doesn't require spicel properties of the small codes, such as $w$-robustness and $p$-resistance for puncturing. 
  
  Our proof also indirectly answers the following question. Why most of the good LDPC codes are known to be bad in terms of detecting errors? In other words, It seems that for most of them, there exist strings that are very far from being in the code and, meanwhile, fail to satisfy only a small number of restrictions.
  While the previous LDPC constructions focused on ensuring that the yielded code would have a good rate and distance parameters, our construction enforces the restrictions collection to have a nontrivial fraction of degeneration. That is, removing a single restriction will not change the code, as any restriction is linearly dependent on the others.



%%Coding theory has emerged by the need to transfer information in noisy communication channels. By embedding a message in higher dimension space, one can guarantee robustness against possible faults. The ratio of the original content length to the passed message \emph{length} is the \emph{rate} of the code, and it measures how consuming our communication protocol is. Furthermore, the \emph{distance} of the code quantifies how many faults the scheme can absorb such that the receiver can recover the original message. We could consider the code as all the strings that satisfy a specified restrictions collection.
%  
%
%  Non-formally, code is good if its distance and rate are scaled linearly in the encoded message length. In practice, one is also interested in implementing those checks efficiently. We say that a code is an LDPC if any bit is involved in a constant number of restrictions, each of which is a linear equation, and if any restriction contains a fixed number of variables.
%
%  Furthermore, finally, another characteristic of the code is its testability, which is the complexity of the number of random checks one should do to negate that a given candidate is in the code. Besides good codes being considered efficient in terms of robustness and overhead, they are also vital components in establishing secure multiparty computation \cite{MultiParty} and have a deep connection to probabilistic proofs.
%
%  First, we state the notations, definitions, and formal theorem in section 2. Then in sections 3 and 4, we review past results and provide their proofs to make this paper self-contained. Readers familiar with the basic concepts of LDPC, Tanner, and Expanders codes construction should consider skipping directly to section 5, in which we provide our proof. 
%

Coding theory has emerged due to the need to transfer information in noisy communication channels. By embedding a message in a higher-dimensional space, one can guarantee robustness against possible faults. The ratio of the original content length to the transmitted message \emph{length} is the \emph{rate} of the code, and it measures how consuming our communication protocol is. Additionally, the \emph{distance} of the code quantifies how many faults the scheme can absorb such that the receiver can recover the original message. We can consider the code as a collection of all strings that satisfy specified restrictions.

Non-formally, a code is good if its distance and rate scale linearly with the encoded message length. In practice, one is also interested in implementing these checks efficiently. We say that a code is an LDPC if any bit is involved in a constant number of restrictions, each of which is a linear equation, and if any restriction contains a fixed number of variables.

Moreover, another characteristic of the code is its testability, which is the complexity of the number of random checks one must do to verify that a given candidate is in the code. Besides being considered efficient in terms of robustness and overhead, good codes are also vital components in establishing secure multiparty computation \cite{MultiParty} and in the proof of the PCP theorem~\cite{PCPoriginal}.

%In Section 2, we state the notations, definitions, and formal theorem. Then, in Sections 3 and 4, we review past results and provide their proofs to make this paper self-contained. Readers familiar with the basic concepts of LDPC, Tanner, and Expanders codes construction may consider skipping directly to Section 5, in which we provide our proof.
%Readers familiar with the basic concepts of LDPC, Tanner, and Expanders codes construction may skip Sections 2, 3, and 4 and proceed directly to Section 5, where we provide our proof.


% commands taken form the original paper. 

\newcommand{\Oh}{{\mathcal{O}}}
\newcommand{\bitsize}{N}
\newcommand{\longversion}[1]{#1}
\newcommand{\abpartization}{{\sc Vertex $(r,\ell)$-Partization}}


\section{Todo.}
\begin{enumerate}
  \item Write the table (sage script).
  \item Add definitions. Problem description.  
  \item Complete the 'proof'. 
  \item Prove lower bound. 
\end{enumerate}

\section{Introduction.} We follow the study of \cite{fomin2015exact}, who relate between the parametrized complexity to the general average case complexity. Crudely put, they shown that for particular wide range of \textbf{NP} hard problems, a solution which run exponentially at some complexity parameter, for example the tree-width of a graph, can be used to derive a batter than bruteforce solution for the general problem. We continue their work by plugin the Grover search   \cite{grover1996fast} routine instead the original sampling process.  We will simplify the definitions given at \cite{fomin2015exact} and use the following definitions instead:
%\begin{definition}[extension problems] 

  Consider a decision problem inside \textbf{NP}, in this paper, we will associate two verifiers $U,V$ with each language. $U$ stands for input validation, conceptually it uses for checking that the solution 'live' inside the problem world. For example, for the $3$-\textbf{SAT}, $U$ checks that the input indeed encode an assignment. Formally the role of $U$ is to restrict the inputs to certain form. And $V$ responsible to verify that the word indeed in the langauge, ie check that the assignment satisfies the formula. We will say that a problem is an \emph{extension problem} if requiring any of the input bits to be $1$ could reduced to another instance of the problem. For example, consider $3$-\textbf{SAT}, fixing an aribtray bit $x_{i}$ to be $1$ could reduced to another $3$-\textbf{SAT} formula by erase any of the closures contain $x_{i}$ and replacing any of the occureents of $\bar{x_{i}}$ by other termianl on the same clouser (i.e $ \bar{x_{i}} \wedge  \bar{y} \wedge z  \mapsto  \bar{y} \wedge \bar{y} \wedge z$).        the input  any instance of the problem could be representated as the bit-wise union of two strings which pass $U$ verification. For example, any assignment satisfies a  $3$-\textbf{SAT} instance could be write as or-wise of two assignments.   
%\end{definition}

\begin{definition}
A \emph{directed graph} $G$ is a pair $(V,E)$ where $V$ is a set of \emph{vertices} and $E$ is a set of \emph{directed edges}.
\end{definition}

\begin{definition}
The \emph{directed shortest path problem} is the problem of finding the directed path with the minimum weight between two given vertices in a directed weighted graph.
\end{definition}


\begin{equation*}
  \begin{split}    
    & \sum_{k' \leq k}   \frac{1}{ \sqrt{ p(k') } } \cdot c^{k'-t} \bitsize^{\Oh(1)} \leq  \max_{k' \leq k} \left( \frac{{n - |X| \choose t}}{{k' \choose t}} \right)^{\frac{1}{2}} \cdot c^{k'-t} \bitsize^{\Oh(1)} = \\ 
    & \left( \max_{k' \leq k} \frac{{n - |X| \choose t}}{{k' \choose t}}  \cdot c^{2 \left( k'-t \right) } \right)^{\frac{1}{2}} \bitsize^{\Oh(1) } =  \left( \max_{k \leq n-|X|} \frac{{n - |X| \choose t}}{{k \choose t}} \right)^{\frac{1}{2}} \cdot c^{ 2 \left(  k-t \right) } \bitsize^{\Oh(1)} \le \\ 
    & \Rightarrow  \left(2-\frac{1}{c^{2}}\right)^{ \frac{ n-|X|}{2} }\bitsize^{\Oh(1)}
  \end{split}
\end{equation*}


%f"{Groverize_complixity(c_values[i]):.3}^{{k}}"
%\begin{sagesilent}
%\end{sagesilent}

\input{sagelocal.py}
\begin{sagesilent}
  c = 8 
  d = 4
  f(x) = (2 - (1/c))^x
  g(x) = (2 - (1/(d*c))^2)^(x/2)
\end{sagesilent}
%\begin{figure}{H}
\scalebox{0.8}{
  \sageplot{plot(f, 0, 7, color = 'red')+ plot(g, 0, 7)}
}
%\end{figure}
\newcommand{\tcite}[1]{\hfill\cite{#1}}
%\scalebox{0.8}{
  \begin{table}[H]
    \centering
    \setlength{\tabcolsep}{4pt}
    {\footnotesize
      \begin{tabular}{l l l l l}
        \toprule
        Problem Name                                  & Parameterized                                 & Groverize                    & New bound                                                                  & Previous Bound       \\
        \midrule
        {\sc Feedback Vertex Set}                     & $3^k$ (r) \tcite{cut-and-count}               & $\sagestr{next(genr)} $ & $1.6667^n$   (r)                                                           & \\
        {\sc Feedback Vertex Set}                     & $3.592^k$            \tcite{KociumakaP13}     & $\sagestr{next(genr)} $                          & $1.7217^n$                                                                 & $1.7347^n$ \tcite{FominTV15}  \\
        {\sc Subset Feedback Vertex Set}              & $4^k$         \tcite{Wahlstrom14}             & $\sagestr{next(genr)} $                           & $1.7500^n$                                                                 & $1.8638^n$ \tcite{FominHKPV14}  \\
        {\sc Feedback Vertex Set in Tournaments}     & $1.6181^k$        \tcite{KumarL16}            & $\sagestr{next(genr)} $ & $1.3820^n$                   & $1.4656^n$  \tcite{KumarL16}  \\
        {\sc  Group Feedback Vertex Set}             & $4^k$           \tcite{Wahlstrom14}           & $\sagestr{next(genr)} $ & $1.7500^n$                   & NPR    \\
        \textsc{Node Unique Label Cover}             & $|\Sigma|^{2k}$           \tcite{Wahlstrom14} & $\sagestr{next(genr)} $ & $(2-\frac{1}{|\Sigma|^2})^n$ & NPR    \\
        {\abpartization} ($r,\ell \leq 2$)           & $3.3146^k$   \tcite{BasteFKS15,KolayP15}      & $\sagestr{next(genr)} $ & $1.6984^n$                   & NPR  \\
      %\textsc{Odd Cycle Transversal}              & $2.3146^k$  \cite{LokshtanovNRRS14}           & $\sagestr{next(genr)} $ & $1.5680^n$                   & $1.4391^n$  \\
        {\sc Interval Vertex Deletion}               & $8^k$       \tcite{Cao8kinterval}             & $\sagestr{next(genr)} $ & $1.8750^n$                   & $(2-\varepsilon)^n$ for $\varepsilon <10^{-20}$  \tcite{BliznetsFPV13} \\
        {\sc Proper Interval Vertex Deletion}        & $6^k$           \tcite{HofV13,Cao15}          & $\sagestr{next(genr)} $ & $1.8334^n$                   & $(2-\varepsilon)^n$ for $\varepsilon <10^{-20}$  \tcite{BliznetsFPV13} \\
        {\sc Block Graph Vertex Deletion}            & $4^k$   \tcite{AgrawalLKS16}                  & $\sagestr{next(genr)} $ & $1.7500^n$                   & $(2-\varepsilon)^n$ for $\varepsilon <10^{-20}$  \tcite{BliznetsFPV13}  \\
      %{\sc Cograph Deletion}                      & $??^k$                                        & $\sagestr{next(genr)} $ &                              & deterministic    \\
        {\sc   Cluster Vertex Deletion}              & $1.9102^k$      \tcite{BoralCKP14}            & $\sagestr{next(genr)} $ & $1.4765^n$                   & $1.6181^n$  \tcite{FominGKLS10}  \\
        {\sc   Thread Graph Vertex Deletion}         & $8^k$    \tcite{Kante0KP15}                   & $\sagestr{next(genr)} $ & $1.8750^n$                   & NPR    \\
        {\sc   Multicut on Trees}                    & $1.5538^k$  \tcite{KanjLLTXXYZZZ14}           & $\sagestr{next(genr)} $ & $1.3565^n$                   & NPR    \\
      %{\sc    Vertex Cover}                       & $1.2738^k$ \cite{ChenKX10}                    & $\sagestr{next(genr)} $ & $1.2150^n$  (DNI)            & $1.1996^n$  \cite{XiaoN13}  \\
        {\sc    $3$-Hitting Set}                     & $2.0755^k$    \tcite{MagnusPhD07}             & $\sagestr{next(genr)} $ & $1.5182^n$                   & $1.6278^n$    \tcite{MagnusPhD07}  \\
        {\sc   $4$-Hitting Set}                      & $3.0755^k$      \tcite{FominGKLS10}           & $\sagestr{next(genr)} $ & $1.6750^n$                   & $1.8704^n$ \tcite{FominGKLS10}     \\
        {\sc   $d$-Hitting Set} ($d\geq 3$)          & $(d-0.9245)^k$     \tcite{FominGKLS10}        & $\sagestr{next(genr)} $ & $(2-\frac{1}{(d-0.9245)})^n$ & \tcite{CochefertCGK16,FominGKLS10}   \\
      %{\sc    Weighted $2$-SAT}                   & $1.2738^k$     \cite{MisraNRS10}              & $\sagestr{next(genr)} $ & $1.2150^n$  (DNI)            & $1.1996^n$    \\ \hline
        {\sc    Min-Ones $3$-SAT}                    & $2.562^k$    \tcite{abs-1007-1166}            & $\sagestr{next(genr)} $ & $1.6097^n$                   & NPR   \\
        {\sc    Min-Ones $d$-SAT} ($d\geq 4$)        & $d^k$                                         & $\sagestr{next(genr)} $ & $(2-\frac{1}{d})^n$          & NPR    \\
        {\sc    Weighted $d$-SAT} ($d\geq 3$)        & $d^k$                                         & $\sagestr{next(genr)} $ & $(2-\frac{1}{d})^n$          & NPR    \\
        {\sc Weighted Feedback Vertex Set}           & $3.6181^k$   \tcite{AgrawalLKS16}             & $\sagestr{next(genr)} $ & $1.7237^n$                   & $1.8638^n$  \tcite{FominGPR08-On} \\
        \textsc{Weighted 3-Hitting Set}              & $2.168^k$ \tcite{ShachnaiZ15}                 & $\sagestr{next(genr)} $ & $1.5388^n$                   & $1.6755^n$ \tcite{CochefertCGK16}\\
        \textsc{Weighted $d$-Hitting Set} ($d\ge 4$) & $(d-0.832)^k$ \tcite{FominGKLS10,ShachnaiZ15} & $\sagestr{next(genr)} $ & $(2-\frac{1}{d-0.932})^n$    & \tcite{CochefertCGK16} \\
        \bottomrule
      %{\almsat}                                    & $2.3146^k$                                    & $\sage{next(genr)} $ & $1.5680^n$                   & NPR    \\ \hline
      \end{tabular}

    }
    \caption{\label{fig:vertexresults}Summary of known and new results for different 
      optimization problems.
    %New results are highlighted in green (last row).
    NPR means  that we are not aware of any previous algorithms faster than brute-force. All bounds suppress factors polynomial in the input size $N$.\longversion{ The algorithms in the first row are randomized (r).}}
  % and DNI means that our approach does not improve over fastest previous results.}
  \end{table}
%}

\printbibliography
\end{document}





