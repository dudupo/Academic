

%\newcommand*{\ACM}{}%

\ifdefined\ACM

%\documentclass[sigplan,screen]{acmart}
  \documentclass[manuscript,screen,review]{acmart}

\else
  \documentclass{article}
  \usepackage[utf8]{inputenc}
\usepackage[a4paper, total={6in, 9in}]{geometry}
\usepackage{braket}
\usepackage{xcolor}
\usepackage{amsmath}
\usepackage{amsfonts}
\usepackage{amsthm}
\usepackage{amssymb}
%\usepackage[ocgcolorlinks]{hyperref}
\usepackage{hyperref}
%\usepackage{hyperref,xcolor}
%\usepackage[ocgcolorlinks]{ocgx2}
\usepackage{cleveref}
\usepackage{graphicx}
\usepackage{svg}
\usepackage{float}
\usepackage{tikz}
\usetikzlibrary{patterns, shapes.arrows}
\usepackage{adjustbox}
%\usepackage{tikz-network}
\usepackage{tkz-graph}
\usepackage{tkz-berge}
\usepackage[linesnumbered]{algorithm2e}
\usepackage{multicol}
\usepackage[backend=biber,style=alphabetic,sorting=ynt]{biblatex}
%\usepackage{xcolor}
%\usepackage{tkz-berge}
%\usepackage{tkz-graph}
\usepackage{pgfplots}
\usepackage{sagetex}
\usepackage{setspace}
\usepackage{etoc}
%\usepackage{wrapfig}
\usepackage{pgfgantt}
\DeclareUnicodeCharacter{2212}{−}
\usepgfplotslibrary{groupplots,dateplot}
\pgfplotsset{compat=newest}

\newtheorem{theorem}{Theorem}
\newtheorem{definition}{Definition}
\newtheorem{example}{Example}
\newtheorem{claim}{Claim}
\newtheorem{fact}{Fact}
\newtheorem{remark}{Remark}
\newtheorem*{theorem*}{Theorem}
\newtheorem{lemma}{Lemma}
\crefname{lemma}{Lemma}{Lemmas}
\hypersetup{colorlinks=true}
% , allcolors=blue,allbordercolors=blue,pdfborderstyle={0 0 1}}
%\hypersetup{pdfborder={2 2 2}}
% pdfpagemode=FullScreen,
% backref 

\newtheorem{problem}{Problem}
\crefname{problem}{Problem}{Problems}

\DeclareMathOperator{\Ima}{Im}


  \addbibresource{./sample.bib} 

\fi

\begin{document}

\newcommand{\commentt}[1]{\textcolor{blue}{ \textbf{[COMMENT]} #1}}
\newcommand{\ctt}[1]{\commentt{#1}}
\newcommand{\prb}[1]{ \mathbf{Pr} \left[ #1 \right]}
\newcommand{\prbm}[2]{ \mathbf{Pr}_{ #2 }\left[ #1 \right]}
\newcommand{\prbc}[3]{ \mathbf{Pr}_{ #2 }\left[ #1 \right | #3]}
\newcommand{\prbcprb}[3]{ \prbc{#2}{#1}{#3} \cdot \prb{#3} } 
\newcommand{\expp}[1]{ \mathbf{E} \left[ {#1} \right]}
\newcommand{\onotation}[1]{\(\mathcal{O} \left( {#1}  \right) \)}
\newcommand{\ona}[1]{\onotation{#1}}
\newcommand{\PSI}{{\ket{\psi}}}
\newcommand{\xij} { X_{ij} } 
\DeclareMathOperator{\Ima}{Im}
%\newcommand{\LESn}{\ket{\psi_n}}
%\newcommand{\LESa}{\ket{\phi_n}}
%\newcommand{\LESs}{\frac{1}{\sqrt{n}}\sum_{i}{\ket{\left(0^{i}10^{n-i}\right)^{n}}}}
%\newcommand{\Hn}{\mathcal{H}_{n}}
%\newcommand{\Ep}{\frac{1}{\sqrt{2^n}}\sum^{2^n}_{x}{ \ket{xx}}}
%\newcommand{\HON}{\ket{\psi_{\text{honest}}}}
%\newcommand{\Lemma}{\paragraph{Lemma.}}
\newcommand{\Cpa}{[n, \rho n, \delta n]}
%\setlength{\columnsep}{0.6cm}
\newcommand{\Jvv}{ \bar{J_{v}} } 
\newcommand{\Cvv}{ \tilde{C_{v}} } 

\newcommand{\Gz}{ G_{z}^{\delta} } 
\newcommand{ \Tann } {  \mathcal{T}\left( G, C_0 \right) }
\newcommand{\ireducable}{ireducable \hyperref[ire]{[\ref{ire}]} }
\newcommand{\cutUU}{E(U_{-1} \bigcup U_{+1} ,U)} 
\newcommand{\wcutUU}{w\left( E(U_{-1} \bigcup U_{+1} ,U)  \right)}
\newcommand{\testgo}{  \mathcal{T}\left(J, q , C_{0}\right) } 

\newcommand{\duC}{\left( C_{A}^{\perp}\otimes C_{B}^{\perp} \right)^{\perp}}
\newcommand{\duduC}{\left( C_{A}\otimes C_{B}\right)^{\perp}}
  





\title{Groverize Monotone Local Search.} 
\author{David Ponarovsky}

\ifdefined\ACM
  \affiliation{%
    \institution{The Th{\o}rv{\"a}ld Group}
    \streetaddress{1 Th{\o}rv{\"a}ld Circle}
    \city{Hekla}
  \country{Iceland}}
  \email{larst@affiliation.org}
\else
  \maketitle
\fi
%
\abstract{We propose an alternative simple construction of good LTC codes. In contrast to previews, constructions made by \cite{Dinur}, \cite{leverrier2022quantum} and \cite{Pavel}, our construction doesn't require spicel properties of the small codes, such as $w$-robustness and $p$-resistance for puncturing.  
} 


\ifdefined\ACM
  \maketitle
\fi

%\begin{multicols*}{2}
% \section{Preambles}

Localy Testable Codes, or LTC, are error correction codes such that verfining a uinformly random cchoosen check, would be enough to detect any error with probability proportional to it's size. Bisdes the clear computional adventage they offer, they took roles at the eriler PCP proofs.  
  In this work, we propose a new construction for good LTC codes, which also have a good testability parameter. In the sense   Our proof also indirectly answers the following question. Why most of the good LDPC codes are known to be bad in terms of detecting errors? In other words, It seems that for most of them, there exist strings that are very far from being in the code and, meanwhile, fail to satisfy only a small number of restrictions.
  While the previous LDPC constructions focused on ensuring that the yielded code would have a good rate and distance parameters, our construction enforces the restrictions collection to have a nontrivial fraction of degeneration. That is, removing a single restriction will not change the code, as any restriction is linearly dependent on the others.



%\section{Introduction.}
.. bla bla bla.. bla bla ..     
\definition[General Entanglement State]{ We say that $\PSI$ is general entanglement \label{def:gEnt} if .. }

\definition[Local-Measure-Circuit] { We say that a quantum circuit $C$ is a local measure circuit \label{def:lmc} if it's can be described as a decomposition of poly classical circuit and a constant depth quantum circuit which contains only 1-qubit gates and measurements. 

We would think about local measure circuits as chip circuits. }

\definition[$p_{0}-\Delta$ Fault Tolerance Circuit]{ We say that $\mathcal{C}$ is $p_{0}-\Delta$ fault tolerance \label{def:gft} presentation of abstract circuit $C$ if there exists a local measure circuit $C_{0}$ \ref{lmc} such it's grunted that for noise $p < p_{0}$ $\mathcal{C}$ compute $C$ w.h.p,
And in addition, if $p \in \left( p_{0}, p_{0} + \varepsilon \right)$ then by applying a $C_{0}$ on $\mathcal{C}$ output yields a general entanglement state \label{def:gEnt}}       

\ctt{We would like to add a complexity parameter for the above definition, for example, ``a general entanglement state over more than $\frac{1}{5}$ of the qubits.}  


\section{Introduction.} We follow the study of \cite{fomin2015exact}, who relate between the parametrized complexity to the general average case complexity. Crudely put, they shown that for particular wide range of $NP$ hard problems, a solution which run exponentially at some complexity parameter, for example the tree-width of a graph, can be used to derive a batter than bruteforce solution for the general problem. We continue their work by plugin the Grover search   \cite{grover1996fast} routine instead the original sampling process.   
%\defproblem{\phiext}{An instance $I$, a set $X \subseteq U_I$, and an integer $k$.}
%{Does there exists a subset $S \subseteq (U_I \setminus X)$ such that $S \cup X \in {\cal F}_I$ and $|S| \leq k$?}

\newcommand{\Oh}{{\mathcal{O}}}
\newcommand{\bitsize}{N}

\begin{equation*}
  \begin{split}    
    & \sum_{k' \leq k}   \frac{1}{ \sqrt{ p(k') } } \cdot c^{k'-t} \bitsize^{\Oh(1)} \leq  \max_{k' \leq k} \left( \frac{{n - |X| \choose t}}{{k' \choose t}} \right)^{\frac{1}{2}} \cdot c^{k'-t} \bitsize^{\Oh(1)} = \\ 
    & \left( \max_{k' \leq k} \frac{{n - |X| \choose t}}{{k' \choose t}}  \cdot c^{2 \left( k'-t \right) } \right)^{\frac{1}{2}} \bitsize^{\Oh(1) } =  \left( \max_{k \leq n-|X|} \frac{{n - |X| \choose t}}{{k \choose t}} \right)^{\frac{1}{2}} \cdot c^{ 2 \left(  k-t \right) } \bitsize^{\Oh(1)} \le \\ 
      & \Rightarrow  \left(2-\frac{1}{c^{2}}\right)^{ \frac{ n-|X|}{2} }\bitsize^{\Oh(1)}
  \end{split}
\end{equation*}

\begin{sagesilent}
  c = 8 
  d = 4
  f(x) = (2 - (1/c))^x
  g(x) = (2 - (1/(d*c))^2)^(x/2)
\end{sagesilent}
%\begin{figure}{H}
  \scalebox{0.8}{
    \sageplot{plot(f, 0, 7, color = 'red')+ plot(g, 0, 7)}
  }
%\end{figure}

\printbibliography
\end{document}





