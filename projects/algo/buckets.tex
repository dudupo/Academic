\documentclass{article}


\usepackage[utf8]{inputenc}
\usepackage[a4paper, total={6in, 9in}]{geometry}
\usepackage{braket}
\usepackage{xcolor}
\usepackage{amsmath}
\usepackage{amsfonts}
\usepackage{amsthm}
\usepackage{amssymb}
%\usepackage[ocgcolorlinks]{hyperref}
\usepackage{hyperref}
%\usepackage{hyperref,xcolor}
%\usepackage[ocgcolorlinks]{ocgx2}
\usepackage{cleveref}
\usepackage{graphicx}
\usepackage{svg}
\usepackage{float}
\usepackage{tikz}
\usetikzlibrary{patterns, shapes.arrows}
\usepackage{adjustbox}
%\usepackage{tikz-network}
\usepackage{tkz-graph}
\usepackage{tkz-berge}
\usepackage[linesnumbered]{algorithm2e}
\usepackage{multicol}
\usepackage[backend=biber,style=alphabetic,sorting=ynt]{biblatex}
%\usepackage{xcolor}
%\usepackage{tkz-berge}
%\usepackage{tkz-graph}
\usepackage{pgfplots}
\usepackage{sagetex}
\usepackage{setspace}
\usepackage{etoc}
%\usepackage{wrapfig}
\usepackage{pgfgantt}
\DeclareUnicodeCharacter{2212}{−}
\usepgfplotslibrary{groupplots,dateplot}
\pgfplotsset{compat=newest}

\newtheorem{theorem}{Theorem}
\newtheorem{definition}{Definition}
\newtheorem{example}{Example}
\newtheorem{claim}{Claim}
\newtheorem{fact}{Fact}
\newtheorem{remark}{Remark}
\newtheorem*{theorem*}{Theorem}
\newtheorem{lemma}{Lemma}
\crefname{lemma}{Lemma}{Lemmas}
\hypersetup{colorlinks=true}
% , allcolors=blue,allbordercolors=blue,pdfborderstyle={0 0 1}}
%\hypersetup{pdfborder={2 2 2}}
% pdfpagemode=FullScreen,
% backref 

\newtheorem{problem}{Problem}
\crefname{problem}{Problem}{Problems}

\DeclareMathOperator{\Ima}{Im}


\addbibresource{../general-tex/sample.bib} %Import the bibliography file
\newcommand{\commentt}[1]{\textcolor{blue}{ \textbf{[COMMENT]} #1}}
\newcommand{\ctt}[1]{\commentt{#1}}
\newcommand{\prb}[1]{ \mathbf{Pr} \left[ #1 \right]}
\newcommand{\prbm}[2]{ \mathbf{Pr}_{ #2 }\left[ #1 \right]}
\newcommand{\prbc}[3]{ \mathbf{Pr}_{ #2 }\left[ #1 \right | #3]}
\newcommand{\prbcprb}[3]{ \prbc{#2}{#1}{#3} \cdot \prb{#3} } 
\newcommand{\expp}[1]{ \mathbf{E} \left[ {#1} \right]}
\newcommand{\onotation}[1]{\(\mathcal{O} \left( {#1}  \right) \)}
\newcommand{\ona}[1]{\onotation{#1}}
\newcommand{\PSI}{{\ket{\psi}}}
\newcommand{\xij} { X_{ij} } 
\DeclareMathOperator{\Ima}{Im}
%\newcommand{\LESn}{\ket{\psi_n}}
%\newcommand{\LESa}{\ket{\phi_n}}
%\newcommand{\LESs}{\frac{1}{\sqrt{n}}\sum_{i}{\ket{\left(0^{i}10^{n-i}\right)^{n}}}}
%\newcommand{\Hn}{\mathcal{H}_{n}}
%\newcommand{\Ep}{\frac{1}{\sqrt{2^n}}\sum^{2^n}_{x}{ \ket{xx}}}
%\newcommand{\HON}{\ket{\psi_{\text{honest}}}}
%\newcommand{\Lemma}{\paragraph{Lemma.}}
\newcommand{\Cpa}{[n, \rho n, \delta n]}
%\setlength{\columnsep}{0.6cm}
\newcommand{\Jvv}{ \bar{J_{v}} } 
\newcommand{\Cvv}{ \tilde{C_{v}} } 

\newcommand{\Gz}{ G_{z}^{\delta} } 
\newcommand{ \Tann } {  \mathcal{T}\left( G, C_0 \right) }
\newcommand{\ireducable}{ireducable \hyperref[ire]{[\ref{ire}]} }
\newcommand{\cutUU}{E(U_{-1} \bigcup U_{+1} ,U)} 
\newcommand{\wcutUU}{w\left( E(U_{-1} \bigcup U_{+1} ,U)  \right)}
\newcommand{\testgo}{  \mathcal{T}\left(J, q , C_{0}\right) } 

\newcommand{\duC}{\left( C_{A}^{\perp}\otimes C_{B}^{\perp} \right)^{\perp}}
\newcommand{\duduC}{\left( C_{A}\otimes C_{B}\right)^{\perp}}
  




\begin{document}


\title{Bucket Sort When You Know The Distribution.} 
\author{David Ponarovsky}
\maketitle
\begin{abstract}None 
\end{abstract}
  \paragraph{The problem.} Let $f: [0,1] \rightarrow [0,1]$ a fixed distribution function. Write an algorithm that sorts $n$ draws $x_1 ... x_{n}$ at linear expectation time.  
  \paragraph{Solution.} We will define a partition of the input into a series of $n$ buckets $\mathcal{B} = \left\{ B_{k} = [t_{k}, t_{k+1} ]: k \in [n]  \right\}$ such that $ \prb{ x \in B_{i}} = \frac{1}{n}$ for any bucket. Assume that we succeed in computing the buckets efficiently. Let the $\xij$ be the indicator of the event that $x_{j}$ falls to $B_i$. Then we have:

  \begin{equation*}
    \begin{split}
      & \ \prb{\sum_{i}{|B_{i}|^{2}} \ge t} = \prb{ \sum_{i}{\left( \sum_{j}X_{ij} \right)^{2}} \ge t   } \\
     = & \ \prb{ \sum_{i,j,j^{\prime}}{X_{i,j}X_{i,j^{\prime}} } \ge t  } = \prb{ \sum_{i,j\neq j^{\prime}}{X_{i,j}{X_{i,j^{\prime}} } \ge t - n   } }\\ 
    \le & \ \frac{\sum_{i,j\neq j^{\prime}}{\expp{ X_{ij}X_{ij^{\prime}}  }}}{ t - n  } = \frac{n}{\left(t-n \right)n^{2}}2\binom{n}{2}  \le \frac{n}{t-n}  
    \end{split}
  \end{equation*}
  It follows that for any function $t: \mathbb{N} \rightarrow \mathbb{R}$, such that $n = o\left( t \right)$, sorting quadric each bucket at turn would last almost surely less than $t(n)$.  
  It shows that knowing the distribution enables one to compute the buckets efficiently. Ensuring the uniform partitioned property leads to the following recursive relation: 
  \begin{equation*}
    \begin{split}
      \frac{1}{n} &= \prb{ x \in B_{k} } = f\left( t_{k+1} \right)- f\left( t_{k} \right) \\
      & \Rightarrow t_{k+1} \leftarrow f^{-1}\left( \frac{1}{n} + f\left( t_{k} \right) \right) 
    \end{split}
  \end{equation*} 
  Hence, if $f$ can be computed in sublinear time, then we obtained an expected linear time algorithm for sorting $\square$
  The result above demonstrates a case when knowing how the input is distributed turns the problem equivalent to facing a uniform distributed input.     
\printbibliography 
\end{document}



