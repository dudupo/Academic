\documentclass{article}

\usepackage[utf8]{inputenc}
\usepackage[a4paper, total={7.5in, 10in} ]{geometry}
\usepackage{braket}
\usepackage{xcolor}
\usepackage{amsmath}
\usepackage{amssymb}
\usepackage{amsfonts}
\usepackage{graphicx}
\usepackage{svg}
\usepackage{float}
\usepackage{tikz}
\usetikzlibrary{patterns,shapes.arrows}
\usepackage{adjustbox}
\usepackage{tikz-network}
\usepackage[ruled,lined,linesnumbered]{algorithm2e}
\usepackage{multicol}
\usepackage[backend=biber,style=alphabetic,sorting=ynt]{biblatex}
\usepackage{xcolor}
\usepackage{pgfplots}
\DeclareUnicodeCharacter{2212}{−}
\usepgfplotslibrary{groupplots,dateplot}
\pgfplotsset{compat=newest}

\addbibresource{sample.bib} %Import the bibliography file

\newcommand{\commentt}[1]{\textcolor{blue}{ \textbf{[COMMENT]} #1}}
\newcommand{\ctt}[1]{\commentt{#1}}
\newcommand{\prb}[1]{ \mathbf{Pr} \left[ {#1} \right]}
\newcommand{\expp}[1]{ \mathbf{E} \left[ {#1} \right]}
\newcommand{\onotation}[1]{\(\mathcal{O} \left( {#1}  \right) \)}
\newcommand{\ona}[1]{\onotation{#1}}
\newcommand{\PSI}{{\ket{\psi}}}
\newcommand{\LESn}{\ket{\psi_n}}
\newcommand{\LESa}{\ket{\phi_n}}
\newcommand{\LESs}{\frac{1}{\sqrt{n}}\sum_{i}{\ket{\left(0^{i}10^{n-i}\right)^{n}}}}
\newcommand{\Hn}{\mathcal{H}_{n}}
\newcommand{\Ep}{\frac{1}{\sqrt{2^n}}\sum^{2^n}_{x}{ \ket{xx}}}
\newcommand{\HON}{\ket{\psi_{\text{honest}}}}
\newcommand{\Lemma}{\paragraph{Lemma.}}
\newcommand{\PonB}{ \rho + \frac{5}{16}\delta\le \frac{3}{4} + \frac{1}{16} } 
\newcommand{\Cpa}{[n, \rho n, \delta n]}
%\setlength{\columnsep}{0.6cm}

\newcommand{\Gz}{ G_{z}^{\delta} } 
\newcommand{ \Tann } {  \mathcal{T}\left( G, C_0 \right) }
\newcommand{\xij} { X_{ij} } 
\begin{document}


\title{Bucket Sort When You Know The Distribution.} 
\author{David Ponarovsky}
\maketitle
\abstract{None.} 
\begin{multicols*}{2}

  \paragraph{The problem.} Let $f: [0,1] \rightarrow [0,1]$ a fixed distribution function. Write an algorithm that sorts $n$ draws $x_1 ... x_{n}$ at linear expectation time.  
  \paragraph{Solution.} We will define a partition of the input into a series of $n$ buckets $\mathcal{B} = \left\{ B_{k} = [t_{k}, t_{k+1} ]: k \in [n]  \right\}$ such that $ \prb{ x \in B_{i}} = \frac{1}{n}$ for any bucket. Assume that we succeed in computing the buckets efficiently. Let the $\xij$ be the indicator of the event that $x_{j}$ falls to $B_i$. Then we have:

  \begin{equation*}
    \begin{split}
      & \ \prb{\sum_{i}{|B_{i}|^{2}} \ge t} = \prb{ \sum_{i}{\left( \sum_{j}X_{ij} \right)^{2}} \ge t   } \\
     = & \ \prb{ \sum_{i,j,j^{\prime}}{X_{i,j}X_{i,j^{\prime}} } \ge t  } = \prb{ \sum_{i,j\neq j^{\prime}}{X_{i,j}{X_{i,j^{\prime}} } \ge t - n   } }\\ 
    \le & \ \frac{\sum_{i,j\neq j^{\prime}}{\expp{ X_{ij}X_{ij^{\prime}}  }}}{ t - n  } = \frac{n}{\left(t-n \right)n^{2}}2\binom{n}{2}  \le \frac{n}{t-n}  
    \end{split}
  \end{equation*}
  It follows that for any function $t: \mathbb{N} \rightarrow \mathbb{R}$, such that $n = o\left( t \right)$, sorting quadric each bucket at turn would last almost surely less than $t(n)$.  
  It shows that knowing the distribution enables one to compute the buckets efficiently. Ensuring the uniform partitioned property leads to the following recursive relation: 
  \begin{equation*}
    \begin{split}
      \frac{1}{n} &= \prb{ x \in B_{k} } = f\left( t_{k+1} \right)- f\left( t_{k} \right) \\
      & \Rightarrow t_{k+1} \leftarrow f^{-1}\left( \frac{1}{n} + f\left( t_{k} \right) \right) 
    \end{split}
  \end{equation*} 
  Hence, if $f$ can be computed in sublinear time, then we obtained an expected linear time algorithm for sorting $\square$
  The resoult above demenostrate a case when knowing how the input is distributed turns the problem equivalents to facing a uniform ditributed input.    
\end{multicols*}
\printbibliography 
\end{document}



