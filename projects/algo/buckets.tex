\documentclass{article}

\usepackage[utf8]{inputenc}
\usepackage[a4paper, total={7.5in, 10in} ]{geometry}
\usepackage{braket}
\usepackage{xcolor}
\usepackage{amsmath}
\usepackage{amssymb}
\usepackage{amsfonts}
\usepackage{graphicx}
\usepackage{svg}
\usepackage{float}
\usepackage{tikz}
\usetikzlibrary{patterns,shapes.arrows}
\usepackage{adjustbox}
\usepackage{tikz-network}
\usepackage[ruled,vlined,linesnumbered]{algorithm2e}
\usepackage{multicol}
\usepackage[backend=biber,style=alphabetic,sorting=ynt]{biblatex}
\usepackage{xcolor}
\usepackage{pgfplots}
\DeclareUnicodeCharacter{2212}{−}
\usepgfplotslibrary{groupplots,dateplot}
\pgfplotsset{compat=newest}

\addbibresource{sample.bib} %Import the bibliography file

\newcommand{\commentt}[1]{\textcolor{blue}{ \textbf{[COMMENT]} #1}}
\newcommand{\ctt}[1]{\commentt{#1}}
\newcommand{\prb}[1]{ \mathbf{Pr} \left[ {#1} \right]}
\newcommand{\expp}[1]{ \mathbf{E} \left[ {#1} \right]}
\newcommand{\onotation}[1]{\(\mathcal{O} \left( {#1}  \right) \)}
\newcommand{\ona}[1]{\onotation{#1}}
\newcommand{\PSI}{{\ket{\psi}}}
\newcommand{\LESn}{\ket{\psi_n}}
\newcommand{\LESa}{\ket{\phi_n}}
\newcommand{\LESs}{\frac{1}{\sqrt{n}}\sum_{i}{\ket{\left(0^{i}10^{n-i}\right)^{n}}}}
\newcommand{\Hn}{\mathcal{H}_{n}}
\newcommand{\Ep}{\frac{1}{\sqrt{2^n}}\sum^{2^n}_{x}{ \ket{xx}}}
\newcommand{\HON}{\ket{\psi_{\text{honest}}}}
\newcommand{\Lemma}{\paragraph{Lemma.}}
\newcommand{\PonB}{ \rho + \frac{5}{16}\delta\le \frac{3}{4} + \frac{1}{16} } 
\newcommand{\Cpa}{[n, \rho n, \delta n]}
%\setlength{\columnsep}{0.6cm}

\newcommand{\Gz}{ G_{z}^{\delta} } 
\newcommand{ \Tann } {  \mathcal{T}\left( G, C_0 \right) }
\newcommand{\xij} { X_{ij} } 
\begin{document}


\title{Bucket Sort When You Know The Distribution.} 
\author{David Ponarovsky}
\maketitle
\abstract{We propose a new simple construction based on Tanner Codes, which yields a good LDPC code with testability query complexity of $\Theta\left( n^{1-\varepsilon} \right)$ for any $\varepsilon> 0$ .} 
\begin{multicols*}{2}

  \paragraph{The problem.} Let $f: [0,1] \rightarrow [0,1]$ a fixed distribution function. Write an algorithm that sort $n$ draws $x_1 ... x_{n}$ at linear expectation time.  
  \paragraph{Solution.} We will define a partition of the input into a seira of $n$ buckets $\mathcal{B} = \left\{ B_{k} = [t_{k}, t_{k+1} ] : k \in [n]  \right\}$ such that $ \prb{ x \in B_{i}} = \frac{1}{n}$ for any bucket. 
  \paragraph{Claim.} The probability that the size of the $i$th bucket excceds $t \in \mathbb{N}$ is bounded by: $\prb{ B_{i} \ge t}  \le \frac{e}{k}t^{-k}$ for every integer $k \le n$.
  \paragraph{Proof.} Let the $\xij$ be the indecator of the event that $x_{j}$ belongs to $B_i$. Then we have:
 %  \begin{equation*}
 %    \begin{split}
 %      \expp{B_{i}^{2} } &= \expp{ \left( \sum_{j}{X_{ij}} \right)^{2}  } \\
 %      &= \expp{ \sum_{j,j^{\prime}}{ X_{ij}X_{ij^{\prime}}}  } 
 %      = \sum_{ j,j^{\prime}}{ \expp{X_{ij}}\expp{X_{ij^{\prime}}}   } \\
 %      &=  \sum_{ j \neq j^{\prime} }{ \expp{X_{ij}}\expp{X_{ij^{\prime}}}   }  + \sum_{j}{ \expp{X_{ij}}}  \\
 %      &= \frac{1}{n^{2}} \binom{n}{2} + 1 = O\left( 1 \right)     
 %    \end{split}
 %  \end{equation*}
 
  \begin{equation*}
    \begin{split}
      \expp{B_{i}^{k} } &= \expp{ \left( \sum_{j}{\xij } \right)^{k}  } =  \expp{ \sum_{ \substack { J \in [n]^k }  }{ \prod_{ l \in [k] } { X_{iJ_{l}} } }   } \\
      &=   \expp{ \sum_{l \in [k]}\sum_{ \substack { J \subset [n] \\ |J| = l }  }{ \prod_{j \in J} { \xij  }   }} \\
      &=   \sum_{l \in [k]}{  \binom{n}{l} \frac{l!  }{n^{l}} } 
      %\expp{ \left( B_{i}^2 \right)^{k} } & \le \left( 1 + \frac{1}{n} \right)^{n} \le e \\ 
      %\prb{ B_{i} \ge t} & \le \frac{e}{t^{k}}\\ 
    \end{split}
  \end{equation*}

  And noitce that quantinue of sequance elements in summation is bounded by:  
  \begin{equation*}
    \begin{split}
      \binom{n}{l+1} \frac{(l+1)!  }{n^{l+1}} /  \binom{n}{l} \frac{l!  }{n^{l}} = \frac{n-l}{n}=1-\frac{l}{n} 
    \end{split}
  \end{equation*}

  \begin{equation*}
    \begin{split}
      \frac{1}{n} &= \prb{ x \in B_{k} } = f\left( t_{k+1} \right)- f\left( t_{k} \right) \\
      & \Rightarrow t_{k+1} \leftarrow f^{-1}\left( \frac{1}{n} + f\left( t_{k} \right) \right) 
    \end{split}
  \end{equation*}



\end{multicols*}
\printbibliography 
\end{document}


