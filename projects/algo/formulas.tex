\documentclass{article}

\usepackage[utf8]{inputenc}
\usepackage[a4paper, total={7.5in, 10in} ]{geometry}
\usepackage{braket}
\usepackage{xcolor}
\usepackage{amsmath}
\usepackage{amssymb}
\usepackage{amsfonts}
\usepackage{graphicx}
\usepackage{svg}
\usepackage{float}
\usepackage{tikz}
\usetikzlibrary{patterns,shapes.arrows}
\usepackage{adjustbox}
\usepackage{tikz-network}
\usepackage[ruled,vlined,linesnumbered]{algorithm2e}
\usepackage{multicol}
\usepackage[backend=biber,style=alphabetic,sorting=ynt]{biblatex}
\usepackage{xcolor}
\usepackage{pgfplots}
\DeclareUnicodeCharacter{2212}{−}
\usepgfplotslibrary{groupplots,dateplot}
\pgfplotsset{compat=newest}

\newcommand{\commentt}[1]{\textcolor{blue}{ \textbf{[COMMENT]} #1}}
\newcommand{\ctt}[1]{\commentt{#1}}
\newcommand{\prb}[1]{ \mathbf{Pr} \left[ {#1} \right]}
\newcommand{\expp}[1]{ \mathbf{E} \left[ {#1} \right]}
\newcommand{\onotation}[1]{\(\mathcal{O} \left( {#1}  \right) \)}
\newcommand{\ona}[1]{\onotation{#1}}
\newcommand{\PSI}{{\ket{\psi}}}
\newcommand{\LESn}{\ket{\psi_n}}
\newcommand{\LESa}{\ket{\phi_n}}
\newcommand{\LESs}{\frac{1}{\sqrt{n}}\sum_{i}{\ket{\left(0^{i}10^{n-i}\right)^{n}}}}
\newcommand{\Hn}{\mathcal{H}_{n}}
\newcommand{\Ep}{\frac{1}{\sqrt{2^n}}\sum^{2^n}_{x}{ \ket{xx}}}
\newcommand{\HON}{\ket{\psi_{\text{honest}}}}
\newcommand{\Lemma}{\paragraph{Lemma.}}
\newcommand{\PonB}{ \rho + \frac{5}{16}\delta\le \frac{3}{4} + \frac{1}{16} } 
\newcommand{\Cpa}{[n, \rho n, \delta n]}
%\setlength{\columnsep}{0.6cm}

\newcommand{\Gz}{ G_{z}^{\delta} } 
\newcommand{ \Tann } {  \mathcal{T}\left( G, C_0 \right) }
\newcommand{\xij} { X_{ij} } 

\begin{document}

\title{Fourmlas Sheet.} 
\author{David Ponarovsky}
\maketitle
\begin{multicols*}{2}
  \subsection*{{\textcolor{orange}{Probability.}} } 
\paragraph{Multiplicative Chernoff bound.} Suppose $ X_1, ..., X_n$ are independence random variables taking values in $\{0, 1\}$ Let $X$ denote their sum and let $\mu = \expp{\sum_{i}^{n}{X_{i}}} $  denote the sum's expected value. Then for any $\delta > 0$: 
%:<math>\Pr ( X > (1+\delta)\mu) < \left(\frac{e^\delta}{(1+\delta)^{1+\delta}}\right)^\mu.</math>
\begin{equation*}
    \begin{split}
      \prb{ X \ge \left( 1+\delta \right) \mu } & \le e^{-2\frac{\delta^2\mu^{2}}{n}} \\ 
      \prb{ |X - \mu| \ge \delta\mu } & \le 2e^{-\delta^2\mu/3}, \qquad 0 \le \delta \le 1
    \end{split}
  \end{equation*}
\end{multicols*}{2}
\end{document} 


