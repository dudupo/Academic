\section{Introduction.}
In this study, we investigate the noise regime near the threshold \cite{aharonov1999faulttolerant}. Specifically, we aim to develop a mapping such that, beneath the threshold, the original circuit is computed, while in situations where the noise is slightly above the threshold, there is still a guarantee that the result of the computation is meaningful. 

 \paragraph{To Do.} Short term tasks:
 \begin{enumerate}
   \item Add an initial generalized entanglement definition.
   \item Describe the quantum teleportation as an example for a simple Local-Measure-Circuit. ``prove'' something about it. Explain the importance of EPR pairs a computation resource. And present the question above as ``is that possible to embed the teleportation inside a general circuit''.  
   \item Given $\PSI$ and a local circuit $C_{0}$, What can we say about the $C_{0}\PSI$. What does it mean in terms of complexities class?   
 \end{enumerate}

\definition[General Entanglement State]{ We say that $\PSI$ is general entanglement \label{def:gEnt} if .. }

\definition[Local-Measure-Circuit] { We say that a quantum circuit $C$ is a local measure circuit \label{def:lmc} if it's can be described as a decomposition of poly classical circuit and a constant depth quantum circuit which contains only 1-qubit gates and measurements. 

We would think about local measure circuits as chip circuits. }

\definition[$p_{0}-\Delta$ Fault Tolerance Circuit]{ We say that $\mathcal{C}$ is $p_{0}-\Delta$ fault tolerance \label{def:gft} presentation of abstract circuit $C$ if there exists a local measure circuit $C_{0}$ \ref{def:lmc} such it's grunted that for noise $p < p_{0}$ $\mathcal{C}$ compute $C$ w.h.p,
And in addition, if $p \in \left( p_{0}, p_{0} + \varepsilon \right)$ then by applying a $C_{0}$ on $\mathcal{C}$ output yields a general entanglement state \ref{def:gEnt}}       

\ctt{We would like to add a complexity parameter for the above definition, for example, ``a general entanglement state over more than $\frac{1}{5}$ of the qubits.}  

\paragraph{The Obvious Solution.} Suppose that $p(C_{0}) \leq p(C_{1}) - \varepsilon$, then we could attach the circuits next to each other by paying an additive cost. So, for the problem to be truly interesting, we have to ask if we can do that while paying less in the width of the circuit. Also, we have a trivial lower bound such that the final width of the circuit must be at least the maximum of the original circuits.
