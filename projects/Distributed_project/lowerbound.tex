\documentclass{article}
\usepackage[utf8]{inputenc}
\usepackage[a4paper, total={7in, 10in}]{geometry}
\usepackage{braket}
\usepackage{xcolor}
\usepackage{amsmath}
\usepackage{amssymb}
\usepackage{amsfonts}
\usepackage{graphicx}
\usepackage{svg}
\usepackage{float}
\usepackage{tikz}
\usepackage[ruled,vlined]{algorithm2e}
\usepackage{multicol}
\usepackage[backend=biber,style=alphabetic,sorting=ynt]{biblatex}

\addbibresource{bibfile.bib} %Import the bibliography file

\newcommand{\commentt}[1]{\textcolor{blue}{ \textbf{[COMMENT]} #1}}
\newcommand{\ctt}[1]{\commentt{#1}}
\newcommand{\prb}[1]{ \mathbf{Pr} \left[ {#1} \right]}
\newcommand{\onotation}[1]{\(\mathcal{O} \left( {#1}  \right) \)}
\newcommand{\ona}[1]{\onotation{#1}}
\newcommand{\PSI}{{\ket{\psi}}}
\newcommand{\LESn}{\ket{\psi_n}}
\newcommand{\LESa}{\ket{\phi_n}}
\newcommand{\LESs}{\frac{1}{\sqrt{n}}\sum_{i}{\ket{\left(0^{i}10^{n-i}\right)^{n}}}}
\newcommand{\Hn}{\mathcal{H}_{n}}
\newcommand{\Ep}{\frac{1}{\sqrt{2^n}}\sum^{2^n}_{x}{ \ket{xx}}}
\newcommand{\HON}{\ket{\psi_{\text{honest}}}}
\newcommand{\Lemma}{\paragraph{Lemma.}}


\setlength{\columnsep}{0.6cm}

\begin{document}

\title{No-Existence Of Generalize Diffusion.}
\author{David Ponarovsky}
% \date{July 2021}
\maketitle
\begin{multicols*}{2}

  \paragraph{Preamble} One of the most promised applications of quantum computation is the Amplitude Amplification algorithm \cite{Brassard_2002}, In which, one can transform a known state with probability $a$ to measure a $\ket{i}$ to a state in which the desired measurement obtained with probability grater than $\frac{1}{2}$ at the cost of less than $\sqrt{a}$ sort of Grover iterations. One question that might rise is whether the above can be done given an single entity of the state. We show that there is no operator that given two state $\ket{\psi},\ket{\phi}$ compute the transformation: 
 
  
\begin{equation*}
  \begin{split}
    D \ket{\psi}\ket{\phi} = \ket{\psi}\left( \mathbb{I} - 2 \ket{\psi}\bra{\psi} \right)\ket{\phi} 
   \end{split}
\end{equation*}

We name the gate above the \textit{Generalize Diffusion} gate, As if such gate were exists it could be used instand of the projection operator to simulate the amplitude amplification procedure. The contradiction of the existence follows by showing that using $D$ two players can compute the disjoints of their sets in single round and $O\left( \sqrt{n} \right)$ communication complexity, which shown by Braverman to be impossible \cite{Braverman}.    
\paragraph{Ex1 - Quantum Communication Complexity of Disjointness.}
Consider the following communication
problem. As inputs Alice gets an \(x\) and Bob get a \(y\), where \(x, y \in \{0, 1\}^n \), and by exchanging information they what to determine if there is an index \(k\) with \(x_k = y_k = 1 \) or not. 
In other words, if \(x\) encodes the
set \(A = \{k | x_k = 1\} \), and \(y\) encodes \(B = \{k | y_k = 1\}\), then Alice and Bob want to determine whether \( A \cap B \) is empty or not.

The classical randomized communication complexity of this problem is \ona{n}.
Assuming Alice and Bob can exchange quantum messages, show how Alice and bob can solve the task
correctly with probability greater than \(2/3\) by exchanging at most \ona{\sqrt{n}\log n } qubits.


\paragraph{Wrong Solution (The original version, the mistake explained beneath).}
Let \( x^{(j)} \) be the \(j\)-th \(\sqrt{n}\)-block of \(x\), e.g \(x^{(j)} = x_{j\sqrt{n}},x_{j\sqrt{n}+1}...,x_{(j+1)(\sqrt{n})-1}  \). And denote by \( \ket{\psi_x} \in \mathcal{H}_{2}^{\bigotimes \sqrt{n}} \bigotimes \mathcal{H}_{\sqrt{n}} \) the uniform superposition state over the \( x^{(j)}\)-'s "tensored" with \(\sqrt{n}\)-qudit (which will correspond to the block number). 
\[ \ket{\psi_x} = \frac{1}{n^\frac{1}{4}}\sum_{j}^{\sqrt{n}}\ket{x^{(j)}}\ket{j} \] Note that the encoding of \( \ket{\psi_x} \) require only \( \sqrt{n} + \log(\sqrt{n}) \) qubits.
Clearly both Alice and Bob can generate the states \( \ket{\psi_x}, \ket{\psi_y} \), then Bob sends he's share to Alice.
We know that there is a classical circuit with logarithmic depth in \( \sqrt{n} \) that act over the pure states \( \ket{x^{(j)}}\ket{j} , \ket{y^{(k)}}\ket{k} \) and decides whether \[ \left( j =  k \right) \ \bigwedge  \ \left( \bigvee_{i \in [ \sqrt{n} ] } x^{(j)}_{i} \ \wedge \  y^{(k)}_{i} \right)   \]


Denote it by \( C \) and by \( U \) the phase flip controled by $C$ i.e $U\ket{i}=\left( -1 \right)^{C\left( i \right)}\ket{i}$.

\paragraph{Lemma.} \textit{ Recall the operator $\mathbf Q  = - {\mathcal A}  {\mathbf S}_0 
  {\mathcal A}^{-1}  {\mathbf S}_\chi$ defined in \cite{Brassard_2002}, such that $ \mathcal A \ket{0} = \ket{\psi_{x}}\ket{\psi_{y}}$ and 
consider the generaliz deffusion gate $D$, Then it holds that for any state $ \ket{\phi} \in \mathcal{H}_{\Psi} $:}
\begin{equation*}
  \left(  \mathbb{I} \otimes \mathbf Q \right) \ket{\psi_{x}}\ket{\psi_{y}} \ket{\phi} =  D \left( \mathbb{I} \otimes U \right)  \ket{\psi_{x}}\ket{\psi_{y}} \ket{\phi} 
\end{equation*}

\paragraph{Proof.} We will show that both operator act the same on $ \mathcal{H}_{\Psi} $ and  $ \mathcal{H}_{\Psi^\perp} $  

% which is operator in Hilbert state at dimension: \[ \mathcal{O} \left( \left( \dim \left( \mathcal{H}_{2}^{\bigotimes \sqrt{n}} \bigotimes \mathcal{H}_{\sqrt{n}} \right) \right) ^2 o(1) \right) = \mathcal{O} \left( 2^{2\sqrt{n}}  \right)  \]

\paragraph{Theorem 3.} \textit{Quadratic speedup without knowing $\mathbf{a}$
There exists a quantum algorithm \textbf{algqsearch} with the following property.
Let $\mathcal A$ be any quantum algorithm that uses no measurements,
and let $\chi : \mathbb{N}  \rightarrow \{0,1\}$ be any Boolean function.
Let $a$ denote the initial success probability of~$\mathcal A$.
Algorithm \textbf{algqsearch} finds a good solution using an expected number
of applications of $\mathcal A$ and ${\mathcal A}^{-1}$ which are in
$\Theta(\sqrt a)$ if $a>0$, and otherwise runs forever.}

\paragraph{}
Suppose that \( A \cap B \neq \emptyset \) then, the support of \( \ket{\psi_x} \otimes \ket{\psi_y} \) contain a state \( \ket{\phi} \) which satisfies \(C\), and therefore  after performing \ona{ 2^{ \sqrt{n}}} Grover iterations, measuring the qubits "type" will collapse (\textbf{w.h.p}) into \( \ket{\phi} \). In another hand, if \( A \cap B = \emptyset \) then no matter what will be the result of the measurement, Alice could could verify it by applying the classical circuit. 
\paragraph{}Summarize the above yields the following protocol,
\begin{enumerate}
    \item Bob create \( \ket{\psi_x} \) and sent it to Alice.
    \item Alice uses Grover search and \(U\) to amplify the probability to measure a state \( \ket{\phi} \in Support \left(   \ket{\psi_x} \otimes \ket{\psi_y}  \right) \) that satisfies \(C\).
    \item denote by \( \tilde{\phi} \) the result (a bit string / pure state) that Alice measured, Alice compute \(C\left(\tilde{\phi}\right) \) and returns the result. 
\end{enumerate}
The corrections, emits directly from Grover corrections and the construction above, the probability to measure an instruction between \(A\) and \(B\) is grater than \( \frac{2}{3} \) in case there is exist such, due to Grover. In the case the instruction is empty Alice will returns false at probability equal 1 (as explained above). The communication cost is exactly  \( \sqrt{n} + \log(\sqrt{n}) \) qubits. 


\paragraph{ The mistake.} \ctt{The mistake.} The fault was that I have ignored the fact that the Diffusion operator in the Grover algorithm might concentrate weight in states which disjoint to the initial support of \( \ket{\psi_{x}}\). In fact this algorithm will always return true. 
Combining the Braverman's communication  lower bound We obtain the following no-go corollary:  
\printbibliography 
\end{multicols*}
\end{document}


