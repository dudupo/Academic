\documentclass{article}
\usepackage[utf8]{inputenc}
\usepackage[a4paper, total={7in, 10in}]{geometry}
\usepackage{braket}
\usepackage{xcolor}
\usepackage{amsmath}
\usepackage{amssymb}
\usepackage{amsfonts}
\usepackage{graphicx}
\usepackage{svg}
\usepackage{float}
\usepackage{tikz}
\usepackage[ruled,vlined]{algorithm2e}
\usepackage{multicol}
\usepackage[backend=biber,style=alphabetic,sorting=ynt]{biblatex}

\addbibresource{bibfile.bib} %Import the bibliography file

\newcommand{\commentt}[1]{\textcolor{blue}{ \textbf{[COMMENT]} #1}}
\newcommand{\ctt}[1]{\commentt{#1}}
\newcommand{\prb}[1]{ \mathbf{Pr} \left[ {#1} \right]}
\newcommand{\onotation}[1]{\(\mathcal{O} \left( {#1}  \right) \)}
\newcommand{\ona}[1]{\onotation{#1}}
\newcommand{\PSI}{{\ket{\psi}}}
\newcommand{\LESn}{\ket{\psi_n}}
\newcommand{\LESa}{\ket{\phi_n}}
\newcommand{\LESs}{\frac{1}{\sqrt{n}}\sum_{i}{\ket{\left(0^{i}10^{n-i}\right)^{n}}}}
\newcommand{\Hn}{\mathcal{H}_{n}}
\newcommand{\Ep}{\frac{1}{\sqrt{2^n}}\sum^{2^n}_{x}{ \ket{xx}}}
\newcommand{\HON}{\ket{\psi_{\text{honest}}}}
\newcommand{\Lemma}{\paragraph{Lemma.}}


\setlength{\columnsep}{0.6cm}

\begin{document}

\title{No-Existence Of Generalize Diffusion.}
\author{David Ponarovsky}
% \date{July 2021}
\maketitle

\begin{abstract}\textit{We show that there is no operator that given two state $\ket{\psi},\ket{\phi}$ compute the transformation: $D \ket{\psi}\ket{\phi} = \ket{\psi}\left( \mathbb{I} - 2 \ket{\psi}\bra{\psi} \right)\ket{\phi} $ The contradiction of the existence follows by showing that using $D$ two players can compute the disjoints of their sets in single round and $O\left( \sqrt{n} \right)$ communication complexity, which shown by Braverman to be impossible \cite{Braverman}. }
\end{abstract}

\begin{multicols*}{2}

  \paragraph{Preamble} One of the most promised applications of quantum computation is the Amplitude Amplification algorithm \cite{Brassard_2002}, In which, one can transform a known state $\ket{\Psi}$  with probability $a$ to measure a $\ket{i}$ to a state in which the desired measurement obtained with probability grater than $\frac{1}{2}$ at the cost of less than $\sqrt{a}$ sort of Grover iterations.
  
  A critical requirement for that precedure is to have the ability to generate a copeis of the initial state, Formulated by \cite{Brassard_2002} as holding an algorithm $\mathcal{A}$, which does not make any mausrements, such $\mathcal{A}\ket{0}=\ket{\Psi}$. Assuming having this ability one could mimic the scattring done in the Grover search, but ristrict himself to be supported on $\ket{\Psi}$. 


  One question that might rise is whether the above amplification process can be done assuming nothing but given a single entity of the initial state. We gave a partly answer for that question by proving the follow theroem: 

  \paragarph{Theorem 1} \textit{ There is no operator that given two state $\ket{\psi},\ket{\phi}$ compute the transformation:} 

\begin{equation*}
  \begin{split}
    D \ket{\psi}\ket{\phi} = \ket{\psi}\left( \mathbb{I} - 2 \ket{\psi}\bra{\psi} \right)\ket{\phi} 
   \end{split}
\end{equation*}

We name the gate above the \textit{Generalize Diffusion} gate, As if such gate were exists it could be used instand of the projection operator to simulate the amplitude amplification procedure. The contradiction of the existence follows by showing that using $D$ two players can compute the disjoints of their sets in single round and $O\left( \sqrt{n} \right)$ communication complexity in contradact to the fact that $r$-rounds two party computation needs at least $\Omega\left( \frac{n}{r} \right)$ communication to compute disjoiness (up to log factors) \cite{Braverman}.    

\paragraph{Quantum Communication Complexity of Disjointness.}
Consider the following communication problem.
As inputs Alice gets an \(x\) and Bob get a \(y\), where \(x, y \in \{0, 1\}^n \), and by exchanging information they want to determine if there is an index \(k\) with \(x_k = y_k = 1 \) or not. 
In other words, if \(x\) encodes the set \(A = \{k | x_k = 1\} \), and \(y\) encodes \(B = \{k | y_k = 1\}\), 
then Alice and Bob want to determine whether \( A \cap B \) is empty or not.

The classical randomized communication complexity of this problem is \ona{n} \cite{v003a011}.
Assuming Alice and Bob can exchange quantum messages, It is known that Alice and bob can solve the task
correctly with probability greater than \(2/3\) by exchanging at most \ona{\sqrt{n}\log n } qubits \ctt{add ciation of the original solution}. 


\paragraph{The reduction.} 
Assume by way of contradiction the existance of $D$ defined above.  
Let \( x^{(j)} \) be the \(j\)-th \(\sqrt{n}\)-block of \(x\), e.g \(x^{(j)} = x_{j\sqrt{n}},x_{j\sqrt{n}+1}...,x_{(j+1)(\sqrt{n})-1}  \). And denote by \( \ket{\psi_x} \in \mathcal{H}_{2}^{\bigotimes \sqrt{n}} \bigotimes \mathcal{H}_{\sqrt{n}} \) the uniform superposition state over the \( x^{(j)}\)-'s "tensored" with \(\sqrt{n}\)-qudit (which will correspond to the block number). 
\[ \ket{\psi_x} = \frac{1}{n^\frac{1}{4}}\sum_{j}^{\sqrt{n}}\ket{x^{(j)}}\ket{j} \] Note that the encoding of \( \ket{\psi_x} \) require only \( \sqrt{n} + \log(\sqrt{n}) \) qubits.
Clearly both Alice and Bob can generate the states \( \ket{\psi_x}, \ket{\psi_y} \), then Bob sends he's share to Alice.
We know that there is a classical circuit with logarithmic depth in \( \sqrt{n} \) that act over the pure states \( \ket{x^{(j)}}\ket{j} , \ket{y^{(k)}}\ket{k} \) and decides whether \[ \left( j =  k \right) \ \bigwedge  \ \left( \bigvee_{i \in [ \sqrt{n} ] } x^{(j)}_{i} \ \wedge \  y^{(k)}_{i} \right)   \]


Denote it by \( C \) and by \( U \) the phase flip controlled by $C$ i.e. $U\ket{i}=\left( -1 \right)^{C\left( i \right)}\ket{i}$. 

The next claim argue that $D,U$ are sufficents to Alice to simulate a single iteration of the amplitutde amplification.  

\paragraph{Claim.} \textit{ Recall the operator $\mathbf Q  = - {\mathcal A}  {\mathbf S}_0 
  {\mathcal A}^{-1}  {\mathbf S}_\chi$ defined in \cite{Brassard_2002}, such that $ \mathcal A \ket{0} = \ket{\Psi} = \ket{\psi_{x}}\ket{\psi_{y}}$ and 
consider the generalize diffusion gate $D$, Denote by $\mathcal{H}_{\Psi}$ the space which is spanned by the $\ket{\Psi}$ support. Then it holds that for any state $ \ket{\phi} \in \mathcal{H}_{\Psi} $:}
\begin{equation*}
  \left(  \mathbb{I} \otimes \mathbf Q \right) \ket{\psi_{x}}\ket{\psi_{y}} \ket{\phi} =  - D \left( \mathbb{I} \otimes U \right)  \ket{\psi_{x}}\ket{\psi_{y}} \ket{\phi} 
\end{equation*}

\paragraph{Proof.} Let $\ket{\Psi_0}, \ket{\Psi_1}$ be the base which span $ \mathcal{H}_{\Psi}$ and in addition $U\ket{\Psi_0} = \ket{\Psi_0}, U\ket{\Psi_1} =- \ket{\Psi_1}$.

First consider the case in which the diminsion of $ \mathcal{H}_{\Psi}$ is exactly 1, If $ \ket{\Psi} $ supported only on non-satisfaing states (i.e $\ket{\Psi} = \ket{\Psi_{0}}) $ then it's clear that $ I \otimes U $ act over the $ \ket{\Psi}\ket{\Psi} $ as identity and therefore $ -D\left( I \otimes U \right) $ act also as identity: 
\begin{equation*}
  -D\left( I \otimes U \right) \ket{\Psi}\ket{\Psi} = -\ket{\Psi}\left( I - 2\ket{\Psi}\bra{\Psi}  \right) \ket{\Psi} = \ket{\Psi}\ket{\Psi}
\end{equation*}
Similar calculation yields that the action is tricial also when  $ \mathcal{H}_{\Psi}$  supported only over $ \ket{\Psi_1} $.  

\paragraph{}

It is left to show the equivaliance when $\ket{\Psi}$ supported both over $\ket{\Psi_0}$ and $\ket{\Psi_1}$. Then it follows that:

    \begin{equation*}
      \begin{split}
    - D & \left( \mathbb{I} \otimes U \right)  \ket{\psi_{x}}\ket{\psi_{y}} \ket{\Psi_1} =    D  \ket{\psi_{x}}\ket{\psi_{y}} \ket{\Psi_1} \\
  &  \ket{\psi_{x}}\ket{\psi_{y}} \left( \mathbb{I} - 2 \ket{\psi_{x}}\ket{\psi_{y}} \bra{ \psi_{x}}\bra{\psi_{y}} \right) \ket{\Psi_1} \\
  &  \ket{\psi_{x}}\ket{\psi_{y}} \left( \mathbb{I} - 2 \ket{\Psi} \bra{\Psi} \right) \ket{\Psi_1}  \\ 
  &  \ket{\psi_{x}}\ket{\psi_{y}} \left( \left( 1 - 2a  \right)\ket{\Psi_1} - 2a \ket{\Psi_0} \right) \\ 
  & \\ 
  - D & \left( \mathbb{I} \otimes U \right)  \ket{\psi_{x}}\ket{\psi_{y}} \ket{\Psi_0} =   - D  \ket{\psi_{x}}\ket{\psi_{y}} \ket{\Psi_0} \\
  &  - \ket{\psi_{x}}\ket{\psi_{y}} \left( \mathbb{I} - 2 \ket{\psi_{x}}\ket{\psi_{y}} \bra{ \psi_{x}}\bra{\psi_{y}} \right) \ket{\Psi_0} \\
  &  - \ket{\psi_{x}}\ket{\psi_{y}} \left( \mathbb{I} - 2 \ket{\Psi} \bra{\Psi} \right) \ket{\Psi_0} \\ 
  &  - \ket{\psi_{x}}\ket{\psi_{y}} \left( \left(  - (2-2a)  \right)\ket{\Psi_1} + 1 -(2 - 2a) \ket{\Psi_0} \right) \\ 
  & \ket{\psi_{x}}\ket{\psi_{y}} \left( \left( 2-2a \right)\ket{\Psi_1} + \left(1  - 2a\right) \ket{\Psi_0} \right) \\ 
  & \square
\end{split}
\end{equation*}
% which is operator in Hilbert state at dimension: \[ \mathcal{O} \left( \left( \dim \left( \mathcal{H}_{2}^{\bigotimes \sqrt{n}} \bigotimes \mathcal{H}_{\sqrt{n}} \right) \right) ^2 o(1) \right) = \mathcal{O} \left( 2^{2\sqrt{n}}  \right)  \]
Now, it's clear that Alice, could simulate the \textbf{algqsearch} algorithm \cite{Brassard_2002}, 

\paragraph{Theorem 3.} \textit{Quadratic speedup without knowing $\mathbf{a}$
There exists a quantum algorithm \textbf{algqsearch} with the following property.
Let $\mathcal A$ be any quantum algorithm that uses no measurements,
and let $\chi : \mathbb{N}  \rightarrow \{0,1\}$ be any Boolean function.
Let $a$ denote the initial success probability of~$\mathcal A$.
Algorithm \textbf{algqsearch} finds a good solution using an expected number
of applications of $\mathcal A$ and ${\mathcal A}^{-1}$ which are in
$\Theta(\sqrt a)$ if $a>0$, and otherwise runs forever.}

\paragraph{} 

\paragraph{Proof of Theorem 1} 
Suppose that \( A \cap B \neq \emptyset \) then, the support of \( \ket{\psi_x} \otimes \ket{\psi_y} \) contain a state \( \ket{\phi} \) which satisfies \(C\), or in other words $a = |\braket{\Psi_1|\Psi}|^2 > 0 $ and therefore by \textit{Theorem 3} there is an explicit procedure which take a $\Theta(\sqrt{a})$ time in expectation, Hence for any \(\varepsilon >0\) we could construct a finite algorithm that fail with probability less than $ \varepsilon $ by rejecting runs that last longer than $\frac{1}{\varepsilon}$. 
  
On the other hand, Consider the case when \(A \cap B = \emptyset\) then $\Rightarrow a = 0 \Rightarrow \mathcal{H}_{\Psi}$ is 1-dimension space spanned only by $\ket{\Psi_0} $, and the operator $ I - 2 \ket{\Psi}\bra{\Psi} $ act over the $ \ket{\Psi_0}  $ as identity and therefore after executing any number of iterations the probability to measure from $\ket{\Psi_0}$ will remain $1$.

\paragraph{}Summarize the above yields the following protocol,
\begin{enumerate}
    \item Bob create \( \ket{\psi_x} \) and send it to Alice.
    \item Alice simulate \textbf{algqsearch} either the algorithm accept or either $n^4$ turns were passed.     
    \item If the algorithm accept then Alice return True otherwise Alice return False. 
\end{enumerate}

The protocol compute the disjointness in single round while requiring transmission of less than $\Theta\left( \sqrt{n} \right)$ qubits. That in contrast to the known lower bound proved by Braverman \cite{Braverman}: 
\paragraph{Theorem A} \textit{The $r$-round quantum communication complexity of Disjointness$_n$ is $ \Omega\left( \frac{n}{r \log^8 r} \right)$.} 
\paragraph{Open question.} 
\printbibliography 
\end{multicols*}
\end{document}


