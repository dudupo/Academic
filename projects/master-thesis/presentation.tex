\documentclass[usenames, aspectratio=169]{beamer}

\usepackage{amsmath}
\usepackage{braket}
\usepackage{amsfonts}
\usepackage{tikz}
\usepackage{adjustbox}
\usepackage{subcaption}
\usepackage{svg}
\usepackage{graphicx}
\usepackage{media9}
\usepackage{float}
\usetikzlibrary{calc}
\usepackage{array}
\usepackage{efbox,graphicx}
\usepackage[normalem]{ulem}
\usepackage{verbatim}
\usepackage{ragged2e}
\usepackage{array}
\efboxsetup{linecolor=Green,linewidth=1.5pt, margin=0pt}

\usetikzlibrary{decorations.pathreplacing}

\newcommand\MemoryLayout[1]{
  \begin{tikzpicture}[scale=0.15]
    \draw[thick](0,0)--++(0,3)node[above]{$0$};
    \foreach \pt/\col/\lab [remember=\pt as \tp (initially 0)] in {#1} {
      \foreach \a [parse=true] in {\tp,...,\pt-1} {
        \draw[fill=\col](-\a, 0) rectangle ++(-1,2);
      }
      \draw[thick](-\pt,0)--++(0,3)node[above]{$\pt$};
      \if\lab\relax\relax\else
        \draw[thick,decorate, decoration={amplitude=1mm}]
        (-\tp,-0.2)--node[below=1mm]{\lab} (-\pt,-0.2);
      \fi
    }
  \end{tikzpicture}
}


\newcommand{\pslsq}[4]{
\begin{frame}
    \frametitle{#1} 
    \includegraphics[width=.7\linewidth]{../source/#3}
    #4  
  \end{frame}
}

\newcommand{\psls}[4]{
  \begin{frame}
    \frametitle{#1} 
    \begin{columns}[t]
      \begin{column}{.48\textwidth}
        #4
      \end{column}
      \begin{column}{.52\textwidth}
        \adjincludegraphics[width=.98\linewidth, valign=t]{../source/#3}
      \end{column} 
    \end{columns}
  \end{frame}
}
\usepackage{sagetex}


\usetheme[progressbar=frametitle]{metropolis}
%\usetheme{EastLansing}
\title[Understanding Quantumness And Testability.] % (optional, only for long titles)
{Understanding Quantumness And Testability.}

\subtitle{  }
\author[D.~Ponarovsky] % (optional, for multiple authors)
	{D.~Ponarovsky\inst{1}}

\institute[HUJI] % (optional)
{  Faculty of Computer Science\newline
  Hebrew University of Jerusalem
}
\date[2023] % (optional)
{Master-Exam-Huji.}
\subject{Understanding Quantumness And Testability.}

\begin{document}
\input{sageutil.py}
\begin{frame}
  \maketitle
\end{frame}
%\pslsq{Today.}{0.3}{controller.png}{}
%\pslsq{Today.}{0.5}{controller-2-out.png}{}

\begin{frame}
  \frametitle{ Today. }
  \begin{itemize}
    \item<1-> Error Correction Codes. 
    \item<2->Quantum Error Correction Codes.
    \item<3->Good Classical Locally Testabile Code.
  \end{itemize} 
\end{frame}

\begin{frame}
\begin{sagesilent}
   
ggs = peter_graphs()
ff = cycle_graph()
%for gg in ggs:
%  gg.set_latex_options(
%          edge_label_sloped = False,
%          edge_labels=True,
%          edge_thickness=0.005,
%          vertex_labels=False,
%          vertex_size= 0.01,
%          format='dot2tex',
%          prog='crico',
%          graphic_size=(7,7),
%          edge_fills=False,
%      )
%  ff.set_latex_options(
%          edge_label_sloped = False,
%          edge_labels=True,
%          edge_thickness=0.005,
%          vertex_labels=False,
%          vertex_size= 0.01,
%          format='dot2tex',
%          prog='crico',
%          graphic_size=(30,8),
%          edge_fills=False,
%      )
% 
%ops = [ gg.latex_options() for gg in ggs ] 
%ops2 = ff.latex_options()
%
%graphs_tex =  ' \ \ \ '.join([  str(op.tkz_picture())  for op in ops[:3 ]])
%graphs_tex_2 = ' \ \ \ \ \ ' +  ' \ \ \ '.join([  str(op.tkz_picture())  for op in ops[3:]])
%graphs_tex_ff  = str(ops2.tkz_picture())
\end{sagesilent}

%\begin{center}
%  \begin{figure}[h]
%    \scalebox{0.65}{
%      \sagestr{graphs_tex} 
%    } 
%    \scalebox{0.65}{
%      \sagestr{graphs_tex_2}
%   }
%  \caption{Peterson Graph.} 
%  \label{fig:pet}
%\end{figure}
%\end{center}

%

blablabla

\end{frame} 

%\psls{BUS.}{0.3}{../source/Pic/1-out.png}{

%  \begin{itemize}[<+->]
%    \item  A BUS is a communication pathway that transfers data between different components of a computer.
%    \item  Buses connect IO devices such as keyboards, monitors, and printers to the computer's central processing unit (CPU).
%    \item  Buses provide a standardized way for different components to exchange data with each other, simplifying device connection and ensuring compatibility.
%\end{itemize}
%}

%\psls{BUS.}{0.3}{../source/Pic/1-out.png}{
%Few examples for Buses: 
%  \begin{itemize}[<+->]
%    \item  Peripheral Component Interconnect (PCI) bus: A type of internal bus that is used to connect various components on the motherboard of a computer, including network adapters, graphics cards, and sound cards.   
%    \item  Universal Serial Bus (USB) bus: A type of external bus that is used to connect various IO devices to a computer, including mice, keyboards, printers, and external hard drives. 
%\end{itemize}
%}
%\psls{BUS.}{0.3}{../source/Pic/1-out.png}{
%  Advantages of using a Bus 
%  \begin{itemize}[<+->]
%\item  Standardization simplifies integration and compatibility.
%\item  Scalability allows for easy system expansion.
%\item  Cost-effective reduces manufacturing expenses.
%\end{itemize}
%}
%
%\psls{BUS.}{0.3}{../source/Pic/1-out.png}{
%  Disadvantages to using a Buses 
%  \begin{itemize}[<+->]
%
%\item  Limited Bandwidth: With many devices sharing the same bus, the available bandwidth can become limited, leading to slower data transfer rates.
%\item  Compatibility Issues: there can still be compatibility issues between newer and older devices, or devices from different manufacturers.
%\end{itemize}
%}
%
%
%\psls{Managing IO.}{0.3}{../source/Pic/2-out.png}{
%
%Operating system's role in managing IO devices:
%\begin{itemize}[<+->]
%  \item  Detect and configure IO devices.
%  \item  Communicate with devices via drivers.
%  \item  Standard interface for input/output operations.
%  \item  Handle interrupts generated by devices.
%  \item  Ensuring that IO devices are used efficiently and conflict-free. 
%\end{itemize}
%}
%
%
%\begin{frame}
%  \frametitle{Controllers} 
%
%  \begin{itemize}[<+->]
%    \item   IO controllers are chips that manage the communication between IO devices and other components of a computer system.
%    \item Different types of devices require different types of controllers.
%    \item Example, hard disk controller: 
%      \begin{itemize}[<+->]
%        \item  Manage read/write operations on hard disks.
%\item  Communicate via specific protocols, such as ATA/SATA.
%\item  Handle functions such as sector mapping and error correction.
%      \end{itemize}
%  \end{itemize}
%\end{frame}
%
%\psls{Drivers.}{0.3}{../source/Pic/3-out.png}{
%  Drivers,
%  \begin{itemize}[<+->]
%    \item  Drivers allow OS to communicate with hardware.
%\item  Different devices use unique communication methods.
%\item  Drivers act as translator for device communication.
%\item  Typically developed by manufacturer and pre-installed.
%  \end{itemize}
%
%}
%
%\psls{Drivers.}{0.3}{../source/Pic/3-out.png}{
%  Drivers,
%  \begin{itemize}[<+->]
%    \item  Drivers allow OS to communicate with hardware.
%\item  Different devices use unique communication methods.
%\item  Drivers act as translator for device communication.
%\item  Typically developed by manufacturer and pre-installed.
%  \end{itemize}
%
%}
%\psls{Drivers.}{0.3}{../source/Pic/3-out.png}{
%Hard disk drivers
%  \begin{itemize}[<+->]
%    \item  Work at a higher level than the hard disk controller.
%      \item Managing overall hard disk access rather than low-level control
%      \item OS sends a series of read/write commands $\rightarrow$ Driver calculates the sector number and the offset and propagate the commands to the controller. 
%    \end{itemize}
%
%}
%
%
%\psls{Drivers.}{0.3}{../source/Pic/3-out.png}{
%  Advantages,
%  \begin{itemize}[<+->]
%    \item Improves hardware function, stability, and security.
%    \item Provides compatibility with older hardware.
%  \end{itemize}
%}
%%\psls{Address Binding.}{0.3}{../source/Pic/4-out.png}{}
%
%\psls{Example, Parallel Port.}{0.3}{../source/Pic/5-out.png}{
%  Controller,
%  \begin{itemize}[<+->]
%    \item Used for old printers and other peripherals
%    \item Sends 8 bits (1 byte) at once
%    \item Connects printer to computer using cable
%    \item Has 25 pins in DB-25 connector
%    \item Uses Centronics interface standard
%  \end{itemize}
%}
%
%\psls{Example, Parallel Port.}{0.3}{../source/Pic/6-out.png}{
%Driver,
%\begin{itemize}[<+->] 
%    \item While the 'busy bit' is on, wait.
%      \item Put a byte on D0-D7, the data registers,
%        \item And transmit a STR pulse to inform the printer.
%\end{itemize}
%}
%%\psls{Address Binding.}{0.3}{../source/Pic/7-out.png}{}
%
%\begin{frame}
%  \frametitle{Example, USB}
%  USB, 
%  \begin{itemize}[<+->]
%    \item  Universal Serial Bus (USB).
%\item  Replaced many legacy ports (e.g. serial), Can transfer data and provide power
%\item  Uses host-device architecture, Host initiates, device responds.
%\item  USB devices classified by USB-IF (non-profit organization) into device classes for standardization.
%\item  Additional class drivers available for communication with specific devices.
%
%  \end{itemize}
%\end{frame}
%
%%\psls{Polling vs Interupts.}{0.3}{../source/Pic/8-out.png}{}
%%\psls{Address Binding.}{0.3}{../source/Pic/9-out.png}{}
%%\psls{}{0.3}{../source/Pic/10-out.png}{}
%\psls{DMA.}{0.3}{../source/Pic/12-out.png}{
%  Direct Memory Access,
%  \begin{itemize}[<+->]
%     \item Mechanism that allows hardware devices, disk controllers, to access memory without going through the CPU. 
%     \item The device driver software on the computer sends a request to the DMA controller.
%     \item The DMA controller receives the request, it reserves memory and takes the bus.
%     \item The device driver can then access the transferred data, and the CPU can resume its normal processing operations. 
%
%  \end{itemize}
%}
%\psls{DMA.}{0.3}{../source/Pic/12-out.png}{
%  Notice, \uncover<2->{The actual execution of the driver code happens on the CPU, \uncover<3->{but it is only responsible for initiating the transfer on the DMA controller. }}
%}
%\psls{Address Binding.}{0.3}{../source/Pic/12-out.png}{}
%\psls{Address Binding.}{0.3}{../source/Pic/13-out.png}{}

%\pslsq{Review Questions.}{0.5}{7-q-8-out}{}
%\pslsq{Review Questions.}{0.5}{7-q-9-out}{}
%\pslsq{Review Questions.}{0.5}{7-q-1-out}{}
%\pslsq{Review Questions.}{0.5}{7-q-3-out}{}
%\pslsq{Review Questions.}{0.5}{7-q-7-out}{}
%

\end{document}
