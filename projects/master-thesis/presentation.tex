\documentclass[usenames, aspectratio=169]{beamer}

\usepackage{amsmath}
\usepackage{braket}
\usepackage{amsfonts}
\usepackage{tikz}
\usepackage{tkz-graph}
\usepackage{tikzpeople}
\usepackage{adjustbox}
\usepackage{subcaption}
\usepackage{svg}
\usepackage{graphicx}
\usepackage{media9}
\usepackage{float}
\usetikzlibrary{calc}
\usepackage{array}
\usepackage{efbox,graphicx}
\usepackage[normalem]{ulem}
\usepackage{verbatim}
\usepackage{ragged2e}
\usepackage{array}
\usepackage[backend=biber,style=alphabetic,sorting=ynt]{biblatex}
%\usepackage{columns}
\addbibresource{./sample.bib} 
  
\efboxsetup{linecolor=Green,linewidth=1.5pt, margin=0pt}

\usetikzlibrary{decorations.pathreplacing}
\usetikzlibrary{shapes}
\theoremstyle{claim}
\newtheorem{claim}[theorem]{Claim}%
\theoremstyle{remark}
\newtheorem{remark}[theorem]{Remark}%

\newcommand\MemoryLayout[1]{
  \begin{tikzpicture}[scale=0.15]
    \draw[thick](0,0)--++(0,3)node[above]{$0$};
    \foreach \pt/\col/\lab [remember=\pt as \tp (initially 0)] in {#1} {
      \foreach \a [parse=true] in {\tp,...,\pt-1} {
        \draw[fill=\col](-\a, 0) rectangle ++(-1,2);
      }
      \draw[thick](-\pt,0)--++(0,3)node[above]{$\pt$};
      \if\lab\relax\relax\else
        \draw[thick,decorate, decoration={amplitude=1mm}]
        (-\tp,-0.2)--node[below=1mm]{\lab} (-\pt,-0.2);
      \fi
    }
  \end{tikzpicture}
}


\newcommand{\pslsq}[4]{
\begin{frame}
    \frametitle{#1} 
    \includegraphics[width=.7\linewidth]{#3}
    #4  
  \end{frame}
}

\newcommand{\psls}[4]{
  \begin{frame}
    \frametitle{#1} 
    \begin{columns}[t]
      \begin{column}{.48\textwidth}
        #4
      \end{column}
      \begin{column}{.52\textwidth}
        \adjincludegraphics[width=.98\linewidth, valign=t]{#3}
      \end{column} 
    \end{columns}
  \end{frame}
}
\input{newcommands.tex}
\usepackage{sagetex}
%\usepackage{libertine}
%\usepackage{emerald}
%\usepackage[T1]{fontenc}
\usetheme[progressbar=frametitle]{metropolis}
\setbeamercolor{block title}{use=structure,fg=white,bg=structure.fg!75!black}
\setbeamercolor{block body}{parent=normal text,use=block title,bg=block title.bg!10!bg}

%\usetheme{EastLansing}
\title[From classical to good quantum LDPC codes.] % (optional, only for long titles)
{From classical to good quantum LDPC codes.}

\subtitle{  }
\author[D.~Ponarovsky] % (optional, for multiple authors)
	{D.~Ponarovsky\inst{1}}

\institute[HUJI] % (optional)
{  Faculty of Computer Science\newline
  Hebrew University of Jerusalem
}
\date[2023] % (optional)
{Master-Exam-Huji.}
\subject{Understanding Quantumness And Testability.}

\begin{document}
\input{sageutil.py}

\tikzset{
    LabelStyle/.append style = {  minimum width = 2em},
    VertexStyle/.append style = { inner sep=5pt,
        font = \Large\bfseries},
    EdgeStyle/.append style = {->} % added blue
}

\begin{frame}
  \maketitle
\end{frame}
%\pslsq{Today.}{0.3}{controller.png}{}
%\pslsq{Today.}{0.5}{controller-2-out.png}{}

\begin{frame}
  \frametitle{ Today. }
  \begin{itemize}
    \item<1-> Brif Review of Coding. \uncover<2->{Tanner and Expander codes. }
    \item<3-> Quantum Error Correction Codes. 
    \item<4->Good Classical Locally Testabile Codes and Good Qauntum LDPC.
  \end{itemize} 
\end{frame}


\begin{frame}
  \begin{center}
\begin{tikzpicture}[scale=0.95]

  \tikzset{
  font={\fontsize{8pt}{12}\selectfont}}

  \draw (13.7,6) node { Future. };
  \alt<4->{ 
    \draw (13.7,5) node { ''We understand quantum complexity. };
    \draw (13.7,4.5) node { as well as we understand classical complexity''. };
}{\draw (13.7,5) node { ''We understand quantum complexity''. };}
\uncover<3-> {\draw (13.7,4) node { \alt<5->{  QMA ? qPCP }{ \alt<3>{BQP (P) ? QMA (NP)}{BQP (P) ? QMA (NP)} ? PSPACE. }};}
\uncover<7->{\draw (13.7,3.5) node { $\downarrow$ };}
\uncover<8->{
  \draw (13.7,3) node { Exsitance of family of statmensts and quantum proofs, };
\draw (13.7,2.5) node {  such that any slightly noisy version of the proofs   };
\draw (13.7,2) node {  is still a proof and cann't be yilded by };
\draw (13.7,1.5) node { 'weak' computations. };
}
\uncover<8->{
  \draw (9.5,3) node { $\rightarrow$ };
  \draw (8,3.8) node { NLTS };
  \draw (8,3.3) node { Hamiltonians };
  \draw (8,2.8) node { from good};
  \draw (8,2.3) node { qLDPC codes \cite{anshu2022nlts}};
  \draw (8,6) node { 22-23 };
}
\uncover<9->{
  \draw (5.5,3) node { $\rightarrow$ };
  \draw (4,3) node { good qLDPC };
  \draw (4,2.5) node { \cite{Dinur}, \cite{Pavel}, \cite{leverrier2022quantum} };
  \draw (4,6) node { 21-22 };
}

        \end{tikzpicture}
    \end{center}
\end{frame}



\begin{frame}
  \frametitle{Introduction.}
  \includegraphics[width=.7\linewidth]{./Assumption-out.png}
\end{frame}



\begin{frame}
  \frametitle{ Coding. }
  Bob is willing to send some \alt<4->{\textbf{state}}{message} to Alice through \uncover<2->{a noisy chanell in which bits might be fliped.} \uncover<3->{By sending extra bits, i.e duplicate any bit tree times, B can ensure that A could still decode the orignal \alt<4->{\textbf{state}}{message} in the presence of a single bit-flip. } 
  \begin{center} 
     \begin{tikzpicture}
    \node[name=b, bob,monitor,minimum size=1cm,xshift=-7.2cm]{};
    \node[name= a, alice,monitor, mirrored,minimum size=1cm]{};
    \node (C) at (-2,0) {};
    \draw[ -> ]  (-6,0) to (-1.5,0); 
    \alt<5->{  \node (D) at (-4,1) { $\ket{1\textcolor{red}{0}0101}$ } ;}{\alt<2->{ \node (D) at (-4,1) { $1\textcolor{red}{0}0101$ } ; }{\node (D) at (-4,1) { $110101$ } ; } }
    \alt<5->{  \node (D) at (-2.85,1.5) { $\ket{1\textcolor{red}{0}0101 1\textcolor{red}{1}0101 1\textcolor{red}{1}0101}$ } ; }{\uncover<3->{ \node (D) at (-2.85,1.5) { $1\textcolor{red}{0}0101 1\textcolor{red}{1}0101 1\textcolor{red}{1}0101$ } ; } }
    %\uncover<2->{\node (D) at (-4,1) { $110101$ } ;  
  \end{tikzpicture}
\end{center}
\end{frame}

\begin{frame}
  \frametitle{Coding.}
  \scalebox{0.15}{
  \includegraphics{code-cloud-out-2.png-out.png}
}
\end{frame}


\begin{frame}
  \frametitle{ Coding. }
  Non formally, We call for the embedding of entitis in a larger space a code. And the questions that we would like to ask are:  
  \begin{itemize}[<+->]
    \item Can we come up with a code that tolerates $*$ bits flip? 
    \item At the cost of at most $*$ extra bits? 
    \item Can we ensure an efficient decoding (and checking) sechme? 
    \item In the assymptotic regime, when the size of the original message grows. 
  \end{itemize}
\end{frame} 

\begin{frame}
  \frametitle{ Coding. }
\begin{definition} 
  Let $n \in \mathbb{N}$ and $\rho, \delta\in \left( 0,1 \right)$. We say that $C$ is a \textbf{binary linear code} with parameters $[n, \rho n, \delta n]$. If $C$ is a subspace of $\mathbb{F}_{2}^{n}$, and the dimension of $C$ is at least $\rho n$ and any pair of distinict elements in $C$ differ in at least $\delta n$ coordinates. We call to the vectors belong to $C$ \textit{codewords}, to $\rho n$ the diminsion of the code, and to $\delta n$ the distance of the code.
  \end{definition}
\end{frame} 

\begin{frame}
  \frametitle{ Coding. }
\begin{definition} 
  A \textbf{family of codes} is an infinite series of codes. Additionally, suppose the rates and relative distances converge into constant values $\rho,\delta$. In that case, we abuse the notation and call that family of codes a code with $[n, \rho n, \delta n]$ for fixed $\rho, \delta\in [ 0,1 )$, and infinite integers $n \in \mathbb{N}$.     
  \end{definition}
\begin{definition} 
  We will say that a family of codes is a \textbf{good code} if its parameters converge into positive values. 
  \end{definition}
\end{frame} 

\begin{frame}
  \frametitle{ Tanner Codes. }
  \begin{definition} Let $\Gamma$ be a graph and $C_{0}$ be a ``small'' linear code with finate parameters $[\Delta, \rho\Delta, \delta\Delta]$. Let $ C = \mathcal{T}\left( \Gamma, C_{0} \right)$  be all the codewords which, for any vertex $v\in \Gamma$, the local view of $v$ is a codeword of $C_{0}$. We say that $C$ is a \textbf{Tanner code}\label{Tan} of $\Gamma, C_{0}$. Notice that if $C_{0}$ is a binary linear code, So $C$ is.  
  \end{definition}
\end{frame}


\begin{frame}
  \frametitle{Coding.} 
  
    Another example, the repttion code can be thought as the tanner graph defind by the parity code on the cyle graph.
    \begin{columns}[t]
      \begin{column}{0.4\textwidth}
    \begin{center}
      \scalebox{0.6}{
      \begin{tikzpicture}[scale=1]
      \draw
        (8.0, 0.0) node[shape=circle,draw=black] (0){}
        (7.825, 0.624) node[shape=circle,draw=black] (1){}
        (7.308, 1.22) node[shape=circle,draw=black] (2){}
        (6.472, 1.763) node[shape=circle,draw=black] (3){}
        (5.353, 2.229) node[shape=circle,draw=black] (4){}
        (4.0, 2.598) node[shape=circle,draw=black] (5){}
        (2.472, 2.853) node[shape=circle,draw=black] (6){}
        (0.836, 2.984) node[shape=circle,draw=black] (7){}
        (-0.836, 2.984) node[shape=circle,draw=black] (8){}
        (-2.472, 2.853) node[shape=circle,draw=black] (9){}
        (-4.0, 2.598) node[shape=circle,draw=black] (10){}
        (-5.353, 2.229) node[shape=circle,draw=black] (11){}
        (-6.472, 1.763) node[shape=circle,draw=black] (12){}
        (-7.308, 1.22) node[shape=circle,draw=black] (13){}
        (-7.825, 0.624) node[shape=circle,draw=black] (14){}
        (-8.0, 0.0) node[shape=circle,draw=black] (15){}
        (-7.825, -0.624) node[shape=circle,draw=black] (16){}
        (-7.308, -1.22) node[shape=circle,draw=black] (17){}
        (-6.472, -1.763) node[shape=circle,draw=black] (18){}
        (-5.353, -2.229) node[shape=circle,draw=black] (19){}
        (-4.0, -2.598) node[shape=circle,draw=black] (20){}
        (-2.472, -2.853) node[shape=circle,draw=black] (21){}
        (-0.836, -2.984) node[shape=circle,draw=black] (22){}
        (0.836, -2.984) node[shape=circle,draw=black] (23){}
        (2.472, -2.853) node[shape=circle,draw=black] (24){}
        (4.0, -2.598) node[shape=circle,draw=black] (25){}
        (5.353, -2.229) node[shape=circle,draw=black] (26){}
        (6.472, -1.763) node[shape=circle,draw=black] (27){}
        (7.308, -1.22) node[shape=circle,draw=black] (28){}
        (7.825, -0.624) node[shape=circle,draw=black] (29){};
      \begin{scope}[-,draw opacity=0.5]
        \draw (0) to node[] {$1$} (1);
        \draw (0) to node[] {$1$} (29);
        \draw (1) to node[] {$1$} (2);
        \draw (2) to node[] {$1$} (3);
        \draw (3) to node[] {$1$} (4);
        \draw (4) to node[] {$1$} (5);
        \draw (5) to node[] {$1$} (6);
        \draw (6) to node[] {$1$} (7);
        \draw (7) to node[] {$1$} (8);
        \draw (8) to node[] {$1$} (9);
        \draw (9) to node[] {$1$} (10);
        \draw (10) to node[] {$1$} (11);
        \draw (11) to node[] {$1$} (12);
        \draw (12) to node[] {$1$} (13);
        \draw (13) to node[] {$1$} (14);
        \draw (14) to node[] {$1$} (15);
        \draw (15) to node[] {$1$} (16);
        \draw (16) to node[] {$1$} (17);
        \draw (17) to node[] {$1$} (18);
        \draw (18) to node[] {$1$} (19);
        \draw (19) to node[] {$1$} (20);
        \draw (20) to node[] {$1$} (21);
        \draw (21) to node[] {$1$} (22);
        \draw (22) to node[] {$1$} (23);
        \draw (23) to node[] {$1$} (24);
        \draw (24) to node[] {$1$} (25);
        \draw (25) to node[] {$1$} (26);
        \draw (26) to node[] {$1$} (27);
        \draw (27) to node[] {$1$} (28);
        \draw (28) to node[] {$1$} (29);
      \end{scope}
    \end{tikzpicture}
 
  }
  \end{center}
\end{column}
\begin{column}{0.25\textwidth}
\begin{equation*}
    \begin{split}
     \overbrace{ 
      \begin{bmatrix}
        1 & 1
      \end{bmatrix}
    }^{ \text{ parity check matrix of } C_{0} }
    % \mathbb{F}_{2}^{1 \times 2} 
  \end{split}
\end{equation*}
\begin{center}
\begin{tikzpicture}
    \draw (0,0) circle (6pt);
    \draw[ - ]  (0,0) to (1,1); 
    \draw[ - ]  (0,0) to (1,-1); 
    %\uncover<2->{\node (D) at (-4,1) { $110101$ } ;  
  \end{tikzpicture}
\end{center}
\end{column}
\begin{column}{0.35\textwidth}
  \begin{equation*}
    \begin{split}
      \overbrace{ 
      \begin{bmatrix}
1 & 1 & 0 & 0 & 0 & 0 \\
0 & 1 & 1 & 0 & 0 & 0 \\
0 & 0 & 1 & 1 & 0 & 0 \\
0 & 0 & 0 & 1 & 1 & 0 \\ 
0 & 0 & 0 & 0 & 1 & 1 \\
1 & 0 & 0 & 0 & 0 & 1 
 \end{bmatrix}
 }^{  \substack{ \text{ Parity check matrix of } \mathcal{T} \left( \Gamma, C_{0} \right) \\ \text{  Each row associated with vertex check. }} }
    \end{split}
  \end{equation*}
\end{column}
\end{columns}
  \end{frame}

\begin{frame}
  \frametitle{ Tanner Codes. }
  Example, the parity code on the Peterson graph.
  \begin{center}
  \scalebox{0.65} {
\begin{tikzpicture}

\Vertex[style={minimum size=0.01cm,shape=circle},NoLabel,x=3.5cm,y=7.0cm]{v0}
\Vertex[style={minimum size=0.01cm,shape=circle},NoLabel,x=0.0cm,y=4.3262cm]{v1}
\Vertex[style={minimum size=0.01cm,shape=circle},NoLabel,x=1.3369cm,y=0.0cm]{v2}
\Vertex[style={minimum size=0.01cm,shape=circle},NoLabel,x=5.6631cm,y=0.0cm]{v3}
\Vertex[style={minimum size=0.01cm,shape=circle},NoLabel,x=7.0cm,y=4.3262cm]{v4}
\Vertex[style={minimum size=0.01cm,shape=circle},NoLabel,x=3.5cm,y=5.0652cm]{v5}
\Vertex[style={minimum size=0.01cm,shape=circle},NoLabel,x=1.75cm,y=3.7284cm]{v6}
\Vertex[style={minimum size=0.01cm,shape=circle},NoLabel,x=2.4184cm,y=1.5652cm]{v7}
\Vertex[style={minimum size=0.01cm,shape=circle},NoLabel,x=4.5816cm,y=1.5652cm]{v8}
\Vertex[style={minimum size=0.01cm,shape=circle},NoLabel,x=5.25cm,y=3.7284cm]{v9}
%
\Edge[lw=0.005cm,labelstyle={pos=0.5},label=\hbox{$0$},](v0)(v1)
\Edge[lw=0.005cm,labelstyle={pos=0.5},label=\hbox{$1$},](v0)(v4)
\Edge[lw=0.005cm,labelstyle={pos=0.5},label=\hbox{$1$},](v0)(v5)
\Edge[lw=0.005cm,labelstyle={pos=0.5},label=\hbox{$0$},](v1)(v2)
\Edge[lw=0.005cm,labelstyle={pos=0.5},label=\hbox{$0$},](v1)(v6)
\Edge[lw=0.005cm,labelstyle={pos=0.5},label=\hbox{$1$},](v2)(v3)
\Edge[lw=0.005cm,labelstyle={pos=0.5},label=\hbox{$1$},](v2)(v7)
\Edge[lw=0.005cm,labelstyle={pos=0.5},label=\hbox{$0$},](v3)(v4)
\Edge[lw=0.005cm,labelstyle={pos=0.5},label=\hbox{$1$},](v3)(v8)
\Edge[lw=0.005cm,labelstyle={pos=0.5},label=\hbox{$1$},](v4)(v9)
\Edge[lw=0.005cm,labelstyle={pos=0.5},label=\hbox{$0$},](v5)(v7)
\Edge[lw=0.005cm,labelstyle={pos=0.5},label=\hbox{$1$},](v5)(v8)
\Edge[lw=0.005cm,labelstyle={pos=0.5},label=\hbox{$0$},](v6)(v8)
\Edge[lw=0.005cm,labelstyle={pos=0.5},label=\hbox{$0$},](v6)(v9)
\Edge[lw=0.005cm,labelstyle={pos=0.5},label=\hbox{$1$},](v7)(v9)
%
\end{tikzpicture} \ \ \ \begin{tikzpicture}
\Vertex[style={minimum size=0.01cm,shape=circle},NoLabel,x=3.5cm,y=7.0cm]{v0}
\Vertex[style={minimum size=0.01cm,shape=circle},NoLabel,x=0.0cm,y=4.3262cm]{v1}
\Vertex[style={minimum size=0.01cm,shape=circle},NoLabel,x=1.3369cm,y=0.0cm]{v2}
\Vertex[style={minimum size=0.01cm,shape=circle},NoLabel,x=5.6631cm,y=0.0cm]{v3}
\Vertex[style={minimum size=0.01cm,shape=circle},NoLabel,x=7.0cm,y=4.3262cm]{v4}
\Vertex[style={minimum size=0.01cm,shape=circle},NoLabel,x=3.5cm,y=5.0652cm]{v5}
\Vertex[style={minimum size=0.01cm,shape=circle},NoLabel,x=1.75cm,y=3.7284cm]{v6}
\Vertex[style={minimum size=0.01cm,shape=circle},NoLabel,x=2.4184cm,y=1.5652cm]{v7}
\Vertex[style={minimum size=0.01cm,shape=circle},NoLabel,x=4.5816cm,y=1.5652cm]{v8}
\Vertex[style={minimum size=0.01cm,shape=circle},NoLabel,x=5.25cm,y=3.7284cm]{v9}
%
\Edge[lw=0.005cm,labelstyle={pos=0.5,},label=\hbox{$0$},](v0)(v1)
\Edge[lw=0.005cm,labelstyle={pos=0.5,},label=\hbox{$0$},](v0)(v4)
\Edge[lw=0.005cm,labelstyle={pos=0.5,},label=\hbox{$0$},](v0)(v5)
\Edge[lw=0.005cm,labelstyle={pos=0.5,},label=\hbox{$1$},](v1)(v2)
\Edge[lw=0.005cm,labelstyle={pos=0.5,},label=\hbox{$1$},](v1)(v6)
\Edge[lw=0.005cm,labelstyle={pos=0.5,},label=\hbox{$0$},](v2)(v3)
\Edge[lw=0.005cm,labelstyle={pos=0.5,},label=\hbox{$1$},](v2)(v7)
\Edge[lw=0.005cm,labelstyle={pos=0.5,},label=\hbox{$1$},](v3)(v4)
\Edge[lw=0.005cm,labelstyle={pos=0.5,},label=\hbox{$1$},](v3)(v8)
\Edge[lw=0.005cm,labelstyle={pos=0.5,},label=\hbox{$1$},](v4)(v9)
\Edge[lw=0.005cm,labelstyle={pos=0.5,},label=\hbox{$0$},](v5)(v7)
\Edge[lw=0.005cm,labelstyle={pos=0.5,},label=\hbox{$0$},](v5)(v8)
\Edge[lw=0.005cm,labelstyle={pos=0.5,},label=\hbox{$1$},](v6)(v8)
\Edge[lw=0.005cm,labelstyle={pos=0.5,},label=\hbox{$0$},](v6)(v9)
\Edge[lw=0.005cm,labelstyle={pos=0.5,},label=\hbox{$1$},](v7)(v9)
%
\end{tikzpicture} 

}
\end{center}
\end{frame}

%\begin{frame}
%  \frametitle{Coding.} 
%  
%    Another example, the repttion code can be thought as the tanner graph defind by the parity code on the cyle graph.
%    \begin{center}
%      \scalebox{0.7}{
%      \begin{tikzpicture}[scale=1]
      \draw
        (8.0, 0.0) node[shape=circle,draw=black] (0){}
        (7.825, 0.624) node[shape=circle,draw=black] (1){}
        (7.308, 1.22) node[shape=circle,draw=black] (2){}
        (6.472, 1.763) node[shape=circle,draw=black] (3){}
        (5.353, 2.229) node[shape=circle,draw=black] (4){}
        (4.0, 2.598) node[shape=circle,draw=black] (5){}
        (2.472, 2.853) node[shape=circle,draw=black] (6){}
        (0.836, 2.984) node[shape=circle,draw=black] (7){}
        (-0.836, 2.984) node[shape=circle,draw=black] (8){}
        (-2.472, 2.853) node[shape=circle,draw=black] (9){}
        (-4.0, 2.598) node[shape=circle,draw=black] (10){}
        (-5.353, 2.229) node[shape=circle,draw=black] (11){}
        (-6.472, 1.763) node[shape=circle,draw=black] (12){}
        (-7.308, 1.22) node[shape=circle,draw=black] (13){}
        (-7.825, 0.624) node[shape=circle,draw=black] (14){}
        (-8.0, 0.0) node[shape=circle,draw=black] (15){}
        (-7.825, -0.624) node[shape=circle,draw=black] (16){}
        (-7.308, -1.22) node[shape=circle,draw=black] (17){}
        (-6.472, -1.763) node[shape=circle,draw=black] (18){}
        (-5.353, -2.229) node[shape=circle,draw=black] (19){}
        (-4.0, -2.598) node[shape=circle,draw=black] (20){}
        (-2.472, -2.853) node[shape=circle,draw=black] (21){}
        (-0.836, -2.984) node[shape=circle,draw=black] (22){}
        (0.836, -2.984) node[shape=circle,draw=black] (23){}
        (2.472, -2.853) node[shape=circle,draw=black] (24){}
        (4.0, -2.598) node[shape=circle,draw=black] (25){}
        (5.353, -2.229) node[shape=circle,draw=black] (26){}
        (6.472, -1.763) node[shape=circle,draw=black] (27){}
        (7.308, -1.22) node[shape=circle,draw=black] (28){}
        (7.825, -0.624) node[shape=circle,draw=black] (29){};
      \begin{scope}[-,draw opacity=0.5]
        \draw (0) to node[] {$1$} (1);
        \draw (0) to node[] {$1$} (29);
        \draw (1) to node[] {$1$} (2);
        \draw (2) to node[] {$1$} (3);
        \draw (3) to node[] {$1$} (4);
        \draw (4) to node[] {$1$} (5);
        \draw (5) to node[] {$1$} (6);
        \draw (6) to node[] {$1$} (7);
        \draw (7) to node[] {$1$} (8);
        \draw (8) to node[] {$1$} (9);
        \draw (9) to node[] {$1$} (10);
        \draw (10) to node[] {$1$} (11);
        \draw (11) to node[] {$1$} (12);
        \draw (12) to node[] {$1$} (13);
        \draw (13) to node[] {$1$} (14);
        \draw (14) to node[] {$1$} (15);
        \draw (15) to node[] {$1$} (16);
        \draw (16) to node[] {$1$} (17);
        \draw (17) to node[] {$1$} (18);
        \draw (18) to node[] {$1$} (19);
        \draw (19) to node[] {$1$} (20);
        \draw (20) to node[] {$1$} (21);
        \draw (21) to node[] {$1$} (22);
        \draw (22) to node[] {$1$} (23);
        \draw (23) to node[] {$1$} (24);
        \draw (24) to node[] {$1$} (25);
        \draw (25) to node[] {$1$} (26);
        \draw (26) to node[] {$1$} (27);
        \draw (27) to node[] {$1$} (28);
        \draw (28) to node[] {$1$} (29);
      \end{scope}
    \end{tikzpicture}
 
%  }
%  \end{center}
%  \end{frame}
%


\begin{frame}
  \frametitle{Coding.}
\begin{lemma}
\label{tanrate} Tanner codes have a rate of at least $2\rho - 1$.
\end{lemma}

\uncover<2->{
  \begin{proof}  The dimension of the subspace is bounded by the dimension of the container minus the number of restrictions. So assuming non-degeneration of the small code restrictions, we have that any vertex count exactly $ \left( 1 - \rho  \right)\Delta $ restrictions. Hence, \begin{equation*}
    \begin{split}
      \dim C & \ge \frac{1}{2}n\Delta - \left( 1-\rho \right)\Delta n = \frac{1}{2}n\Delta\left( 2\rho - 1 \right)  
    \end{split}
  \end{equation*} Clearly, any small code with rate $> \frac{1}{2}$ will yield a code with an asymptotically positive rate \end{proof} 
}
\end{frame}

\begin{frame}
  \frametitle{Coding.}
  Note, that if the $\Gamma$ is a family of $\Delta$-regular graphs then, the size (length, dim, and dis) of $C_{0}$ is $O(1)$ as $n$ grows and the hamming weight of any row in the parity chack matrix of $\mathcal{T}(\Gamma,C_{0})$ is finte. We say that a family of codes with a parity cheack matrices having a constant row weight is a Low Density Parity Check code (LDPC). LDPC can be view as the first local property and also has practical value as it implies a linear time algorithm for correctness verification.   
\end{frame}



\begin{frame}
  \frametitle{Coding.}
\begin{definition} Denote by $\lambda$ the second eigenvalue of the adjacency matrix of the $\Delta$-regular graph. For our uses, it will be satisfied to define expander as a graph $G = \left( V,E \right)$ such that for any two subsets of vertices $T,S \subset V$, the number of edges between $S$ and $T$ is at most:
  \begin{equation*}
    \begin{split}
      \mid E\left( S,T \right) - \frac{\Delta}{n}|S||T| \mid \le \lambda\sqrt{|S| |T|} 
    \end{split}
  \end{equation*}
\end{definition}
\end{frame}


\begin{frame}
  \frametitle{Codeing.}
\begin{lemma} Theorem, let $C$ be the Tanner Code defined by the small code $C_{0} = [\Delta,\delta\Delta, \rho\Delta ]$ such that $\rho \ge \frac{1}{2}$ and the expander graph $G$ such that $\delta\Delta \ge \lambda$. $C$ is a good  LDPC code.
  \end{lemma}
  \uncover<2->{
  \begin{proof} Fix a codeword $x \in C$ and denote By $S$ the support of $x$ over the edges. Namely, a vertex $v\in V$ belongs to $S$ if it connects to nonzero edges regarding the assignment by $x$, Assume towards contradiction that $|x| = o\left( n \right)$. And notice that $|S|$ is at most $2|x|$, Then by The Expander Mixining Lemma we have that: 
  \begin{equation*}
    \begin{split}
      \frac{E\left( S,S \right)}{|S|} & \le \frac{\Delta}{n}|S|  + \lambda \\
      & \le_{ n \rightarrow \infty} o\left( 1 \right) + \lambda
    \end{split}
  \end{equation*}
   \end{proof}
}


\end{frame}

\begin{frame}
  \frametitle{Coding.}
  \begin{proof}
  \begin{equation*}
    \begin{split}
      \frac{E\left( S,S \right)}{|S|} & \le \frac{\Delta}{n}|S|  + \lambda \\
      & \le_{ n \rightarrow \infty} o\left( 1 \right) + \lambda
    \end{split}
  \end{equation*}

  Namely, for any such sublinear weight string, $x$, the average of nontrivial edges for the vertex is less than $\lambda$. So there must be at least one vertex $v \in S$ that, on his local view, sets a  string at a weight less than $\lambda$. By the definition of $S$, this string cannot be trivial. Combining the fact that any nontrivial codeword of the $C_{0}$ is at weight at least $\delta\Delta$, we get a contradiction to the assumption that $v$ is satisfied, videlicet, $x$ can't be a codeword \end{proof}

\end{frame}
\begin{frame}
  \frametitle{Quantum In Our Presentation.}
  For presanting as simple as possible, we will refer to quantum state $\ket{\psi}$ as a linear combination of classical states with $\pm 1$ coffeicents scaled by a normalization factor such the $l_{2}$ is $1$. For example, let $\ket{11}$, $\ket{00}$ and $\ket{01}$ be classical states, then $\frac{1}{\sqrt{3}}(\ket{11} + \ket{00} - \ket{01})$ is a quantum state.


  \uncover<2->{
    \begin{enumerate}
      \item bit flip $X \ket{0} \rightarrow \ket{1}$, $X \ket{1} \rightarrow \ket{0}$ 

        \uncover<3-> {\item phase flip $Z \ket{0} \rightarrow \ket{0}$, $Z \ket{1} \rightarrow -\ket{1}$ }
    \end{enumerate}

  }
\end{frame}


\begin{frame}
  \frametitle{Quantum Error Correction Codes.}
  Observation, $\ket{0}$ and $\ket{1}$ cann't both encoded to a single ket codeword. 
\end{frame}


%
%  \begin{frame}
%    \frametitle{ Quantum Error Correction Codes. }
%    Back to the quantum noise. 
%  \end{frame} 


%
%\begin{frame}
%  \frametitle{ Quantum Noise. }
%  \begin{center}
%  \begin{tikzpicture}
%    \node[name=b, bob,monitor,minimum size=1cm,xshift=-7.2cm]{};
%    \node[name= a, alice,monitor, mirrored,minimum size=1cm]{};
%    \node (C) at (-2,0) {};
%    \draw[ -> ] (b.mouth) + (1,0) to (C)  ; 
%    \alt<3->{ 
%      \node(D) at (-4,1.5) { $\ket{\textcolor{red}{0}10101} + \ket{\textcolor{red}{1}10100} \textcolor{blue}{-} \ket{\textcolor{red}{1}1\textcolor{blue}{1}110}$} ;
%  \node(E) at (-4,1) { $\ket{110101} + \ket{010100} + \ket{011110}$} ;
%  }{
%      \node(D) at (-4,1) { $\ket{110101} + \ket{010100} + \ket{011110}$} ;
%    }
%    %\uncover<2->{\node (D) at (-4,1) { $110101$ } ;  
%  \end{tikzpicture}
%\end{center}
%\end{frame}
%


\begin{frame}
  \frametitle{ Quantum Error Correction Codes. }
  The clouds perspective. 
\end{frame}


\begin{frame}
  \frametitle{ Quantum Error Correction Codes. }
\begin{definition}[CSS Code]
  Let $C_{X}, C_{Z}$ classical linear codes such that $C_{Z}^{\perp} \subset C_{X}$ define the $Q\left( C_{X},C_{Z} \right)$ to be all the codewords with following structure:
  \begin{equation*}
    \begin{split}
    \ket { \mathbf{ x } } := \ket { x + C_{Z}^{\perp} } = \frac{1}{\sqrt{C_{Z}^{\perp}}} \sum_{z \in C_{Z}^{\perp}}{ \ket{ x + z }} 
    \end{split}
  \end{equation*}
\end{definition}
\end{frame} 



\begin{frame}
  \frametitle{ Quantum Error Correction Codes. }
  Observation, $C_{X}, C_{Z}$ cann't be both good codes and qLDPC codes.

Furthermore, if one is willing to has an qLDPC code, then $H_{X}$ and $H_{Z}$ can't be parity check matrices of good classical code as any column of $H_{z}^{\top}$ is a codeword of $C_{X}$. 
  \begin{equation*}
    \begin{split}
      C_{Z}^{\perp} \subset C_{X} \Rightarrow H_{X}H_{Z}^{\top} = 0 
    \end{split}
  \end{equation*}
  And by being an LDPC code, the rows wights of $H_{Z}$ is bounded by constant. Therefore there is a codeword $\in C_{X}$ which is also a row of $H_{Z}$ that has a constant weight.  

\end{frame}

\begin{frame}
  \frametitle{ Quantum Error Correction Codes. }
  Observation, setting a small qauntum code on a graph, in similar manner to tanner code construction doesn't give a CSS code. two adjoint vertices define checks that intersect in excactly one coordinate. 
  %$C_{X}, C_{Z}$ cann't be both good codes and qLDPC codes.
\end{frame}

\begin{frame}
  \frametitle{ Quantum Error Correction Codes. }
  \begin{center}
    \begin{figure}[H]
  \begin{tikzpicture}
  \draw[step=1cm,gray,very thin] (0,0) grid (5,5);
  \foreach \x in {0,1,2,3,4,5}
  \foreach \y in {0,1,2,3,4,5}
  {
  \node[draw,circle,inner sep=2pt,fill] at (\x,\y) {};
}
\draw[ -> ]  (0,0) to [out=50, in=130] (0,5);
\draw[ -> ]  (1,0) to [out=50, in=130] (1,5);
\draw[ -> ]  (2,0) to [out=50, in=130] (2,5);
\draw[ -> ]  (3,0) to [out=50, in=130] (3,5);
\draw[ -> ]  (4,0) to [out=50, in=130] (4,5);
\draw[ -> ]  (5,0) to [out=50, in=130] (5,5);
\draw[ -> ]  (0,5) to [out=130, in=220] (5,5);
\draw[ -> ]  (0,4) to [out=130, in=220] (5,4);
\draw[ -> ]  (0,3) to [out=130, in=220] (5,3);
\draw[ -> ]  (0,2) to [out=130, in=220] (5,2);
\draw[ -> ]  (0,1) to [out=130, in=220] (5,1);
\draw[ -> ]  (0,0) to [out=130, in=220] (5,0);

\node[draw,circle,inner sep=2pt,fill] at (9,2) {};
\node[draw,circle,inner sep=2pt,fill] at (10,2) {};
\node[draw,circle,inner sep=2pt,fill] at (8,2) {};
\node[draw,circle,inner sep=2pt,fill] at (9,1) {};
\node[draw,circle,inner sep=2pt,fill] at (9,3) {};
\draw[ -> ]  (9,2) to (10,2);
\draw[ -> ]  (9,2) to (8,2);
\draw[ -> ]  (9,2) to (9,1);
\draw[ -> ]  (9,2) to (9,3);
%\draw[ -> ]  (9,2) to (5,0);

\node[draw,circle,inner sep=2pt,fill] at (10,1) {};
\node[draw,circle,inner sep=2pt,fill] at (11,1) {};
\node[draw,circle,inner sep=2pt,fill] at (10,0) {};
\node[draw,circle,inner sep=2pt,fill] at (11,0) {};
\draw[ -> ]  (10,1) to (11,1);
\draw[ -> ]  (10,1) to (10,0);
\draw[ -> ]  (10,0) to (11,0);
\draw[ -> ]  (11,1) to (11,0);
\end{tikzpicture}
\caption{On the left is the Toric Graph. On the right are cross and face checks.}
\label{fig:Toric}
\end{figure}
\end{center}

\end{frame} 

\begin{frame}
  \frametitle{ Quantum Error Correction Codes. }
\begin{definition}[$w$-Robustness] 
  \label{def:wrobust}
  Let $C_{A}$ and $C_{B}$ be codes of length $\Delta$ with minimum distance $\delta_{0}\Delta$. $C = \duC $ will be said to be $w$-robust if for any codeword $c \in C$ of weight less than $w$, it follows that $c$ can be decomposed into a sum of $c = t + s$ such that $t \in C_A \otimes \mathbb{F}^{B}$ and $s \in \mathbb{F}^A \otimes C_B$, where $s$ and $t$ are each supported on at most $\frac{w}{\delta_0\Delta}$ rows and columns. For convenience, we will denote by $B'$ ($A'$) the rows (columns) supporting $t$ ($s$) and use the notation $t \in C_A \otimes \mathbb{F}^{B'}$.
\end{definition}
\end{frame} 

\begin{frame}
  \frametitle{ Quantum Error Correction Codes. }
  \begin{center}
  \scalebox{0.65}{
  \begin{tikzpicture}[scale=0.95]

\draw (5.5,6.3) node[scale=2]  { $=$ };
\draw (11.5,6.3) node[scale=2]  { $+$ };

\draw [decorate,decoration={brace,amplitude=5pt,raise=4ex}] 
(0.5,5) -- (4.5,5) node[scale=2,midway,yshift=2em]{$c \in \left(C_{A}^{\perp}\otimes C_{B}^{\perp}\right)^{\perp} $};
    \filldraw [fill=white!80!black](0,0) rectangle (5,5);
   \fill [fill=gray!80!black] (1,0) rectangle (2,5);
    \fill [fill=gray!80!black] (0,1) rectangle (5,3);
\filldraw [fill=white!80!black](6,0) rectangle (11,5);
\draw [decorate,decoration={brace,amplitude=5pt,raise=4ex}] 
(6.5,5) -- (10.5,5) node[scale=2,midway,yshift=2em]{$t \in C_{A}\otimes \mathbb{F}^{B}$};
   \fill [fill=gray!80!black,draw opacity=0.5] (6,1) rectangle (11,3);
\filldraw [fill=white!80!black](12,0) rectangle (17,5);
\draw [decorate,decoration={brace,amplitude=5pt,raise=4ex}] 
(12.5,5) -- (16.5,5) node[scale=2,midway,yshift=2em]{$s \in \mathbb{F}^{A}\otimes C_{B}$};
 \fill [fill=gray!80!black,draw opacity=0.5] (13,0) rectangle (14,5);
        \end{tikzpicture}
      }
    \end{center}
\end{frame} 

\begin{frame}
  \frametitle{ Quantum Error Correction Codes. }
  \begin{center}
  \scalebox{0.55}{
      \begin{tikzpicture}[scale=0.8]
            \draw[thick](0,0)(0, 0) -- (1.9049911888377424,4.002079117547451) -- (5.81285084955138,4.602079117547451) -- (5.212850849551381,0.2192498173032679) -- (0, 0)
(0, 0) -- (2.879157898699928,2.130954290533681) -- (6.412850849551381,3.330954290533681) -- (5.212850849551381,0.2192498173032679) -- (0, 0)
(0, 0) -- (4.588270563356833,4.1850314105127575) -- (7.012850849551381,5.985031410512757) -- (5.212850849551381,0.2192498173032679) -- (0, 0)
(0, 0) -- (1.9049911888377424,4.002079117547451) -- (5.890590913653101,4.602079117547451) -- (5.2905909136531015,0.6119546131629411) -- (0, 0)
(0, 0) -- (2.879157898699928,2.130954290533681) -- (6.490590913653102,3.330954290533681) -- (5.2905909136531015,0.6119546131629411) -- (0, 0)
(0, 0) -- (4.588270563356833,4.1850314105127575) -- (7.090590913653101,5.985031410512757) -- (5.2905909136531015,0.6119546131629411) -- (0, 0)
(0, 0) -- (1.9818949186177792,-4.217396385887875) -- (5.81285084955138,-4.817396385887875) -- (5.212850849551381,0.2192498173032679) -- (0, 0)
(0, 0) -- (3.9944175473955323,-4.094159016671296) -- (6.412850849551381,-5.294159016671296) -- (5.212850849551381,0.2192498173032679) -- (0, 0)
(0, 0) -- (2.2401823432002748,-5.388664191428558) -- (7.012850849551381,-7.188664191428558) -- (5.212850849551381,0.2192498173032679) -- (0, 0)
(0, 0) -- (1.9818949186177792,-4.217396385887875) -- (5.890590913653101,-4.817396385887875) -- (5.2905909136531015,0.6119546131629411) -- (0, 0)
(0, 0) -- (3.9944175473955323,-4.094159016671296) -- (6.490590913653102,-5.294159016671296) -- (5.2905909136531015,0.6119546131629411) -- (0, 0)
(0, 0) -- (2.2401823432002748,-5.388664191428558) -- (7.090590913653101,-7.188664191428558) -- (5.2905909136531015,0.6119546131629411) -- (0, 0)
(5.2905909136531015, 0.6119546131629411) -- (5.890590913653101,4.602079117547451) -- (10.291575076901802,5.502079117547451) -- (9.391575076901802,1.4619219538089987) -- (5.2905909136531015, 0.6119546131629411)
(5.2905909136531015, 0.6119546131629411) -- (6.490590913653102,3.330954290533681) -- (11.191575076901803,5.130954290533681) -- (9.391575076901802,1.4619219538089987) -- (5.2905909136531015, 0.6119546131629411)
(5.2905909136531015, 0.6119546131629411) -- (7.090590913653101,5.985031410512757) -- (12.091575076901801,8.685031410512757) -- (9.391575076901802,1.4619219538089987) -- (5.2905909136531015, 0.6119546131629411)
(5.2905909136531015, 0.6119546131629411) -- (5.890590913653101,4.602079117547451) -- (11.4486392380322,5.502079117547451) -- (10.5486392380322,0.7771825477033486) -- (5.2905909136531015, 0.6119546131629411)
(5.2905909136531015, 0.6119546131629411) -- (6.490590913653102,3.330954290533681) -- (12.3486392380322,5.130954290533681) -- (10.5486392380322,0.7771825477033486) -- (5.2905909136531015, 0.6119546131629411)
(5.2905909136531015, 0.6119546131629411) -- (7.090590913653101,5.985031410512757) -- (13.248639238032201,8.685031410512757) -- (10.5486392380322,0.7771825477033486) -- (5.2905909136531015, 0.6119546131629411)
(5.2905909136531015, 0.6119546131629411) -- (5.890590913653101,-4.817396385887875) -- (10.291575076901802,-3.917396385887875) -- (9.391575076901802,1.4619219538089987) -- (5.2905909136531015, 0.6119546131629411)
(5.2905909136531015, 0.6119546131629411) -- (6.490590913653102,-5.294159016671296) -- (11.191575076901803,-3.494159016671296) -- (9.391575076901802,1.4619219538089987) -- (5.2905909136531015, 0.6119546131629411)
(5.2905909136531015, 0.6119546131629411) -- (7.090590913653101,-7.188664191428558) -- (12.091575076901801,-4.488664191428557) -- (9.391575076901802,1.4619219538089987) -- (5.2905909136531015, 0.6119546131629411)
(5.2905909136531015, 0.6119546131629411) -- (5.890590913653101,-4.817396385887875) -- (11.4486392380322,-3.917396385887875) -- (10.5486392380322,0.7771825477033486) -- (5.2905909136531015, 0.6119546131629411)
(5.2905909136531015, 0.6119546131629411) -- (6.490590913653102,-5.294159016671296) -- (12.3486392380322,-3.494159016671296) -- (10.5486392380322,0.7771825477033486) -- (5.2905909136531015, 0.6119546131629411)
(5.2905909136531015, 0.6119546131629411) -- (7.090590913653101,-7.188664191428558) -- (13.248639238032201,-4.488664191428557) -- (10.5486392380322,0.7771825477033486) -- (5.2905909136531015, 0.6119546131629411)
;
\node at (5.91285084955138,4.602079117547451) {$ a_{ 0  } gb_{ 0 } $};
\node at (6.512850849551381,3.330954290533681) {$ a_{ 0  } gb_{ 1 } $};
\node at (7.11285084955138,5.985031410512757) {$ a_{ 0  } gb_{ 2 } $};
\node at (5.990590913653101,4.802079117547451) {$ a_{ 1  } gb_{ 0 } $};
\node at (6.590590913653101,3.530954290533681) {$ a_{ 1  } gb_{ 1 } $};
\node at (7.190590913653101,6.1850314105127575) {$ a_{ 1  } gb_{ 2 } $};
\node at (5.91285084955138,-4.817396385887875) {$ a_{ 0  } gb_{ 4 } $};
\node at (6.512850849551381,-5.294159016671296) {$ a_{ 0  } gb_{ 5 } $};
\node at (7.11285084955138,-7.188664191428558) {$ a_{ 0  } gb_{ 6 } $};
\node at (5.990590913653101,-4.617396385887875) {$ a_{ 1  } gb_{ 4 } $};
\node at (6.590590913653101,-5.094159016671296) {$ a_{ 1  } gb_{ 5 } $};
\node at (7.190590913653101,-6.988664191428557) {$ a_{ 1  } gb_{ 6 } $};
\node at (10.391575076901802,5.502079117547451) {$ a_{ 0  } a_{ 1 }gb_{ 0 } $};
\node at (11.291575076901802,5.130954290533681) {$ a_{ 0  } a_{ 1 }gb_{ 1 } $};
\node at (12.1915750769018,8.685031410512757) {$ a_{ 0  } a_{ 1 }gb_{ 2 } $};
\node at (11.5486392380322,5.702079117547451) {$ a_{ 1  } a_{ 1 }gb_{ 0 } $};
\node at (12.4486392380322,5.330954290533681) {$ a_{ 1  } a_{ 1 }gb_{ 1 } $};
\node at (13.3486392380322,8.885031410512756) {$ a_{ 1  } a_{ 1 }gb_{ 2 } $};
\node at (10.391575076901802,-3.917396385887875) {$ a_{ 0  } a_{ 1 }ga_{ 0 } $};
\node at (11.291575076901802,-3.494159016671296) {$ a_{ 0  } a_{ 1 }ga_{ 1 } $};
\node at (12.1915750769018,-4.488664191428557) {$ a_{ 0  } a_{ 1 }ga_{ 2 } $};
\node at (11.5486392380322,-3.717396385887875) {$ a_{ 1  } a_{ 1 }ga_{ 0 } $};
\node at (12.4486392380322,-3.294159016671296) {$ a_{ 1  } a_{ 1 }ga_{ 1 } $};
\node at (13.3486392380322,-4.288664191428557) {$ a_{ 1  } a_{ 1 }ga_{ 2 } $};
\node at (-0.1,0) {$ g $};
\node at (5.31285084955138,0.3192498173032679) {$ a_{ 0 }g $};
\node at (5.390590913653101,0.7119546131629411) {$ a_{ 1 }g $};
\node at (2.0049911888377423,4.102079117547451) {$ gb_{ 0 } $};
\node at (2.979157898699928,2.230954290533681) {$ gb_{ 1 } $};
\node at (4.6882705633568325,4.285031410512757) {$ gb_{ 2 } $};
\node at (2.0818949186177793,-4.117396385887876) {$ gb_{ 4 } $};
\node at (4.094417547395532,-3.9941590166712957) {$ gb_{ 5 } $};
\node at (2.340182343200275,-5.288664191428558) {$ gb_{ 6 } $};          
\end{tikzpicture}}
\end{center}
\end{frame} 

\begin{frame}
  \frametitle{ Quantum Error Correction Codes. }
\begin{definition}[$p$-Resistance to Puncturing.]
%  \label{def:resistance}
  Let $p,w$ be integers. We will say that the dual tensor code $C_{A} \otimes \mathbb{F} + \mathbb{F} \otimes C_{B}$ is $w$-robust with $p$-resistance to puncturing, if the code obtained by removing (puncturing) a subset of at most $p$ rows and columns is $w$-robust.   
\end{definition}
\end{frame} 

\begin{frame}
  \frametitle{ Quantum Error Correction Codes. }
\begin{definition}[Quantum Tanner Code.]

%  Let $\Gamma$ be a group on $n$ verices. And let $A,B$ be a two generator set of $\Gamma$ such that if $a \in A$ ($B$) then also $a^{-1}\in A$ and that for any $g\in \Gamma,a \in A, b \in B$ it holds that $g \neq agb$ . Define the left right Cayley complex to be the graph $G = \left( \Gamma, E \right) $ obtain by taking the union of the two Cayley graphs generated by $A$ and $B$. So the vertices pair $u,v$ are set on square diaginal only if there are $a\in A$ and $b \in B$ such that $u = avb$. We can assume that $G$ is a bipartite graph (otherise just take $\Gamma^{\prime} = \Gamma \times \mathbb{Z}_{2}$ and define the product to be $a\left( u,\pm \right) = \left( au, \mp \right)$). 
%
%  Now divide the graph into postivie and negative vertices according their coloring $V_{-}$ and $V_{+}$. And define the positive graph to be $G^{+} = \left( V_{+}, E \right)$ and by $G^{-} = \left( V_{-}, E \right)$ the negative graph, when $E$denotes the sqaures, put it defrently their is an edge between $v$ and $u$ in $G^{+}$ if both vertices are positive and they are lay on the ends of square's diongal.
%
%  The quantum tanner code is a CSS code, such $C_{X}$ defined to be the classical tanner code $\mathcal{T}\left(G^{+}, \left(C_{A}^\perp\otimes C_{B}^{\perp}\right)^{\perp} \right)$ and $C_{Z}$ define as $\mathcal{T}\left(G^{-}, \left(C_{A}\otimes C_{B}\right)^{\perp} \right)$. Noice that in contrast to classical tanner code, in the quantum case it will be more convinent to think codewords as assiments set on the squaers and not on the edges.  
%
  Let $\Gamma$ be a group at size $n$. And let $A,B$ be a two generator set of $\Gamma$ such that if $a \in A$ ($B$) then also $a^{-1}\in A$ ($B^{-1}$) and that for any $g\in \Gamma, a \in A, b \in B$ it holds that $g \neq agb$. Define the left-right Cayley complex to be the graph $G = \left( \Gamma, E \right)$ obtained by taking the union of the two Cayley graphs generated by $A$ and $B$. So the vertices pair $u,v$ are set on a square diagonal only if there are $a\in A$ and $b \in B$ such that $u = avb$. We can assume that $G$ is a bipartite graph (otherwise just take $\Gamma^{\prime} = \Gamma \times \mathbb{Z}_{2}$ and define the product to be $a\left( u,\pm \right) = \left( au, \mp \right)$).
\end{definition}
\end{frame}


\begin{frame}
  \frametitle{ Quantum Error Correction Codes. }
\begin{definition}[Quantum Tanner Code.]


Now divide the graph into positive and negative vertices according to their coloring $V_{-}$ and $V_{+}$. And define the positive graph to be $G^{+} = \left( V_{+}, E \right)$ and by $G^{-} = \left( V_{-}, E \right)$ the negative graph, where $E$ denotes the squares, put differently there is an edge between $v$ and $u$ in $G^{+}$ if both vertices are positive and they are laid on the ends of a square's diagonal.


The quantum Tanner code is a CSS code, such that $C_{X}$ is defined to be the classical Tanner code $\mathcal{T}\left(G^{+}, \left(C_{A}^\perp\otimes C_{B}^{\perp}\right)^{\perp} \right)$ and $C_{Z}$ is defined as $\mathcal{T}\left(G^{-}, \left(C_{A}\otimes C_{B}\right)^{\perp} \right)$. Note that in contrast to the classical Tanner code, in the quantum case it will be more convenient to think of codewords as assignments set on the squares and not on the edges.
\end{definition}

\end{frame} 

\begin{frame}
The existence of good quantum LDPC codes and locally testable codes (LTCs) was considered an open problem for roughly two decades. Although they seemed to be related only by containing the word "code" in their names, they were proven to exist by the same construction. They first appeared in~\cite{Dinur} as good locally testable codes and not long after that in~\cite{Pavel}, in which they also extended and derived the result to obtain the quantum code. We emphasize that even though they developed the same codes, their proofs are not similar at all. Here, we follow the~\cite{leverrier2022quantum} work, which simplifies the original proof and does not rely on any concept more complicated than what we have already seen in the previous chapters. They also coined the term "Quantum Tanner Codes" referring to the fact that $C_{X}$ and $C_{Z}$ are classical Tanner codes. Yet, the proof we present is not exactly the same, as we use a small code that requires satisfying a stronger assumption (the $w$-robustness property~\ref{def:wrobust}) relative to the original work. The reason why they had to use a weaker assumption is because the existence of codes satisfying the stronger one was proven a year later~\cite{kalachev2022twosided}. Relying on the stronger assumption allows us to simplify the proof even more and get rid of another requirement that the small code has to satisfy (The $p$-resistance to puncturing~\ref{def:resistance}).
\end{frame}

\begin{frame}
Recall our insight that for a pair of LDPC codes to define a good CSS code, they must both be poor codes in the sense that they must have a constant distance. Therefore, we understand that any codeword in $C_{X}$ with small weight belongs to $C_{Z}^\perp$. To prove this, we will construct a proof such that if $x \in C_{X}$ and $|x|$ is small, then there is a small codeword $z \in C_{Z}^{\perp}$ such that $|x+z| < |x|$; by repeating this process recursively, it follows that $x\in C_{Z}^{\perp}$. To formulate this theorem, we will need to define more definitions.
  
 The next two definitions are concerned with the properties of the small code that will be set on the edges. Using them, one can characterize cases in which a local view can be reduced by subtracting a codeword from the dual code. 
\end{frame}
\begin{frame}
\begin{definition}[$w$-Robustness] 
  \label{def:wrobust}
  Let $C_{A}$ and $C_{B}$ be codes of length $\Delta$ with minimum distance $\delta_{0}\Delta$. $C = \duC $ will be said to be $w$-robust if for any codeword $c \in C$ of weight less than $w$, it follows that $c$ can be decomposed into a sum of $c = t + s$ such that $t \in C_A \otimes \mathbb{F}^{B}$ and $s \in \mathbb{F}^A \otimes C_B$, where $s$ and $t$ are each supported on at most $\frac{w}{\delta_0\Delta}$ rows and columns. For convenience, we will denote by $B'$ ($A'$) the rows (columns) supporting $t$ ($s$) and use the notation $t \in C_A \otimes \mathbb{F}^{B'}$.
\end{definition}


\begin{figure}[h]
  \label{fig:wrobustf}
  \begin{tikzpicture}[scale=0.95]

    \draw (5.5,6.3) node[scale=2]  { $=$ };
    \draw (11.5,6.3) node[scale=2]  { $+$ };

\draw [decorate,decoration={brace,amplitude=5pt,raise=4ex}] 
(0.5,5) -- (4.5,5) node[scale=2,midway,yshift=2em]{$c \in \left(C_{A}^{\perp}\otimes C_{B}^{\perp}\right)^{\perp} $};
        \filldraw [fill=white!80!black](0,0) rectangle (5,5);
       \fill [fill=gray!80!black] (1,0) rectangle (2,5);
        \fill [fill=gray!80!black] (0,1) rectangle (5,3);
\filldraw [fill=white!80!black](6,0) rectangle (11,5);
 \draw [decorate,decoration={brace,amplitude=5pt,raise=4ex}] 
 (6.5,5) -- (10.5,5) node[scale=2,midway,yshift=2em]{$t \in C_{A}\otimes \mathbb{F}^{B}$};
       \fill [fill=gray!80!black,draw opacity=0.5] (6,1) rectangle (11,3);
\filldraw [fill=white!80!black](12,0) rectangle (17,5);
  \draw [decorate,decoration={brace,amplitude=5pt,raise=4ex}] 
  (12.5,5) -- (16.5,5) node[scale=2,midway,yshift=2em]{$s \in \mathbb{F}^{A}\otimes C_{B}$};
     \fill [fill=gray!80!black,draw opacity=0.5] (13,0) rectangle (14,5);
            \end{tikzpicture}
            \caption{$w$-Robustness, Any low-weight codeword of the dual tensor code $c$ can be decomposed into a sum $t+s$, where $t$ is a collection of rows, each of which is a codeword in $C_A$, and similarly $s$ is a collection of columns, each of which is a codeword of $C_B$. }
\end{figure}

\end{frame}
\begin{frame}

The definition we gave for $w$-Robustness is identical to the one stated by Zemor and Leverrier, but we also included the decomposition property in the definition. We refer readers to the appendix section in~\cite{leverrier2022quantum} for an existence proof of $w$-robustness codes for $w = \Delta^{3/2 - \varepsilon}, \varepsilon > 0$, through random construction. We note that the random construction also yields the Gilbert-Varshamov bound. Though, we need a $w$-robustness codes for $\Delta^{3/2+\varepsilon}$ where $\varepsilon$ is positive. For achiving that we use the theorem proven in~\cite{kalachev2022twosided}:  

\end{frame}

\begin{frame}

\begin{theorem}
Fix $\rho_A, \rho_B \in (0,1)$, and let $\kappa$ be:
\begin{equation*}
  \begin{split}
    \kappa = \frac{1}{2}\min\left( \frac{1}{4}H_{2}^{-1}\left( \frac{\rho_{A}}{8} \right) H_{2}^{-1}\left( \frac{\rho_{B}}{8} \right), H_{2}^{-1}\left( \frac{\rho_{A}\rho_{B}}{8} \right)\right)
  \end{split}
\end{equation*}
where $H^{-1}_{2}$ denotes the inverse of the binary entropy function. Let $C_A, C_B$ be $\Delta$-length and $\rho_A \Delta, \rho_B \Delta$ codimension codes sampled uniformly at random. Then, with high probability as $\Delta \rightarrow \infty$, for any codeword of their dual tensor code $c \in \duC$, there exists a decomposition of $c$ into a sum of $c = t + s$ such that $t \in C_A \otimes \mathbb{F}^B$ and $s \in \mathbb{F}^A \otimes C_B$, where $s$ and $t$ are each supported on at most $\frac{c}{\kappa \Delta}$ rows and columns. We call such codes $\kappa$ product expanding.
\end{theorem}
\end{frame}

\begin{frame}

Note that the fact that sampling succeeds with high probability implies that, with high probability, the codes that are sampled have a good distance, as well as their duals. By denoting $\delta \leftarrow \min\{\kappa, \delta\}$, we can say that, for any rate $\rho$ and large enough $\Delta$, there exists $\delta > 0$ such that $\duC$ is $\Delta^{3/2+\varepsilon}$-robust for $\varepsilon < \frac{1}{2}$ and $C_{A},C_{B},C_{A}^{\perp},C_{B}^{\perp}$ have rate and relative distance of at least $\rho$ and $\delta$, respectively.

\begin{definition}[$p$-Resistance to Puncturing.]
  \label{def:resistance}
  Let $p,w$ be integers. We will say that the dual tensor code $C_{A} \otimes \mathbb{F} + \mathbb{F} \otimes C_{B}$ is $w$-robust with $p$-resistance to puncturing, if the code obtained by removing (puncturing) a subset of at most $p$ rows and columns is $w$-robust.   
\end{definition}
Our proof does not utilize $p$-resistance to puncturing, yet it is a fundamental component in~\cite{leverrier2022quantum}. Therefore, we will later indicate where and how precisely the $p$-resistance is being used. Now, we will define exactly what the code is.
\end{frame}
\begin{frame}
\begin{definition}[Quantum Tanner Code.]

%  Let $\Gamma$ be a group on $n$ verices. And let $A,B$ be a two generator set of $\Gamma$ such that if $a \in A$ ($B$) then also $a^{-1}\in A$ and that for any $g\in \Gamma,a \in A, b \in B$ it holds that $g \neq agb$ . Define the left right Cayley complex to be the graph $G = \left( \Gamma, E \right) $ obtain by taking the union of the two Cayley graphs generated by $A$ and $B$. So the vertices pair $u,v$ are set on square diaginal only if there are $a\in A$ and $b \in B$ such that $u = avb$. We can assume that $G$ is a bipartite graph (otherise just take $\Gamma^{\prime} = \Gamma \times \mathbb{Z}_{2}$ and define the product to be $a\left( u,\pm \right) = \left( au, \mp \right)$). 
%
%  Now divide the graph into postivie and negative vertices according their coloring $V_{-}$ and $V_{+}$. And define the positive graph to be $G^{+} = \left( V_{+}, E \right)$ and by $G^{-} = \left( V_{-}, E \right)$ the negative graph, when $E$denotes the sqaures, put it defrently their is an edge between $v$ and $u$ in $G^{+}$ if both vertices are positive and they are lay on the ends of square's diongal.
%
%  The quantum tanner code is a CSS code, such $C_{X}$ defined to be the classical tanner code $\mathcal{T}\left(G^{+}, \left(C_{A}^\perp\otimes C_{B}^{\perp}\right)^{\perp} \right)$ and $C_{Z}$ define as $\mathcal{T}\left(G^{-}, \left(C_{A}\otimes C_{B}\right)^{\perp} \right)$. Noice that in contrast to classical tanner code, in the quantum case it will be more convinent to think codewords as assiments set on the squaers and not on the edges.  
%
  Let $\Gamma$ be a group at size $n$. And let $A,B$ be a two generator set of $\Gamma$ such that if $a \in A$ ($B$) then also $a^{-1}\in A$ ($B^{-1}$) and that for any $g\in \Gamma, a \in A, b \in B$ it holds that $g \neq agb$. Define the left-right Cayley complex to be the graph $G = \left( \Gamma, E \right)$ obtained by taking the union of the two Cayley graphs generated by $A$ and $B$. So the vertices pair $u,v$ are set on a square diagonal only if there are $a\in A$ and $b \in B$ such that $u = avb$. We can assume that $G$ is a bipartite graph (otherwise just take $\Gamma^{\prime} = \Gamma \times \mathbb{Z}_{2}$ and define the product to be $a\left( u,\pm \right) = \left( au, \mp \right)$).


Now divide the graph into positive and negative vertices according to their coloring $V_{-}$ and $V_{+}$. And define the positive graph to be $G^{+} = \left( V_{+}, E \right)$ and by $G^{-} = \left( V_{-}, E \right)$ the negative graph, where $E$ denotes the squares, put differently there is an edge between $v$ and $u$ in $G^{+}$ if both vertices are positive and they are laid on the ends of a square's diagonal.


The quantum Tanner code is a CSS code, such that $C_{X}$ is defined to be the classical Tanner code $\mathcal{T}\left(G^{+}, \left(C_{A}^\perp\otimes C_{B}^{\perp}\right)^{\perp} \right)$ and $C_{Z}$ is defined as $\mathcal{T}\left(G^{-}, \left(C_{A}\otimes C_{B}\right)^{\perp} \right)$. Note that in contrast to the classical Tanner code, in the quantum case it will be more convenient to think of codewords as assignments set on the squares and not on the edges.
\end{definition}
\end{frame}
\begin{frame}
Note that any check of $C_X$ can be thought of as multiplying one of the matrices in $C_A^\perp \otimes C_B^\perp$ by the local view of some positive vertex. Similarly, the checks of $C_Z$ are obtained by multiplying the matrices in $C_A \otimes C_B$ in the local view of the negative vertices. As any two sibling positive and negative vertices share either a row or a column, it is easy to see that the checks commute. Therefore, by definition, $H_{X}H_{Z}^{\top} =0$, which implies that the code is a CSS code.
\end{frame}

\begin{frame}

\begin{figure}[h]
            %\label{fig:square}
            \begin{center}
              \begin{tikzpicture}[scale=0.8]
            \draw[thick](0,0)(0, 0) -- (1.9049911888377424,4.002079117547451) -- (5.81285084955138,4.602079117547451) -- (5.212850849551381,0.2192498173032679) -- (0, 0)
(0, 0) -- (2.879157898699928,2.130954290533681) -- (6.412850849551381,3.330954290533681) -- (5.212850849551381,0.2192498173032679) -- (0, 0)
(0, 0) -- (4.588270563356833,4.1850314105127575) -- (7.012850849551381,5.985031410512757) -- (5.212850849551381,0.2192498173032679) -- (0, 0)
(0, 0) -- (1.9049911888377424,4.002079117547451) -- (5.890590913653101,4.602079117547451) -- (5.2905909136531015,0.6119546131629411) -- (0, 0)
(0, 0) -- (2.879157898699928,2.130954290533681) -- (6.490590913653102,3.330954290533681) -- (5.2905909136531015,0.6119546131629411) -- (0, 0)
(0, 0) -- (4.588270563356833,4.1850314105127575) -- (7.090590913653101,5.985031410512757) -- (5.2905909136531015,0.6119546131629411) -- (0, 0)
(0, 0) -- (1.9818949186177792,-4.217396385887875) -- (5.81285084955138,-4.817396385887875) -- (5.212850849551381,0.2192498173032679) -- (0, 0)
(0, 0) -- (3.9944175473955323,-4.094159016671296) -- (6.412850849551381,-5.294159016671296) -- (5.212850849551381,0.2192498173032679) -- (0, 0)
(0, 0) -- (2.2401823432002748,-5.388664191428558) -- (7.012850849551381,-7.188664191428558) -- (5.212850849551381,0.2192498173032679) -- (0, 0)
(0, 0) -- (1.9818949186177792,-4.217396385887875) -- (5.890590913653101,-4.817396385887875) -- (5.2905909136531015,0.6119546131629411) -- (0, 0)
(0, 0) -- (3.9944175473955323,-4.094159016671296) -- (6.490590913653102,-5.294159016671296) -- (5.2905909136531015,0.6119546131629411) -- (0, 0)
(0, 0) -- (2.2401823432002748,-5.388664191428558) -- (7.090590913653101,-7.188664191428558) -- (5.2905909136531015,0.6119546131629411) -- (0, 0)
(5.2905909136531015, 0.6119546131629411) -- (5.890590913653101,4.602079117547451) -- (10.291575076901802,5.502079117547451) -- (9.391575076901802,1.4619219538089987) -- (5.2905909136531015, 0.6119546131629411)
(5.2905909136531015, 0.6119546131629411) -- (6.490590913653102,3.330954290533681) -- (11.191575076901803,5.130954290533681) -- (9.391575076901802,1.4619219538089987) -- (5.2905909136531015, 0.6119546131629411)
(5.2905909136531015, 0.6119546131629411) -- (7.090590913653101,5.985031410512757) -- (12.091575076901801,8.685031410512757) -- (9.391575076901802,1.4619219538089987) -- (5.2905909136531015, 0.6119546131629411)
(5.2905909136531015, 0.6119546131629411) -- (5.890590913653101,4.602079117547451) -- (11.4486392380322,5.502079117547451) -- (10.5486392380322,0.7771825477033486) -- (5.2905909136531015, 0.6119546131629411)
(5.2905909136531015, 0.6119546131629411) -- (6.490590913653102,3.330954290533681) -- (12.3486392380322,5.130954290533681) -- (10.5486392380322,0.7771825477033486) -- (5.2905909136531015, 0.6119546131629411)
(5.2905909136531015, 0.6119546131629411) -- (7.090590913653101,5.985031410512757) -- (13.248639238032201,8.685031410512757) -- (10.5486392380322,0.7771825477033486) -- (5.2905909136531015, 0.6119546131629411)
(5.2905909136531015, 0.6119546131629411) -- (5.890590913653101,-4.817396385887875) -- (10.291575076901802,-3.917396385887875) -- (9.391575076901802,1.4619219538089987) -- (5.2905909136531015, 0.6119546131629411)
(5.2905909136531015, 0.6119546131629411) -- (6.490590913653102,-5.294159016671296) -- (11.191575076901803,-3.494159016671296) -- (9.391575076901802,1.4619219538089987) -- (5.2905909136531015, 0.6119546131629411)
(5.2905909136531015, 0.6119546131629411) -- (7.090590913653101,-7.188664191428558) -- (12.091575076901801,-4.488664191428557) -- (9.391575076901802,1.4619219538089987) -- (5.2905909136531015, 0.6119546131629411)
(5.2905909136531015, 0.6119546131629411) -- (5.890590913653101,-4.817396385887875) -- (11.4486392380322,-3.917396385887875) -- (10.5486392380322,0.7771825477033486) -- (5.2905909136531015, 0.6119546131629411)
(5.2905909136531015, 0.6119546131629411) -- (6.490590913653102,-5.294159016671296) -- (12.3486392380322,-3.494159016671296) -- (10.5486392380322,0.7771825477033486) -- (5.2905909136531015, 0.6119546131629411)
(5.2905909136531015, 0.6119546131629411) -- (7.090590913653101,-7.188664191428558) -- (13.248639238032201,-4.488664191428557) -- (10.5486392380322,0.7771825477033486) -- (5.2905909136531015, 0.6119546131629411)
;
\node at (5.91285084955138,4.602079117547451) {$ a_{ 0  } gb_{ 0 } $};
\node at (6.512850849551381,3.330954290533681) {$ a_{ 0  } gb_{ 1 } $};
\node at (7.11285084955138,5.985031410512757) {$ a_{ 0  } gb_{ 2 } $};
\node at (5.990590913653101,4.802079117547451) {$ a_{ 1  } gb_{ 0 } $};
\node at (6.590590913653101,3.530954290533681) {$ a_{ 1  } gb_{ 1 } $};
\node at (7.190590913653101,6.1850314105127575) {$ a_{ 1  } gb_{ 2 } $};
\node at (5.91285084955138,-4.817396385887875) {$ a_{ 0  } gb_{ 4 } $};
\node at (6.512850849551381,-5.294159016671296) {$ a_{ 0  } gb_{ 5 } $};
\node at (7.11285084955138,-7.188664191428558) {$ a_{ 0  } gb_{ 6 } $};
\node at (5.990590913653101,-4.617396385887875) {$ a_{ 1  } gb_{ 4 } $};
\node at (6.590590913653101,-5.094159016671296) {$ a_{ 1  } gb_{ 5 } $};
\node at (7.190590913653101,-6.988664191428557) {$ a_{ 1  } gb_{ 6 } $};
\node at (10.391575076901802,5.502079117547451) {$ a_{ 0  } a_{ 1 }gb_{ 0 } $};
\node at (11.291575076901802,5.130954290533681) {$ a_{ 0  } a_{ 1 }gb_{ 1 } $};
\node at (12.1915750769018,8.685031410512757) {$ a_{ 0  } a_{ 1 }gb_{ 2 } $};
\node at (11.5486392380322,5.702079117547451) {$ a_{ 1  } a_{ 1 }gb_{ 0 } $};
\node at (12.4486392380322,5.330954290533681) {$ a_{ 1  } a_{ 1 }gb_{ 1 } $};
\node at (13.3486392380322,8.885031410512756) {$ a_{ 1  } a_{ 1 }gb_{ 2 } $};
\node at (10.391575076901802,-3.917396385887875) {$ a_{ 0  } a_{ 1 }ga_{ 0 } $};
\node at (11.291575076901802,-3.494159016671296) {$ a_{ 0  } a_{ 1 }ga_{ 1 } $};
\node at (12.1915750769018,-4.488664191428557) {$ a_{ 0  } a_{ 1 }ga_{ 2 } $};
\node at (11.5486392380322,-3.717396385887875) {$ a_{ 1  } a_{ 1 }ga_{ 0 } $};
\node at (12.4486392380322,-3.294159016671296) {$ a_{ 1  } a_{ 1 }ga_{ 1 } $};
\node at (13.3486392380322,-4.288664191428557) {$ a_{ 1  } a_{ 1 }ga_{ 2 } $};
\node at (-0.1,0) {$ g $};
\node at (5.31285084955138,0.3192498173032679) {$ a_{ 0 }g $};
\node at (5.390590913653101,0.7119546131629411) {$ a_{ 1 }g $};
\node at (2.0049911888377423,4.102079117547451) {$ gb_{ 0 } $};
\node at (2.979157898699928,2.230954290533681) {$ gb_{ 1 } $};
\node at (4.6882705633568325,4.285031410512757) {$ gb_{ 2 } $};
\node at (2.0818949186177793,-4.117396385887876) {$ gb_{ 4 } $};
\node at (4.094417547395532,-3.9941590166712957) {$ gb_{ 5 } $};
\node at (2.340182343200275,-5.288664191428558) {$ gb_{ 6 } $};          
\end{tikzpicture}
\end{center}

            \caption{Local environment of a square complex.}
            \label{fig:square}
           
            \end{figure}

          \end{frame}

\begin{frame}
Now we can state formally the theorem: 

\begin{theorem}
  \label{theorem:maint}
  For $\varepsilon \in \left( 0,\frac{1}{2} \right)$, $\delta_{0}> 0$, large enough $\Delta$, and small codes $C_{A},C_{B}$ with distance at least $\delta_{0}\Delta$ there exist constant $\zeta > 0 $ such if the dual tensor code of $C_{A},C_{B}$ is $\Delta^{1\frac{1}{2}}+\varepsilon$-robust then there exists an infinite family of square complexes for which the Tanner code $ \mathcal{T}\left(G^{+}, \left(C_{A}^\perp\otimes C_{B}^{\perp}\right)^{\perp} \right)$  defined by the complexes and the dual tensor code such that for any codeword $c$ with weigh less than $ \zeta n \Delta^{2} $ there exist a negative vertex in $v \in V^{-}$ and code word in $ y \in  C_{A}\otimes C_{B}$ supported only on the squers adjoint to $v$ such that $|c + y| < |c|$.
\end{theorem}


Observe that $y \in C_{Z}^{\perp} \subset C_{X}$, so $c + y \in C_{X}$ and $|c + y| < |c| < \zeta \cdot n \Delta^{2}$. Therefore, by~\ref{theorem:maint}, there is another $y_{1} \in C_{Z}^{\perp}$ such that $|c + y + y_{1}| < |c + y| < |c|$. Repeating this process enough times yields a series of $y, y_{1}, y_{2}, \dots, y_{l}$, all of them in $C_{Z}^{\perp}$, such that:
\begin{equation*}
  \begin{split}
    |c + y + \dots + y_{l}| = 0 \Rightarrow c = y + y_{1} + y_{2} + \dots + y_{l} \Rightarrow c \in C_{Z}^{\perp}
  \end{split}
\end{equation*}


\end{frame}

\begin{frame}
\begin{claim}
  The distance of the dual tensor code is at least $\delta_{0}\Delta$.
  \label{claim:duweight}
\end{claim}
\begin{proof}
By the robustness property, any codeword of the dual tensor code with a weight less than $\delta_{0}\Delta$ is supported on at most one row. Let $c$ be such a codeword and denote by $i$ the number of the non-trivial row. Fix a $c^{\prime} \in C_{A}^{\perp}$ such that the $i$th coordinate of  $c^{\prime}$ is non-zero and consider the multiplication of $c$ with the codewords of $C_{A}^\perp \otimes C_{B}^\perp$ of the following form:
  \begin{equation*}
    \begin{split}
      J = \left\{ c^{\prime} \otimes c_{b} : c_{b}\in C_{B}^{\perp} \right\} 
    \end{split}
  \end{equation*}
  So the $i$th row of any $x \in J$ is a codeword of $C_{B}^{\perp}$ and in total, collecting all the $i$th rows of codewords in $J$ sums up to all the code words in $C_{B}^{\perp}$. On the other hand, $c\cdot x = 0$ for all $x \in J$; that is, $c\cdot x = c_{i} \cdot c_{b} = 0$. Thus we obtain that $c_{i} \in C_{B}$ and therefore $|c_{i}| \ge \delta_{0}\Delta \Rightarrow |c| \ge \delta_{0}\Delta$, which is a contradiction.
\end{proof}
\end{frame}

\begin{frame}

\begin{definition}
Let $S$ and $S_{-}$ denote the positive and negative vertices that support the codeword $c \in C_{X}$, respectively. Furthermore, let $S_e$ and $S_n$ denote the exceptional and normal vertices, respectively, where the weight of the local view for any vertex in $S_e$ is greater than $\Delta^{3/2 + \varepsilon}$, and $S_n$ is the complementary set of vertices. An edge in $G$ will be said to be heavy if it supports more than $\delta_{0}\Delta - \Delta^{\frac{1}{2} + \varepsilon}/\delta_{0}$ squares in $G$. Let $T \subset S_{-}$ denote the negative vertices connected to $S_{n}$ by at least one heavy edge. Additionally, let $T_s \subset T$ denote the vertices in $S_-$ that are surrounded by only normal vertices. Finally, for any pair of vertex subsets $A,B$ such that $A \subset V_{+}$ and $B \subset V_{-}$, let $d_{B\rightarrow A}$ denote the average number of heavy edges leaving $B$ and going to $A$.
\end{definition}

\end{frame}
\begin{frame}
\begin{claim}
  \label{claim:epss}
  for any $\varepsilon \in \left( 0,1 \right)$ and large enough $\Delta$  it holds that $ |S| \le \Delta^{\varepsilon}|S_{-}| $ 
\end{claim}
\begin{proof}
  Suppose not, namely that $|S| > \Delta^{\varepsilon}|S_{-}|$, then $|x|/|S_{-}| > \Delta^{\varepsilon}|x|/|S| > \Delta^{\varepsilon} \cdot \delta_{0}\Delta $ But:  
\begin{equation*}
  \begin{split}
    \frac{|x|}{|S_{-}|} = \frac{\Theta \left(E(S_{-},S_{-}) \right)}{|S_{-}|} \le \Theta(\Delta^{2})\frac{|S_{-}|}{n}  + \Theta(\Delta)  \rightarrow_{n\rightarrow \infty} \Theta(\Delta)
  \end{split}
\end{equation*}
\end{proof}
\end{frame}
\begin{frame}

\begin{claim}
  \label{claim:portion}
  At least $ 1- \Delta^{-\frac{\varepsilon}{4}}$ portion of the negative vertices adjoin to only normal vertices. 
\end{claim}
% Define the graph $G^{\star} = (V_{-}, E^{\star})$ such that $E^{\star} = E \cup \left\{ \left\{g, xyg \right\} x\neq y, \in A\cup B \right\}$ And notice that the assumption follows that  any 
\begin{proof}
  Suppose through contradiction that for $\Delta^{-p}$ portion of the negative vertices $v_{-}\in V_{-}$ have at least one ($\Delta^{\gamma}$) sibling in $S_{e}$. Therefore $ \Delta^{-p} |S_{-}| \le \Delta |S_{e}|$ combining with \ref{claim:epss} it follows that $|S| \le \Delta^{1+\varepsilon + p }|S_{e}|$ : 
  \begin{equation*}
    \begin{split}
      \Delta^{3/2 + \varepsilon} & \le \frac{E(S,S_{e})}{|S_{e}|} = \Theta\left( \Delta^{2} \right)\frac{|S|}{n} + \Theta\left( \Delta \right)\sqrt{ \frac{|S|}{|S_{e}|}  }\\ 
      & \le \Theta(\Delta^{2}) \frac{|S|}{n} + \Theta(\Delta) \Theta\left( \Delta^{\frac{1+\varepsilon + p}{2}} \right)  
    \end{split}
  \end{equation*} 
  Thus we obtain contradiction for any $p < \varepsilon/2$. In particular for $p = \varepsilon/4$ we obtain that at least $1 - \Delta^{-\varepsilon/4}$ portion of the negative vertices are surounded by only normal vertices. 
 \end{proof}
 \end{frame}

 % \begin{claim}
%   Any $x \in \duC$ such that $|x| \le w$, and denote by $A^{\prime},B^{\prime}$ the rows and cols support $x$. Then $x$ can decompise into a sum of $x = t + s$ such that $t \in C_{A}\otimes \mathbb{F}^{B^\prime}$ and $s \in  \mathbb{F}^{A^\prime} \otimes  C_{B}$. 
% \end{claim}
% \begin{proof}
%   By induction on the number of raws supported $x$. Denote $x$ by 
%
%   \begin{equation*}
%     \begin{split}
%       x & = \sum_{i, c_{a}}{ c_{a} \otimes e_{i}  } + \sum_{i, c_{b} }{ e_{i} \otimes c_{b}} \\
%       & =  c_{a^{\prime}} \otimes e_{\tau} + \sum_{i / \tau, c_{a}}{ c_{a} \otimes e_{i}  } + \sum_{i, c_{b} }{ e_{i} \otimes c_{b}} \\
%      \end{split}
%   \end{equation*} 
%   
%   Now, if $c_{a^{\prime}} \otimes e_{\tau}$ decrese the summation weight then it must holds that $x$  
%
%    \begin{equation*}
%     \begin{split}
%       & =  c_{a^{\prime}} \otimes e_{\tau} + t^{\prime} + s^{\prime} = \overbrace{\left( c_{a^{\prime}} \otimes e_{\tau} + t^{\prime} \right) }^{ C_{A} \otimes \mathbb{F}^{B^{\prime}}}+ s^{\prime}
%     \end{split}
%   \end{equation*}
% \end{proof}
% 

 \begin{frame}
 \begin{claim}  
   \label{claim:close}
   Let $x$ be a codeword of $\duC$ and $\xi < w$ such that $d(x, \mathbb{F}^{A} \otimes C_{B}) + d(x, C_{A}\otimes\mathbb{F}^{B}) \le \xi $ . Then $d(x, C_{A} \otimes C_{B}) < 3\xi $. 
 \end{claim} 
 \begin{proof}
Denote by $R$ the closest codeword of $C_{A}\otimes\mathbb{F}^{B}$ to $x$. Similarly, denote by $C$ the closest codeword of $\mathbb{F}^{A} \otimes C_{B}$ to $x$. Notice that $C + R \in \duC$. In addition, the weight of $C+R$ is bounded by:
   \begin{equation*}
     \begin{split}
       |R + C| &=  |x + \left(x + R\right) + x + \left( x + C\right)| \\
       & \le |\left(x + R\right)|+ |\left( x + C\right)| \le d\left( x ,   C_{A}\otimes\mathbb{F}^{B} \right) +  d\left( x ,   \mathbb{F}^{A} \otimes C_{B} \right) \\ 
       & \le w
     \end{split}
   \end{equation*}
   Therefore, by the robustness property, there are $r \in C_{A}\otimes\mathbb{F}^{B}$ and $c \in \mathbb{F}^{A} \otimes C_{B} $ such that $ R + C  = r + c$. And $r,c$ are supported on at most $|R+C|/\delta_{0}\Delta$ rows and columns. (Here $r$ and $c$ play the role of $s,t$ in \ref{def:wrobust}.)  

   Now observe that on one hand $C + c = R + r$, and on the other hand $ C + c \in \mathbb{F}^{A} \otimes C_{B} $ and $R+r \in C_{A}\otimes\mathbb{F}^{B}$. Therefore, $C+c \in C_{A}\otimes\mathbb{F}^{B} \cap \mathbb{F}^{A} \otimes C_{B}$. Namely, $C +c \in C_{A} \otimes C_{B}$. Thus we have:  
   
   \begin{equation*}
     \begin{split}
       d\left(x, C_{A}\otimes C_{B}\right) & \le d\left(x, C\right) + d\left(C, C_{A}\otimes C_{B} \right) \\
       &\le \xi + |c|
     \end{split}
   \end{equation*}
   And in the same way we obtain also that $d\left( x, C_{A} \otimes C_{B} \right) \le \xi + |r|$. Since $c,r$ are supported on at most $|R+C|/\delta_{0}\Delta$ rows and columns, the weight of the string obtained by joining a single row of $r$ with $c$ grows by at least $\delta_{0}\Delta - |R+C|/\delta_{0}\Delta > 0$. Therefore, $|c| < |c + r| = |R + C|$. Thus, in total, $d\left( x, C_{A} \otimes C_{B} \right) \le 3\xi$.
 \end{proof}

\end{frame}

\begin{frame}
\begin{claim}
  \label{claim:closeto}
  Suppose that $v \in T_{s}$, Namely $v$ is surounded by only normal vertices. Then:
  \begin{equation*}
    \begin{split}
      d\left( c_{v}, C_{A}\otimes C_{B}\right) < \Theta\left( \Delta^{3/2+\varepsilon} \right)
    \end{split}
  \end{equation*} 
 \end{claim}
\begin{proof}
  By being surrounded only by normal vertices any row in the local view of $v$ is codeword of $C_{A}$ plus at most $\Delta^{3/2 + \varepsilon}/\Delta = \Delta^{\frac{1}{2}+\varepsilon}$ faults. So correcting the rows require flipping at most $\Delta \cdot \Delta^{\frac{1}{2} + \varepsilon}$ bits in total.  Thus $d\left(c_{v}, C_{A}\otimes \mathbb{F}^{B}\right) < \Delta^{3/2 + \varepsilon}$. In same way we obtain that $d\left(c_{v},  \mathbb{F}^{A} \otimes C_{B}\right) < \Delta^{3/2 + \varepsilon}$. Noitce that, in parituclar, $d(c_{v}, \duC) \le \Delta^{3/2 + \varepsilon}$.

  Denote by $y$ the closest codeword of $\duC$ to $c_{v}$. And observes that the distance between $y$ to either $C_{A} \otimes \mathbb{F}^{B}$ or $\mathbb{F}^{A}\otimes C_{B}$ is at most $2 \cdot \Delta^{3/2 + \varepsilon}$. To see it consider the decoding: 
  \begin{equation*}
    \begin{split}
  y \rightarrow x \rightarrow C_{A} \otimes \mathbb{F}^{B}   
    \end{split}
  \end{equation*}
  
  Therefore form \ref{claim:close} it follows that $d\left(y, C_{A} \otimes C_{B}\right) < 3 \xi $, So $d\left(x, C_{A} \otimes C_{B}\right) < \Delta^{3/2 + \varepsilon} +  3 \xi $.
     \end{proof}
   
   \end{frame}
   \begin{frame}
 \begin{claim}[The Technical Lemma]  
   \label{cliam:tech} 
Let $A \subset S$ and $B \subset S_{-}$ be subsets of the positive and negative vertices supported by a codeword $x \in C_{X}$ such that $x < \zeta n \Delta^2$ and $\alpha \le \Delta^2, \beta \le \Delta$ are the minimum degrees in $G^{+}, G$ induced by $x$ (note that in $G^{+}$ the edges are associated with the squares of the left-right Cayley graph). Assume the following conditions hold:
   \begin{enumerate}
     \item $\beta = \frac{\delta}{4 \sqrt{\Delta}}\alpha + \Theta\left( \Delta \right)$
     \item $B$ defined to be all the vertices connected to $\bar{A}$ by at least one heavy edge.
     \item Any vertex in $\bar{A}$ has at least one heavy edge. 
   \end{enumerate}
   Then: $d_{B\rightarrow \bar{A}} = \Omega\left( \Delta \right)$.  
 \end{claim}

 \begin{proof}
   By the given $|S| \le \frac{2|x|}{\delta_{0}\Delta} \le \zeta\frac{2n\Delta}{\delta}$ we have that $|S|/n \le \zeta\cdot \frac{2\Delta}{\delta}$.
   %Let $A \subset S$ and $B \subset S_{-}$ subsets of the positive and the negetive vertices support $x$, and $\alpha \le \Delta^{2},\beta \le \Delta$ their minimal degrees in $G, G$.
   Then by the Mixing Expander Lemma we have that:   
   \begin{equation*}
     \begin{split}
       \alpha |A| & \le |E(A,S)| \le \frac{\Delta^{2}}{n}|A||S| + 4 \Delta \sqrt{|A||S|} \le |A| \cdot \zeta \frac{2\Delta^{3}}{\delta} +  4 \Delta \sqrt{|A||S|}\\ 
       & \Rightarrow \sqrt{|A|}\left(\alpha-\zeta \frac{2\Delta^{3}}{\delta}  \right) \le 4\Delta\sqrt{|S|} \\
       & \Rightarrow |A| \le \left( \alpha -  \zeta \frac{2\Delta^{3}}{\delta}  \right)^{-2} \cdot 16\Delta^{2}|S|
     \end{split}
   \end{equation*}
   And by repetting the same calculation but consider $B$ in the $G$ graph we obtain: 
\begin{equation*}
     \begin{split}
       \Rightarrow  |B| \le & \left( \beta -  \zeta \frac{4 \cdot 2\Delta^{2} }{\delta}  \right)^{-2} \cdot 16\Delta|S|\\
       \Rightarrow  |B| \le & \left( \frac{\delta}{\sqrt{\Delta}}\alpha   -  \zeta \frac{4 \cdot 2\Delta^{2} }{\delta}  \right)^{-2} \cdot 16\Delta|S|\\
       = & \frac{\Delta}{\delta^{2}} \left(  \alpha -  4\cdot 2\zeta \frac{\Delta^{2\frac{1}{2}}}{\delta^{2}}   \right)^{-2} \cdot 16\Delta|S| 
     \end{split}
   \end{equation*}
   And for large enugh $\Delta$ the above is bounded by:
   \begin{equation*}
     \begin{split}
     \left(  \alpha -  4\cdot 2\zeta \frac{\Delta^{2\frac{1}{2}}}{\delta^{2}}   \right) \ge \left( \alpha -  \zeta \frac{2\Delta^{3}}{\delta}  \right) \Rightarrow |B|  \le  & \frac{1}{\delta^{2}} \left( \alpha -  \zeta \frac{2\Delta^{3}}{\delta}  \right)^{-2} \cdot 16\Delta^{2}|S| 
     \end{split}
   \end{equation*}
   Now, choose $\zeta$ such $\left( \alpha -  \zeta \frac{2\Delta^{3}}{\delta}  \right) \ge 16^{\frac{1}{2}} \cdot 100 \Delta^{1\frac{1}{2}}$ yields that: $|A| \le 10^{-4}\Delta^{-1} |S| \Rightarrow $$ |\bar{A}| \ge \left( 1 - 10^{-4} \Delta^{-1}\right)|S|$, And $|B| \le 10^{-4} \frac{|S|}{16 \delta^{2}_{0}\Delta}$. Conditions (2) and (3) garunte that any vertex in $\bar{A}$ is connected to at least on vertex of $B$. And therfore, $B$ covers $\bar{A}$, that is, $ d_{B\rightarrow\bar{A}}\cdot |B| \ge \bar{A}$, Hence:

   \begin{equation*}
     \begin{split}
       d_{B\rightarrow \bar{A}}  \ge \frac{|\bar{A}|}{|B|} \ge \left( 1 - 10^{-4}\Delta^{-1}  \right) 10^{4} \cdot \delta^{2}_{0}\Delta  = \Theta\left( \Delta \right)
     \end{split}
   \end{equation*}
 \end{proof}
 
\end{frame}
\begin{frame}

 \begin{claim}
   \label{claim:satis}
   $S_{e}$ and $T$ satisfies the requrirments of \ref{cliam:tech} with $A = S_{e}$, $B = T$, $\alpha = \Delta^{3/2 + \varepsilon}$ and $\beta = \delta_{0}\Delta + \Delta^{\frac{1}{2} + \varepsilon}$. That is, the average of havey edges form $T$ to $S_{n}$ is $\Theta\left( \Delta \right)$. 
 \end{claim}

 \begin{proof}
   Conditions (1) and (2) holds by definition of $S_{e},T$ for values $\alpha = \Delta^{3/2 + \varepsilon}$ and $\beta = \delta_{0}\Delta - \Delta^{\frac{1}{2}+\varepsilon}/\delta_{0}$. It left to show that any normal vertex has at least on heaviy edge. By \ref{claim:duweight} any normal vertex $v$ has weight at least $\delta_{0}\Delta$, yet by robustness there are $t,s \in C_{A}\otimes \mathbb{F}^{B}, \mathbb{F}^{B}\otimes  C_{B}$ such that $t+s = c_{v}$. Assume that that any row of $c_{v}$ has weight less than $\delta_{0}\Delta - \Delta^{\frac{1}{2}+\varepsilon}/\delta_{0}$. Pick an arbitery row, and denote it by $\tau$, Now observes that by the fact that $c_{v}$ has support on at most $\Delta^{\frac{1}{2}+\varepsilon}/\delta_{0}$ columns, then $\tau$ is at distance at most $\Delta^{\frac{1}{2}+\varepsilon}/\delta_{0}$ from $C_{A}$. But, by assumption, $|\tau| < \delta_{0}\Delta -\Delta^{\frac{1}{2}+\varepsilon}/\delta_{0}$ and therefore the closet codeword to $\tau$ in $C_{A}$ has weight less than  $\delta_{0}\Delta$ in contradiction to the fact that the distance of $C_{A}$ is at least $\delta_{0}\Delta$.  

Conditions (1) and (2) hold by definition of $S_e,T$ for values $\alpha = \Delta^{3/2 + \varepsilon}$ and $\beta = \delta_{0}\Delta - \Delta^{\frac{1}{2}+\varepsilon}/\delta_{0}$. It remains to show that any normal vertex has at least one heavy edge. By robustness there are $t,s \in C_{A}\otimes \mathbb{F}^{B}, \mathbb{F}^{B}\otimes  C_{B}$ such that $t+s = c_{v}$. Assume that any row of $c_{v}$ has weight less than $\delta_{0}\Delta - \Delta^{\frac{1}{2}+\varepsilon}/\delta_{0}$. Pick an arbitrary row, and denote it by $\tau$. Now observe that by the fact that $c_{v}$ has support on at most $\Delta^{\frac{1}{2}+\varepsilon}/\delta_{0}$ columns, then $\tau$ is at a distance of at most $\Delta^{\frac{1}{2}+\varepsilon}/\delta_{0}$ from $C_{A}$. But, by assumption, $|\tau| < \delta_{0}\Delta -\Delta^{\frac{1}{2}+\varepsilon}/\delta_{0}$ and therefore the closest codeword to $\tau$ in $C_{A}$ has weight less than $\delta_{0}\Delta$, in contradiction to the fact that the distance of $C_{A}$ is at least $\delta_{0}\Delta$.
 \end{proof}

\end{frame}

\begin{frame}

\begin{claim}
  \label{claim:linear}
  There is a normal-surounded vertex in $v \in T_{s}$ in weight at least $\Theta\left(\Delta^{2}\right)$.   
 \end{claim}
 \begin{proof}
   We know from \ref{claim:satis} that $d_{T\rightarrow S_{n} }$ is linear in $\Delta$. Morever \ref{claim:portion} tell us that most of the vertices in $S_{-}$ are surrounded only by normal vertices. Denote by $\expp{ d(v) | v \in T_{s} }$ the expected degree of heavy edges connected to vertex in $T_{s}$. Using the conditional expection formula we get:
  
   \begin{equation*}
     \begin{split}
       d_{T \rightarrow S_{n}} &= \expp{ d(v) | v \in T_{s} } \prb{ v \in T_{s}} + \expp{d\left( v \right) | v \in T / T_{s}} \prb{ v \in T / T_{s}} \\
       & \le \expp{ d(v) | v \in T_{s} } \prb{ v \in T_{s}} + \expp{d\left( v \right) | v \in T / T_{s}} \prb{ v \in S_{-} / T_{s}} \\
       & \le \expp{ d(v) | v \in T_{s} } \cdot 1   + \Delta \cdot  \Delta^{-\varepsilon/4} \\
       & \Rightarrow  \expp{ d(v) | v \in T_{s} } \ge d_{T \rightarrow S_{n}} - \Delta^{\frac{3}{4} \varepsilon} = \Theta\left( \Delta \right) 
     \end{split}
   \end{equation*}
   Therefore there is at least a single vertex in $T_{s}$ connected to $\Theta\left( \Delta \right)$ havey edges. Combining the fact that edge is heavy edge if there are at least $\delta_{0}\Delta - \Delta^{\frac{1}{2} + \varepsilon}$ non trival bits on it's squares we get the desired.  
 \end{proof}
\end{frame}


\begin{frame}

 \begin{remark}
For small codes which are robust for $w < \Delta^{3/2}$, as in the original proof of \cite{leverrier2022quantum}, \ref{claim:portion} no longer holds. However, assuming the dual tensor code is also $p$-resistant to puncturing~\ref{def:resistance}, one can still prove \ref{claim:linear}.
 \end{remark}

 We are about to finish the proof of the theorem. Combining \ref{claim:linear} and \ref{claim:closeto} we obtain the the existnes of a negative vertex which is both at distance $\Theta\left(\Delta^{3/2 + \varepsilon}\right)$ form $C_{A}\otimes C_{B}$ and weight at least $\Theta\left( \Delta^{2} \right)$. Denote by $v \in T_{s}$ that vertex, by $c_{v}$ it's local view and by $y \in C_{A}\otimes C_{B}$ the closest codeword to $c_{v}$. Subtructing $y$ from $c_{v}$ yilds: 
 
 \begin{equation*}
   \begin{split}
     \left|   c_{v} + y   \right| &= d\left( c_{v}, y  \right) = \Theta\left( \Delta^{3/2 + \varepsilon} \right) < \left|   c_{v}   \right|     \\
     \Rightarrow & | c + y| < |c| 
   \end{split}
 \end{equation*}

\end{frame}


% \section{LTC.} 
%As exactly as in the polynomial code case, we will prove that a code is locally testable by presenting a decoder that can correct small errors and reject errors greater than a linear threshold. But before that, let us define the LTC code.
%
%\begin{definition}
%  \label{def:ltcode}
%Consider the Tanner code defined on the square complex and the negative vertices as presented above, but instead of taking the dual tensor as the local code, we take the product code: $\mathcal{T} \left(G^{+}, C_{A} \otimes C_{B} \right)$.
%\end{definition}
%Now, let us define the disagreement code, which we can conceptually think of as the strings obtained by an attempt to decode by local correction. That is, any vertex chooses the closest codeword to its local view and the summation of the suggestions is the disagreement. Notice that if the initial assignment was a valid codeword then each of the vertices suggest the codeword it sees on its local view. Hence, for any edge will suggest exactly the same bit and the summation on the edge will be equal to zero. Because it is true for all the edges, the total string obtained will be the zero codeword. Nevertheless, any assignment on an edge which equals $1$ points to a disagreement between the vertices at the edge's ends.
%../NLTES_project/ltc_ldpc/disag.tex
%
%Furthermore, in the case of the LTC code \ref{def:ltcode}, the disagreement code is simply $C_X$ (or $C_Z$). This is because any vertex contributes to the local view of the positive vertices codewords in either $C_A \otimes \mathbb{F}^B$ or $\mathbb{F}^A \otimes C_B$. 
%\section{Decoding and Testing}
  For completeness, we show exactly how Theorem 1 implies testability. The following section repeats Leiverar's and Zemor's proof \cite{leverrier2022quantum}. Consider a binary string $x$ that is not a codeword. The main idea is the observation that the number of bits filliped by (any) decoder, while decoding $x$, bounds the distance $d\left( x, C \right)$ from above. In addition, the number of positive checks in the first iteration is exactly the number of violated restrictions.
%\begin{figure*}[h]
%\begin{adjustbox}{width=\textwidth}
  \begin{definition}Let $L = \{L_{i}\}^{2|E|}_{0}$  be a series of $2|E|$. Such that for each vertex $ v \in V$ $\sum_{ e = \{u,v\} }{ L_{e_v} } \in C_{0}$. We will call $L$ a \textit{Potential list} and refer to the $e_{v}$'the element of $L$ as a suggestion made by the vertex $v \in V$ for the edge $e \in E$. Sometimes we will use the notation $L_{v}$ to denote all the $L$'s coordinates of the form $ L_{e_{v}} \forall e \in \text{Support} \left( v \right) $. Define the \textit{Force} of $L$ to be the following sum $  F\left( L \right) = \sum_{e = \{v,u\} \in E }{ \left(L_{e_v} + L_{e_u}\right) }$ and notice that $ F\left( L \right) \in C_{\oplus}$. And define the \textit{state} $S(L) \subset \mathbb{F}^{|E|}_{2}$ of $L$ as the vector obtained by choosing an arbitrary value from $ \{ L_{e_v}, L_{e_u} \}$ for each edge $e \in E$.  
  \end{definition}
  \begin{claim} \label{claim:pot} Let $L$ be the Potential list. If $F(L)=0$ then $S(L)\in C$. \end{claim}
  \begin{proof} Denote by $\phi\left( e \right) \subset \{ L_{e_v}, L_{e_u} \}$ the value which was chosen to $e = \{v,u\} \in E$. By $F\left(L\right) = 0$ , it follows that $ L_{e_v} + L_{e_u} = 0 \Rightarrow L_{e_v} = L_{e_u} = \phi\left( e \right) $ for any $e \in E$. Hence for every $v\in V$ we have that $ S\left( L \right)|_{v} = \sum_{u \sim v}{ \phi\left( \{v,u\} \right) } =  \sum_{u \sim v}{ L_{e_v }} \in C_{0}$ $ \Rightarrow S\left( L \right) \in C$   
  \end{proof}
  The decoding goes as follows. First, each vertex suggests the closet $C_{0}$'s codeword to his local view. Those suggestions define a Potential list, denote it by $L$, then if $F\left( L \right) <\tau$, by Theorem 1, one could find a suggestion of vertex $v$ and a codeword $c_v$ such that updating the value of $L_{v} \leftarrow L_{v} + c_{v}$ yields a Potential list with lower force. Therefore repeating the process till the force vanishes, obtain a Potential list in which its state is a codeword. 
  \begin{definition} Let $\tau > 0, f : \mathbb{N} \rightarrow \mathbb{R^{+}}$, and consider a Tanner Code $C = \mathcal{T}\left( G, C_{0} \right)$. Let us Define the following decoder and denote it by $\mathcal{D}$.  
  \end{definition}

  \begin{algorithm}[h]
    \caption{Decoding}
    \label{alg:three}
    \KwData{ $x \in \mathbb{F}_{2}^{n}$ }
    \KwResult{ $\arg\min {\left\{  y \in C : |y + x|  \right\} }$ if $d(y,C) < \tau $ and False otherwise. }
    $ L \leftarrow \text{Array} \{ \} $\\
    \For { $ v \in V$} {
      $c^{\prime}_{v} \leftarrow \arg\min {\left\{  y \in C_{0} : |y + x|_{v} |  \right\} } $\\
      $ L_{v} \leftarrow c^{\prime}_{v}$
    }
    $ z \leftarrow \sum_{v \in V}{c^{\prime}_{v}} $\\
    \eIf{ $ |z| < \tau \frac{n}{f\left( n \right)} $}{
      \While{ $|z| > 0$ }{
	find $v$ and $c \in C_{0}$ such that $|z + c_{v}| < |z|$\\
	$z \leftarrow z + c_{v}$ \\
	$ L_{v} \leftarrow  L_{v} + c_{v}$
      }
    }{
      reject. 
    }
    \Return  $S(L) $

  \end{algorithm}

  \begin{theorem}
Consider a Tanner Code $C = [n, n\rho, n\delta]$ and the corresponding disagreement code $C_{\oplus}$. Suppose that for every codeword $z \in C_{\oplus}$ such that $|z| < \frac{\tau^{\prime} n}{f\left(n\right)}$, there exists a vertex $v$ and a suggestion for $v$ which is another codeword $y \in C_{\oplus}$ such that $|z + y| < |z|$. Set $\tau \leftarrow \frac{\tau^{\prime}}{6 \Delta} \delta$ then.

  \begin{enumerate}
    \item $\mathcal{D}$ corrects any error at a weight less than $\tau n / f\left(n\right)$.   
    \item $C$ is $f\left( n \right)$ testable code.
  \end{enumerate}
\end{theorem}

\begin{proof} So it is clear from the~\cref{claim:pot} above that if the condition at line (6) is satisfied, then $\mathcal{D}$  will converge into some codeword in $C$. Hence, to complete the first section, it left to show that $\mathcal{D}$ returns the closest codeword. Denote by $e$ the error, and by simple counting arguments; we have that $\mathcal{D}$ flips at most:  
  \begin{equation*}
    \begin{split}
      d_{\mathcal{D}}\left( x, C \right) & \le 2|e|\Delta + \tau \frac{n}{f\left( n \right)}\Delta
    \end{split}
  \end{equation*}
  bits. Hence, by the assumption, 
  \begin{equation*}
    \begin{split}
      d_{\mathcal{D}}\left( x, C \right) & \le 3\Delta \tau \frac{n}{f\left( n \right)} \le 3\Delta \tau\delta n < \frac{1}{2} \delta n  
    \end{split}
  \end{equation*}
  Therefore the code word returned by $\mathcal{D}$ must be the closet. Otherwise, it contradicts the fact that the relative distance of the code is $\delta$.
  To obtain the correctness of the second section, we will separate when the conditional at the line (5) holds and not. And prove that the testability inequality holds in both cases. 
  Let $x \in \mathbb{F}_{2}^{n}$ and consider the running of $\mathcal{D}$ over $x$. Assume the first case, in which the conditional at line (5) is satisfied. In that case, $\mathcal{D}$ decodes $x$ into its closest codeword in $C$. Therefore:
  \begin{equation*}
    \begin{split}
      d\left( x, C \right) \le & \ d_{D} \left( x, C \right) \le m\xi\left( x \right)\Delta +  |z|\Delta  \\ \le &  \  m\xi\left( x \right)\Delta + m\xi\left( x \right)  \Delta^{2} \\ 
      \frac{d\left( x, C \right)}{n} \le & \  \kappa_{1} \xi\left( x \right)    
    \end{split}
  \end{equation*}
  Now, consider the other case in which: $ |z| \ge \tau \frac{n}{f\left( n \right)}  $.
  \begin{equation*}
    \begin{split}
      \frac{d\left( x, C \right)}{n} & \le 1 \le \frac{|z|}{\tau n}f\left( n \right) \le \frac{m}{n} \frac{1}{\tau} \Delta \xi\left( x\right)f\left( n \right) \\ & \le \kappa_{2} \xi\left( x \right)f\left( n \right)  
    \end{split}
  \end{equation*}
  Picking $ \kappa \leftarrow \max \{ \kappa_{1}, \kappa_{2} \}$ proves $f\left( n \right)$-testability
\end{proof}


%
%\printbibliography[heading=subbibliography]

\end{document}
