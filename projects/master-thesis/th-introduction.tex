\chapter{Introduction}


A deterministic test for comparing two numbers can be done by simply comparing each digit in the numbers one by one. This comparison can be done in polynomial time, making it an efficient algorithm for small inputs. However, as the input size becomes larger, the time required to compare each digit becomes infeasible.

A classical probabilistic test for comparing two numbers is the coordinate picking test. In this test, a uniformly random digit is picked from each number, and it is checked if both numbers agree on this chosen coordinate. This test can be repeated multiple times to increase the probability of determining if the two numbers are the same or different. However, even with multiple repeated tests, there is always a chance of both numbers appearing identical even when they are not.


% ------- %

The coordinate picking test is a probabilistic test for comparing the equality of two numbers. It's efficient because it provides a high probability of correctly determining if two numbers are different, especially if the numbers are far apart from each other in terms of their digits.

To use the coordinate picking test, a random digit from each number is chosen, and the test checks if the chosen digits from both numbers are the same or different. The process is repeated for multiple positions or coordinates, and if all the chosen coordinates match, the numbers are considered equal.

When the numbers being compared are far apart in terms of their digits, the probability of detecting their difference with the coordinate picking test becomes higher. This is because it's more likely to find a position where the numbers differ upon choosing random coordinates for comparison.

Overall, the coordinate picking test is an efficient probabilistic algorithm for comparing the equality of two numbers, especially if the numbers are far apart in terms of their digits.


% ------ % 

The quantum swap test is an algorithm used in quantum computing to compare the similarity of two large strings of information. It works by manipulating qubits to determine whether two large strings are identical or different. The swap test can be used on product states, which are tensor products of individual qubits. 

To use the quantum swap test, two large strings of information (in quantum form) are prepared, and the algorithm compares them by swapping qubits and measuring the resulting state. The process is repeated multiple times to improve the accuracy of the test. 

In contrast to the probabilistic coordinate picking test, the quantum swap test is a deterministic algorithm that can determine the equality of two numbers with high probability. However, the quantum swap test requires complex quantum operations that may be difficult to implement, and it may not be efficient for very large inputs.

Regarding a hypothetical test where a random coordinate is picked and only the corresponding qubits are swapped, such a test would likely fail with high probability in mimicking the coordinate picking test, as it would not take into account the correlation between qubits that exist in quantum states. Unlike classical bits, qubits can exist in superpositions, which means they can represent multiple states simultaneously. Therefore, a test that only swaps particular qubits without taking into account their entanglement is unlikely to give accurate results in a way that mimics the classical coordinate picking test.



%\printbibliography[heading=subbibliography]
