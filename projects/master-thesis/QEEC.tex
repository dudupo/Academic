\chapter{Quantum Error Correction Codes.}
\section{Introduction.}
It's wide believed that quantum machines have a significant advantage over the classical in range of computational tasks\cite{grover1996fast}, \cite{ahuja1999quantum}. Simple algorithms, which could be interpreted as the quantum version of scanning all the options cut the running time by square root of the classical magnitude. 

Nevertheless, Shore shown a polynomial depth quantum circuit that solve the hidden abelian subgroup \cite{Shor_1997}, what is considered as breakthrough, as it made the computer science community to believe that a quantum computer might offer an exponential advantage.

Yet, even those that there is a general consensus about the superiority of ideal quantum computation model,it is still unclear that it feasible to implement such machine in the presence of noise.   
Still just point about the existences of noise is not powerful enough to cancel feasibility of computation and evidence of this is the fact that classical computers are also suffer form a certain rate of faults. Thus, for getting a full understandings of the hardness, let us compare two main reasons that made realize an hard task. 
First is the magnitude of the error rate, the classical computers also have errors, and sometimes when that happen we are wetness for systems failures (blue screen for example). The error rate of modern computers is so low such that the probability for error to propagate stay negligible even if the length of the computation is polynomial in the scale of what considered as reasonable input size. It's worth to mention, that in the area of exascale computing, when super computers preform around $10^{18}$ operations per second, It is hard to miss the faults. In quantum we become aware to their existences much more before.      

The second difference, which is a really tricky point, is that quantum sates are sensitive for additional type of error. Along with the chance for bit flip error, quantum state might change their phase. For example, consider the initial state $\ket{+} = \frac{1}{\sqrt{2}}\left( \ket{0} + \ket{1} \right)$, and suppose that due to noise the state transformed into $\frac{1}{\sqrt{4}}\left( \sqrt{3}\ket{0} + \ket{1} \right)$. While classical circuits are blind to such faults, namely their run would stay identical as no error occurs, Quantum circuits, usually, would affect and might fail. Furthermore, when planing a decoder for quantum error correction codes, If one is willing to use a classical code to defend against phase flips he has to make sure that the decoding itself doesn't translate bit flip errors. 
\begin{definition}[Bit And Phase Filp] \label{def:bphf}  
 Consider a quantum state $\ket{\psi}$ encoded in the computation base. We will say that a bit-flip occurred in a scenario the operator $X$ applied on our state. The bit-flip event could be thought and be treated as exactly as the standard bit-flip error in the classical regime.     
  
\end{definition}

 
%\printbibliography[heading=subbibliography]
