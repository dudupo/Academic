\chapter{Quantum Error Correction Codes.}
\section{Introduction.}
It's wide believed that quantum machines have a significant advantage over the classical in range of computational tasks\cite{grover1996fast}, \cite{ahuja1999quantum}. Simple algorithms, which could be interpreted as the quantum version of scanning all the options cut the running time by square root of the classical magnitude. 

Nevertheless, Shore shown a polynomial depth quantum circuit that solve the hidden abelian subgroup \cite{Shor_1997}, what is considered as breakthrough, as it made the computer science community to believe that a quantum computer might offer an exponential advantage. Yet, even those that there is a general consensus about the superiority of ideal quantum computation model,it is still unclear that it feasible to implement such machine in the presence of noise.   
Still just point about the existences of noise is not powerful enough to cancel feasibility of computation and evidence of this is the fact that classical computers are also suffer form a certain rate of faults. So what are the difference? For getting a full understandings of the hardness of the problem let us compare two main reasons that made realize an hard task. 
First is the magnitude of the error rate, the classical computers also have errors, and sometimes when that happen we are wetness for systems failures (blue screen for example).     

 
%\printbibliography[heading=subbibliography]
