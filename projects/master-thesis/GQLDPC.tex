\chapter{Good qLDPC and LTC.}
We are now ready to present the Good qLDPC codes. 
  \begin{definition} Let $C = \Tann$. We say that $x \in C_{\oplus}$ is \textbf{reducible} if there exists a vertex $v$ and a small codeword $c_v$, for which, adding the assignment of $c_v$ over the $v$'s edges to $x$ decreases the weight. Namely, $|x + c_{v}| < |x|$. If $x \in C_{\oplus}$ is not a reducible codeword then we say that $x$ is \textbf{irreducible} \label{ire}. \end{definition}


\begin{definition}[$w$-Robustness] Let $C_{0}$ be code of length $\Delta$ with minimum distance $\delta_{0}\Delta$. $C = C_{0} \otimes \mathbb{F} + \mathbb{F}\otimes C_{0}^{\perp}$ will be said $w$-robust if any codeword $c \in C$ at weight less than $w$ it follows that $c$ is supported on at most $2\cdot w/\delta_{0}\Delta$ rows and cols.
\end{definition}

\begin{definition}[$p$-Resistance to Puncturing.] Let $p,w$ be integers. We will say that the dual tensor code $C_{A} \otimes \mathbb{F} + \mathbb{F} \otimes C_{B}$ is $w$-robust with $p$-resistance to puncturing, if the code obtained by removing (puncturing) a subset of at most $p$ rows and columns is $w$-robust.   
\end{definition}

\begin{theorem}
  For $\varepsilon \in \left( 0,\frac{1}{2} \right)$, $\gamma\in \left( \frac{1}{2} + \varepsilon, 1 \right)$, $\delta> 0$, large enough $\Delta$, and small codes $C_{A},C_{B}$ with distance at least $\delta\Delta$ if the dual tensor code of $C_{A},C_{B}$ is $w$-robust and $\Delta^{\gamma}$ resistance to puncturing, then there exists an infinite family of square complexes for which the Tanner code defined by the complexes and the dual tensor code such that any codeword with weigh less than $ \frac{\delta n}{4\Delta^{3/2 + \varepsilon}} $ is reducible \cref{ire}.
\end{theorem}

\begin{claim}
  The distance of the dual tensor code is at least $\delta\Delta$.
\end{claim}
\begin{proof}
  By the robustness any codeword with weight less than $\delta\Delta$ supported on at most one row. Now as the multiplication with any code word of $C_{A}^\perp \otimes C_{B}^\perp$ can be decomposed to a pairwise multiplication over the rows We obtain that only non trivial row must to be in $ \left( C_{B}^{\perp} \right)^{\perp}$ and there fore at weight at least $\delta\Delta$ and that is contradiction.  
\end{proof}

\begin{claim}
  \label{claim:epss}
  for any $\varepsilon \in \left( 0,1 \right)$ and large enough $\Delta$  it holds that $ |S| \le \Delta^{\varepsilon}|T| $ 
\end{claim}
\begin{proof}
Suppose not, namely that $|S| > \Delta^{\varepsilon}|T|$, then $|x|/|T| > \Delta^{\varepsilon}|x|/|S| > \Delta^{\varepsilon} \cdot \Delta $ But:  
\begin{equation*}
  \begin{split}
    \frac{|x|}{|T|} = \frac{\Theta \left(E(T,T) \right)}{|T|} \le \Theta(\Delta^{2})\frac{|T|}{n}  + \Theta(\Delta)  \rightarrow_{n\rightarrow \infty} \Theta(\Delta)
  \end{split}
\end{equation*}
\end{proof}

\begin{claim}
  There is a negative vertex which adjoins to only normal vertices. 
\end{claim}
% Define the graph $G^{\star} = (V_{-}, E^{\star})$ such that $E^{\star} = E \cup \left\{ \left\{g, xyg \right\} x\neq y, \in A\cup B \right\}$ And notice that the assumption follows that  any 
\begin{proof}
  Denote by $S_{e},S_{n}$ the exceptional and the normal vertices. Namely the weight of the local view for any vertex in $S_{e}$ is grater than $\Delta^{3/2 + \varepsilon}$, and the normal vertices is the complementary vertices set. Suppose through contradiction that any negative vertex $v_{-}\in V_{-}$ has at least one sibling in $S_{e}$. Therefore $|T| \le \Delta |S_{e}|$ combining with \cref{claim:epss} it follows that $|S| \le \Delta^{1+\varepsilon}|S_{e}|$ : 
  \begin{equation*}
    \begin{split}
      \Delta^{3/2 + \varepsilon} & \le \frac{E(S,S_{e})}{|S_{e}|} = \Theta\left( \Delta^{2} \right)\frac{|S|}{n} + \Theta\left( \Delta \right)\sqrt{ \frac{|S|}{|S_{e}|}  }\\ 
      & \le \Theta(\Delta^{2}) \frac{|S|}{n} + \Theta(\Delta) \Theta\left( \Delta^{\frac{1+\varepsilon}{2}} \right)  
    \end{split}
  \end{equation*} 
  Thus we have that: 
  \begin{equation*}
    \begin{split}
      \Delta^{3/2+\varepsilon} & \le \frac{E_{G_{A\times B}}\left( S_{e},S \right)}{|S_{e}|}\le  \frac{E_{G^{\star}}\left( S_{e},S \right)}{|S_{e}|} \\
      & \le \Theta\left( \Delta^{2} \right) \frac{  |S| }{n} +  \lambda\left( G^{\star} \right) \sqrt{ \frac{ |S|}{ |S_{e}|} } \\
      & \le \Theta\left( \Delta^{2} \right) \frac{|S|}{n}   
    \end{split}
  \end{equation*}
\end{proof}

\begin{equation*}
  \begin{split}
    \begin{matrix}
   \alpha  |S_{e}|  \le &\theta^{2}\frac{|S_{e}||S|}{|V|} - \lambda^{2}\sqrt{ |S_{e}||S| } \\
   \beta |T|   \le  &2\theta \frac{|T||S|}{2|V|} +2 \lambda\sqrt{ |T||S| }  
 \end{matrix}
 & \Rightarrow \begin{matrix}
 |S_{e}|  \le &\left( \alpha -   \theta^{2} \frac{|S|}{|V|} \right)^{-2} \lambda^{4} |S| \\
 |T|  \le &\left( \beta -   2 \theta \frac{|S|}{2|V|} \right)^{-2} 4\lambda^{2} |S|
 \end{matrix}
 \end{split}
\end{equation*}



\begin{definition}[The Disagreement Code] Given a Tanner code $C = \Tann$, define the code $C_{\oplus}$ to contain all the words equal to the formal summation $ \sum_{v \in V\left( G \right)} {c_{v} }$ when $c_{v}$ is an assignment of a codeword $ c_{v} \in C_0 $  on the edges of the vertex $ v \in V\left( G \right)$.
  We call to such code the \textbf{disagreement code} of $C$, as edges are set to 1 only if their connected vertices contribute to the summation codewords that are different on the corresponding bit to that edge. In addition, we will call to any contribute $c_v$, the \textbf{suggestion} of $v$. And notice that by linearity, each vertex suggests, at most, a single suggestion.   

  Finally, given a bits assessment $x \in \mathbb{F}_{2}^{E}$ over the edges of $G$, we will denote by $x^{\oplus} \in C_{\oplus} $ the codeword which obtained by summing up suggestions set such each vertex suggests the closet codeword to his local view. Namely, for each $v \in V$ define:   
  \begin{equation*}
    \begin{split}
      c_{v} & \leftarrow \arg_{ \tilde{c} \in C_{0}} \min{ d( x|_{v} , \tilde{c} ) } \ \ \forall v\in V   \\
      x^{\oplus} & \leftarrow \sum_{v \in V}{c_{v}} 
    \end{split}
  \end{equation*}
  We will think about $x^{\oplus}$ as the disagreement between the vertices over $x$. 

\end{definition}

\begin{definition} Let $C = \Tann$. We say that $x \in C_{\oplus}$ is \textbf{reducible} if there exists a vertex $v$ and a small codeword $c_v$, for which, adding the assignment of $c_v$ over the $v$'s edges to $x$ decreases the weight. Namely, $|x + c_{v}| < |x|$. If $x \in C_{\oplus}$ is not a reducible codeword then we say that $x$ is \textbf{irreducible} \label{ire}. \end{definition}

The following lemma states that the disagreement is invariant when adding codewords, resulting in any decoder that can correct errors occurring to the trivial codeword by taking the derived disagreement as input being able to correct the same errors when they occur to any codeword.

\begin{lemma}[Linearity of The Disagreement] \label{lemma:lin} Consider the code $C = \Tann$. Let $ x \in \mathbb{F}_{2}^{E}$ then for any $ y \in C$ it holds that: 
  \begin{equation*}
    \begin{split}
      \left( x + y  \right)^{\oplus} = \left( x  \right)^{\oplus} 
    \end{split}
  \end{equation*}
\end{lemma}
  \begin{proof} Having that $y \in C$ followes $y|_v \in C_{0}$ and therefore 


    \begin{equation*}
      \begin{split}
        \arg_{ \tilde{c} \in C_{0}} \min{ d( z  , \tilde{c} ) } = y|_{v} + \arg_{ \tilde{c} \in C_{0}} \min{ d( z, \tilde{c} + y|_{v} ) } 
      \end{split}
    \end{equation*}
     Hence the suggestion made by vertrx $v$ is: 
  \begin{equation*}
    \begin{split}
      c_{v}\leftarrow &  \arg_{ \tilde{c} \in C_{0}} \min{ d( (x+y)|_{v}  , \tilde{c} ) } \\
      \leftarrow &  y|_{v} +  \arg_{ \tilde{c} \in C_{0}} \min{ d( (x+y)|_{v}  , \tilde{c} + y|_{v} ) } \\
      \leftarrow &  y|_{v} +  \arg_{ \tilde{c} \in C_{0}} \min{ d( x|_{v} , \tilde{c} ) } 
    \end{split}
  \end{equation*}
  It follows that: 

  \begin{equation*}
    \begin{split}
      \left( x + y \right)^{\oplus} =& \sum_{v\in V}{c_{v}} = \sum_{v \in V}{y|_{v}} + \sum_{v\in V}{ \arg_{ \tilde{c} \in C_{0}} \min{ d( x|_{v} , \tilde{c} ) } } \\ 
      =& y^{\oplus} + x^{\oplus} = x^{\oplus}
    \end{split}
  \end{equation*}
  When the last transition follows immediately by the fact that $y \in C$ and therefore any pair of connected vertices contribute the same value for their associated edge \end{proof}
%
%  \begin{definition} Let $C = \Tann$. We say that $x \in C_{\oplus}$ is \textbf{reducable} if there exists a vertex $v$ and a small codeword $c_v$, for which, adding the assignment of $c_v$ over the $v$'s edges to $x$ decreases the weight. Namely, $|x + c_{v}| < |x|$. If $x \in C_{\oplus}$ is not a reducable codeword then we say that $x$ is \textbf{ireducable} \label{ire}. \end{definition}
%
%


\section{Decoding and Testing}
  For completeness, we show exactly how Theorem 1 implies testability. The following section repeats Leiverar's and Zemor's proof \cite{leverrier2022quantum}. Consider a binary string $x$ that is not a codeword. The main idea is the observation that the number of bits filliped by (any) decoder, while decoding $x$, bounds the distance $d\left( x, C \right)$ from above. In addition, the number of positive checks in the first iteration is exactly the number of violated restrictions.
%\begin{figure*}[h]
%\begin{adjustbox}{width=\textwidth}
  \begin{definition}Let $L = \{L_{i}\}^{2|E|}_{0}$  be a series of $2|E|$. Such that for each vertex $ v \in V$ $\sum_{ e = \{u,v\} }{ L_{e_v} } \in C_{0}$. We will call $L$ a \textit{Potential list} and refer to the $e_{v}$'the element of $L$ as a suggestion made by the vertex $v \in V$ for the edge $e \in E$. Sometimes we will use the notation $L_{v}$ to denote all the $L$'s coordinates of the form $ L_{e_{v}} \forall e \in \text{Support} \left( v \right) $. Define the \textit{Force} of $L$ to be the following sum $  F\left( L \right) = \sum_{e = \{v,u\} \in E }{ \left(L_{e_v} + L_{e_u}\right) }$ and notice that $ F\left( L \right) \in C_{\oplus}$. And define the \textit{state} $S(L) \subset \mathbb{F}^{|E|}_{2}$ of $L$ as the vector obtained by choosing an arbitrary value from $ \{ L_{e_v}, L_{e_u} \}$ for each edge $e \in E$.  
  \end{definition}
  \begin{claim} \label{claim:pot} Let $L$ be the Potential list. If $F(L)=0$ then $S(L)\in C$. \end{claim}
  \begin{proof} Denote by $\phi\left( e \right) \subset \{ L_{e_v}, L_{e_u} \}$ the value which was chosen to $e = \{v,u\} \in E$. By $F\left(L\right) = 0$ , it follows that $ L_{e_v} + L_{e_u} = 0 \Rightarrow L_{e_v} = L_{e_u} = \phi\left( e \right) $ for any $e \in E$. Hence for every $v\in V$ we have that $ S\left( L \right)|_{v} = \sum_{u \sim v}{ \phi\left( \{v,u\} \right) } =  \sum_{u \sim v}{ L_{e_v }} \in C_{0}$ $ \Rightarrow S\left( L \right) \in C$   
  \end{proof}
  The decoding goes as follows. First, each vertex suggests the closet $C_{0}$'s codeword to his local view. Those suggestions define a Potential list, denote it by $L$, then if $F\left( L \right) <\tau$, by Theorem 1, one could find a suggestion of vertex $v$ and a codeword $c_v$ such that updating the value of $L_{v} \leftarrow L_{v} + c_{v}$ yields a Potential list with lower force. Therefore repeating the process till the force vanishes, obtain a Potential list in which its state is a codeword. 
  \begin{definition} Let $\tau > 0, f : \mathbb{N} \rightarrow \mathbb{R^{+}}$, and consider a Tanner Code $C = \mathcal{T}\left( G, C_{0} \right)$. Let us Define the following decoder and denote it by $\mathcal{D}$.  
  \end{definition}

  \begin{algorithm}[h]
    \caption{Decoding}
    \label{alg:three}
    \KwData{ $x \in \mathbb{F}_{2}^{n}$ }
    \KwResult{ $\arg\min {\left\{  y \in C : |y + x|  \right\} }$ if $d(y,C) < \tau $ and False otherwise. }
    $ L \leftarrow \text{Array} \{ \} $\\
    \For { $ v \in V$} {
      $c^{\prime}_{v} \leftarrow \arg\min {\left\{  y \in C_{0} : |y + x|_{v} |  \right\} } $\\
      $ L_{v} \leftarrow c^{\prime}_{v}$
    }
    $ z \leftarrow \sum_{v \in V}{c^{\prime}_{v}} $\\
    \eIf{ $ |z| < \tau \frac{n}{f\left( n \right)} $}{
      \While{ $|z| > 0$ }{
	find $v$ and $c \in C_{0}$ such that $|z + c_{v}| < |z|$\\
	$z \leftarrow z + c_{v}$ \\
	$ L_{v} \leftarrow  L_{v} + c_{v}$
      }
    }{
      reject. 
    }
    \Return  $S(L) $

  \end{algorithm}

  \begin{theorem}
Consider a Tanner Code $C = [n, n\rho, n\delta]$ and the corresponding disagreement code $C_{\oplus}$. Suppose that for every codeword $z \in C_{\oplus}$ such that $|z| < \frac{\tau^{\prime} n}{f\left(n\right)}$, there exists a vertex $v$ and a suggestion for $v$ which is another codeword $y \in C_{\oplus}$ such that $|z + y| < |z|$. Set $\tau \leftarrow \frac{\tau^{\prime}}{6 \Delta} \delta$ then.

  \begin{enumerate}
    \item $\mathcal{D}$ corrects any error at a weight less than $\tau n / f\left(n\right)$.   
    \item $C$ is $f\left( n \right)$ testable code.
  \end{enumerate}
\end{theorem}

\begin{proof} So it is clear from the claim \cref{claim:pot} above that if the condition at line (6) is satisfied, then $\mathcal{D}$  will converge into some codeword in $C$. Hence, to complete the first section, it left to show that $\mathcal{D}$ returns the closest codeword. Denote by $e$ the error, and by simple counting arguments; we have that $\mathcal{D}$ flips at most:  
  \begin{equation*}
    \begin{split}
      d_{\mathcal{D}}\left( x, C \right) & \le 2|e|\Delta + \tau \frac{n}{f\left( n \right)}\Delta
    \end{split}
  \end{equation*}
  bits. Hence, by the assumption, 
  \begin{equation*}
    \begin{split}
      d_{\mathcal{D}}\left( x, C \right) & \le 3\Delta \tau \frac{n}{f\left( n \right)} \le 3\Delta \tau\delta n < \frac{1}{2} \delta n  
    \end{split}
  \end{equation*}
  Therefore the code word returned by $\mathcal{D}$ must be the closet. Otherwise, it contradicts the fact that the relative distance of the code is $\delta$.
  To obtain the correctness of the second section, we will separate when the conditional at the line (5) holds and not. And prove that the testability inequality holds in both cases. 
  Let $x \in \mathbb{F}_{2}^{n}$ and consider the running of $\mathcal{D}$ over $x$. Assume the first case, in which the conditional at line (5) is satisfied. In that case, $\mathcal{D}$ decodes $x$ into its closest codeword in $C$. Therefore:
  \begin{equation*}
    \begin{split}
      d\left( x, C \right) \le & \ d_{D} \left( x, C \right) \le m\xi\left( x \right)\Delta +  |z|\Delta  \\ \le &  \  m\xi\left( x \right)\Delta + m\xi\left( x \right)  \Delta^{2} \\ 
      \frac{d\left( x, C \right)}{n} \le & \  \kappa_{1} \xi\left( x \right)    
    \end{split}
  \end{equation*}
  Now, consider the other case in which: $ |z| \ge \tau \frac{n}{f\left( n \right)}  $.
  \begin{equation*}
    \begin{split}
      \frac{d\left( x, C \right)}{n} & \le 1 \le \frac{|z|}{\tau n}f\left( n \right) \le \frac{m}{n} \frac{1}{\tau} \Delta \xi\left( x\right)f\left( n \right) \\ & \le \kappa_{2} \xi\left( x \right)f\left( n \right)  
    \end{split}
  \end{equation*}
  Picking $ \kappa \leftarrow \max \{ \kappa_{1}, \kappa_{2} \}$ proves $f\left( n \right)$-testability
\end{proof}



%\printbibliography[heading=subbibliography]
