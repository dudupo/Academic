
\chapter{Further Research.}
Both good quantum LDPC and good LTC had been open problems for more than decades, so the fact that they have been solved in one stroke raises two interesting questions. The first one is whether the construction that obtains each of them alone can also yield good quantum LTC? What is the exact challenge that prevents the inventors from proceeding into quantum testability?

The second question is whether one of the codes can be obtained without giving the other. Either positive or negative answers to these questions will shed light on our understanding of what quantumness is.

In this chapter, we present our attempts to provide answers. The chapter is divided into two parts. In the first, we demonstrate a variation of the classical Tanner code that defines many equations. We believe that this variation contains enough structure to guarantee degeneration of equations similarly to what occurs in the polynomial code. In the second chapter, we show that the polynomial code is not $w$-robust. If it were not the case, then one could hope to obtain quantum testability by considering a Tanner code in which any vertex restricts its local view to be a codeword of a quantum code.



\section{Local Majority $=?$ Local Testability.}

We begin by demonstrating that selecting $C_0$, the small code in the Tanner code, to have a large distance, which can also be considered as adding numerous restrictions, yields testability. However, the amount one will have to enlarge the distance to cannot be achieved with a rate greater than $\frac{1}{2}$ by the Singleton bound.
  %\section{Construction}
  \subsection{Almost LTC With Zero Rate}
%  \begin{definition}[The Disagreement Code] Given a Tanner code $C = \Tann$, define the code $C_{\oplus}$ to contain all the words equal to the formal summation $ \sum_{v \in V\left( G \right)} {c_{v} }$ when $c_{v}$ is an assignment of a codeword $ c_{v} \in C_0 $  on the edges of the vertex $ v \in V\left( G \right)$.
  We call to such code the \textbf{disagreement code} of $C$, as edges are set to 1 only if their connected vertices contribute to the summation codewords that are different on the corresponding bit to that edge. In addition, we will call to any contribute $c_v$, the \textbf{suggestion} of $v$. And notice that by linearity, each vertex suggests, at most, a single suggestion.   

  Finally, given a bits assessment $x \in \mathbb{F}_{2}^{E}$ over the edges of $G$, we will denote by $x^{\oplus} \in C_{\oplus} $ the codeword which obtained by summing up suggestions set such each vertex suggests the closet codeword to his local view. Namely, for each $v \in V$ define:   
  \begin{equation*}
    \begin{split}
      c_{v} & \leftarrow \arg_{ \tilde{c} \in C_{0}} \min{ d( x|_{v} , \tilde{c} ) } \ \ \forall v\in V   \\
      x^{\oplus} & \leftarrow \sum_{v \in V}{c_{v}} 
    \end{split}
  \end{equation*}
  We will think about $x^{\oplus}$ as the disagreement between the vertices over $x$. 

\end{definition}

\begin{definition} Let $C = \Tann$. We say that $x \in C_{\oplus}$ is \textbf{reducible} if there exists a vertex $v$ and a small codeword $c_v$, for which, adding the assignment of $c_v$ over the $v$'s edges to $x$ decreases the weight. Namely, $|x + c_{v}| < |x|$. If $x \in C_{\oplus}$ is not a reducible codeword then we say that $x$ is \textbf{irreducible} \label{ire}. \end{definition}

The following lemma states that the disagreement is invariant when adding codewords, resulting in any decoder that can correct errors occurring to the trivial codeword by taking the derived disagreement as input being able to correct the same errors when they occur to any codeword.

\begin{lemma}[Linearity of The Disagreement] \label{lemma:lin} Consider the code $C = \Tann$. Let $ x \in \mathbb{F}_{2}^{E}$ then for any $ y \in C$ it holds that: 
  \begin{equation*}
    \begin{split}
      \left( x + y  \right)^{\oplus} = \left( x  \right)^{\oplus} 
    \end{split}
  \end{equation*}
\end{lemma}
  \begin{proof} Having that $y \in C$ followes $y|_v \in C_{0}$ and therefore 


    \begin{equation*}
      \begin{split}
        \arg_{ \tilde{c} \in C_{0}} \min{ d( z  , \tilde{c} ) } = y|_{v} + \arg_{ \tilde{c} \in C_{0}} \min{ d( z, \tilde{c} + y|_{v} ) } 
      \end{split}
    \end{equation*}
     Hence the suggestion made by vertrx $v$ is: 
  \begin{equation*}
    \begin{split}
      c_{v}\leftarrow &  \arg_{ \tilde{c} \in C_{0}} \min{ d( (x+y)|_{v}  , \tilde{c} ) } \\
      \leftarrow &  y|_{v} +  \arg_{ \tilde{c} \in C_{0}} \min{ d( (x+y)|_{v}  , \tilde{c} + y|_{v} ) } \\
      \leftarrow &  y|_{v} +  \arg_{ \tilde{c} \in C_{0}} \min{ d( x|_{v} , \tilde{c} ) } 
    \end{split}
  \end{equation*}
  It follows that: 

  \begin{equation*}
    \begin{split}
      \left( x + y \right)^{\oplus} =& \sum_{v\in V}{c_{v}} = \sum_{v \in V}{y|_{v}} + \sum_{v\in V}{ \arg_{ \tilde{c} \in C_{0}} \min{ d( x|_{v} , \tilde{c} ) } } \\ 
      =& y^{\oplus} + x^{\oplus} = x^{\oplus}
    \end{split}
  \end{equation*}
  When the last transition follows immediately by the fact that $y \in C$ and therefore any pair of connected vertices contribute the same value for their associated edge \end{proof}
%
%  \begin{definition} Let $C = \Tann$. We say that $x \in C_{\oplus}$ is \textbf{reducable} if there exists a vertex $v$ and a small codeword $c_v$, for which, adding the assignment of $c_v$ over the $v$'s edges to $x$ decreases the weight. Namely, $|x + c_{v}| < |x|$. If $x \in C_{\oplus}$ is not a reducable codeword then we say that $x$ is \textbf{ireducable} \label{ire}. \end{definition}
%
%


  \begin{theorem}[LTC Zero Rate] 
      \label{theorem:ltczerorate}
    There exist a constant $\alpha > 0$ and an infinte family of Tanner Codes $C = \Tann$ such that any \ireducable codeword $x$ of a coresponding disagreement code $x \in C_{\oplus}$ at length $n$, weight at least $\alpha n$. \end{theorem}



  \paragraph{Proof.} By induction over the number of vertices $V^\prime \subset V$, which suggest a nontrivial codeword to $x$. Base, assume that a single vertex $v \in V$ suggests a nontrivial codeword $c_{v} \in C_{0}$. Then it's clear that $x = c_{v}$. And therefore, we have that $|x +c_{v}| = 0 < |x|$.

  Assume the correctness of the argument for every codeword defined by at most $m$ nontrivial suggestions made by $V^\prime \subset V$. And consider the graph $\left( V^\prime, E^\prime \right)$ induced by them. If the graph has more than a single connectivity component, then any of them is also a codeword of $C_{\oplus}$  but composed of at most $m-1$ nontrivial suggestions. Therefore, by the assumption, we could find a vertex $v$ and a proper small codeword $c_v \in C_0 $, such that the addition of the suggestion will decrease the weight of the codeword defined on that component and therefore decrease the total weight of $x$.

  So, we can assume that the vertices in $V^\prime$ compose a single connectivity component. Let be $x|_{v} \in \mathbb{F}_{2}^{\Delta}$ the bits of $x$ on the indices corresponding to $v$'s edges. For any $S \subset E$, define $w_{S}\left( x \right)$ as the weight that $x$ induces over $S$. Sometimes we will refer to $w_{S}\left( x \right)$ as the \textbf{flux} induced by $x$ over $S$.

  The genreal idea of the proof is to show that if the distance of the small code is large ($ \ge \frac{2}{3}$ ) and $x$ is \ireducable codeword then there exist an indepandent subset of vertices $U \subset V^{\prime}$, at linear size, that induce a significant flux over $E/E^{\prime}$. If $U$ has linear size than also $x$ has a linear size, And if not, Then we will show that no serious interface has been occurred. ~\cref{claim:deg} and ~\cref{claim:tree-size} state that if one is willing to hide an ireducable \hyperref[ire]{[\ref{ire}]} error then he has to touch at least a linar number of verties. ~\cref{claim:flu1} and  ~\cref{claim:flu2} quanitify the flux that induced by such errors. 

\begin{claim}\label{claim:deg} For any $v \in V^\prime$ and corresponded suggestion $c_{v}$ it holds that: $w_{E^\prime}\left( c_{v} \right) \ge \frac{1}{2}\delta_{0}\Delta$. \end{claim}
  \begin{proof} Notice that any edge of $E$ connected only to a single vertex in $V^\prime$ equals the corresponding bit in the original suggestion made by $c_{v}$. Hence for every $v\in V^\prime$, it holds that: 
\begin{equation*}
      \begin{split}
	    w_{E / E^\prime}\left(x|_{v}\right) = w_{E / E^\prime}\left(c_{v}\right) \Rightarrow  w_{E / E^\prime}\left(x|_{v}\right) \le | x \cap c_{v} |
      \end{split}
    \end{equation*}
     Now consider the weight of $x + c_{v}$, By the assumption that $x$ is ireducable code word of $c_{\oplus}$ we have that: 
  
 \begin{equation*}
    \begin{split}
       |x + c_{v}| & = |x| + |c_{v}| - 2|x \cap c_{v}| > |x| \\
       \Rightarrow &   |x \cap c_{v}|  < \frac{1}{2} |c_{v}| \\
      w_{E^\prime}\left( c_{v} \right) &= |c_{v}| - w_{E / E^\prime}\left( c_{v} \right) =  |c_{v} | - w_{E / E^\prime}\left( x|_{v} \right) \\ 
      & \ge | c_{v} | - | x \cap c_{v} |  \ge \frac{1}{2}|c_{v}| = \frac{1}{2}\delta_{0}\Delta 
    \end{split}
  \end{equation*}
  \end{proof}

  Consider an arbitrary vertex $r \in V^\prime$, and consider the DAG obtained by the BFS walk over the subgraph $\left(V^\prime, E^\prime \right)$ starting at $r$. Denote this directed tree by $T$.

%Let $g$ be the girth of the graph and consider a layer $U$ in $T$ at height $h\left( U \right)$ satisfies the inequality $ h\left( U \right) < \frac{1}{2}g + l$ for some integer $l$.
  %\begin{adjustbox}{width=150pt}%\columnwidth}
%\begin{figure*}[t]%{width=150pt} %0.3\textwidth}

  
%\end{figure*}
%\end{adjustbox} 
  \begin{claim} \label{claim:tree-size} The size of $T$ is at least:
  \begin{equation*}
    \begin{split}
      |T| & \ge \left( \frac{1}{4}\delta_{0} - \frac{\lambda}{\Delta} \right)n 
    \end{split}
  \end{equation*}
\end{claim}
\begin{proof}By~\cref{claim:deg}  any $v \in T$ the degree of $v$ is at least $\frac{1}{2}\delta_{0}\Delta$ we have that: $E\left( T,T \right) \ge \frac{1}{2}\cdot \frac{1}{2}\delta_{0}\Delta |T|$. Combine the Mixining Expander Lemma we obtain:
  \begin{equation*}
    \begin{split}
      \frac{1}{4}\delta_{0}\Delta |T| & \le \frac{\Delta}{n}|T|^2  + \lambda|T| \\ 
      \Rightarrow & \left( \frac{\Delta}{n}|T| + \lambda -  \frac{1}{4}\delta_{0}\Delta \right)|T| \ge 0 \\ 
      \Rightarrow & |T| \ge \left( \frac{1}{4}\delta_{0} - \frac{\lambda}{\Delta} \right)n 
    \end{split}
  \end{equation*}
  \end{proof}



\begin{claim} \label{claim:flu1}  Suppose that $G$ is an expander graph with a second eigenvalue $\lambda$, then For any layer $U$ there exist a layer $U^{\prime}$ such that:
  \begin{equation*}
    \begin{split}
      (1) & \ \ |U^{\prime}| \ge |U| \\
      (2) & \ \ w_{E/E^{\prime}}\left( x|_{U^{\prime}} \right)  \ge\Delta|U^{\prime}|\left( \delta_{0}-\frac{2}{3}-\frac{2\lambda}{\Delta} \right)
    \end{split}
  \end{equation*}
\end{claim} 
  \begin{proof} Consider layer $U$ and denote by $U_{-1}$ and $U_{+1}$ the preceding and the following layers to $U$ in $T$. It follows from the expander mixing lemma that:
  \begin{equation*}
    \begin{split}
      w_{E/E^{\prime}}\left( x|_{U} \right) & \ge \delta_{0}\Delta|U| -\wcutUU \ge \\ 
      & \delta_{0}\Delta|U| - \cutUU \\ 
      &  \delta_{0}\Delta|U| - \Delta\frac{|U||U_{-1}|}{n} - \Delta\frac{|U||U_{+1}|}{n} \\
      & -\lambda\sqrt{|U||U_{-1}|} - \lambda\sqrt{|U||U_{+1}|}
    \end{split}
  \end{equation*}

  \begin{claim} \label{claim:maxu} We can assume that $|U| \ge |U_{-1}|, |U_{+1}|$. \end{claim}
  \begin{proof} Suppose that $|U_{+1}| > |U|$, so we could choose $U$ to be $U_{+1}$. Continuing stepping deeper till we have that $|U| > |U_{+1}|, |U_{-1}|$. Simiraly, if $|U| > |U_{+1}|$ but $|U_{-1}| > |U|$, the we could take steps upward by replacing $U_{-1}$ with $U$. At the end of the process, we will be left with $U$ at a size greater than the initial layer and $|U| > |U_{+1}|, |U_{-1}|$ \end{proof}

  Using ~\cref{claim:maxu}, we have that $\left( |U_{+1}| + |U_{-1}| \right)/n <\frac{2}{3} $ and therefore:
  \begin{equation*}
    \begin{split}
      w_{E/E^{\prime}}\left( x|_{U} \right) & \ge \left( \delta_{0} - \frac{2}{3} - \frac{2\lambda}{\Delta} \right) \Delta |U| \ \   
    \end{split}
  \end{equation*}
\end{proof}
  That immediately yields the following: let $U_{\text{max}} = \text{arg} \max_{U \text{ layer in }  T } |U|  $  then: 
  \begin{equation*}
    \begin{split}
      |x| \ge  w_{E/E^{\prime}}\left( x|_{U_{\text{max}}} \right) \ge \left( \delta_{0} - \frac{2}{3} - \frac{2\lambda}{\Delta} \right)\Delta |U_{\text{max}}|
    \end{split}
  \end{equation*}
  \begin{claim}  \label{claim:flu2}Consider again the maximal layer $U_{\max}$ then: 
  \begin{equation*}
    \begin{split}
      w_{E/E^{\prime}}\left( x \right) \ge \left( \delta_{0} - \frac{|U_{\max}|}{n} - \frac{\lambda}{\Delta} \right) \Delta|T| 
    \end{split}
  \end{equation*}
\end{claim}

  \begin{proof} Similarly to above, now we will bound the flux that all the nodes in $T$ induce over $E/E^{\prime}$. Denote by $U_{0}, U_{1} .. U_{m}$ the layers of $T$ ordered corresponded to their height, thus we obtain: 
  \begin{equation*}
    \begin{split}
      w_{E/E^{\prime}}\left( x \right) & \ge \delta_{0}\Delta|T| - \sum_{i\in [m]}{ w \left( E\left( U_{i}, U_{i+1}  \right) \right)  } \\ 
      \ge & \delta_{0}\Delta|T|  - \sum_{i \in [m]}{ E\left( U_{i}, U_{i+1}  \right)  } \\ 
      \ge & \delta_{0}\Delta|T|  -  \sum_{i \in [m]}{ \frac{\Delta}{n}|U_{i}| |U_{i+1}| + \lambda \sqrt{ |U_{i}| |U_{i+1}|} }\\ 
      \ge & \delta_{0}\Delta|T|  -  \sum_{i \in [m]}{ \frac{\Delta}{n}|U_{i}| |U_{i+1}| + \lambda \frac{ |U_{i}|+ |U_{i+1}|}{2 } }\\ 
      \ge & \delta_{0}\Delta|T|  - \frac{\Delta}{n}|T||U_{\max}| -  \lambda |T| \\ 
      \ge & \left( \delta_{0} - \frac{|U_{\max}| }{n}-  \frac{\lambda}{\Delta} \right) \Delta|T| 
    \end{split}
  \end{equation*}
  \end{proof}

  \begin{claim}  Alternate proof of fulx inequality, which dosn't assume that there is no interference inside the layers. $w\left( E\left( U,U \right) \right) > 0 $. 
  \end{claim}
  \newcommand{\mxU}{U_{\text{max}}}
  \begin{proof}
    Separeate into the following cases, First assume that $ \mxU / n   > \frac{1}{3}  $ then we have that the total interference with $\mxU$ layers is at most: 
    \begin{equation*}
      \begin{split}
	& \frac{\Delta|\mxU|\left( n- |\mxU| \right)}{n} + \lambda\sqrt{\mxU n } \le \left( 1 - \frac{\mxU}{n} + \sqrt{3} \frac{\lambda}{\Delta}  \right) \Delta |\mxU |  \\ 
	& \le \left( \frac{2}{3} + \sqrt{3} \frac{\lambda}{\Delta}  \right) \Delta |\mxU |  
      \end{split}
    \end{equation*}
    And therefore we have that the flux induced by $\mxU$ is at least: 
    \begin{equation*}
      \begin{split}
	\left( \delta_{0}\Delta -  \frac{2}{3} + \sqrt{3} \frac{\lambda}{\Delta}  \right)\Delta\mxU
      \end{split}
    \end{equation*}
   
    So it lefts to consider the case in which for every layer it holds that $\mxU \le \frac{1}{3}n$. At that case we count the fulx induced by the whole three $T$ which is what exactly we have prove in ~\cref{cliam:flu2} minus the inner interference at the tree, That it we need only to subtract $ \sum{ \frac{\Delta|U_{i}|^{2}}{n} + \lambda|U_{i}| } \le \left(\frac{|\mxU|}{n} + \lambda/\Delta  \right)  |T| $ So we obtained that in that case: 
    \begin{equation*}
      \begin{split}
	w_{E/E^{\prime}}\left( x \right)\ge\left( \delta_{0} - 2 \frac{\mxU }{n} - 2\lambda/\Delta \right) \Delta |T| \ge \left( \delta_{0} - \frac{2}{3} - 2 \frac{\lambda}{\Delta}    \right)\Delta |T|
      \end{split}
    \end{equation*}<++>
  \end{proof}

  \begin{proof}[Proof of Theorem 1.] Consider the size of the maxiaml layer $|U_{\max}|$ and sepearte to the following two cases. First, consider the case that $|U_{\max}| \ge  \alpha n $ in that case it follows immedily by~\cref{claim:flu1} that if $\delta_{0} > \frac{2}{3} - \frac{2\lambda}{\Delta}$ there exists $\alpha^{\prime} > 0 $ such that:  
  \begin{equation*}
    \begin{split}
      |x| \ge \left( \delta_{0} - \frac{2}{3} - \frac{2}{\lambda}\Delta \right)\Delta|U_{\max}| \ge  \alpha^{\prime} n 
    \end{split}
  \end{equation*}
  So, it is lefts to consider the second case in which $ |U_{\max}| < \alpha n $ in that case, we have from~\cref{claim:flu2} inequality that: 

  \begin{equation*}
    \begin{split}
      |x| & \ge  w_{E/E^{\prime}}\left( x \right)  \ge \left( \delta_{0} - \frac{|U_{\max}|}{n} - \frac{\lambda}{\Delta} \right) \Delta|T| \\ 
      & \ge \left( \delta_{0} - \alpha - \frac{\lambda}{\Delta} \right) \Delta|T| 
    \end{split}
  \end{equation*}
  Setting $\alpha \ge \frac{2}{3}$ we complete the proof
\end{proof}

Unfortunately, Singelton bound doesn't allow both $\delta_0 > \frac{2}{3}$ and $\rho_0 \ge \frac{1}{2}$, so in total, we prove the existence of code LDPC code which is good in terms of testability and distance yet has a zero rate. In the following subsection, we will prove that one can overcome this problem, by considering a variton of Tanner code, in which every vertex cheks only a $\frac{2}{3}$ fraction of the edges in his support.      



%\printbibliography[heading=subbibliography]
