 
  %\section{Construction}
  \subsection{Almost LTC With Zero Rate}
  \begin{definition}[The Disagreement Code] Given a Tanner code $C = \Tann$, define the code $C_{\oplus}$ to contain all the words equal to the formal summation $ \sum_{v \in V\left( G \right)} {c_{v} }$ when $c_{v}$ is an assignment of a codeword $ c_{v} \in C_0 $  on the edges of the vertex $ v \in V\left( G \right)$.
  We call to such code the \textbf{disagreement code} of $C$, as edges are set to 1 only if their connected vertices contribute to the summation codewords that are different on the corresponding bit to that edge. In addition, we will call to any contribute $c_v$, the \textbf{suggestion} of $v$. And notice that by linearity, each vertex suggests, at most, a single suggestion.   

  Finally, given a bits assessment $x \in \mathbb{F}_{2}^{E}$ over the edges of $G$, we will denote by $x^{\oplus} \in C_{\oplus} $ the codeword which obtained by summing up suggestions set such each vertex suggests the closet codeword to his local view. Namely, for each $v \in V$ define:   
  \begin{equation*}
    \begin{split}
      c_{v} & \leftarrow \arg_{ \tilde{c} \in C_{0}} \min{ d( x|_{v} , \tilde{c} ) } \ \ \forall v\in V   \\
      x^{\oplus} & \leftarrow \sum_{v \in V}{c_{v}} 
    \end{split}
  \end{equation*}
  We will think about $x^{\oplus}$ as the disagreement between the vertices over $x$. 

\end{definition}

\begin{definition} Let $C = \Tann$. We say that $x \in C_{\oplus}$ is \textbf{reducible} if there exists a vertex $v$ and a small codeword $c_v$, for which, adding the assignment of $c_v$ over the $v$'s edges to $x$ decreases the weight. Namely, $|x + c_{v}| < |x|$. If $x \in C_{\oplus}$ is not a reducible codeword then we say that $x$ is \textbf{irreducible} \label{ire}. \end{definition}

The following lemma states that the disagreement is invariant when adding codewords, resulting in any decoder that can correct errors occurring to the trivial codeword by taking the derived disagreement as input being able to correct the same errors when they occur to any codeword.

\begin{lemma}[Linearity of The Disagreement] \label{lemma:lin} Consider the code $C = \Tann$. Let $ x \in \mathbb{F}_{2}^{E}$ then for any $ y \in C$ it holds that: 
  \begin{equation*}
    \begin{split}
      \left( x + y  \right)^{\oplus} = \left( x  \right)^{\oplus} 
    \end{split}
  \end{equation*}
\end{lemma}
  \begin{proof} Having that $y \in C$ followes $y|_v \in C_{0}$ and therefore 


    \begin{equation*}
      \begin{split}
        \arg_{ \tilde{c} \in C_{0}} \min{ d( z  , \tilde{c} ) } = y|_{v} + \arg_{ \tilde{c} \in C_{0}} \min{ d( z, \tilde{c} + y|_{v} ) } 
      \end{split}
    \end{equation*}
     Hence the suggestion made by vertrx $v$ is: 
  \begin{equation*}
    \begin{split}
      c_{v}\leftarrow &  \arg_{ \tilde{c} \in C_{0}} \min{ d( (x+y)|_{v}  , \tilde{c} ) } \\
      \leftarrow &  y|_{v} +  \arg_{ \tilde{c} \in C_{0}} \min{ d( (x+y)|_{v}  , \tilde{c} + y|_{v} ) } \\
      \leftarrow &  y|_{v} +  \arg_{ \tilde{c} \in C_{0}} \min{ d( x|_{v} , \tilde{c} ) } 
    \end{split}
  \end{equation*}
  It follows that: 

  \begin{equation*}
    \begin{split}
      \left( x + y \right)^{\oplus} =& \sum_{v\in V}{c_{v}} = \sum_{v \in V}{y|_{v}} + \sum_{v\in V}{ \arg_{ \tilde{c} \in C_{0}} \min{ d( x|_{v} , \tilde{c} ) } } \\ 
      =& y^{\oplus} + x^{\oplus} = x^{\oplus}
    \end{split}
  \end{equation*}
  When the last transition follows immediately by the fact that $y \in C$ and therefore any pair of connected vertices contribute the same value for their associated edge \end{proof}
%
%  \begin{definition} Let $C = \Tann$. We say that $x \in C_{\oplus}$ is \textbf{reducable} if there exists a vertex $v$ and a small codeword $c_v$, for which, adding the assignment of $c_v$ over the $v$'s edges to $x$ decreases the weight. Namely, $|x + c_{v}| < |x|$. If $x \in C_{\oplus}$ is not a reducable codeword then we say that $x$ is \textbf{ireducable} \label{ire}. \end{definition}
%
%


    \begin{theorem*}[Theorem 1] There exist a constant $\alpha > 0$ and an infinte family of Tanner Codes $C = \Tann$ such that any \ireducable codeword $x$ of a coresponding disagreement code $x \in C_{\oplus}$ at length $n$, weight at least $\alpha n$. \end{theorem*}


  \paragraph{Proof.} By induction over the number of vertices $V^\prime \subset V$, which suggest a nontrivial codeword to $x$. Base, assume that a single vertex $v \in V$ suggests a nontrivial codeword $c_{v} \in C_{0}$. Then it's clear that $x = c_{v}$. And therefore, we have that $|x +c_{v}| = 0 < |x|$.

  \paragraph{}

  Assume the correctness of the argument for every codeword defined by at most $m$ nontrivial suggestions made by $V^\prime \subset V$. And consider the graph $\left( V^\prime, E^\prime \right)$ induced by them. If the graph has more than a single connectivity component, then any of them is also a codeword of $C_{\oplus}$  but composed of at most $m-1$ nontrivial suggestions. Therefore, by the assumption, we could find a vertex $v$ and a proper small codeword $c_v \in C_0 $, such that the addition of the suggestion will decrease the weight of the codeword defined on that component and therefore decrease the total weight of $x$.

  So, we can assume that the vertices in $V^\prime$ compose a single connectivity component. Let be $x|_{v} \in \mathbb{F}_{2}^{\Delta}$ the bits of $x$ on the indices corresponding to $v$'s edges. For any $S \subset E$, define $w_{S}\left( x \right)$ as the weight that $x$ induces over $S$. Sometimes we will refer to $w_{S}\left( x \right)$ as the \textbf{flux} induced by $x$ over $S$.

  \paragraph{}

  The genreal idea of the proof is to show that if the distance of the small code is large ($ \ge \frac{2}{3}$ ) and $x$ is \ireducable codeword then there exist an indepandent subset of vertices $U \subset V^{\prime}$, at linear size, that induce a significant flux over $E/E^{\prime}$. If $U$ has linear size than also $x$ has a linear size, And if not, Then we will show that no serious interface has been occurred. ~\cref{claim:deg} and ~\cref{claim:tree-size} state that if one is willing to hide an ireducable \hyperref[ire]{[\ref{ire}]} error then he has to touch at least a linar number of verties. ~\cref{claim:flu1} and  ~\cref{claim:flu2} quanitify the flux that induced by such errors. 

\begin{claim}\label{claim:deg} For any $v \in V^\prime$ and corresponded suggestion $c_{v}$ it holds that: $w_{E^\prime}\left( c_{v} \right) \ge \frac{1}{2}\delta_{0}\Delta$. \end{claim}
  \begin{proof} Notice that any edge of $E$ connected only to a single vertex in $V^\prime$ equals the corresponding bit in the original suggestion made by $c_{v}$. Hence for every $v\in V^\prime$, it holds that: 
\begin{equation*}
      \begin{split}
	    w_{E / E^\prime}\left(x|_{v}\right) = w_{E / E^\prime}\left(c_{v}\right) \Rightarrow  w_{E / E^\prime}\left(x|_{v}\right) \le | x \cap c_{v} |
      \end{split}
    \end{equation*}
     Now consider the weight of $x + c_{v}$, By the assumption that $x$ is ireducable code word of $c_{\oplus}$ we have that: 
  
 \begin{equation*}
    \begin{split}
       |x + c_{v}| & = |x| + |c_{v}| - 2|x \cap c_{v}| > |x| \\
       \Rightarrow &   |x \cap c_{v}|  < \frac{1}{2} |c_{v}| \\
      w_{E^\prime}\left( c_{v} \right) &= |c_{v}| - w_{E / E^\prime}\left( c_{v} \right) =  |c_{v} | - w_{E / E^\prime}\left( x|_{v} \right) \\ 
      & \ge | c_{v} | - | x \cap c_{v} |  \ge \frac{1}{2}|c_{v}| = \frac{1}{2}\delta_{0}\Delta 
    \end{split}
  \end{equation*}
  \end{proof}

  Consider an arbitrary vertex $r \in V^\prime$, and consider the DAG obtained by the BFS walk over the subgraph $\left(V^\prime, E^\prime \right)$ starting at $r$. Denote this directed tree by $T$.

%Let $g$ be the girth of the graph and consider a layer $U$ in $T$ at height $h\left( U \right)$ satisfies the inequality $ h\left( U \right) < \frac{1}{2}g + l$ for some integer $l$.
  %\begin{adjustbox}{width=150pt}%\columnwidth}
%\begin{figure*}[t]%{width=150pt} %0.3\textwidth}

%\end{figure*}
%\end{adjustbox} 
  \begin{claim} \label{claim:tree-size} The size of $T$ is at least:
  \begin{equation*}
    \begin{split}
      |T| & \ge \left( \frac{1}{4}\delta_{0} - \frac{\lambda}{\Delta} \right)n 
    \end{split}
  \end{equation*}
\end{claim}
\begin{proof}By~\cref{claim:deg}  any $v \in T$ the degree of $v$ is at least $\frac{1}{2}\delta_{0}\Delta$ we have that: $E\left( T,T \right) \ge \frac{1}{2}\cdot \frac{1}{2}\delta_{0}\Delta |T|$. Combine the Mixining Expander Lemma we obtain:
  \begin{equation*}
    \begin{split}
      \frac{1}{4}\delta_{0}\Delta |T| & \le \frac{\Delta}{n}|T|^2  + \lambda|T| \\ 
      \Rightarrow & \left( \frac{\Delta}{n}|T| + \lambda -  \frac{1}{4}\delta_{0}\Delta \right)|T| \ge 0 \\ 
      \Rightarrow & |T| \ge \left( \frac{1}{4}\delta_{0} - \frac{\lambda}{\Delta} \right)n 
    \end{split}
  \end{equation*}
  \end{proof}

