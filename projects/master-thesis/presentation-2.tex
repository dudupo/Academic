\documentclass[usenames, aspectratio=169]{beamer}

\usepackage{amsmath}
\usepackage{braket}
\usepackage{amsfonts}
\usepackage{tikz}
\usepackage{tkz-graph}
\usepackage{tikzpeople}
\usepackage{adjustbox}
\usepackage{subcaption}
\usepackage{svg}
\usepackage{graphicx}
\usepackage{media9}
\usepackage{float}
\usetikzlibrary{calc}
\usepackage{array}
\usepackage{efbox,graphicx}
\usepackage[normalem]{ulem}
\usepackage{verbatim}
\usepackage{ragged2e}
\usepackage{array}
\usepackage[backend=biber,style=alphabetic,sorting=ynt]{biblatex}
%\usepackage{columns}
\addbibresource{./sample.bib} 
  
\efboxsetup{linecolor=Green,linewidth=1.5pt, margin=0pt}

\usetikzlibrary{decorations.pathreplacing}
\usetikzlibrary{shapes}
\theoremstyle{claim}
\newtheorem{claim}[theorem]{Claim}%
\theoremstyle{remark}
\newtheorem{remark}[theorem]{Remark}%

\newcommand\MemoryLayout[1]{
  \begin{tikzpicture}[scale=0.15]
    \draw[thick](0,0)--++(0,3)node[above]{$0$};
    \foreach \pt/\col/\lab [remember=\pt as \tp (initially 0)] in {#1} {
      \foreach \a [parse=true] in {\tp,...,\pt-1} {
        \draw[fill=\col](-\a, 0) rectangle ++(-1,2);
      }
      \draw[thick](-\pt,0)--++(0,3)node[above]{$\pt$};
      \if\lab\relax\relax\else
        \draw[thick,decorate, decoration={amplitude=1mm}]
        (-\tp,-0.2)--node[below=1mm]{\lab} (-\pt,-0.2);
      \fi
    }
  \end{tikzpicture}
}


\newcommand{\pslsq}[4]{
\begin{frame}
    \frametitle{#1} 
    \includegraphics[width=.7\linewidth]{#3}
    #4  
  \end{frame}
}

\newcommand{\psls}[4]{
  \begin{frame}
    \frametitle{#1} 
    \begin{columns}[t]
      \begin{column}{.48\textwidth}
        #4
      \end{column}
      \begin{column}{.52\textwidth}
        \adjincludegraphics[width=.98\linewidth, valign=t]{#3}
      \end{column} 
    \end{columns}
  \end{frame}
}
\newcommand{\commentt}[1]{\textcolor{blue}{ \textbf{[COMMENT]} #1}}
\newcommand{\ctt}[1]{\commentt{#1}}
\newcommand{\prb}[1]{ \mathbf{Pr} \left[ #1 \right]}
\newcommand{\prbm}[2]{ \mathbf{Pr}_{ #2 }\left[ #1 \right]}
\newcommand{\prbc}[3]{ \mathbf{Pr}_{ #2 }\left[ #1 \right | #3]}
\newcommand{\prbcprb}[3]{ \prbc{#2}{#1}{#3} \cdot \prb{#3} } 
\newcommand{\expp}[1]{ \mathbf{E} \left[ {#1} \right]}
\newcommand{\onotation}[1]{\(\mathcal{O} \left( {#1}  \right) \)}
\newcommand{\ona}[1]{\onotation{#1}}
\newcommand{\PSI}{{\ket{\psi}}}
\newcommand{\xij} { X_{ij} } 
\DeclareMathOperator{\Ima}{Im}
%\newcommand{\LESn}{\ket{\psi_n}}
%\newcommand{\LESa}{\ket{\phi_n}}
%\newcommand{\LESs}{\frac{1}{\sqrt{n}}\sum_{i}{\ket{\left(0^{i}10^{n-i}\right)^{n}}}}
%\newcommand{\Hn}{\mathcal{H}_{n}}
%\newcommand{\Ep}{\frac{1}{\sqrt{2^n}}\sum^{2^n}_{x}{ \ket{xx}}}
%\newcommand{\HON}{\ket{\psi_{\text{honest}}}}
%\newcommand{\Lemma}{\paragraph{Lemma.}}
\newcommand{\Cpa}{[n, \rho n, \delta n]}
%\setlength{\columnsep}{0.6cm}
\newcommand{\Jvv}{ \bar{J_{v}} } 
\newcommand{\Cvv}{ \tilde{C_{v}} } 

\newcommand{\Gz}{ G_{z}^{\delta} } 
\newcommand{ \Tann } {  \mathcal{T}\left( G, C_0 \right) }
\newcommand{\ireducable}{ireducable \hyperref[ire]{[\ref{ire}]} }
\newcommand{\cutUU}{E(U_{-1} \bigcup U_{+1} ,U)} 
\newcommand{\wcutUU}{w\left( E(U_{-1} \bigcup U_{+1} ,U)  \right)}
\newcommand{\testgo}{  \mathcal{T}\left(J, q , C_{0}\right) } 

\newcommand{\duC}{\left( C_{A}^{\perp}\otimes C_{B}^{\perp} \right)^{\perp}}
\newcommand{\duduC}{\left( C_{A}\otimes C_{B}\right)^{\perp}}
  




\usepackage{sagetex}
%\usepackage{libertine}
%\usepackage{emerald}
%\usepackage[T1]{fontenc}
\usetheme[progressbar=frametitle]{metropolis}
\setbeamercolor{block title}{use=structure,fg=white,bg=structure.fg!75!black}
\setbeamercolor{block body}{parent=normal text,use=block title,bg=block title.bg!10!bg}

%\usetheme{EastLansing}
\title[From classical to good quantum LDPC codes.] % (optional, only for long titles)
{From classical to good quantum LDPC codes.}

\subtitle{  }
\author[D.~Ponarovsky] % (optional, for multiple authors)
	{D.~Ponarovsky\inst{1}}

\institute[HUJI] % (optional)
{  Faculty of Computer Science\newline
  Hebrew University of Jerusalem
}
\date[2023] % (optional)
{Master-Exam-Huji.}
\subject{Understanding Quantumness And Testability.}

\begin{document}
\input{sageutil.py}

\tikzset{
    LabelStyle/.append style = {  minimum width = 2em},
    VertexStyle/.append style = { inner sep=5pt,
        font = \Large\bfseries},
    EdgeStyle/.append style = {->} % added blue
}

\begin{frame}
  \maketitle
\end{frame}
%\pslsq{Today.}{0.3}{controller.png}{}
%\pslsq{Today.}{0.5}{controller-2-out.png}{}

\begin{frame}
  \frametitle{ Today. }
  \begin{itemize}
    \item<1-> Brif Review of Coding. \uncover<2->{Tanner and Expander codes. }
    \item<3-> Quantum Error Correction Codes. 
    \item<4->Good Classical Locally Testabile Codes and Good Qauntum LDPC.
  \end{itemize} 
\end{frame}

\begin{frame}
  \frametitle{Classical Vs Quantum Encoding.}

  \begin{columns}[t]
    \begin{column}{0.5\textwidth}
        Classical:
      \begin{center} 
        

        \scalebox{0.7}{
         \begin{tikzpicture}
        \node[name=b, bob,monitor,minimum size=1cm,xshift=-7.2cm]{};
        \node[name= a, alice,monitor, mirrored,minimum size=1cm]{};
        \node (C) at (-2,0) {};
        \draw[ -> ]  (-6,0) to (-1.5,0); 
        \alt<5->{  \node (D) at (-4,1) { $\ket{\textcolor{red}{0}}$ } ;}{\alt<2->{ \node (D) at (-4,1) { $\textcolor{red}{0}$ } ; }{\node (D) at (-4,1) { $1$ } ; } }
          \alt<5->{  \node (D) at (-3.8,1.8) { $\ket{\textcolor{red}{0}11}$ } ; }{\uncover<3->{ \node (D) at (-3.8,1.8) { $\textcolor{red}{0}11$ } ; } }
        %\uncover<2->{\node (D) at (-4,1) { $110101$ } ;  
      \end{tikzpicture}
    }
    \end{center}
    \end{column}

    \begin{column}{0.5\textwidth}
\uncover<6->{
        Quantum:
      \begin{center} 
        

        \scalebox{0.7}{
         \begin{tikzpicture}
        \node[name=b, bob,monitor,minimum size=1cm,xshift=-7.2cm]{};
        \node[name= a, alice,monitor, mirrored,minimum size=1cm]{};
        \node (C) at (-2,0) {};
        \draw[ -> ]  (-6,0) to (-1.5,0); 
        \alt<7->{ \node (D) at (-4,1) { $\frac{1}{\sqrt{2}}\left(\ket{0} \textcolor{red}{-} \ket{1}\right)$ } ; }{\node (D) at (-4,1) { $\frac{1}{\sqrt{2}}\left(\ket{0} + \ket{1}\right)$ } ; }
        \uncover<8->{ \node (D) at (-3.2,1.8) { $ \frac{1}{\sqrt{2}}\left( \ket{000} \textcolor{red}{-} \ket{111}\right)$ } ; }  
        %\uncover<2->{\node (D) at (-4,1) { $110101$ } ;  
      \end{tikzpicture}    }
    \end{center}
  }
    \end{column}
  \end{columns}


\end{frame}

\begin{frame}
  \frametitle{Good Classical LDPC Code.}
\end{frame}
\begin{frame}
  \frametitle{Good Classical LDPC Code.}
\end{frame}
\begin{frame}
  \frametitle{Good Classical LDPC Code.}
\end{frame}
\begin{frame}
  \frametitle{Good Classical LDPC Code.}
\end{frame}


 \begin{frame}
   \frametitle{Quantum Encoding.}
\end{frame}
 \begin{frame}
   \frametitle{Quantum Encoding.}
\end{frame}
 \begin{frame}
   \frametitle{Quantum Encoding.}
\end{frame}

\begin{frame}
  \frametitle{ Idea I - (Uncertainty) Clouds as States. }
\end{frame}
\begin{frame}
  \frametitle{ CSS Code.  }
\end{frame}

\begin{frame}
  \frametitle{ 'Idea II' - Tanner Checks are 'Too Much' Interdependence.}
\end{frame}

\begin{frame}
  \frametitle{ 'Idea III' - Impossibility of Both $C_{X},C_{Z}$ being Good.}
\end{frame}

\begin{frame}
  \frametitle{ Quantum Tanner Code Construction.}
\end{frame}

\begin{frame}
  \frametitle{ Proving Strategy. }
\end{frame}

%\begin{frame}
%  \frametitle{   }
%\end{frame}



\end{document}
