\documentclass[usenames, aspectratio=169]{beamer}

\usepackage{amsmath}
\usepackage{braket}
\usepackage{amsfonts}
\usepackage{tikz}
\usepackage{tkz-graph}
\usepackage{tikzpeople}
\usepackage{adjustbox}
\usepackage{subcaption}
\usepackage{svg}
\usepackage{graphicx}
\usepackage{media9}
\usepackage{float}
\usetikzlibrary{calc}
\usepackage{array}
\usepackage{efbox,graphicx}
\usepackage[normalem]{ulem}
\usepackage{verbatim}
\usepackage{ragged2e}
\usepackage{array}
\usepackage[backend=biber,style=alphabetic,sorting=ynt]{biblatex}
%\usepackage{columns}
\addbibresource{./sample.bib} 
  
\efboxsetup{linecolor=Green,linewidth=1.5pt, margin=0pt}

\usetikzlibrary{decorations.pathreplacing}
\usetikzlibrary{shapes}
\theoremstyle{claim}
\newtheorem{claim}[theorem]{Claim}%
\theoremstyle{remark}
\newtheorem{remark}[theorem]{Remark}%

\newcommand\MemoryLayout[1]{
  \begin{tikzpicture}[scale=0.15]
    \draw[thick](0,0)--++(0,3)node[above]{$0$};
    \foreach \pt/\col/\lab [remember=\pt as \tp (initially 0)] in {#1} {
      \foreach \a [parse=true] in {\tp,...,\pt-1} {
        \draw[fill=\col](-\a, 0) rectangle ++(-1,2);
      }
      \draw[thick](-\pt,0)--++(0,3)node[above]{$\pt$};
      \if\lab\relax\relax\else
        \draw[thick,decorate, decoration={amplitude=1mm}]
        (-\tp,-0.2)--node[below=1mm]{\lab} (-\pt,-0.2);
      \fi
    }
  \end{tikzpicture}
}


\newcommand{\pslsq}[4]{
\begin{frame}
    \frametitle{#1} 
    \includegraphics[width=.7\linewidth]{#3}
    #4  
  \end{frame}
}

\newcommand{\psls}[4]{
  \begin{frame}
    \frametitle{#1} 
    \begin{columns}[t]
      \begin{column}{.48\textwidth}
        #4
      \end{column}
      \begin{column}{.52\textwidth}
        \adjincludegraphics[width=.98\linewidth, valign=t]{#3}
      \end{column} 
    \end{columns}
  \end{frame}
}
\newcommand{\commentt}[1]{\textcolor{blue}{ \textbf{[COMMENT]} #1}}
\newcommand{\ctt}[1]{\commentt{#1}}
\newcommand{\prb}[1]{ \mathbf{Pr} \left[ #1 \right]}
\newcommand{\prbm}[2]{ \mathbf{Pr}_{ #2 }\left[ #1 \right]}
\newcommand{\prbc}[3]{ \mathbf{Pr}_{ #2 }\left[ #1 \right | #3]}
\newcommand{\prbcprb}[3]{ \prbc{#2}{#1}{#3} \cdot \prb{#3} } 
\newcommand{\expp}[1]{ \mathbf{E} \left[ {#1} \right]}
\newcommand{\onotation}[1]{\(\mathcal{O} \left( {#1}  \right) \)}
\newcommand{\ona}[1]{\onotation{#1}}
\newcommand{\PSI}{{\ket{\psi}}}
\newcommand{\xij} { X_{ij} } 
\DeclareMathOperator{\Ima}{Im}
%\newcommand{\LESn}{\ket{\psi_n}}
%\newcommand{\LESa}{\ket{\phi_n}}
%\newcommand{\LESs}{\frac{1}{\sqrt{n}}\sum_{i}{\ket{\left(0^{i}10^{n-i}\right)^{n}}}}
%\newcommand{\Hn}{\mathcal{H}_{n}}
%\newcommand{\Ep}{\frac{1}{\sqrt{2^n}}\sum^{2^n}_{x}{ \ket{xx}}}
%\newcommand{\HON}{\ket{\psi_{\text{honest}}}}
%\newcommand{\Lemma}{\paragraph{Lemma.}}
\newcommand{\Cpa}{[n, \rho n, \delta n]}
%\setlength{\columnsep}{0.6cm}
\newcommand{\Jvv}{ \bar{J_{v}} } 
\newcommand{\Cvv}{ \tilde{C_{v}} } 

\newcommand{\Gz}{ G_{z}^{\delta} } 
\newcommand{ \Tann } {  \mathcal{T}\left( G, C_0 \right) }
\newcommand{\ireducable}{ireducable \hyperref[ire]{[\ref{ire}]} }
\newcommand{\cutUU}{E(U_{-1} \bigcup U_{+1} ,U)} 
\newcommand{\wcutUU}{w\left( E(U_{-1} \bigcup U_{+1} ,U)  \right)}
\newcommand{\testgo}{  \mathcal{T}\left(J, q , C_{0}\right) } 

\newcommand{\duC}{\left( C_{A}^{\perp}\otimes C_{B}^{\perp} \right)^{\perp}}
\newcommand{\duduC}{\left( C_{A}\otimes C_{B}\right)^{\perp}}
  




\usepackage{sagetex}
%\usepackage{libertine}
%\usepackage{emerald}
%\usepackage[T1]{fontenc}
\usetheme[progressbar=frametitle]{metropolis}
\setbeamercolor{block title}{use=structure,fg=white,bg=structure.fg!75!black}
\setbeamercolor{block body}{parent=normal text,use=block title,bg=block title.bg!10!bg}

%\usetheme{EastLansing}
\title[From classical to good quantum LDPC codes.] % (optional, only for long titles)
{From classical to good quantum LDPC codes.}

\subtitle{  }
\author[D.~Ponarovsky] % (optional, for multiple authors)
	{D.~Ponarovsky\inst{1}}

\institute[HUJI] % (optional)
{  Faculty of Computer Science\newline
  Hebrew University of Jerusalem
}
\date[2023] % (optional)
{Master-Exam-Huji.}
\subject{Understanding Quantumness And Testability.}

\begin{document}
\input{sageutil.py}

\tikzset{
    LabelStyle/.append style = {  minimum width = 2em},
    VertexStyle/.append style = { inner sep=5pt,
        font = \Large\bfseries},
    EdgeStyle/.append style = {->} % added blue
}

\begin{frame}
  \maketitle
\end{frame}
%\pslsq{Today.}{0.3}{controller.png}{}
%\pslsq{Today.}{0.5}{controller-2-out.png}{}

\begin{frame}
  \frametitle{ Today. }
  \begin{itemize}
    \item<1-> Brif Review of Coding. \uncover<2->{Tanner and Expander codes. }
    \item<3-> Quantum Error Correction Codes. 
    \item<4->Good Classical Locally Testabile Codes and Good Qauntum LDPC.
  \end{itemize} 
\end{frame}

\begin{frame}
  \frametitle{Classical Vs Quantum Encoding.}

  \begin{columns}[t]
    \begin{column}{0.5\textwidth}
        Classical:
      \begin{center} 
        \scalebox{0.7}{
         \begin{tikzpicture}
        \node[name=b, bob,monitor,minimum size=1cm,xshift=-7.2cm]{};
        \node[name= a, alice,monitor, mirrored,minimum size=1cm]{};
        \node (C) at (-2,0) {};
        \draw[ -> ]  (-6,0) to (-1.5,0); 
        \alt<5->{  \node (D) at (-4,1) { $\ket{\textcolor{red}{0}}$ } ;}{\alt<2->{ \node (D) at (-4,1) { $\textcolor{red}{0}$ } ; }{\node (D) at (-4,1) { $1$ } ; } }
          \alt<5->{  \node (D) at (-3.8,1.8) { $\ket{\textcolor{red}{0}11}$ } ; }{\uncover<3->{ \node (D) at (-3.8,1.8) { $\textcolor{red}{0}11$ } ; } }
        %\uncover<2->{\node (D) at (-4,1) { $110101$ } ;  
      \end{tikzpicture}
    }
    \end{center}
    \end{column}

    \begin{column}{0.5\textwidth}
\uncover<6->{
        Quantum:
      \begin{center} 
        \scalebox{0.7}{
         \begin{tikzpicture}
        \node[name=b, bob,monitor,minimum size=1cm,xshift=-7.2cm]{};
        \node[name= a, alice,monitor, mirrored,minimum size=1cm]{};
        \node (C) at (-2,0) {};
        \draw[ -> ]  (-6,0) to (-1.5,0); 
        \alt<7->{ \node (D) at (-4,1) { $\frac{1}{\sqrt{2}}\left(\ket{0} \textcolor{red}{-} \ket{1}\right)$ } ; }{\node (D) at (-4,1) { $\frac{1}{\sqrt{2}}\left(\ket{0} + \ket{1}\right)$ } ; }
        \uncover<8->{ \node (D) at (-3.6,1.8) { $ \frac{1}{\sqrt{2}}\left( \ket{000} \textcolor{red}{-} \ket{111}\right)$ } ; }  
        %\uncover<2->{\node (D) at (-4,1) { $110101$ } ;  
      \end{tikzpicture}    }
    \end{center}
  }
    \end{column}
  \end{columns}

\uncover<9->{
  \begin{block}{The C.S Questions.}
    In the asymptotic regime, can we encode quantum states in codes robust against many errors, as our original massage grows? And in what costs?    
  \end{block}
}

\end{frame}

\begin{frame}
  \frametitle{Good Classical LDPC Code.}
\begin{definition} 
  Let $n \in \mathbb{N}$ and $\rho, \delta\in \left( 0,1 \right)$. We say that $C$ is a \textbf{binary linear code} with parameters $[n, \rho n, \delta n]$. If $C$ is a subspace of $\mathbb{F}_{2}^{n}$, and the dimension of $C$ is at least $\rho n$ and any pair of distinct elements in $C$ differ in at least $\delta n$ coordinates. We call to the vectors belong to $C$ \textit{codewords}, to $\rho n$ the dimension of the code, and to $\delta n$ the distance of the code.
  \end{definition}
  \uncover<2->{ 
    \begin{definition} 
      A \textbf{family of codes} is an infinite series of codes.. 
    \end{definition}
}
\uncover<3->{
    \begin{definition}
  We will say that a family of codes is a \textbf{good code} if its parameters converge into positive values. 
    \end{definition}
}
\end{frame}
\begin{frame}
  \frametitle{Good Classical LDPC Code.}
  \begin{block}{Parity Check Matrix.}
    Code $C$ is a linear subspace $\Rightarrow$ There is a matrix $H$ such: 
    \begin{equation*}
      \begin{split}
        x \in C \Leftrightarrow Hx = 0 
      \end{split}
    \end{equation*}
    We will call $H$ the parity check matrix. 
  \end{block}

  \begin{definition}
    A codes family will be called LDPC code if weight of any row (col) in $H$ is $O(1)$.
  \end{definition}
\end{frame}
\begin{frame}
  \frametitle{Good Classical LDPC Code.}
  \begin{block}{Example. Repetition code.}
    Let the Repetition code, $[n, 1, n]$ be the mapping $0 \rightarrow 0^{n}$ and $1 \rightarrow 1^{n}$.   
  \end{block}
\end{frame}
\begin{frame}
  \frametitle{Good Classical LDPC Code.}
  Technic for design LDPC families with positive rate. 
 
  \begin{definition} Let $\Gamma$ be a graph and $C_{0}$ be a ``small'' linear code with finate parameters $[\Delta, \rho\Delta, \delta\Delta]$. Let $ C = \mathcal{T}\left( \Gamma, C_{0} \right)$  be all the codewords which, for any vertex $v\in \Gamma$, the local view of $v$ is a codeword of $C_{0}$. We say that $C$ is a \textbf{Tanner code}\label{Tan} of $\Gamma, C_{0}$. Notice that if $C_{0}$ is a binary linear code, So $C$ is.  
  \end{definition}

\end{frame}

\begin{frame}
  \frametitle{Good Classical LDPC Code.}
  Example, the parity code on the Peterson graph.
  \begin{center}
  \scalebox{0.65} {
\begin{tikzpicture}

\Vertex[style={minimum size=0.01cm,shape=circle},NoLabel,x=3.5cm,y=7.0cm]{v0}
\Vertex[style={minimum size=0.01cm,shape=circle},NoLabel,x=0.0cm,y=4.3262cm]{v1}
\Vertex[style={minimum size=0.01cm,shape=circle},NoLabel,x=1.3369cm,y=0.0cm]{v2}
\Vertex[style={minimum size=0.01cm,shape=circle},NoLabel,x=5.6631cm,y=0.0cm]{v3}
\Vertex[style={minimum size=0.01cm,shape=circle},NoLabel,x=7.0cm,y=4.3262cm]{v4}
\Vertex[style={minimum size=0.01cm,shape=circle},NoLabel,x=3.5cm,y=5.0652cm]{v5}
\Vertex[style={minimum size=0.01cm,shape=circle},NoLabel,x=1.75cm,y=3.7284cm]{v6}
\Vertex[style={minimum size=0.01cm,shape=circle},NoLabel,x=2.4184cm,y=1.5652cm]{v7}
\Vertex[style={minimum size=0.01cm,shape=circle},NoLabel,x=4.5816cm,y=1.5652cm]{v8}
\Vertex[style={minimum size=0.01cm,shape=circle},NoLabel,x=5.25cm,y=3.7284cm]{v9}
%
\Edge[lw=0.005cm,labelstyle={pos=0.5},label=\hbox{$0$},](v0)(v1)
\Edge[lw=0.005cm,labelstyle={pos=0.5},label=\hbox{$1$},](v0)(v4)
\Edge[lw=0.005cm,labelstyle={pos=0.5},label=\hbox{$1$},](v0)(v5)
\Edge[lw=0.005cm,labelstyle={pos=0.5},label=\hbox{$0$},](v1)(v2)
\Edge[lw=0.005cm,labelstyle={pos=0.5},label=\hbox{$0$},](v1)(v6)
\Edge[lw=0.005cm,labelstyle={pos=0.5},label=\hbox{$1$},](v2)(v3)
\Edge[lw=0.005cm,labelstyle={pos=0.5},label=\hbox{$1$},](v2)(v7)
\Edge[lw=0.005cm,labelstyle={pos=0.5},label=\hbox{$0$},](v3)(v4)
\Edge[lw=0.005cm,labelstyle={pos=0.5},label=\hbox{$1$},](v3)(v8)
\Edge[lw=0.005cm,labelstyle={pos=0.5},label=\hbox{$1$},](v4)(v9)
\Edge[lw=0.005cm,labelstyle={pos=0.5},label=\hbox{$0$},](v5)(v7)
\Edge[lw=0.005cm,labelstyle={pos=0.5},label=\hbox{$1$},](v5)(v8)
\Edge[lw=0.005cm,labelstyle={pos=0.5},label=\hbox{$0$},](v6)(v8)
\Edge[lw=0.005cm,labelstyle={pos=0.5},label=\hbox{$0$},](v6)(v9)
\Edge[lw=0.005cm,labelstyle={pos=0.5},label=\hbox{$1$},](v7)(v9)
%
\end{tikzpicture} \ \ \ \begin{tikzpicture}
\Vertex[style={minimum size=0.01cm,shape=circle},NoLabel,x=3.5cm,y=7.0cm]{v0}
\Vertex[style={minimum size=0.01cm,shape=circle},NoLabel,x=0.0cm,y=4.3262cm]{v1}
\Vertex[style={minimum size=0.01cm,shape=circle},NoLabel,x=1.3369cm,y=0.0cm]{v2}
\Vertex[style={minimum size=0.01cm,shape=circle},NoLabel,x=5.6631cm,y=0.0cm]{v3}
\Vertex[style={minimum size=0.01cm,shape=circle},NoLabel,x=7.0cm,y=4.3262cm]{v4}
\Vertex[style={minimum size=0.01cm,shape=circle},NoLabel,x=3.5cm,y=5.0652cm]{v5}
\Vertex[style={minimum size=0.01cm,shape=circle},NoLabel,x=1.75cm,y=3.7284cm]{v6}
\Vertex[style={minimum size=0.01cm,shape=circle},NoLabel,x=2.4184cm,y=1.5652cm]{v7}
\Vertex[style={minimum size=0.01cm,shape=circle},NoLabel,x=4.5816cm,y=1.5652cm]{v8}
\Vertex[style={minimum size=0.01cm,shape=circle},NoLabel,x=5.25cm,y=3.7284cm]{v9}
%
\Edge[lw=0.005cm,labelstyle={pos=0.5,},label=\hbox{$0$},](v0)(v1)
\Edge[lw=0.005cm,labelstyle={pos=0.5,},label=\hbox{$0$},](v0)(v4)
\Edge[lw=0.005cm,labelstyle={pos=0.5,},label=\hbox{$0$},](v0)(v5)
\Edge[lw=0.005cm,labelstyle={pos=0.5,},label=\hbox{$1$},](v1)(v2)
\Edge[lw=0.005cm,labelstyle={pos=0.5,},label=\hbox{$1$},](v1)(v6)
\Edge[lw=0.005cm,labelstyle={pos=0.5,},label=\hbox{$0$},](v2)(v3)
\Edge[lw=0.005cm,labelstyle={pos=0.5,},label=\hbox{$1$},](v2)(v7)
\Edge[lw=0.005cm,labelstyle={pos=0.5,},label=\hbox{$1$},](v3)(v4)
\Edge[lw=0.005cm,labelstyle={pos=0.5,},label=\hbox{$1$},](v3)(v8)
\Edge[lw=0.005cm,labelstyle={pos=0.5,},label=\hbox{$1$},](v4)(v9)
\Edge[lw=0.005cm,labelstyle={pos=0.5,},label=\hbox{$0$},](v5)(v7)
\Edge[lw=0.005cm,labelstyle={pos=0.5,},label=\hbox{$0$},](v5)(v8)
\Edge[lw=0.005cm,labelstyle={pos=0.5,},label=\hbox{$1$},](v6)(v8)
\Edge[lw=0.005cm,labelstyle={pos=0.5,},label=\hbox{$0$},](v6)(v9)
\Edge[lw=0.005cm,labelstyle={pos=0.5,},label=\hbox{$1$},](v7)(v9)
%
\end{tikzpicture} 

}
\end{center}
\end{frame}

\begin{frame}
  \frametitle{Good Classical LDPC Code.} 
  
    Another example, the repttion code can be thought as the tanner graph defind by the parity code on the cyle graph.
    \begin{columns}[t]
      \begin{column}{0.4\textwidth}
    \begin{center}
      \scalebox{0.6}{
      \begin{tikzpicture}[scale=1]
      \draw
        (8.0, 0.0) node[shape=circle,draw=black] (0){}
        (7.825, 0.624) node[shape=circle,draw=black] (1){}
        (7.308, 1.22) node[shape=circle,draw=black] (2){}
        (6.472, 1.763) node[shape=circle,draw=black] (3){}
        (5.353, 2.229) node[shape=circle,draw=black] (4){}
        (4.0, 2.598) node[shape=circle,draw=black] (5){}
        (2.472, 2.853) node[shape=circle,draw=black] (6){}
        (0.836, 2.984) node[shape=circle,draw=black] (7){}
        (-0.836, 2.984) node[shape=circle,draw=black] (8){}
        (-2.472, 2.853) node[shape=circle,draw=black] (9){}
        (-4.0, 2.598) node[shape=circle,draw=black] (10){}
        (-5.353, 2.229) node[shape=circle,draw=black] (11){}
        (-6.472, 1.763) node[shape=circle,draw=black] (12){}
        (-7.308, 1.22) node[shape=circle,draw=black] (13){}
        (-7.825, 0.624) node[shape=circle,draw=black] (14){}
        (-8.0, 0.0) node[shape=circle,draw=black] (15){}
        (-7.825, -0.624) node[shape=circle,draw=black] (16){}
        (-7.308, -1.22) node[shape=circle,draw=black] (17){}
        (-6.472, -1.763) node[shape=circle,draw=black] (18){}
        (-5.353, -2.229) node[shape=circle,draw=black] (19){}
        (-4.0, -2.598) node[shape=circle,draw=black] (20){}
        (-2.472, -2.853) node[shape=circle,draw=black] (21){}
        (-0.836, -2.984) node[shape=circle,draw=black] (22){}
        (0.836, -2.984) node[shape=circle,draw=black] (23){}
        (2.472, -2.853) node[shape=circle,draw=black] (24){}
        (4.0, -2.598) node[shape=circle,draw=black] (25){}
        (5.353, -2.229) node[shape=circle,draw=black] (26){}
        (6.472, -1.763) node[shape=circle,draw=black] (27){}
        (7.308, -1.22) node[shape=circle,draw=black] (28){}
        (7.825, -0.624) node[shape=circle,draw=black] (29){};
      \begin{scope}[-,draw opacity=0.5]
        \draw (0) to node[] {$1$} (1);
        \draw (0) to node[] {$1$} (29);
        \draw (1) to node[] {$1$} (2);
        \draw (2) to node[] {$1$} (3);
        \draw (3) to node[] {$1$} (4);
        \draw (4) to node[] {$1$} (5);
        \draw (5) to node[] {$1$} (6);
        \draw (6) to node[] {$1$} (7);
        \draw (7) to node[] {$1$} (8);
        \draw (8) to node[] {$1$} (9);
        \draw (9) to node[] {$1$} (10);
        \draw (10) to node[] {$1$} (11);
        \draw (11) to node[] {$1$} (12);
        \draw (12) to node[] {$1$} (13);
        \draw (13) to node[] {$1$} (14);
        \draw (14) to node[] {$1$} (15);
        \draw (15) to node[] {$1$} (16);
        \draw (16) to node[] {$1$} (17);
        \draw (17) to node[] {$1$} (18);
        \draw (18) to node[] {$1$} (19);
        \draw (19) to node[] {$1$} (20);
        \draw (20) to node[] {$1$} (21);
        \draw (21) to node[] {$1$} (22);
        \draw (22) to node[] {$1$} (23);
        \draw (23) to node[] {$1$} (24);
        \draw (24) to node[] {$1$} (25);
        \draw (25) to node[] {$1$} (26);
        \draw (26) to node[] {$1$} (27);
        \draw (27) to node[] {$1$} (28);
        \draw (28) to node[] {$1$} (29);
      \end{scope}
    \end{tikzpicture}
 
  }
  \end{center}
\end{column}
\begin{column}{0.25\textwidth}
\begin{equation*}
    \begin{split}
     \overbrace{ 
      \begin{bmatrix}
        1 & 1
      \end{bmatrix}
    }^{ \text{ parity check matrix of } C_{0} }
    % \mathbb{F}_{2}^{1 \times 2} 
  \end{split}
\end{equation*}
\begin{center}
\begin{tikzpicture}
    \draw (0,0) circle (6pt);
    \draw[ - ]  (0,0) to (1,1); 
    \draw[ - ]  (0,0) to (1,-1); 
    %\uncover<2->{\node (D) at (-4,1) { $110101$ } ;  
  \end{tikzpicture}
\end{center}
\end{column}
\begin{column}{0.35\textwidth}
  \begin{equation*}
    \begin{split}
      \overbrace{ 
      \begin{bmatrix}
1 & 1 & 0 & 0 & 0 & 0 \\
0 & 1 & 1 & 0 & 0 & 0 \\
0 & 0 & 1 & 1 & 0 & 0 \\
0 & 0 & 0 & 1 & 1 & 0 \\ 
0 & 0 & 0 & 0 & 1 & 1 \\
1 & 0 & 0 & 0 & 0 & 1 
 \end{bmatrix}
 }^{  \substack{ \text{ Parity check matrix of } \mathcal{T} \left( \Gamma, C_{0} \right) \\ \text{  Each row associated with vertex check. }} }
    \end{split}
  \end{equation*}
\end{column}
\end{columns}
  \end{frame}




\begin{frame}
  \frametitle{Good Classical LDPC Code.}
\begin{lemma}
\label{tanrate} Tanner codes have a rate of at least $2\rho - 1$.
\end{lemma}

\uncover<2->{
  \begin{proof}  The dimension of the subspace is bounded by the dimension of the container minus the number of restrictions. So assuming non-degeneration of the small code restrictions, we have that any vertex count exactly $ \left( 1 - \rho  \right)\Delta $ restrictions. Hence, \begin{equation*}
    \begin{split}
      \dim C & \ge \frac{1}{2}n\Delta - \left( 1-\rho \right)\Delta n = \frac{1}{2}n\Delta\left( 2\rho - 1 \right)  
    \end{split}
  \end{equation*} Clearly, any small code with rate $> \frac{1}{2}$ will yield a code with an asymptotically positive rate \end{proof} 
}
\end{frame}

\begin{frame}
  \frametitle{Good Classical LDPC Code.}
  Technic for design LDPC families with positive relative distance. 
  \uncover<2->{
    \begin{definition} Denote by $\lambda$ the second eigenvalue of the adjacency matrix of the $\Delta$-regular graph. For our uses, it will be satisfied to define $\lambda$-Expander as a graph $G = \left( V,E \right)$ such that for any two subsets of vertices $T,S \subset V$, the number of edges between $S$ and $T$ is at most:
  \begin{equation*}
    \begin{split}
      \mid E\left( S,T \right) - \frac{\Delta}{n}|S||T| \mid \le \lambda\sqrt{|S| |T|} 
    \end{split}
  \end{equation*}
\end{definition}
  }
\end{frame}
\begin{frame}
  \frametitle{Good Classical LDPC Code.}
  \begin{lemma} Using $\lambda$-Expander, the Tanner Code defined bit is a good LDPC code.  
  \end{lemma}
  \uncover<2->{
  \begin{proof} Fix a codeword $x \in C$ and denote By $S$ the support of $x$ over the edges. Namely, a vertex $v\in V$ belongs to $S$ if it connects to nonzero edges regarding the assignment by $x$, Assume towards contradiction that $|x| = o\left( n \right)$. And notice that $|S|$ is at most $2|x|$, Then by The Expander Mixining Lemma we have that: 
  \begin{equation*}
    \begin{split}
      \frac{E\left( S,S \right)}{|S|} & \le \frac{\Delta}{n}|S|  + \lambda \\
      & \le_{ n \rightarrow \infty} o\left( 1 \right) + \lambda
    \end{split}
  \end{equation*}
   \end{proof}
}
\end{frame}

\begin{frame}
  \frametitle{Good Classical LDPC Code.}
\end{frame}


 \begin{frame}
   \frametitle{Quantum Encoding.}
\end{frame}
 \begin{frame}
   \frametitle{Quantum Encoding.}
\end{frame}
 \begin{frame}
   \frametitle{Quantum Encoding.}
\end{frame}

\begin{frame}
  \frametitle{ Idea I - (Uncertainty) Clouds as States. }
\end{frame}
\begin{frame}
  \frametitle{ CSS Code.  }
\end{frame}

\begin{frame}
  \frametitle{ 'Idea II' - Tanner Checks are 'Too Much' Interdependence.}
\end{frame}

\begin{frame}
  \frametitle{ 'Idea III' - Impossibility of Both $C_{X},C_{Z}$ being Good.}
\end{frame}

\begin{frame}
  \frametitle{ Quantum Tanner Code Construction.}
\end{frame}

\begin{frame}
  \frametitle{ Proving Strategy. }
\end{frame}

%\begin{frame}
%  \frametitle{   }
%\end{frame}



\end{document}
