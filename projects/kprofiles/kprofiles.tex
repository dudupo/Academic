

%\newcommand*{\ACM}{}%

\ifdefined\ACM

%\documentclass[sigplan,screen]{acmart}
  \documentclass[manuscript,screen,review]{acmart}

\else
  \documentclass{article}
  \usepackage[utf8]{inputenc}
\usepackage[a4paper, total={6in, 9in}]{geometry}
\usepackage{braket}
\usepackage{xcolor}
\usepackage{amsmath}
\usepackage{amsfonts}
\usepackage{amsthm}
\usepackage{amssymb}
%\usepackage[ocgcolorlinks]{hyperref}
\usepackage{hyperref}
%\usepackage{hyperref,xcolor}
%\usepackage[ocgcolorlinks]{ocgx2}
\usepackage{cleveref}
\usepackage{graphicx}
\usepackage{svg}
\usepackage{float}
\usepackage{tikz}
\usetikzlibrary{patterns, shapes.arrows}
\usepackage{adjustbox}
%\usepackage{tikz-network}
\usepackage{tkz-graph}
\usepackage{tkz-berge}
\usepackage[linesnumbered]{algorithm2e}
\usepackage{multicol}
\usepackage[backend=biber,style=alphabetic,sorting=ynt]{biblatex}
%\usepackage{xcolor}
%\usepackage{tkz-berge}
%\usepackage{tkz-graph}
\usepackage{pgfplots}
\usepackage{sagetex}
\usepackage{setspace}
\usepackage{etoc}
%\usepackage{wrapfig}
\usepackage{pgfgantt}
\DeclareUnicodeCharacter{2212}{−}
\usepgfplotslibrary{groupplots,dateplot}
\pgfplotsset{compat=newest}

\newtheorem{theorem}{Theorem}
\newtheorem{definition}{Definition}
\newtheorem{example}{Example}
\newtheorem{claim}{Claim}
\newtheorem{fact}{Fact}
\newtheorem{remark}{Remark}
\newtheorem*{theorem*}{Theorem}
\newtheorem{lemma}{Lemma}
\crefname{lemma}{Lemma}{Lemmas}
\hypersetup{colorlinks=true}
% , allcolors=blue,allbordercolors=blue,pdfborderstyle={0 0 1}}
%\hypersetup{pdfborder={2 2 2}}
% pdfpagemode=FullScreen,
% backref 

\newtheorem{problem}{Problem}
\crefname{problem}{Problem}{Problems}

\DeclareMathOperator{\Ima}{Im}


  \addbibresource{./sample.bib} 

\fi

\begin{document}

\newcommand{\commentt}[1]{\textcolor{blue}{ \textbf{[COMMENT]} #1}}
\newcommand{\ctt}[1]{\commentt{#1}}
\newcommand{\prb}[1]{ \mathbf{Pr} \left[ #1 \right]}
\newcommand{\prbm}[2]{ \mathbf{Pr}_{ #2 }\left[ #1 \right]}
\newcommand{\prbc}[3]{ \mathbf{Pr}_{ #2 }\left[ #1 \right | #3]}
\newcommand{\prbcprb}[3]{ \prbc{#2}{#1}{#3} \cdot \prb{#3} } 
\newcommand{\expp}[1]{ \mathbf{E} \left[ {#1} \right]}
\newcommand{\onotation}[1]{\(\mathcal{O} \left( {#1}  \right) \)}
\newcommand{\ona}[1]{\onotation{#1}}
\newcommand{\PSI}{{\ket{\psi}}}
\newcommand{\xij} { X_{ij} } 
\DeclareMathOperator{\Ima}{Im}
%\newcommand{\LESn}{\ket{\psi_n}}
%\newcommand{\LESa}{\ket{\phi_n}}
%\newcommand{\LESs}{\frac{1}{\sqrt{n}}\sum_{i}{\ket{\left(0^{i}10^{n-i}\right)^{n}}}}
%\newcommand{\Hn}{\mathcal{H}_{n}}
%\newcommand{\Ep}{\frac{1}{\sqrt{2^n}}\sum^{2^n}_{x}{ \ket{xx}}}
%\newcommand{\HON}{\ket{\psi_{\text{honest}}}}
%\newcommand{\Lemma}{\paragraph{Lemma.}}
\newcommand{\Cpa}{[n, \rho n, \delta n]}
%\setlength{\columnsep}{0.6cm}
\newcommand{\Jvv}{ \bar{J_{v}} } 
\newcommand{\Cvv}{ \tilde{C_{v}} } 

\newcommand{\Gz}{ G_{z}^{\delta} } 
\newcommand{ \Tann } {  \mathcal{T}\left( G, C_0 \right) }
\newcommand{\ireducable}{ireducable \hyperref[ire]{[\ref{ire}]} }
\newcommand{\cutUU}{E(U_{-1} \bigcup U_{+1} ,U)} 
\newcommand{\wcutUU}{w\left( E(U_{-1} \bigcup U_{+1} ,U)  \right)}
\newcommand{\testgo}{  \mathcal{T}\left(J, q , C_{0}\right) } 

\newcommand{\duC}{\left( C_{A}^{\perp}\otimes C_{B}^{\perp} \right)^{\perp}}
\newcommand{\duduC}{\left( C_{A}\otimes C_{B}\right)^{\perp}}
  





\title{Short Note On The Kprofile Problem.} 
\author{David Ponarovsky}
%\author{Noa Viner, David Ponarovsky}

\ifdefined\ACM
  \affiliation{%
    \institution{The Th{\o}rv{\"a}ld Group}
    \streetaddress{1 Th{\o}rv{\"a}ld Circle}
    \city{Hekla}
  \country{Iceland}}
  \email{larst@affiliation.org}
\else
  \maketitle
\fi
%
\abstract{We propose an alternative simple construction of good LTC codes. In contrast to previews, constructions made by \cite{Dinur}, \cite{leverrier2022quantum} and \cite{Pavel}, our construction doesn't require spicel properties of the small codes, such as $w$-robustness and $p$-resistance for puncturing.  
} 


\ifdefined\ACM
  \maketitle
\fi

%\begin{multicols*}{2}
% \section{Preambles}

Localy Testable Codes, or LTC, are error correction codes such that verfining a uinformly random cchoosen check, would be enough to detect any error with probability proportional to it's size. Bisdes the clear computional adventage they offer, they took roles at the eriler PCP proofs.  
  In this work, we propose a new construction for good LTC codes, which also have a good testability parameter. In the sense   Our proof also indirectly answers the following question. Why most of the good LDPC codes are known to be bad in terms of detecting errors? In other words, It seems that for most of them, there exist strings that are very far from being in the code and, meanwhile, fail to satisfy only a small number of restrictions.
  While the previous LDPC constructions focused on ensuring that the yielded code would have a good rate and distance parameters, our construction enforces the restrictions collection to have a nontrivial fraction of degeneration. That is, removing a single restriction will not change the code, as any restriction is linearly dependent on the others.




%\section{Introduction.}
.. bla bla bla.. bla bla ..     
\definition[General Entanglement State]{ We say that $\PSI$ is general entanglement \label{def:gEnt} if .. }

\definition[Local-Measure-Circuit] { We say that a quantum circuit $C$ is a local measure circuit \label{def:lmc} if it's can be described as a decomposition of poly classical circuit and a constant depth quantum circuit which contains only 1-qubit gates and measurements. 

We would think about local measure circuits as chip circuits. }

\definition[$p_{0}-\Delta$ Fault Tolerance Circuit]{ We say that $\mathcal{C}$ is $p_{0}-\Delta$ fault tolerance \label{def:gft} presentation of abstract circuit $C$ if there exists a local measure circuit $C_{0}$ \ref{lmc} such it's grunted that for noise $p < p_{0}$ $\mathcal{C}$ compute $C$ w.h.p,
And in addition, if $p \in \left( p_{0}, p_{0} + \varepsilon \right)$ then by applying a $C_{0}$ on $\mathcal{C}$ output yields a general entanglement state \label{def:gEnt}}       

\ctt{We would like to add a complexity parameter for the above definition, for example, ``a general entanglement state over more than $\frac{1}{5}$ of the qubits.}  


\newcommand{\precom}[3]{ \braket{#1,#2,#3} }
\newcommand{\ora}{ \mathcal{O} }

%\section{The Problem.} Given a permutation $\sigma \in S_{n}$ one can map each of the her subsequent at length $k$ to a permutation in $S_{k}$. The histogram obtained by counting how many subsequences are mapped to permutation in $S_{k}$ calls the k-profile of $\sigma$. We study the problem in the setting on which one is allowed to perform preprocessing independent on $\sigma$.        
%
%\subsection{Trivial Example.} Suppose that we don't limit the preprocessing running time, then one can just compute for head the k-profile of any of the permutations in $S_{n}$, as the size of $S_{n}$ equals $n!$ we have that the depth of the binary tree stores the answer is tightly $\Theta(n\log n)$. We denote it by $\precom{n!}{n!}{n \log n}$ for $n!$ memory and time preprocessing and $n \log n$ for query time cost.
%
%\subsection{Quantum.} Assume that we have an oracle $\ora$ that compute for any $k$ distinct numbers the matched permutation in $S_{k}$.
%
%\paragraph{Preprocessing.} In the preprocessing stage we compute classically the $n$ length binary strings contain at weight exactly $k$, we can do and store it at $\Theta(n^{k}) \cdot n$ time and memory. Denote by $M$ the classically gate which on $x \in \{0,1\}^{\Theta\left( \log(n^k) \right)}$ return the $n$ length bit string at weight $k$  corresponds to $x$. In addition denote by $\sigma_{M(x)}$ the restriction of $\sigma$ on the non-zero bits of $M\left( x \right)$.    
%
%Recall that for any classical circuit $C : x \mapsto C(x) $ one can construct a traversal circuit $C : \ket{x}\ket{y} \mapsto \ket{x}\ket{y \oplus C(x)}$. From now on, we refer to $\ora$ and $M$ as quantum circuits.   
%
%\paragraph{Query.} For query we prepare $ \Theta (\log(n^k) )$ qubits, and apply them $H^{\log(n^k)}$ to obtain a uniform super position over all the strings at length $\log(n^k)$.  Now we associate each of the strings with a $k$ length subsequent of $\sigma$ by taking the coordinates $\sigma_{i}$ such that digit $i$ of string is not $0$, namely: 
%
%\begin{equation*}
%  \begin{split}
%    \sum_{x\in \{0,1\}^{*}}{\ket{x}} \overbrace{\mapsto}^M \sum_{x\in \{0,1\}^{*}}{\ket{x}\ket{M(x)}} \mapsto  \sum_{x\in \{0,1\}^{*}}{\ket{x}\ket{M(x)}\ket{\sigma_{M(x)} }}
%  \end{split}
%\end{equation*}
%
%When the feaching of $ \sigma_{i_0},\sigma_{i_1},\sigma_{i_2},\cdots \sigma_{i_k}$ is done by  taking the Toffoli gate on $x_{i},\sigma_{i}, (0^{*})_{i}$. Noitce that that is the only operator which is not in the Clifford group.  We finish by computing $f$ and measure.
%
%\begin{equation*}
%  \begin{split}
%    \sum_{x\in \{0,1\}^{*}}{\ket{x}\ket{M(x)}\ket{\sigma_{M(x)} }}\overbrace{\mapsto}^f & \sum_{x\in \{0,1\}^{*}}{\ket{x}\ket{M(x)}\ket{\sigma_{M(x)} }\ket{f\left( \sigma_{M(x)}  \right) } } \\ 
%    =  &  \sum_{ \tau \in S_{k}}{ \left(\sum{\ket{junk}} \right) \ket{\tau}}
%  \end{split}
%\end{equation*}
%
%In fact we show a reduction from the kprofile problem to  estimation of classic dice with $|S_{k}|= k!$ faces. So if it suffices to estimate up to precision depend only on $k$, we obtained a query time which only linear at $n$ (and exponential on $k$). 
%

\section{The Problem.} Given a permutation $\sigma \in S_{n}$, one can map each of its subsequences of length $k$ to a permutation in $S_{k}$. The histogram obtained by counting how many subsequences are mapped to permutations in $S_{k}$ is called the $k$-profile of $\sigma$. We study the problem in the setting in which one is allowed to perform preprocessing independent of $\sigma$.        

\subsection{Trivial Example.} Suppose that we do not limit the preprocessing running time, then one can just compute ahead of time the $k$-profile of any of the permutations in $S_{n}$. As the size of $S_{n}$ equals $n!$, we have that the depth of the binary tree storing the answer is tightly $\Theta(n\log n)$. We denote it by $\precom{n!}{n!}{n \log n}$ for $n!$ memory and time preprocessing and $n \log n$ for query time cost.

\subsection{Quantum.} Assume that we have an oracle $\ora$ that computes for any $k$ distinct numbers the matched permutation in $S_{k}$.

\paragraph{Preprocessing.} In the preprocessing stage, we compute classically the $n$ length binary strings containing a weight of exactly $k$. We can do and store this at $\Theta(n^{k}) \cdot n$ time and memory. Denote by $M$ the classical gate which, on $x \in \{0,1\}^{\Theta\left( \log(n^k) \right)}$, returns the $n$ length bit string of weight $k$ that corresponds to $x$. In addition, denote by $\sigma_{M(x)}$ the restriction of $\sigma$ on the non-zero bits of $M\left( x \right)$.    

Recall that for any classical circuit $C : x \mapsto C(x) $ one can construct a traversal circuit $C : \ket{x}\ket{y} \mapsto \ket{x}\ket{y \oplus C(x)}$. From now on, we refer to $\ora$ and $M$ as quantum circuits.   

\paragraph{Query.} For the query, we prepare $\Theta (\log(n^k) )$ qubits, and apply $H^{\log(n^k)}$ to them to obtain a uniform superposition over all the strings of length $\log(n^k)$. Now we associate each of the strings with a $k$ length subsequence of $\sigma$ by taking the coordinates $\sigma_{i}$ such that the digit $i$ of the string is not $0$, namely: 

\begin{equation*}
  \begin{split}
    \sum_{x\in \{0,1\}^{*}}{\ket{x}} \overbrace{\mapsto}^M \sum_{x\in \{0,1\}^{*}}{\ket{x}\ket{M(x)}} \mapsto  \sum_{x\in \{0,1\}^{*}}{\ket{x}\ket{M(x)}\ket{\sigma_{M(x)} }}
  \end{split}
\end{equation*}


When the fetching of $\sigma_{i_0}, \sigma_{i_1}, \sigma_{i_2}, \cdots, \sigma_{i_k}$ is done by taking the Toffoli gate on $x_i, \sigma_i, (0^*)_i$, note that this is the only operator which is not in the Clifford group. We finish by computing $f$ and measuring.

\begin{equation*}
  \begin{split}
    \sum_{x\in \{0,1\}^{*}}{\ket{x}\ket{M(x)}\ket{\sigma_{M(x)} }}\overbrace{\mapsto}^f & \sum_{x\in \{0,1\}^{*}}{\ket{x}\ket{M(x)}\ket{\sigma_{M(x)} }\ket{f\left( \sigma_{M(x)}  \right) } } \\ 
    =  &  \sum_{ \tau \in S_{k}}{ \left(\sum{\ket{junk}} \right) \ket{\tau}}
  \end{split}
\end{equation*}

We show a reduction from the k-profile problem to the estimation of a classical die with $|S_k| = k!$ faces. Thus, if it suffices to estimate up to a precision that depends only on $k$, we obtain a query time that is linear in $n$ (and exponential in $k$). \ctt{ And now I realize that there is no quantum advantage here.} 
%The k-profile of a permutation is a measure of the complexity of the permutation. It is defined as the number of distinct subsequences of length k that appear in the permutation. Computing the k-profile of a permutation is a useful tool for analyzing the structure of the permutation and can be used to identify patterns and symmetries. It can also be used to compare the complexity of different permutations.
%\end{multicols*}{2}




\printbibliography
\end{document}





