

%\newcommand*{\ACM}{}%

\ifdefined\ACM

%\documentclass[sigplan,screen]{acmart}
  \documentclass[manuscript,screen,review]{acmart}

\else
  \documentclass{article}
  \usepackage[utf8]{inputenc}
\usepackage[a4paper, total={6.5in, 10in} ]{geometry}
\usepackage{braket}
\usepackage{xcolor}
\usepackage{amsmath}
\usepackage{amssymb}
\usepackage{amsfonts}
\usepackage{graphicx}
\usepackage{svg}
\usepackage{float}
\usepackage{tikz}
\usetikzlibrary{patterns, shapes.arrows}
\usepackage{adjustbox}
\usepackage{tikz-network}
\usepackage[ruled,lined,linesnumbered]{algorithm2e}
\usepackage{multicol}
\usepackage[backend=biber,style=alphabetic,sorting=ynt]{biblatex}
\usepackage{xcolor}
\usepackage{pgfplots}
\DeclareUnicodeCharacter{2212}{−}
\usepgfplotslibrary{groupplots,dateplot}
\pgfplotsset{compat=newest}



  \addbibresource{./sample.bib} 

\fi

\begin{document}

\input{newcommands}

\title{Short Note On The Kprofile Problem.} 
\author{David Ponarovsky}
%\author{Noa Viner, David Ponarovsky}

\ifdefined\ACM
  \affiliation{%
    \institution{The Th{\o}rv{\"a}ld Group}
    \streetaddress{1 Th{\o}rv{\"a}ld Circle}
    \city{Hekla}
  \country{Iceland}}
  \email{larst@affiliation.org}
\else
  \maketitle
\fi
%\begin{abstract}
  Quantum feasibility hinges on the assumption that the basic gate's noise rate is below a certain threshold. Here we study the behavior of computation models when the noise is slightly greater than that threshold. In particular, We ask if one can design a fault tolerance schema such that if the noise is above the threshold, it is still grunted that the final generated state would have a value. 
\end{abstract}

\ifdefined\ACM
  \maketitle
\fi

%\begin{multicols*}{2}
% \section{Preambles}
  In this work, we propose a new construction for good LDPC codes, which also have a good testability parameter. In the sense that verfining a constant number of random checks, would be enough to detect any error with probability proportional to the error size. In contrast to previews, constructions made by \cite{Dinur}, \cite{leverrier2022quantum} and \cite{Pavel}, our construction doesn't require spicel properties of the small codes, such as $w$-robustness and $p$-resistance for puncturing. 
  
  Our proof also indirectly answers the following question. Why most of the good LDPC codes are known to be bad in terms of detecting errors? In other words, It seems that for most of them, there exist strings that are very far from being in the code and, meanwhile, fail to satisfy only a small number of restrictions.
  While the previous LDPC constructions focused on ensuring that the yielded code would have a good rate and distance parameters, our construction enforces the restrictions collection to have a nontrivial fraction of degeneration. That is, removing a single restriction will not change the code, as any restriction is linearly dependent on the others.




%%Coding theory has emerged by the need to transfer information in noisy communication channels. By embedding a message in higher dimension space, one can guarantee robustness against possible faults. The ratio of the original content length to the passed message \emph{length} is the \emph{rate} of the code, and it measures how consuming our communication protocol is. Furthermore, the \emph{distance} of the code quantifies how many faults the scheme can absorb such that the receiver can recover the original message. We could consider the code as all the strings that satisfy a specified restrictions collection.
%  
%
%  Non-formally, code is good if its distance and rate are scaled linearly in the encoded message length. In practice, one is also interested in implementing those checks efficiently. We say that a code is an LDPC if any bit is involved in a constant number of restrictions, each of which is a linear equation, and if any restriction contains a fixed number of variables.
%
%  Furthermore, finally, another characteristic of the code is its testability, which is the complexity of the number of random checks one should do to negate that a given candidate is in the code. Besides good codes being considered efficient in terms of robustness and overhead, they are also vital components in establishing secure multiparty computation \cite{MultiParty} and have a deep connection to probabilistic proofs.
%
%  First, we state the notations, definitions, and formal theorem in section 2. Then in sections 3 and 4, we review past results and provide their proofs to make this paper self-contained. Readers familiar with the basic concepts of LDPC, Tanner, and Expanders codes construction should consider skipping directly to section 5, in which we provide our proof. 
%

Coding theory has emerged due to the need to transfer information in noisy communication channels. By embedding a message in a higher-dimensional space, one can guarantee robustness against possible faults. The ratio of the original content length to the transmitted message \emph{length} is the \emph{rate} of the code, and it measures how consuming our communication protocol is. Additionally, the \emph{distance} of the code quantifies how many faults the scheme can absorb such that the receiver can recover the original message. We can consider the code as a collection of all strings that satisfy specified restrictions.

Non-formally, a code is good if its distance and rate scale linearly with the encoded message length. In practice, one is also interested in implementing these checks efficiently. We say that a code is an LDPC if any bit is involved in a constant number of restrictions, each of which is a linear equation, and if any restriction contains a fixed number of variables.

Moreover, another characteristic of the code is its testability, which is the complexity of the number of random checks one must do to verify that a given candidate is in the code. Besides being considered efficient in terms of robustness and overhead, good codes are also vital components in establishing secure multiparty computation \cite{MultiParty} and in the proof of the PCP theorem~\cite{PCPoriginal}.

%In Section 2, we state the notations, definitions, and formal theorem. Then, in Sections 3 and 4, we review past results and provide their proofs to make this paper self-contained. Readers familiar with the basic concepts of LDPC, Tanner, and Expanders codes construction may consider skipping directly to Section 5, in which we provide our proof.
%Readers familiar with the basic concepts of LDPC, Tanner, and Expanders codes construction may skip Sections 2, 3, and 4 and proceed directly to Section 5, where we provide our proof.


\newcommand{\precom}[3]{ \braket{#1,#2,#3} }
\newcommand{\ora}{ \mathcal{O} }

%\section{The Problem.} Given a permutation $\sigma \in S_{n}$ one can map each of the her subsequent at length $k$ to a permutation in $S_{k}$. The histogram obtained by counting how many subsequences are mapped to permutation in $S_{k}$ calls the k-profile of $\sigma$. We study the problem in the setting on which one is allowed to perform preprocessing independent on $\sigma$.        
%
%\subsection{Trivial Example.} Suppose that we don't limit the preprocessing running time, then one can just compute for head the k-profile of any of the permutations in $S_{n}$, as the size of $S_{n}$ equals $n!$ we have that the depth of the binary tree stores the answer is tightly $\Theta(n\log n)$. We denote it by $\precom{n!}{n!}{n \log n}$ for $n!$ memory and time preprocessing and $n \log n$ for query time cost.
%
%\subsection{Quantum.} Assume that we have an oracle $\ora$ that compute for any $k$ distinct numbers the matched permutation in $S_{k}$.
%
%\paragraph{Preprocessing.} In the preprocessing stage we compute classically the $n$ length binary strings contain at weight exactly $k$, we can do and store it at $\Theta(n^{k}) \cdot n$ time and memory. Denote by $M$ the classically gate which on $x \in \{0,1\}^{\Theta\left( \log(n^k) \right)}$ return the $n$ length bit string at weight $k$  corresponds to $x$. In addition denote by $\sigma_{M(x)}$ the restriction of $\sigma$ on the non-zero bits of $M\left( x \right)$.    
%
%Recall that for any classical circuit $C : x \mapsto C(x) $ one can construct a traversal circuit $C : \ket{x}\ket{y} \mapsto \ket{x}\ket{y \oplus C(x)}$. From now on, we refer to $\ora$ and $M$ as quantum circuits.   
%
%\paragraph{Query.} For query we prepare $ \Theta (\log(n^k) )$ qubits, and apply them $H^{\log(n^k)}$ to obtain a uniform super position over all the strings at length $\log(n^k)$.  Now we associate each of the strings with a $k$ length subsequent of $\sigma$ by taking the coordinates $\sigma_{i}$ such that digit $i$ of string is not $0$, namely: 
%
%\begin{equation*}
%  \begin{split}
%    \sum_{x\in \{0,1\}^{*}}{\ket{x}} \overbrace{\mapsto}^M \sum_{x\in \{0,1\}^{*}}{\ket{x}\ket{M(x)}} \mapsto  \sum_{x\in \{0,1\}^{*}}{\ket{x}\ket{M(x)}\ket{\sigma_{M(x)} }}
%  \end{split}
%\end{equation*}
%
%When the feaching of $ \sigma_{i_0},\sigma_{i_1},\sigma_{i_2},\cdots \sigma_{i_k}$ is done by  taking the Toffoli gate on $x_{i},\sigma_{i}, (0^{*})_{i}$. Noitce that that is the only operator which is not in the Clifford group.  We finish by computing $f$ and measure.
%
%\begin{equation*}
%  \begin{split}
%    \sum_{x\in \{0,1\}^{*}}{\ket{x}\ket{M(x)}\ket{\sigma_{M(x)} }}\overbrace{\mapsto}^f & \sum_{x\in \{0,1\}^{*}}{\ket{x}\ket{M(x)}\ket{\sigma_{M(x)} }\ket{f\left( \sigma_{M(x)}  \right) } } \\ 
%    =  &  \sum_{ \tau \in S_{k}}{ \left(\sum{\ket{junk}} \right) \ket{\tau}}
%  \end{split}
%\end{equation*}
%
%In fact we show a reduction from the kprofile problem to  estimation of classic dice with $|S_{k}|= k!$ faces. So if it suffices to estimate up to precision depend only on $k$, we obtained a query time which only linear at $n$ (and exponential on $k$). 
%

\section{The Problem.} Given a permutation $\sigma \in S_{n}$, one can map each of its subsequences of length $k$ to a permutation in $S_{k}$. The histogram obtained by counting how many subsequences are mapped to permutations in $S_{k}$ is called the $k$-profile of $\sigma$. We study the problem in the setting in which one is allowed to perform preprocessing independent of $\sigma$.        

\subsection{Trivial Example.} Suppose that we do not limit the preprocessing running time, then one can just compute ahead of time the $k$-profile of any of the permutations in $S_{n}$. As the size of $S_{n}$ equals $n!$, we have that the depth of the binary tree storing the answer is tightly $\Theta(n\log n)$. We denote it by $\precom{n!}{n!}{n \log n}$ for $n!$ memory and time preprocessing and $n \log n$ for query time cost.

\subsection{Quantum.} Assume that we have an oracle $\ora$ that computes for any $k$ distinct numbers the matched permutation in $S_{k}$.

\paragraph{Preprocessing.} In the preprocessing stage, we compute classically the $n$ length binary strings containing a weight of exactly $k$. We can do and store this at $\Theta(n^{k}) \cdot n$ time and memory. Denote by $M$ the classical gate which, on $x \in \{0,1\}^{\Theta\left( \log(n^k) \right)}$, returns the $n$ length bit string of weight $k$ that corresponds to $x$. In addition, denote by $\sigma_{M(x)}$ the restriction of $\sigma$ on the non-zero bits of $M\left( x \right)$.    

Recall that for any classical circuit $C : x \mapsto C(x) $ one can construct a traversal circuit $C : \ket{x}\ket{y} \mapsto \ket{x}\ket{y \oplus C(x)}$. From now on, we refer to $\ora$ and $M$ as quantum circuits.   

\paragraph{Query.} For the query, we prepare $\Theta (\log(n^k) )$ qubits, and apply $H^{\log(n^k)}$ to them to obtain a uniform superposition over all the strings of length $\log(n^k)$. Now we associate each of the strings with a $k$ length subsequence of $\sigma$ by taking the coordinates $\sigma_{i}$ such that the digit $i$ of the string is not $0$, namely: 

\begin{equation*}
  \begin{split}
    \sum_{x\in \{0,1\}^{*}}{\ket{x}} \overbrace{\mapsto}^M \sum_{x\in \{0,1\}^{*}}{\ket{x}\ket{M(x)}} \mapsto  \sum_{x\in \{0,1\}^{*}}{\ket{x}\ket{M(x)}\ket{\sigma_{M(x)} }}
  \end{split}
\end{equation*}


When the fetching of $\sigma_{i_0}, \sigma_{i_1}, \sigma_{i_2}, \cdots, \sigma_{i_k}$ is done by taking the Toffoli gate on $x_i, \sigma_i, (0^*)_i$, note that this is the only operator which is not in the Clifford group. We finish by computing $f$ and measuring.

\begin{equation*}
  \begin{split}
    \sum_{x\in \{0,1\}^{*}}{\ket{x}\ket{M(x)}\ket{\sigma_{M(x)} }}\overbrace{\mapsto}^f & \sum_{x\in \{0,1\}^{*}}{\ket{x}\ket{M(x)}\ket{\sigma_{M(x)} }\ket{f\left( \sigma_{M(x)}  \right) } } \\ 
    =  &  \sum_{ \tau \in S_{k}}{ \left(\sum{\ket{junk}} \right) \ket{\tau}}
  \end{split}
\end{equation*}

We show a reduction from the k-profile problem to the estimation of a classical die with $|S_k| = k!$ faces. Thus, if it suffices to estimate up to a precision that depends only on $k$, we obtain a query time that is linear in $n$ (and exponential in $k$). \ctt{ And now I realize that there is no quantum advantage here.} 
%The k-profile of a permutation is a measure of the complexity of the permutation. It is defined as the number of distinct subsequences of length k that appear in the permutation. Computing the k-profile of a permutation is a useful tool for analyzing the structure of the permutation and can be used to identify patterns and symmetries. It can also be used to compare the complexity of different permutations.
%\end{multicols*}{2}




\printbibliography
\end{document}





