\documentclass{article}

\usepackage[utf8]{inputenc}
\usepackage[a4paper, total={7.5in, 10in} ]{geometry}
\usepackage{braket}
\usepackage{xcolor}
\usepackage{amsmath}
\usepackage{amssymb}
\usepackage{amsfonts}
\usepackage{graphicx}
\usepackage{svg}
\usepackage{float}
\usepackage{tikz}
\usetikzlibrary{patterns,shapes.arrows}
\usepackage{adjustbox}
\usepackage{tikz-network}
\usepackage[ruled,vlined,linesnumbered]{algorithm2e}
\usepackage{multicol}
\usepackage[backend=biber,style=alphabetic,sorting=ynt]{biblatex}
\usepackage{xcolor}
\usepackage{pgfplots}
\DeclareUnicodeCharacter{2212}{−}
\usepgfplotslibrary{groupplots,dateplot}
\pgfplotsset{compat=newest}

\addbibresource{sample.bib} %Import the bibliography file

\newcommand{\commentt}[1]{\textcolor{blue}{ \textbf{[COMMENT]} #1}}
\newcommand{\ctt}[1]{\commentt{#1}}
\newcommand{\prb}[1]{ \mathbf{Pr} \left[ {#1} \right]}
\newcommand{\onotation}[1]{\(\mathcal{O} \left( {#1}  \right) \)}
\newcommand{\ona}[1]{\onotation{#1}}
\newcommand{\PSI}{{\ket{\psi}}}
\newcommand{\LESn}{\ket{\psi_n}}
\newcommand{\LESa}{\ket{\phi_n}}
\newcommand{\LESs}{\frac{1}{\sqrt{n}}\sum_{i}{\ket{\left(0^{i}10^{n-i}\right)^{n}}}}
\newcommand{\Hn}{\mathcal{H}_{n}}
\newcommand{\Ep}{\frac{1}{\sqrt{2^n}}\sum^{2^n}_{x}{ \ket{xx}}}
\newcommand{\HON}{\ket{\psi_{\text{honest}}}}
\newcommand{\Lemma}{\paragraph{Lemma.}}
\newcommand{\PonB}{ \rho + \frac{5}{16}\delta\le \frac{3}{4} + \frac{1}{16} } 
\newcommand{\Cpa}{[n, \rho n, \delta n]}
%\setlength{\columnsep}{0.6cm}

\newcommand{\Gz}{ G_{z}^{\delta} } 
\newcommand{ \Tann } {  \mathcal{T}\left( G, C_0 \right) }
\begin{document}




\title{Simple Almost LTC Good LDPC Codes} 
\author{David Ponarovsky}
\maketitle
\abstract{We propose a new simple construction based on Tanner Codes, which yields a good LDPC code with testability query complexity of $\Theta\left( n^{1-\varepsilon} \right)$ for any $\varepsilon> 0$ .} 
\begin{multicols*}{2}
  \subsection{Quantum Code.}
  \paragraph{Definition.} \ctt{ defintion of the square complex. } 
  \paragraph{Construction.}  Let $x \in C$ a non-reducible codeword. Then it has a linear weight. 
  \paragraph{Proof.} As shown eriler $x$ has represntion as codeword of the disgreement code over the negetive graph: $x = \sum_{u^{-} \in V^{-}}{c_{u^{-}}}$ where $c_{u^{-}} \in C_{A}\otimes C_{B}$. By $x$ been non-reducilbe codeword and by Lemma \ctt{add number} we have that at least linear fruction of the negative vertices contribute a non-trivial codeword.     


  \subsection{ Quantum Codes. }
%
%\paragraph{Upper-Side-Sub-Expander Lemma.} Let $G=(V,E)$ a $\Delta$-regular graph with $\lambda$-algebric expansioin. Consider a subgraph $G^{\prime} \subset G $ such that $ G^{\prime}= \left( U, E^{\prime} \right)$. Then there exist $\lambda^{\prime} \in \left( 0, \lambda \right)$ that for any subset pair of $S, T \subset U$ satesfys: 
%
%\begin{equation*}
%  \begin{split}
%    E_{G^{\prime}}\left(S,T \right) \le \frac{\Delta}{|U|}|S||T| + \lambda^{\prime}\sqrt{|S|||T|}
%  \end{split}
%\end{equation*}
%\paragraph{Proof.} Recall that the first eigenvector of en undireacted graph adggecny matrix is distirubuted as $ x_{v} = deg\left( v \right)$. Define the $J$ matrix to be $J_{u,v} =\frac{1}{2|E^{\prime}|} deg_{G^{\prime}}\left( u \right)$ and observes that: $\left(Jx\right)_{v} = \frac{1}{2|E^{\prime}|}\sum_{u \in U}{deg_{G^{\prime}}\left( v \right)  deg_{G^{\prime}}\left( v \right)} = x_{v} $ Therefore for any normalized vecotrs $x,y$ we have that: $ \braket{x\left( A_{G^{\prime}} - J  \right)y} \le  \lambda^{\prime} \sqrt{\left( |x||y| \right)}$. Where $\lambda^{\prime}$ is the seconed largest eigen value of $A_{G^{\prime}}$. 
%
%Note both $A_{G^{\prime}}, A_{G}$ are symmetric and thefore by the Interlacing principle (\ctt{add cition}) we have that $\lambda^{\prime} < \lambda$. Finaly consider the indectors vecotrs $\chi_{S}, \chi_{T}$ for $S,T$ gettering all yields:     
%\begin{equation*}
%  \begin{split}
%    E\left( S,T \right) &= \braket{\chi_{S} A_{G^{\prime}} \chi_{T} } \le \braket{\chi_{S}J\chi_{T}} + \lambda^{\prime}\sqrt{|S||T|}  \\ 
%    & \le  \frac{1}{2|E^{\prime}|}\sum{\chi_{S_{i}}\chi_{T_{j}}deg(i)} +  \lambda^{\prime}\sqrt{|S||T|}  \\
%    & \le \frac{\Delta}{2|E^{\prime}|}|S||T| +  \lambda^{\prime}\sqrt{|S||T|} \\
%    & \le \frac{\Delta}{|U|}|S||T| +  \lambda^{\prime}\sqrt{|S||T|} 
%  \end{split}
%\end{equation*}
%$\square$

  \paragraph{Draft.} By the Discrete Cheeger's inequality it follows that, 
  \begin{equation*}
    \begin{split}
      \frac{1}{2}\lambda^{\prime} & \le \frac{E_{G^{\prime}}\left(S^{\prime}, \cdot  \right)}{|S|}\le \left( 1-\frac{1}{2}\delta_{1} \right)\Delta \le \frac{1}{2}\delta_{2}\Delta \\
      \Rightarrow |S| & \ge \left( \delta_{2} - \frac{\lambda^{\prime}}{\Delta}  \right)\Delta|T| 
    \end{split}
  \end{equation*}

  \begin{equation*}
    \begin{split}
      \frac{1}{2}\lambda^{\prime} \le \frac{ \sum_{u\sim v }{ \left( x(u)-x(v) \right)^{2}  }}{\sum{x\left( v \right)^{2}}} \le \frac{\left( \beta+\alpha \right)^{2}\left( 1-\frac{1}{2}\delta_{1} \right)\Delta}{\left( \alpha^{2}-\beta^{2} \right) } \frac{|S|}{|T|} 
    \end{split}
  \end{equation*}
  \begin{equation*}
    \begin{split}
      \delta_{2}\Delta|S| & \le \braket{ \chi_{S^{\prime}}   J \chi_{S}    } + \lambda^{\prime}\sqrt{|S^{\prime}||S|} \\ 
      & \le \frac{\left( 1- \frac{1}{2}\delta_{1} \right)\Delta^{2}|S|+|S|^{2}\Delta^{2}}{\frac{1}{2}|T|\Delta} + \lambda^{\prime}\sqrt{|S^{\prime}||S|} \\ 
      \Rightarrow |S| & \ge \frac{|T|}{2}\left( \delta_{2} - \left( 1 - \frac{1}{2}\delta_{1} \right)-\frac{\lambda^{\prime}}{\Delta} \right)
    \end{split}
  \end{equation*}

  \paragraph{Lemma.} Let $C_{1} , C_{2}$ be Tanner codes over the graph $G$ and small codes $C_{0i} = \Delta[1, \rho_{i}, \delta_{i}]$. Let's define the code $C$ to be all the non-reducible words in the intersection between $C_{1}^{\oplus}$ and $C_{2}$. Then $C$ has linear distance. 
  \paragraph{Proof.} Consider a vaild codeword $x \in C$ and denote by $S$ the support of $x$ on the vertecis which do not suggest a trival codeword. We have seen that the degree of the vertices of $S$ in the indueced subgraph $\left( T, \cdot \right)$ is at least $\frac{1}{2}\delta_{1}\Delta$. Denote by $S^{\prime} \subset T $ the vertices such their neighborhod is also contained in $T$ and consider the subgraph $G^{\prime}=\left( T, E^{\prime} \right)$ obtaind by taking the vertics which suggested non-trival codewords and the edges which are fully supported on those vertices. 


  and therefore the weight of any $v \in S$ upon the edges of the induced graph is at least $\left(\delta_{2} - \left( 1 - \frac{1}{2}\delta_{1} \right)\right)\Delta$. Otherwise there exists a vertex which see less than $\delta_{2}\Delta$ bits. Using the Expander Mixining Lemma we have that: 
  \begin{equation*}
    \begin{split}
      \left( \delta_{2} - \left( 1- \frac{1}{2}\delta_{ 1}  \right) \right)\Delta & \le \frac{E\left( S,S \right)}{|S|} \le \frac{\Delta}{n}|S|^2+ \lambda|S| \\
      |S| & \ge \left( \delta_{2} + \frac{1}{2}\delta_{1} -  1 - \frac{\lambda}{\Delta} \right)n 
    \end{split}
  \end{equation*}
  \ctt{ $ \delta^2 + \frac{1}{2}\delta - 1 > 0 \Rightarrow \delta \in \left( 0, \frac{\sqrt{2} - 1 }{2} \right) $. So in the end it will be fine. } $\square$

  In the following section we will construct a family of complexes on which we will define a pairs of Tanner Codes, evently, they will used to compose a CSS pairs of good quantum codes.  
  \paragraph{Inifinte Family Of Tanner Quantum Codes.} 
  Let $p$ be a prime and $\delta \in \left( 0,1 \right)$. Consider the Cayly graphs obtained by taking uniformly a $c\left( \delta \right)\log n$ generators of the cyclic group at order $p$, denote that set by $S$. It was shown by N.Alon that with high probability that process yield a Graph with $\delta$-algebric expansion. Now, consider the double cover of that graph and denote it by $G = \left( V = V^{+} \cup V^{-},E \right)$. And define the folowing graph denoted by $\Gamma^{\pm} = \left(V^{\pm}, E^{\prime}\right)$:
  \begin{equation*}
    \begin{split}
      \left( \left(u , \pm  \right), \left( v, \pm \right) \right) \in  E^{\prime} \Leftrightarrow \exists a\neq b \in S \ s.t \ abu = v     
    \end{split}
  \end{equation*}
  clearly $|E^{\prime}| =  \frac{1}{2} {|S| \choose 2} |V|$. \ctt{ We need to show expansion, One elgante way is first to pick $\sqrt{\log n }$ elements and then show that they match to expansion generator set.}    
\end{multicols*}
\end{document}
