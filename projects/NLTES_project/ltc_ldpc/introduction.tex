%Coding theory has emerged by the need to transfer information in noisy communication channels. By embedding a message in higher dimension space, one can guarantee robustness against possible faults. The ratio of the original content length to the passed message \emph{length} is the \emph{rate} of the code, and it measures how consuming our communication protocol is. Furthermore, the \emph{distance} of the code quantifies how many faults the scheme can absorb such that the receiver can recover the original message. We could consider the code as all the strings that satisfy a specified restrictions collection.
%  
%
%  Non-formally, code is good if its distance and rate are scaled linearly in the encoded message length. In practice, one is also interested in implementing those checks efficiently. We say that a code is an LDPC if any bit is involved in a constant number of restrictions, each of which is a linear equation, and if any restriction contains a fixed number of variables.
%
%  Furthermore, finally, another characteristic of the code is its testability, which is the complexity of the number of random checks one should do to negate that a given candidate is in the code. Besides good codes being considered efficient in terms of robustness and overhead, they are also vital components in establishing secure multiparty computation \cite{MultiParty} and have a deep connection to probabilistic proofs.
%
%  First, we state the notations, definitions, and formal theorem in section 2. Then in sections 3 and 4, we review past results and provide their proofs to make this paper self-contained. Readers familiar with the basic concepts of LDPC, Tanner, and Expanders codes construction should consider skipping directly to section 5, in which we provide our proof. 
%

Coding theory has emerged due to the need to transfer information in noisy communication channels. By embedding a message in a higher-dimensional space, one can guarantee robustness against possible faults. The ratio of the original content length to the transmitted message \emph{length} is the \emph{rate} of the code, and it measures how consuming our communication protocol is. Additionally, the \emph{distance} of the code quantifies how many faults the scheme can absorb such that the receiver can recover the original message. We can consider the code as a collection of all strings that satisfy specified restrictions.

Non-formally, a code is good if its distance and rate scale linearly with the encoded message length. In practice, one is also interested in implementing these checks efficiently. We say that a code is an LDPC if any bit is involved in a constant number of restrictions, each of which is a linear equation, and if any restriction contains a fixed number of variables.

Moreover, another characteristic of the code is its testability, which is the complexity of the number of random checks one must do to verify that a given candidate is in the code. Besides being considered efficient in terms of robustness and overhead, good codes are also vital components in establishing secure multiparty computation \cite{MultiParty} and have a deep connection to probabilistic proofs.

In Section 2, we state the notations, definitions, and formal theorem. Then, in Sections 3 and 4, we review past results and provide their proofs to make this paper self-contained. Readers familiar with the basic concepts of LDPC, Tanner, and Expanders codes construction may consider skipping directly to Section 5, in which we provide our proof.
Readers familiar with the basic concepts of LDPC, Tanner, and Expanders codes construction may skip Sections 2, 3, and 4 and proceed directly to Section 5, where we provide our proof.
