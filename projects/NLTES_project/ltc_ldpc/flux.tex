
\begin{claim} \label{claim:flu1}  Suppose that $G$ is an expander graph with a second eigenvalue $\lambda$, then For any layer $U$ there exist a layer $U^{\prime}$ such that:
  \begin{equation*}
    \begin{split}
      (1) & \ \ |U^{\prime}| \ge |U| \\
      (2) & \ \ w_{E/E^{\prime}}\left( x|_{U^{\prime}} \right)  \ge\Delta|U^{\prime}|\left( \delta_{0}-\frac{2}{3}-\frac{2\lambda}{\Delta} \right)
    \end{split}
  \end{equation*}
\end{claim} 
  \begin{proof} Consider layer $U$ and denote by $U_{-1}$ and $U_{+1}$ the preceding and the following layers to $U$ in $T$. It follows from the expander mixing lemma that:
  \begin{equation*}
    \begin{split}
      w_{E/E^{\prime}}\left( x|_{U} \right) & \ge \delta_{0}\Delta|U| -\wcutUU \ge \\ 
      & \delta_{0}\Delta|U| - \cutUU \\ 
      &  \delta_{0}\Delta|U| - \Delta\frac{|U||U_{-1}|}{n} - \Delta\frac{|U||U_{+1}|}{n} \\
      & -\lambda\sqrt{|U||U_{-1}|} - \lambda\sqrt{|U||U_{+1}|}
    \end{split}
  \end{equation*}

  \begin{claim} \label{claim:maxu} We can assume that $|U| \ge |U_{-1}|, |U_{+1}|$. \end{claim}
  \begin{proof} Suppose that $|U_{+1}| > |U|$, so we could choose $U$ to be $U_{+1}$. Continuing stepping deeper till we have that $|U| > |U_{+1}|, |U_{-1}|$. Simiraly, if $|U| > |U_{+1}|$ but $|U_{-1}| > |U|$, the we could take steps upward by replacing $U_{-1}$ with $U$. At the end of the process, we will be left with $U$ at a size greater than the initial layer and $|U| > |U_{+1}|, |U_{-1}|$ \end{proof}

  Using ~\cref{claim:maxu}, we have that $\left( |U_{+1}| + |U_{-1}| \right)/n <\frac{2}{3} $ and therefore:
  \begin{equation*}
    \begin{split}
      w_{E/E^{\prime}}\left( x|_{U} \right) & \ge \left( \delta_{0} - \frac{2}{3} - \frac{2\lambda}{\Delta} \right) \Delta |U| \ \   
    \end{split}
  \end{equation*}
\end{proof}
  That immediately yields the following: let $U_{\text{max}} = \text{arg} \max_{U \text{ layer in }  T } |U|  $  then: 
  \begin{equation*}
    \begin{split}
      |x| \ge  w_{E/E^{\prime}}\left( x|_{U_{\text{max}}} \right) \ge \left( \delta_{0} - \frac{2}{3} - \frac{2\lambda}{\Delta} \right)\Delta |U_{\text{max}}|
    \end{split}
  \end{equation*}
  \begin{claim}  \label{claim:flu2}Consider again the maximal layer $U_{\max}$ then: 
  \begin{equation*}
    \begin{split}
      w_{E/E^{\prime}}\left( x \right) \ge \left( \delta_{0} - \frac{|U_{\max}|}{n} - \frac{\lambda}{\Delta} \right) \Delta|T| 
    \end{split}
  \end{equation*}
\end{claim}

  \begin{proof} Similarly to above, now we will bound the flux that all the nodes in $T$ induce over $E/E^{\prime}$. Denote by $U_{0}, U_{1} .. U_{m}$ the layers of $T$ ordered corresponded to their height, thus we obtain: 
  \begin{equation*}
    \begin{split}
      w_{E/E^{\prime}}\left( x \right) & \ge \delta_{0}\Delta|T| - \sum_{i\in [m]}{ w \left( E\left( U_{i}, U_{i+1}  \right) \right)  } \\ 
      \ge & \delta_{0}\Delta|T|  - \sum_{i \in [m]}{ E\left( U_{i}, U_{i+1}  \right)  } \\ 
      \ge & \delta_{0}\Delta|T|  -  \sum_{i \in [m]}{ \frac{\Delta}{n}|U_{i}| |U_{i+1}| + \lambda \sqrt{ |U_{i}| |U_{i+1}|} }\\ 
      \ge & \delta_{0}\Delta|T|  -  \sum_{i \in [m]}{ \frac{\Delta}{n}|U_{i}| |U_{i+1}| + \lambda \frac{ |U_{i}|+ |U_{i+1}|}{2 } }\\ 
      \ge & \delta_{0}\Delta|T|  - \frac{\Delta}{n}|T||U_{\max}| -  \lambda |T| \\ 
      \ge & \left( \delta_{0} - \frac{|U_{\max}| }{n}-  \frac{\lambda}{\Delta} \right) \Delta|T| 
    \end{split}
  \end{equation*}
  \end{proof}

  \begin{claim}  Alternate proof of fulx inequality, which dosn't assume that there is no interference inside the layers. $w\left( E\left( U,U \right) \right) > 0 $. 
  \end{claim}
  \newcommand{\mxU}{U_{\text{max}}}
  \begin{proof}
    Separeate into the following cases, First assume that $ |\mxU| / n   > \frac{1}{3}  $ then we have that the total interference with $\mxU$ layers is at most: 
    \begin{equation*}
      \begin{split}
	& \frac{\Delta|\mxU|\left( n- |\mxU| \right)}{n} + \lambda\sqrt{|\mxU| n } \le \left( 1 - \frac{|\mxU|}{n} + \sqrt{3} \frac{\lambda}{\Delta}  \right) \Delta |\mxU |  \\ 
	& \le \left( \frac{2}{3} + \sqrt{3} \frac{\lambda}{\Delta}  \right) \Delta |\mxU |  
      \end{split}
    \end{equation*}
    And therefore we have that the flux induced by $\mxU$ is at least: 
    \begin{equation*}
      \begin{split}
	\left( \delta_{0}\Delta -  \frac{2}{3} + \sqrt{3} \frac{\lambda}{\Delta}  \right)\Delta|\mxU|
      \end{split}
    \end{equation*}
   
    So it lefts to consider the case in which for every layer it holds that $|\mxU| \le \frac{1}{3}n$. At that case we count the fulx induced by the whole three $T$ which is what exactly we have prove in ~\cref{cliam:flu2} minus the inner interference at the tree, That it we need only to subtract $ \sum{ \frac{\Delta|U_{i}|^{2}}{n} + \lambda|U_{i}| } \le \left(\frac{|\mxU|}{n} + \lambda/\Delta  \right)  |T| $ So we obtained that in that case: 
    \begin{equation*}
      \begin{split}
	w_{E/E^{\prime}}\left( x \right)\ge\left( \delta_{0} - 2 \frac{|\mxU| }{n} - 2\lambda/\Delta \right) \Delta |T| \ge \left( \delta_{0} - \frac{2}{3} - 2 \frac{\lambda}{\Delta}    \right)\Delta |T|
      \end{split}
    \end{equation*}<++>
  \end{proof}

  \begin{proof}[Proof of Theorem 1.] Consider the size of the maxiaml layer $|U_{\max}|$ and sepearte to the following two cases. First, consider the case that $|U_{\max}| \ge  \alpha n $ in that case it follows immedily by~\cref{claim:flu1} that if $\delta_{0} > \frac{2}{3} - \frac{2\lambda}{\Delta}$ there exists $\alpha^{\prime} > 0 $ such that:  
  \begin{equation*}
    \begin{split}
      |x| \ge \left( \delta_{0} - \frac{2}{3} - \frac{2}{\lambda}\Delta \right)\Delta|U_{\max}| \ge  \alpha^{\prime} n 
    \end{split}
  \end{equation*}
  So, it is lefts to consider the second case in which $ |U_{\max}| < \alpha n $ in that case, we have from~\cref{claim:flu2} inequality that: 

  \begin{equation*}
    \begin{split}
      |x| & \ge  w_{E/E^{\prime}}\left( x \right)  \ge \left( \delta_{0} - \frac{|U_{\max}|}{n} - \frac{\lambda}{\Delta} \right) \Delta|T| \\ 
      & \ge \left( \delta_{0} - \alpha - \frac{\lambda}{\Delta} \right) \Delta|T| 
    \end{split}
  \end{equation*}
  Setting $\alpha \ge \frac{2}{3}$ we complete the proof
\end{proof}

Unfortunately, Singelton bound doesn't allow both $\delta_0 > \frac{2}{3}$ and $\rho_0 \ge \frac{1}{2}$, so in total, we prove the existence of code LDPC code which is good in terms of testability and distance yet has a zero rate. In the following subsection, we will prove that one can overcome this problem, by considering a variton of Tanner code, in which every vertex cheks only a $\frac{2}{3}$ fraction of the edges in his support.      

