\section{Draft.} 

\subsection{Claim 13.}

Denote By $S_{e}$ the exceptional vertices in $\Gamma^{+}_{\square}$ which their degree restricted to $x$ is more than $\Delta^{1 \frac{1}{2} + \varepsilon}$ and we will say that any vertex which is not exceptional is normal.  

Let $T \subset V^{-}$ be the negative vertices which are connected to at least one normal vertex through an heavy edge i.e an edge which adjoins to at least $\delta\Delta - \Delta^{\frac{1}{2}}$ squares. 


\begin{equation*}
  \begin{split}
    & |S_{e}|\Delta^{1 \frac{1}{2}+ \varepsilon} \le \Delta^{2}|S_{e}|\frac{|S|}{n} +  \lambda \sqrt{|S||S_{e}|} \\ 
    & \left(  \Delta^{1 \frac{1}{2}+ \varepsilon} - \Delta^{2} \frac{|S|}{n}\right)|S_{e}| \le  \lambda \sqrt{|S||S_{e}|}\\
    & |S_{e}| \le \frac{\lambda^{2}}{\left(  \Delta^{1 \frac{1}{2}+ \varepsilon} - \Delta^{2} \frac{|S|}{n} \right)^{2}} |S| \le \frac{3}{4} |S|
  \end{split}
\end{equation*}


\begin{equation*}
  \begin{split}
    & \delta\Delta|T|\left( 1 - \frac{1}{\sqrt{\Delta}} \right) \le \Delta |T| \frac{|S|}{n} + \lambda_{\cup} \sqrt{|S||T|}\\ 
    & |T| \le \frac{\lambda_{\cup}^{2}}{ \left( \delta \Delta \left( 1 - \frac{1}{\sqrt{\Delta}} \right) -  \Delta \frac{|S|}{n} \right)^{2} } |S| \le \frac{1}{\Delta}|S|
  \end{split}
\end{equation*}


\begin{equation*}
  \begin{split}
    d_{T} = \frac{E(S_{n}, T)}{|T|} \ge \frac{|S_{n|}}{|T|} \ge   \left( 1 -  \frac{1}{\Delta} \right) \delta\Delta
  \end{split}
\end{equation*}

\begin{equation*}
  \begin{split}
    (1- \frac{1}{2}\delta)\Delta^{2} \le \frac{3}{4}\left( 1 - \frac{1}{\sqrt{\Delta}}  \right)\Delta^{2} 
  \end{split}
\end{equation*}

Suppose that we have vertex $v_{-}$ which any of his positive neighbors contribute a row (col) with of poly-code with a small amount of noise (lets say below $\sqrt{\Delta}$). What could we say about the existence of polynomial that close to that manifold. We could think about that a linear number of rows has at most $\frac{1}{3}$ of zeros.  $ f\left( x,y \right)= \prod{ \left( x - a_{i} \right) \left(  y - b_{j} \right) } $


\ctt{ Seems like if $d_{T} \ge \theta \Delta^{2}$ then there is a little number of cols which are zero. Therefore we can think about some $f\left( x,y  \right) = \prod{ \left( y - a_{i} \right)   } g(x,y) $ such the degree of the first product is not so heigh and in addition if we write $f= l\left( x,y \right)g\left( x,y \right)$ then $l$ grantee that $f$ will agree on most to the zeros.  Now as the distance of the polynomial is relative large, we have that any term in $g$ after fixing $y$ is a low degree polynomial at $x$. Hence, it looks like ewe can only improve by adding the suit polynomial.}

\ctt{Notice that the product of two polynomial codes can be though as the polynomial in $2D$ in which the degree of each term is less than $2d$.} 

\subsection{Simplified Claim 13.} 
\newcommand{\Gtt}{\tilde{G}}
% \text{ if the row (col) } [uv] \text{ weight is } \Teta\left( \Delta \right) 
Denote by $\Gtt$ the graph obtained by $\Gtt  = \left\{ \left\{ u^{-}, v^{+} \right\}  \right\} $. By the assumption that $x$ is not a reducible we have that there exists $\gamma$ ( $ 1- \frac{1}{2}\delta$  )  such that the degree of any negative vertex in $\Gtt$ is less than $\gamma\Delta$. Denote by $S$ and $T$ the subsets of positive and the negative vertices on the support of $x$. By counting arguments we obtain that: 
\begin{equation*}
  \begin{split}
    |S| 2\Delta \ge E|_{\Gtt} \ge  |T| \Rightarrow \frac{|S|}{|T|} \ge \frac{\Delta}{2}
  \end{split}
\end{equation*}
We could also think on $S$ the normal vertices defined at \cite{leverrier2022quantum} (as been with positive degree in $\Gtt$ equivalence for been connected with heavy edge. On the other hand that quotient is also lower bound for the average degree of the negative vertices, Since $E_{\Gtt}(S,T)/|T| \ge |S|/|T| \ge \gamma/2$. For any random variable $X$ such that $X \le \Delta$ we have that: 

  \begin{equation*}
    \begin{split}
      & \prb{X\le x}x+\prb{X > x}\Delta\ge \expp{X}\Rightarrow \\ 
      & \prb{X \ge \frac{1}{2} \expp{X} } \ge \frac{\expp{X}}{2\Delta}
    \end{split}
  \end{equation*}

