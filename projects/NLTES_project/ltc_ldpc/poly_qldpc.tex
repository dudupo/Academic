

%\newcommand*{\ACM}{}%

\ifdefined\ACM

%\documentclass[sigplan,screen]{acmart}
\documentclass[manuscript,screen,review]{acmart}

\else
\documentclass{article}
\usepackage[utf8]{inputenc}
\usepackage[a4paper, total={6in, 9in}]{geometry}
\usepackage{braket}
\usepackage{xcolor}
\usepackage{amsmath}
\usepackage{amsfonts}
\usepackage{amsthm}
\usepackage{amssymb}
%\usepackage[ocgcolorlinks]{hyperref}
\usepackage{hyperref}
%\usepackage{hyperref,xcolor}
%\usepackage[ocgcolorlinks]{ocgx2}
\usepackage{cleveref}
\usepackage{graphicx}
\usepackage{svg}
\usepackage{float}
\usepackage{tikz}
\usetikzlibrary{patterns, shapes.arrows}
\usepackage{adjustbox}
%\usepackage{tikz-network}
\usepackage{tkz-graph}
\usepackage{tkz-berge}
\usepackage[linesnumbered]{algorithm2e}
\usepackage{multicol}
\usepackage[backend=biber,style=alphabetic,sorting=ynt]{biblatex}
%\usepackage{xcolor}
%\usepackage{tkz-berge}
%\usepackage{tkz-graph}
\usepackage{pgfplots}
\usepackage{sagetex}
\usepackage{setspace}
\usepackage{etoc}
%\usepackage{wrapfig}
\usepackage{pgfgantt}
\DeclareUnicodeCharacter{2212}{−}
\usepgfplotslibrary{groupplots,dateplot}
\pgfplotsset{compat=newest}

\newtheorem{theorem}{Theorem}
\newtheorem{definition}{Definition}
\newtheorem{example}{Example}
\newtheorem{claim}{Claim}
\newtheorem{fact}{Fact}
\newtheorem{remark}{Remark}
\newtheorem*{theorem*}{Theorem}
\newtheorem{lemma}{Lemma}
\crefname{lemma}{Lemma}{Lemmas}
\hypersetup{colorlinks=true}
% , allcolors=blue,allbordercolors=blue,pdfborderstyle={0 0 1}}
%\hypersetup{pdfborder={2 2 2}}
% pdfpagemode=FullScreen,
% backref 

\newtheorem{problem}{Problem}
\crefname{problem}{Problem}{Problems}

\DeclareMathOperator{\Ima}{Im}


\addbibresource{sample.bib} %Import the bibliography file

\fi
\begin{document}

\newcommand{\commentt}[1]{\textcolor{blue}{ \textbf{[COMMENT]} #1}}
\newcommand{\ctt}[1]{\commentt{#1}}
\newcommand{\prb}[1]{ \mathbf{Pr} \left[ {#1} \right]}
\newcommand{\expp}[1]{ \mathbf{E} \left[ {#1} \right]  }
\newcommand{\onotation}[1]{\(\mathcal{O} \left( {#1}  \right) \)}
\newcommand{\ona}[1]{\onotation{#1}}
%\newcommand{\PSI}{{\ket{\psi}}}
%\newcommand{\LESn}{\ket{\psi_n}}
%\newcommand{\LESa}{\ket{\phi_n}}
%\newcommand{\LESs}{\frac{1}{\sqrt{n}}\sum_{i}{\ket{\left(0^{i}10^{n-i}\right)^{n}}}}
%\newcommand{\Hn}{\mathcal{H}_{n}}
%\newcommand{\Ep}{\frac{1}{\sqrt{2^n}}\sum^{2^n}_{x}{ \ket{xx}}}
%\newcommand{\HON}{\ket{\psi_{\text{honest}}}}
%\newcommand{\Lemma}{\paragraph{Lemma.}}
\newcommand{\Cpa}{[n, \rho n, \delta n]}
%\setlength{\columnsep}{0.6cm}
\newcommand{\Jvv}{ \bar{J_{v}} } 
\newcommand{\Cvv}{ \tilde{C_{v}} } 

\newcommand{\Gz}{ G_{z}^{\delta} } 
\newcommand{ \Tann } {  \mathcal{T}\left( G, C_0 \right) }
\newcommand{\ireducable}{ireducable \hyperref[ire]{[\ref{ire}]} }
\newcommand{\cutUU}{E(U_{-1} \bigcup U_{+1} ,U)} 
\newcommand{\wcutUU}{w\left( E(U_{-1} \bigcup U_{+1} ,U)  \right)}
\newcommand{\testgo}{  \mathcal{T}\left(J, q , C_{0}\right) } 

\newcommand{\duC}{\left( C_{A}^{\perp}\otimes C_{B}^{\perp} \right)^{\perp}}
\newcommand{\duduC}{\left( C_{A}\otimes C_{B}\right)^{\perp}}
\title{The Dual-Tensor Polynomial Code Is Not $w$-Robust. } 
\author{David Ponarovsky}

\ifdefined\ACM
\affiliation{%
  \institution{The Th{\o}rv{\"a}ld Group}
  \streetaddress{1 Th{\o}rv{\"a}ld Circle}
  \city{Hekla}
  \country{Iceland}}
\email{larst@affiliation.org}
\else
\maketitle
\fi
%\input{abstract-poly}
\ifdefined\ACM
\maketitle
\fi
%\begin{multicols*}{2}
%\input{preamble-poly}
%   \input{backgroud-poly}

\begin{abstract}
  $w$-Robust codes are among the main ingredients in the novel constructions of good Quantum LDPC and LTC codes made by \cite{Dinur}, \cite{leverrier2022quantum}, and \cite{Pavel}. The Robustness property grants that any small-weight local view of the codeword will spread a fraction of it in both directions of the Left-Right Cayley Complex. On our way to construct Locally Testable Quantum Codes, we have tested a particular case on which the small code set on each local view is the polynomial code and focus on whether it can be $w$-robust code. Unfortunately, our answer to that question is negative. In this work, we share our experiences, ideas, and insights. We hope that all those would serve others in bringing closer a Quantum PCP Theorem.
\end{abstract}  

\section{Preambles}

Locally Testable Codes (LTC) are error correction codes such that verifying a uniformly chosen check would be enough to detect any error with probability proportional to its size. Simply put, one can imagine puzzle parts such that any trial to connect pieces in order far from a correct assignment would fail (w.p) at an early step of the process. The analogy for not testability is the case in which the contradiction is observed only in the attempt to putting the last piece.
 
 Likewise, Quantum LDPC codes (qLDPC) are also error correction codes; though that qLDPC encode qubits instead bits, for been considered good, they have to protect against two types of error, and obviously, their decoders have to be designed such that any attempt to detect or fix a one type error would not cause a second type error.
 
 Good LTC and qLDPC have more in common besides the fact that their existences were open questions for a long time that were solved at once. For example, It has shown that sampling uniformly a code would be, with probability 1, neither LTC \cite{Sasson} nor qLDPC code. That stands in total contrast to many other valuable entities in computer science,  such as good classic LDPC codes, which a random process can achieve. Thus, it is unsurprising that the recent constructions hinge on a complex that is relatively rich with algebraic structure. And even though those results are indeed used for proving the NLTS conjecture \cite{anshu2022nlts}, one could expect that the construction of a qLTC will follow soon after them. 

 Here we shatter light on that wondering by point on one reason that cause the straightforward approach to fail. In detail,.. \ctt{complete this.}   



\section{Background.}

\subsection{Polynomial Code.} Consider the field $\mathbb{F}_{m}$ for an arbitrary prime power $m=q^{l}$ greater than $n$. The polynomial codes relay on the fact that any two different polynomials in the ring $\mathbb{F}_{m}\left[ x \right]$ at degree at most $d$ different by at least $n - d + 1$ points. By define the code to be the subspace contains all the polynomials at degree at most $d$ encoded by $n$ numbers associated with their values. Formally we define:     
\begin{definition}
  Fix $m > n $ to be a prime power and let $a_{0},a_{1},a_{2},\ldots a_{n}$ distinct points of the field $\mathbb{F}_{m} = R$  and define the code $C \subset R $ as follows:  
  \begin{equation*}
    \begin{split}
      C = \left\{p\left(a_{0}\right),p\left(a_{1}\right),p\left(a_{2}\right),\cdots p\left(a_{n}\right) : p \text{ is polynomial at degree at most } k \right\}
    \end{split}
  \end{equation*}
\end{definition}
\begin{lemma}
  Fix the degree of the polynomial code to be at most $d$. Then the parameters of the code are $[n,d + 1, n - d]$.  
  \label{polycode}
\end{lemma}
\begin{proof}
  The dimension of the code equals to the dimension of the polynomials space at degree at most $d$ which is spanned by the vectors $e_{1}, e_{2} .. e_{d} = 1, x .. x^{d}$ and therefore is $d+1$. In addition suppose that $f,g$ are different polynomials i.e $f\neq g$.

  Hence $h = f-g$ is a non-$0$ polynomial at degree at most $d$ and therefore has at most $d$ roots. Namely at most $d$ points in which $f$ equals $g$ and at least $n-d$ in which they disagree. Put in another way the distance between any two different codewords of the code is at least $n-d$.  
\end{proof}
Notice that encoding naively the aleph-bet of $\mathbb{F}_{p}$ in binary strings require to pay a factor $\log n$ bits, So in general the asymptotic rate of the polynomial code attends to zero. Yet in our case, as we use the code for encoding only local views, The total factor that we have to pay is $\log\Delta$, which is constant relative to the number of bits (faces in the complex).   

\subsection{Quantum Polynomial Code.} 
Let's define the code $C$ such that any state in $C$ is a coset of the polynomials at degree at most $d$ shifted by $x \in \mathbb{F}_{p}$. In other words the codeword associated with $x$ is the state $\ket{\underline{c}} = \sum_{ \substack{ f \in \mathbb{F}_{d}[x] \\  f(0) = 0}}{ \ket{ c + f}} $. The inner product between any $d$-degree polynomial with zero free coefficient is:
\begin{equation*}
  \begin{split}
    \braket{ f | x^{j} } = \sum_{i \le d }{ \braket{a_{i} x^{i} | x^{j}}} = \sum_{i \le d}{ a_{i} \expp{ x^{i}x^{j}   }} =  \sum_{i \le d}{ a_{i } \mathbf{1}_{ i + j =_{n} 0 }}
  \end{split}
\end{equation*}
\ctt{Say some words about the classily testability of the polynomial code, and why for quantum it doesn't work. (The dual space of polynomials of low degree is the subspace of all the polynomials with heigh degree.)}


Next, we will review Tanner's construction, that in addition to being a critical element to our proof, also serves as an example of how one can construct a code with arbitrary length and positive rate.


\begin{definition}[$w$-Robustness] Let $C_{0}$ be code of length $\Delta$ over the aleph-bet $\Sigma$ with minimum distance $\delta_{0}\Delta$. $C = C_{0} \otimes \mathbb{F} + \mathbb{F}\otimes C_{0}^{\perp}$ will be said $w$-robust if any codeword $c \in C$ at weight less than $w$ it follows that $c$ is supported on at most $2\cdot w/\delta_{0}\Delta$ rows and cols.
\end{definition}

This definition is exactly identical to the one found in \cite{leverrier2022quantum}, expect that here we leave a room for consider also a non-binary codes. We note that, at least for proving the existence of negative vertex adjoins to many normal vertices via heavy edges, the aleph-bet is not matter.  
\section{The Polynomial-Code Is Not $w$-Robust.}
One idea for constructing is to use the polynomial code instead of $C_{0}$. This follows from the fact that if one picks a degree strictly greater than $\Delta/2$, then $C_{0}^{\perp} \subset C_{0}$ and therefore one could choose $C_{z}$ to be the same code defined on the negative vertices of the graph.  Here we prove that the dual-tensor code, in that case, is not $w$-robust, meaning that any such construction should be considered another way of proving the Reduction Lemma.

%We can construct a dual-tensor code by using a polynomial code instead of $C_{0}$. This is because if we choose a degree strictly greater than $\Delta/2$, then $C_{0}^{\perp} \subset C_{0}$. This means that we can choose $C_{z}$ to be the same code defined on the negative vertices of the graph. However, this construction is not $w$-robust, meaning that it does not satisfy the Reduction Lemma. Therefore, we must consider other methods of proving the Reduction Lemma.

%We can construct a dual-tensor code by using a polynomial code instead of $C_{0}$. This is because if we choose a degree strictly greater than $\Delta/2$, then $C_{0}^{\perp} \subset C_{0}$. This means that we can choose $C_{z}$ to be the same code defined on the negative vertices of the graph. However, this construction is not $w$-robust, meaning that any such construction should be considered another way of proving the Reduction Lemma.

\begin{claim}
  \label{claim:poldu}
  Let $C_{0}$ be the $[\Delta,d, \Delta-d]$ polynomial code. Then any code word in $\left( C_{0}^{\perp} \otimes C_{0}^{\perp} \right)^{\perp}$ is a polynomial in $F[x,y]$ at degree at most $\Delta + d$
\end{claim}
\begin{proof}
Consider base element $ C_{0} \otimes \mathbb{F} $, denote it by $c = g_{i} \otimes e_{j}$. And notice that $c$ has representation in $F[x,y]$ of $\prod_{y^{\prime} \neq j}{\left( y - y^{\prime} \right) }g_{i}\left( x \right)$. By the fact that $g_{i}\left( x \right) \in C_{0} $ we have that degree of $c$ is at most $\Delta + \delta$. Hence any element in the subspace of $C_{0} \otimes \mathbb{F}$ is a polynomial at degree at most $\Delta + d$.   
\end{proof}

\begin{claim}
  \label{claim:nowr} The dual-tensor polynomial code is not $\Delta^{1 +\varepsilon}$-robust, for any $\varepsilon > 0$.  
\end{claim}

\begin{proof}
Consider the following polynomial 
  \begin{equation*}
    \begin{split}
%      & P(x,y) = \Delta - 1 + \mathbf{1}_{ x = y } \\ 
      P(x,y) = \prod_{i \neq \Delta - 1}{ \left( x + i y \right)  }=\prod_{i \neq  1}{ \left( x - i y \right)  }  
         \end{split}
  \end{equation*}
  The degree of any monomial is at most $\Delta-1$, Thus it clear that $P \in \left( C_{0}^{\perp} \otimes C_{0}^{\perp} \right)^{\perp}$. And by the fact that for any $x \neq y$ there exists $i\neq 1 $ such that $x = iy$ we have that $P(x,y) = 0$. Hence the weight of $P$ is at most $|\{ (x,y) : x = y \}|= \Delta$. Yet, for any $x = y$ it follows:      
  \begin{equation*}
    \begin{split}
       P(x,x) = \prod_{i \neq \Delta - 1}{ \left( x + i x \right)  }= x^{\Delta-1}\prod_{i \neq \Delta - 1}{ \left( 1 + i \right) } = \left( \Delta-1 \right)!  =_{\Delta} -1  \neq_{\Delta} 0 
    \end{split}
  \end{equation*}
  Put it differently, the diagonal of the matrix, and only it, has only non-zero values, therefore the $P$ cannot decompised to a sum of $s + t$ such that $s \in C_{0}\otimes \mathbb{F}$, $t \in \mathbb{F}\otimes C_{0}$ and $s$ ($t$) is supported at most of $\Delta^{\varepsilon}$ rows (columns). But, $P$ is a codeword of the dual tensor code at weight less than $\Delta^{1 + \varepsilon}$. So, the dual-tensor polynomial code is not a $w$-robust code,  for $w = \Delta^{1 +\varepsilon}$.   
  Put it differently, the diagonal of the matrix, and only it, has non-zero values, so the $P$ cannot be decomposed into a sum of $s + t$ such that $s \in C_{0}\otimes \mathbb{F}$, $t \in \mathbb{F}\otimes C_{0}$ and $s$ ($t$) is supported on at most $\Delta^{\varepsilon}$ rows (columns). However, $P$ is a codeword of the dual tensor code at a weight less than $\Delta^{1 + \varepsilon}$. Therefore, the dual-tensor polynomial code is not a $w$-robust code, for $w = \Delta^{1 +\varepsilon}, \varepsilon> 0 $.
\end{proof}

\begin{openproblem} 
Is there a quantum CSS code that is also $\Delta^{1\frac{1}{2}}+ \varepsilon$-robust, for some $\varepsilon > 0$ ? And if so, can it be set on the square complex such that all the checks still commute?
\end{openproblem}



\printbibliography 
\end{document}

 



