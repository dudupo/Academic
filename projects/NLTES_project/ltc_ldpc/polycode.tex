
\newcommand{\FF}{\mathbb{F}_{q}}
\section{Polynomial Code.} Consider the field $\FF$ for an arbitrary prime power $q$ greater than $n$. The polynomial codes relay on the fact that any two different polynomials in the ring $\FF\left[ x \right]$ at degree at most $d$ different by at least $q - d + 1$ points. For example consider a polynomials pair at degree 1, namely two linear straight lines. If they are not identical than they have at most single intersection point, and the disagree on each of the $n-1$ remaining points.  

So by define the code to be the subspace contains all the polynomials at degree at most $d$, in such way that any codeword is an image of such polynomail encoded by $n$ numbers, one can garntee a lower bound on the code's distance. Formally we define:     
\begin{definition}[Polynomial Code. \cite{Reed1960PolynomialCO}]
  Fix $m > n $ to be a prime power and let $a_{0},a_{1},a_{2},\ldots a_{n}$ distinct points of the field $\FF$  and define the code $C$ as follows:  
  \begin{equation*}
    \begin{split}
      C = \left\{p\left(a_{0}\right),p\left(a_{1}\right),p\left(a_{2}\right),\cdots p\left(a_{n}\right) : p \text{ is polynomial at degree at most } d \right\}
    \end{split}
  \end{equation*}
\end{definition}
Observe that $C$ is a linear code at length $n$ over the aleph-bet $\FF$. The following Lemma states the realtion between the maximal degree of the polynomials and the properites of the code.   
\begin{lemma}
  Fix the degree of the polynomial code to be at most $d$. Then the parameters of the code are $[n,d + 1, n - d]$.  
  \label{polycode}
\end{lemma}
\begin{proof}
  The dimension of the code equals to the dimension of the polynomials space at degree at most $d$ which is spanned by the monomial base $e_{0}, e_{1}, e_{2} ... e_{d} = 1, x ... x^{d}$ and therefore is $d+1$. In addition suppose that $f,g$ are different polynomials i.e $f\neq g$.

  Hence $h = f-g$ is a non-$0$ polynomial at degree at most $d$ and therefore has at most $d$ roots. Namely at most $d$ points in which $f$ equals $g$ and at least $n-d$ in which they disagree. Put in another way the distance between any two different codewords of the code is at least $n-d$.  
\end{proof}
\begin{sagesilent}
R.<t> = PowerSeriesRing(GF(17));
polyf = (t-1) * (t-2)
polyg = (t-1) * (t-4)
f(x) = (x-1) * (x-2)
g(x) = (x-1) * (x-4)
pplot_t = finate_poly_plot(f)
pplot_t2 = finate_poly_plot(g)
\end{sagesilent}
 
\begin{figure}[H]
  \sageplot{ pplot_t + pplot_t2 }
    \sagestr{print_capt('$' + latex(f) + '$ and $' + latex(g) +'$')}
    %\caption{ The plot presents the extension of the polynomials $\sage{f}$ and  in the filed $\mathbb{F}_{17}$. }
  \label{fig:polyexample}
\end{figure}

\begin{fact} \label{fact:poly}
  Given $d+1$ points, there is a unique polynomial at degree at most $d$ that pass through all those points. Nevertheless, there is an algorithm $G$ that takes those points as input and outputs the corresponding polynomial.   
\end{fact}


