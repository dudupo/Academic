\subsection{Polynomial Code.} Consider the field $\mathbb{F}_{m}$ for an arbitrary prime power $m=q^{l}$ greater than $n$. The polynomial codes relay on the fact that any two different polynomials in the ring $\mathbb{F}_{m}\left[ x \right]$ at degree at most $d$ different by at least $m - d + 1$ points. For example consider a polynomials pair at degree 1, namely two linear straight lines. If they are not identical than they have at most single intersection point, and the disagree on each of the $n-1$ remaining points.  

So by define the code to be the subspace contains all the polynomials at degree at most $d$, in such way that any codeword is an image of such polynomail encoded by $n$ numbers, one can garntee a lower bound on the code's distance. Formally we define:     
\begin{definition}[Polynomial Code. \cite{Reed1960PolynomialCO}]
  Fix $m > n $ to be a prime power and let $a_{0},a_{1},a_{2},\ldots a_{n}$ distinct points of the field $\mathbb{F}_{m} = R$  and define the code $C \subset R $ as follows:  
  \begin{equation*}
    \begin{split}
      C = \left\{p\left(a_{0}\right),p\left(a_{1}\right),p\left(a_{2}\right),\cdots p\left(a_{n}\right) : p \text{ is polynomial at degree at most } d \right\}
    \end{split}
  \end{equation*}
\end{definition}
Observe that $C$ is a linear code at length $n$ over the aleph-bet $\mathbb{F}_{m}$. The following Lemma states the realtion between the maximal degree of the polynomials and the properites of the code.   
\begin{lemma}
  Fix the degree of the polynomial code to be at most $d$. Then the parameters of the code are $[n,d + 1, n - d]$.  
  \label{polycode}
\end{lemma}
\begin{proof}
  The dimension of the code equals to the dimension of the polynomials space at degree at most $d$ which is spanned by the monomial base $e_{0}, e_{1}, e_{2} ... e_{d} = 1, x ... x^{d}$ and therefore is $d+1$. In addition suppose that $f,g$ are different polynomials i.e $f\neq g$.

  Hence $h = f-g$ is a non-$0$ polynomial at degree at most $d$ and therefore has at most $d$ roots. Namely at most $d$ points in which $f$ equals $g$ and at least $n-d$ in which they disagree. Put in another way the distance between any two different codewords of the code is at least $n-d$.  
\end{proof}

\begin{fact} \label{fact:poly}
  Given $d+1$ points, there is a unique polynomial at degree at most $d$ that pass through all those points. Nevertheless, there is an algorithm $G$ that takes those points as input and outputs the corresponding polynomial.   
\end{fact}

\begin{lemma}[Testability Of Polynomail Code.] The polynomial code is $\left(d \cdot \log\left( n \right), \varepsilon  \right) $ testable.   

\end{lemma}
\begin{proof}
Denote by $f \in \mathbb{F}_{m}^{n}$ a codeword candidate and consider the following test. First, choose uniformly at random $d+1$ points $x_{1}, x_{2}, x_{3}, ... x_{d+1}$ and use $G$ to output a polynomial that agree on that points, denote it by $G\left( f \right)$. Then uniformly choose an additional point, return \textbf{True} if $G\left( f \right)(x_{d+2}) = f\left( x_{d+2} \right)$ and \textbf{False} otherwise.   

  Now, observe that if $f\in C$ then by ~\cref{fact:poly} we have that $G\left( f \right) = f$. In particular $G\left( f \right)(x_{d+2}) = f\left( x_{d+2} \right)$, Namely the test validate a codeword with probability $1$.   

  Now consider the case that $f \neq C$. And observes that $1 - \frac{1}{n} d\left( f, C \right)$ is the maximal fraction $\eta$ such that there exists a polynomial agree with $f$ on at least $\eta n$ points. Then we have:

  \begin{equation*}
    \begin{split}
      \eta & \le \mathbf{Pr}_{x_{1}, x_{2} ... x_{d} \sim_{U}, x_{d+1}\sim \mathbb{F}_{}}\left[ (Gf)(x) = f(x) \right] \\ & \le \prb{ (Gf)(x) = f(x)} \le \varepsilon \\
       1 - \eta & \ge 1 - \varepsilon 
    \end{split}
  \end{equation*}
\end{proof}
  
\begin{sagesilent}
  from math import fmod
  from numpy import linspace
  def res(X):
    Y = [(X[0][0], X[0][1])]
    for x,y in X[1:]:
      midx = (Y[-1][0] + x)/2
      midy = (Y[-1][1] + y)/2
      Y.append( (midx, midy) )
      Y.append((x,y))
    return Y
  def split_lines(X):
    ret = [ [ (X[0][0], X[0][1])  ] ]
    for x,y in X[1:]:
      if abs(y - ret[-1][-1][1]) > 1:
        ret +=  [ [ (x,y) ] ] 
      else:
        ret[-1].append( (x,y) )
    return ret 
  R.<t> = PowerSeriesRing(GF(17));
  poly = (t+1) * (t+2)
  f(x) = (x-1) * (x-2)
  g(x) = (x-1) * (x-4)
  p_list =  split_lines([ (y, fmod(f(y),17)) for y in linspace(0,17,100) ])
  pplot_t =  line(p_list[0]) 
  for l in p_list[1:]:
    p_list += line(l) 
  p_list2 = [ (y, fmod(g(y),17)) for y in linspace(0,17,100) ]
\end{sagesilent}
Here's a plot of $f$ from $-1$ to $1$ and:  $\sage{poly}$ %and $\sage{p_list}$: 
\begin{center}
  \sageplot{pplot_t }
\end{center}
%+  point(p_list2, size=1, color='green')
