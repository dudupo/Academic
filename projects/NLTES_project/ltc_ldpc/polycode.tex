
\newcommand{\FF}{\mathbb{F}_{q}}
\section{Polynomial Code.} Consider the field $\FF$ for an arbitrary prime power $q$ greater than $n$. The polynomial codes rely on the fact that any two different polynomials in the ring $\FF\left[ x \right]$ of degree at most $d$ differ by at least $q - d + 1$ points. For example, consider a pair of polynomials of degree~1, namely two linear straight lines. If they are not identical, then they have at most one intersection point, and they disagree on each of the $n-1$ remaining points. By defining the code to be the subspace containing all polynomials of degree at most $d$, such that any codeword is an image of such a polynomial encoded by $n$ numbers, we can guarantee a lower bound on the code's distance. Formally, we define:

\begin{definition}[Polynomial Code. \cite{Reed1960PolynomialCO}]
  Fix $m > n $ to be a prime power and let $a_{0},a_{1},a_{2},\ldots a_{n}$ distinct points of the field $\FF$  and define the code $C$ as follows:  
  \begin{equation*}
    \begin{split}
      C = \left\{p\left(a_{0}\right),p\left(a_{1}\right),p\left(a_{2}\right),\cdots p\left(a_{n}\right) : p \text{ is polynomial at degree at most } d \right\}
    \end{split}
  \end{equation*}
\end{definition}
Observe that $C$ is a linear code of length $n$ over the alphabet $\FF$. The following Lemma states the relation between the maximal degree of the polynomials and the properties of the code.
\begin{lemma}
Fix the degree of the polynomial code to be at most $d$. Then the parameters of the code are $[n, d+1, n-d]$.
  \label{polycode}
\end{lemma}
\begin{proof}
The dimension of the code is equal to the dimension of the polynomial space of degree at most $d$, which is spanned by the monomial basis $e_{0}, e_{1}, e_{2}, \dots, e_{d} = 1, x, \dots, x^{d}$, and is therefore $d+1$. Furthermore, suppose that $f$ and $g$ are two different polynomials, i.e. $f \neq g$. Thus, $h = f-g$ is a non-zero polynomial of degree at most $d$, and therefore has at most $d$ roots. This implies that there are at most $d$ points in which $f$ and $g$ are equal, and at least $n-d$ points in which they disagree. In other words, the distance between any two different codewords of the code is at least $n-d$.
\end{proof}
\begin{sagesilent}
R.<t> = PowerSeriesRing(GF(17));
polyf = (t-1) * (t-2)
polyg = (t-1) * (t-4)
f(x) = (x-1) * (x-2)
g(x) = (x-1) * (x-4)
pplot_t = finate_poly_plot(f)
pplot_t2 = finate_poly_plot(g)
\end{sagesilent}
 
\begin{figure}[H]
  \sageplot{ pplot_t + pplot_t2 }
    \sagestr{print_capt('$' + latex(f) + '$ and $' + latex(g) +'$')}
  \label{fig:polyexample}
\end{figure}



