  \section{Introduction}
  \section{Introduction.}
.. bla bla bla.. bla bla ..     
\definition[General Entanglement State]{ We say that $\PSI$ is general entanglement \label{def:gEnt} if .. }

\definition[Local-Measure-Circuit] { We say that a quantum circuit $C$ is a local measure circuit \label{def:lmc} if it's can be described as a decomposition of poly classical circuit and a constant depth quantum circuit which contains only 1-qubit gates and measurements. 

We would think about local measure circuits as chip circuits. }

\definition[$p_{0}-\Delta$ Fault Tolerance Circuit]{ We say that $\mathcal{C}$ is $p_{0}-\Delta$ fault tolerance \label{def:gft} presentation of abstract circuit $C$ if there exists a local measure circuit $C_{0}$ \ref{lmc} such it's grunted that for noise $p < p_{0}$ $\mathcal{C}$ compute $C$ w.h.p,
And in addition, if $p \in \left( p_{0}, p_{0} + \varepsilon \right)$ then by applying a $C_{0}$ on $\mathcal{C}$ output yields a general entanglement state \label{def:gEnt}}       

\ctt{We would like to add a complexity parameter for the above definition, for example, ``a general entanglement state over more than $\frac{1}{5}$ of the qubits.}  

  %Linear Error Correction Codes, 
  \subsection{Notations, Definitions, And Our Contribution}
  Here we focus only on linear binary codes, which one could think about as linear subspaces of $\mathbb{F}_{2}^{n}$. A common way to measure resilience is to ask how many bits an evil entity needs to flip such that the corrupted vector will be closer to another vector in that space than the original one. Those ideas were formulated by Hamming \cite{Hamming}, who presented the following definitions. 
  \paragraph{Definition.} Let $n \in \mathbb{N}$ and $\rho, \delta\in \left( 0,1 \right)$. We say that $C$ is a \textit{binary linear code} with parameters $[n, \rho n, \delta n]$. If $C$ is a subspace of $\mathbb{F}_{2}^{n}$, and the dimension of $C$ is at least $\rho n$. In addition, we call the vectors belong to $C$ \textit{codewords} and define the distance of $C$ to be the minimal number of different bits between any codewords pair of $C$.   

  From now on, we will use the term code to refer to linear binary codes, as we don't deal with any other types of codes. Also, even though it is customary to use the above parameters to analyze codes, we will use their percent forms called the relative distance and the rate of code, matching $\delta$ and $\rho$ correspondingly.     
  \paragraph{Definition.} A \textit{family of codes} is an infinite series of codes. Additionally, suppose the rates and relative distances converge into constant values $\rho,\delta$. In that case, we abuse the notation and call that family of codes a code with $[n, \rho n, \delta n]$ for fixed $\rho, \delta\in [ 0,1 )$, and infinite integers $n \in \mathbb{N}$.     

  Notice that the above definition contains codes with parameters attending to zero. From a practical view, it means that either we send too many bits, more than a constant amount, on each bit in the original message. Or that for big enough $n$, adversarial, limited to changing only a constant fraction of the bits, could disrupt the transmission. That distinction raises the definition of good codes.

  \paragraph{Definition.}We will say that a family of codes is a \textit{good code} if its parameters converge into positive values. 

  Apart from distance and rate here, we interest also that the checking process will be robust. In particular,  we wish that against significant errors, forgetting to perform a single check will sabotage the computation only with a tiny probability.  
  \paragraph{Definition.} Consider a code $C$  a string $x$, and denote by $\xi\left( x \right)$ the fraction of the checks in which $x$ fails. $C$ will be called a \textit{local-testability $f\left( n \right)$} If there exists $\kappa > 0$ such that 
  \begin{equation*}
    \begin{split}
      \frac{d\left( x, C \right)}{n} \le \kappa \cdot  \xi\left( x \right) f\left( n \right)
    \end{split}
  \end{equation*}

  
  Nowadays, we are aware of a wide range of constructions yield good codes, including the expander codes of Sipser and Spilman \cite{ExpanderCodes} and the LTC codes of Dinur \cite{Dinur}, \cite{Pavel}, \cite{leverrier2022quantum}. Thus if a decade ago, the main question was the existence of a good code and its construction, now, and particularly in this work, we concentrate on getting a deep understanding of what makes those constructions work. By utilizing those insights, we succussed in achieving significantly simpler construction. Our results: 

 \begin{theorem*}[Theorem 1] There exist a constant $\alpha > 0$ and an infinte family of Tanner Codes $C = \Tann$ such that any ireducable codeword $x$ of a coresponding disagreement code $x \in C_{\oplus}$ at length $n$, weight at least $\alpha n$. \end{theorem*}

\begin{theorem*}[Theorem 1+] There exist a constant $\alpha > 0 $ and infinite familiy of codes which satesfies Theroem 1 and also good.
  \end{theorem*}




  \subsection{Singleton Bound}  
  To get a feeling of the behavior of the distance-rate trade-of, Let us consider the following two codes; each demonstrates a different extreme case. First, define the repetition code $C_{r} \subset \mathbb{F}_{2}^{n \cdot r}$, In which, for a fixed integer $r$, any bit of the original string is duplicated $r$ times. Second, consider the parity check code $C_{p} \subset \mathbb{F}_{2}^{n+1}$, in which its codewords are only the vectors with even parity. Let us analyze the repetition code. Clearly, any two $n$-bits different messages must have at least a single different bit. Therefore their corresponding encoded codewords have to differ in at least $r$ bits. Hence, by scaling $r$, one could achieve a higher distance as he wishes. Sadly the rate of the code decays as $n/nr = 1/r$. In contrast, the parity check code adds only a single extra bit for the original message. Therefore scaling $n$ gives a family which has a rate attends to $\rho \rightarrow 1$. However, flipping any two different bits of a valid codeword is conversing the parity and, as a result, leads to another valid codeword.

  To summarize the above, we have that, using a simple construction, one could construct the codes $[r, 1, r]$, $[r, r-1, 2]$. Each has a single perfect parameter, while the other decays to the worst. In the next section, we will review the Singleton bound, which states that for any code (not necessarily good), there must be a zero-sum game between the relative distance and the rate.
  Now, we are ready to formulate our contribution. 


  Besides being the first bound, Singleton bound demonstrates how one could get results by using relatively simple elementary arguments. It is also engaging to ask why the proof yields a bound that, empirically, seems far from being tight.
  \begin{theorem*}[Singleton Bound.]\label{Sing}  For any linear code with parameter $[n,k,d]$, the following inequality holds:
  \begin{equation*}
    k+ d \le n + 1
  \end{equation*} 
  \end{theorem*}

\begin{proof} Since any two codewords of $C$ differ by at least $d$ coordinates, we know that by ignoring the first $d-1$ coordinate of any vector, we obtain a new code with one-to-one corresponding to the original code. In other words, we have found a new code with the same dimension embedded in $\mathbb{F}_{2}^{n-d+1}$. Combine the fact that dimension is, at most, the dimension of the container space, we get that:  
  \begin{equation*}
    \begin{split}
      \dim C &= 2^{k} \le 2^{n-d+1} \Rightarrow k+d \le n + 1
    \end{split}
  \end{equation*}
  $\square$
\end{proof}

  It is also well known that the only binary codes that reach the bound are: $[n,1,n]$, $[n,n-1,2]$,$[n,n,1]$ \cite{eczoo_mds}. In particular, there are no good binary codes that obtain equality. Next, we will review Tanner's construction, that in addition to being a critical element to our proof, also serves as an example of how one can construct a code with arbitrary length and positive rate.
  \subsection{Tanner Code}
  The constructions require two main ingredients: a graph $\Gamma$, and for simplicity, we will restrict ourselves to a $\Delta$ regular graph. Secondly, a small code $C_{0}$ at length equals the graph's regularity, namely $C_{0} = [\Delta,\rho\Delta, \delta\Delta]$. We can think about any bit string at length E as an assignment over the edges of the graph. Furthermore, for every vertex $v \in \Gamma$, we will call the bit string, which is set on its edges, the local view of $v$. Then we can define, \cite{Tanner}:
  \paragraph{Definition.}  Let $ C = \mathcal{T}\left( \Gamma, C_{0} \right)$  be all the codewords which, for any vertex $v\in \Gamma$, the local view of $v$ is a codeword of $C_{0}$. We say that $C$ is a Tanner code of $\Gamma, C_{0}$. Notice that if $C_{0}$ is a binary linear code, So $C$ is.  


  It's also worth mentioning that the first construction of good classical codes, due to Sipser and Shpilman, are Tanner codes over expanders graphs \cite{ExpanderCodes}.
  \paragraph{Theorem} Tanner codes have a rate of at least $2\rho - 1$.
  \paragraph{Proof.}  The dimension of the subspace is bounded by the dimension of the container minus the number of restrictions. So assuming non-degeneration of the small code restrictions, we have that any vertex count exactly $ \left( 1 - \rho  \right)\Delta $ restrictions. Hence, \begin{equation*}
    \begin{split}
      \dim C & \ge \frac{1}{2}n\Delta - \left( 1-\rho \right)\Delta n = \frac{1}{2}n\Delta\left( 2\rho - 1 \right)  
    \end{split}
  \end{equation*} Clearly, any small code with rate $> \frac{1}{2}$ will yield a code with an asymptotically positive rate $\square$ 
  \subsection{Expander Codes}
  We saw how a graph could give us arbitrarily long codes with a positive rate. We will show, Sipser's result that if the graph is also an expander, we can guarantee a positive relative distance. We notice that the name expander codes is coined for a more general version than the one we will present.   
  \paragraph{Definition.} Denote by $\lambda$ the second eigenvalue of the adjacency matrix of the $\Delta$-regular graph. For our uses, it will be satisfied to define expander as a graph $G = \left( V,E \right)$ such that for any two subsets of vertices $T,S \subset V$, the number of edges between $S$ and $T$ is at most:
  \begin{equation*}
    \begin{split}
      \mid E\left( S,T \right) - \frac{\Delta}{n}|S||T| \mid \le \lambda\sqrt{|S| |T|} 
    \end{split}
  \end{equation*}
  This bound is known as the Expander Mixining Lemma. 
  \paragraph{Theorem.} Theorem, let $C$ be the Tanner Code defined by the small code $C_{0} = [\Delta,\delta\Delta, \rho\Delta ]$ such that $\rho \ge \frac{1}{2}$ and the expander graph $G$ such that $\delta\Delta \ge \lambda$. $C$ is a good  LDPC code.
  \paragraph{Proof.} We have already shown that the graph has a positive rate due to the Tunner construction. So it's left to show also the code has a linear distance. Fix a codeword $x \in C$ and denote By $S$ the support of $x$ over the edges. Namely, a vertex $v\in V$ belongs to $S$ if it connects to nonzero edges regarding the assignment by $x$, Assume towards contradiction that $|x| = o\left( n \right)$. And notice that $|S|$ is at most $2|x|$, Then by The Expander Mixining Lemma we have that: 
  \begin{equation*}
    \begin{split}
      \frac{E\left( S,S \right)}{|S|} & \le \frac{\Delta}{n}|S|  + \lambda \\
      & \le_{ n \rightarrow \infty} o\left( 1 \right) + \lambda
    \end{split}
  \end{equation*}

  Namely, for any such sublinear weight string, $x$, the average of nontrivial edges for the vertex is less than $\lambda$. So there must be at least one vertex $v \in S$ that, on his local view, sets a  string at a weight less than $\lambda$. By the definition of $S$, this string cannot be trivial. Combining the fact that any nontrivial codeword of the $C_{0}$ is at weight at least $\delta\Delta$, we get a contradiction to the assumption that $v$ is satisfied, videlicet, $x$ can't be a codeword $\square$
 
