\section{Preambles}

Localy Testable Codes, or LTC, are error correction codes such that verfining a uinformly random choosen check, would be enough to detect any error with probability proportional to it's size. Simply put, one can imagine a puzzle parts such that any trieal to connect pieces at a order that far from a correct assignment, would fail (w.p) at an early step of the process. The a anlogy for not teastblility, is the case which the contrudiciton observed only the attempt to putting the last piece.     

Besides the clear computional adventage that they offer, they are known by their sagnificent roles at the eirler PCP theorems proofs. And still, the existaness of good LTC, was considered as an open question for a decades. Moreover, Sasson proved that codes obtaind by the standart randomized constrctions can not be LTC \cite{Sasson}, what rasie the suspicion that mabye codes can not be both good and locally testabile. However, receent works, by \cite{Dinur}, \cite{Pavel} and \cite{leverrier2022quantum}, yield a positive answer.

In a nutshell their sophisticated constructions ensure that no a subilinear depandency of restriction exists and yet gurnte that the restrctions are linear far form been indepandet. Namely, no restrication is important than other, and removing a linear number of consratints would yield the same code.  

Their constructions requires that the local restractions, or the local codes, have two properities, stated as the $w$-robustness and $p$-resistance for puncturing. Even those they showed a probalistic proof for existence of inifinte familiy such codes, they are bigger for any practical use. we would not resate them formally here, and instend, we refer the reader to \cite{leverrier2022quantum}. Yet, we want to point out, that any assumption over local structure of the code might be also an obstacle for encode a univirasal compution at the code. 

In this work, we propose a new construction for good LTC that demands from the samll codes only to have a large distance. In short, by assosicate each check with a small code over $2/3$-fraction of the vertex'a edges, instead all of them as in the standart Tanner code, we secusses to obtian an LTC with constant rate. Then by considering graphs, such that both the graph and he's subgraph obtained by taking $\frac{1}{2}$-fracation of the edges of each vertex are good expander we also success to prove that the codes have linear distance. 

Finaly, we show how one can constract shuch graph given a Ramnujan \emph{Cayley} graph. Nevertheless, despite the fact that we seucessed to simplifiy the LTC, we still did not understand how they can be used for encoding an univerisal computation.  

