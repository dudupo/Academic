\section{Decodeing and Testing}
  For completeness, we show exactly how Theorem 1 implies testability. The following section repeats Leiverar's and Zemor's proof \cite{leverrier2022quantum}. Consider a binary string $x$ that is not a codeword. The main idea is the observation that the number of bits filliped by (any) decoder, while decoding $x$, bounds the distance $d\left( x, C \right)$ from above. In addition, the number of positive checks in the first iteration is exactly the number of violated restrictions.
%\begin{figure*}[h]
%\begin{adjustbox}{width=\textwidth}
  \begin{definition}Let $L = \{L_{i}\}^{2|E|}_{0}$  be a series of $2|E|$. Such that for each vertex $ v \in V$ $\sum_{ e = \{u,v\} }{ L_{e_v} } \in C_{0}$. We will call $L$ a \textit{Potential list} and refer to the $e_{v}$'the element of $L$ as a suggestion made by the vertex $v \in V$ for the edge $e \in E$. Sometimes we will use the notation $L_{v}$ to denote all the $L$'s coordinates of the form $ L_{e_{v}} \forall e \in \text{Support} \left( v \right) $. Define the \textit{Force} of $L$ to be the following sum $  F\left( L \right) = \sum_{e = \{v,u\} \in E }{ \left(L_{e_v} + L_{e_u}\right) }$ and notice that $ F\left( L \right) \in C_{\oplus}$. And define the \textit{state} $S(L) \subset \mathbb{F}^{|E|}_{2}$ of $L$ as the vector obtained by choosing an arbitrary value from $ \{ L_{e_v}, L_{e_u} \}$ for each edge $e \in E$.  
  \end{definition}
  \paragraph{Claim.} Let $L$ be the Potential list. If $F(L)=0$ then $S(L)\in C$.
  \begin{proof} Denote by $\phi\left( e \right) \subset \{ L_{e_v}, L_{e_u} \}$ the value which was chosen to $e = \{v,u\} \in E$. By $F\left(L\right) = 0$ , it follows that $ L_{e_v} + L_{e_u} = 0 \Rightarrow L_{e_v} = L_{e_u} = \phi\left( e \right) $ for any $e \in E$. Hence for every $v\in V$ we have that $ S\left( L \right)|_{v} = \sum_{u \sim v}{ \phi\left( \{v,u\} \right) } =  \sum_{u \sim v}{ L_{e_v }} \in C_{0}$ $ \Rightarrow S\left( L \right) \in C$   
  \end{proof}
  The decoding goes as follows. First, each vertex suggests the closet $C_{0}$'s codeword to his local view. Those suggestions define a Potential list, denote it by $L$, then if $F\left( L \right) <\tau$, by Theorem 1, one could find a suggestion of vertex $v$ and a codeword $c_v$ such that updating the value of $L_{v} \leftarrow L_{v} + c_{v}$ yields a Potential list with lower force. Therefore repeating the process till the force vanishes, obtain a Potential list in which its state is a codeword. 
  \begin{definition} Let $\tau > 0, f : \mathbb{N} \rightarrow \mathbb{R^{+}}$, and consider a Tanner Code $C = \mathcal{T}\left( G, C_{0} \right)$. Let us Define the following decoder and denote it by $\mathcal{D}$.  
  \end{definition}

  \begin{algorithm}[H]
    \caption{Decoding}
    \label{alg:three}
    \KwData{ $x \in \mathbb{F}_{2}^{n}$ }
    \KwResult{ $\arg\min {\left\{  y \in C : |y + x|  \right\} }$ if $d(y,C) < \tau $ and False otherwise. }
    $ L \leftarrow \text{Array} \{ \} $\\
    \For { $ v \in V$} {
      $c^{\prime}_{v} \leftarrow \arg\min {\left\{  y \in C_{0} : |y + x|_{v} |  \right\} } $\\
      $ L_{v} \leftarrow c^{\prime}_{v}$
    }
    $ z \leftarrow \sum_{v \in V}{c^{\prime}_{v}} $\\
    \eIf{ $ |z| < \tau \frac{n}{f\left( n \right)} $}{
      \While{ $|z| > 0$ }{
	find $v$ and $c \in C_{0}$ such that $|z + c_{v}| < |z|$\\
	$z \leftarrow z + c_{v}$ \\
	$ L_{v} \leftarrow  L_{v} + c_{v}$
      }
    }{
      reject. 
    }
    \Return  $S(L) $

  \end{algorithm}

  \paragraph{Theorem.} Consider a Tanner Code $C = [n, n\rho, n\delta]$ and the disagreement code $C_{\oplus}$ defined by it. Suppose that for every codeword $ z \in C_{\oplus}$ in $C_{\oplus}$ such that $|z| < \tau^{\prime} n / f\left(n\right)$, there exists another codeword $y \in C_{\oplus} $ such that $|y| < |z|$, set $\tau \leftarrow \frac{\tau^{\prime} }{6 \Delta} \delta$ then, 

  \begin{enumerate}
    \item $\mathcal{D}$ corrects any error at a weight less than $\tau n / f\left(n\right)$.   
    \item $C$ is $f\left( n \right)$ testable code.
  \end{enumerate}


  \begin{proof} So it is clear from the claim above that if the condition at line (6) is satisfied, then $\mathcal{D}$  will converge into some codeword in $C$. Hence, to complete the first section, it left to show that $\mathcal{D}$ returns the closest codeword. Denote by $e$ the error, and by simple counting arguments; we have that $\mathcal{D}$ flips at most:  
  \begin{equation*}
    \begin{split}
      d_{\mathcal{D}}\left( x, C \right) & \le 2|e|\Delta + \tau \frac{n}{f\left( n \right)}\Delta
    \end{split}
  \end{equation*}
  bits. Hence, by the assumption, 
  \begin{equation*}
    \begin{split}
      d_{\mathcal{D}}\left( x, C \right) & \le 3\Delta \tau \frac{n}{f\left( n \right)} \le 3\Delta \tau\delta n < \frac{1}{2} \delta n  
    \end{split}
  \end{equation*}
  Therefore the code word returned by $\mathcal{D}$ must be the closet. Otherwise, it contradicts the fact that the relative distance of the code is $\delta$.
  To obtain the correctness of the second section, we will separate when the conditional at the line (5) holds and not. And prove that the testability inequality holds in both cases. 
  Let $x \in \mathbb{F}_{2}^{n}$ and consider the running of $\mathcal{D}$ over $x$. Assume the first case, in which the conditional at line (5) is satisfied. In that case, $\mathcal{D}$ decodes $x$ into its closest codeword in $C$. Therefore:
  \begin{equation*}
    \begin{split}
      d\left( x, C \right) \le & \ d_{D} \left( x, C \right) \le m\xi\left( x \right)\Delta +  |z|\Delta  \\ \le &  \  m\xi\left( x \right)\Delta + m\xi\left( x \right)  \Delta^{2} \\ 
      \frac{d\left( x, C \right)}{n} \le & \  \kappa_{1} \xi\left( x \right)    
    \end{split}
  \end{equation*}
  Now, consider the other case in which: $ |z| \ge \tau \frac{n}{f\left( n \right)}  $.
  \begin{equation*}
    \begin{split}
      \frac{d\left( x, C \right)}{n} & \le 1 \le \frac{|z|}{\tau n}f\left( n \right) \le \frac{m}{n} \frac{1}{\tau} \Delta \xi\left( x\right)f\left( n \right) \\ & \le \kappa_{2} \xi\left( x \right)f\left( n \right)  
    \end{split}
  \end{equation*}
  Picking $ \kappa \leftarrow \max \{ \kappa_{1}, \kappa_{2} \}$ proves $f\left( n \right)$-testability
\end{proof}

