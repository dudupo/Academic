\begin{figure*}[t]
% Drawing a graph using the PG 3.0 graphdrawing library
% Author: Mark Wibrow
\documentclass[tikz,border=10pt]{article}
\usepackage{tikz}
\usetikzlibrary{positioning}
\begin{document}

\tikzset{%
  every neuron/.style={
    circle,
    draw,
    fill=black,
    radius=0.05mm
  },
  neuron missing/.style={
    draw=none,
    scale=2,
    text height=0.25cm,
    execute at begin node=\color{black}$\vdots$
  },
}

\begin{tikzpicture}[x=0.3cm, y=0.5cm, >=stealth]

  \node [every  neuron ] (v-0) at (0,0) {};  
  \node [every  neuron ] (v-1) at (-4,-2) {};
  \node [every  neuron ] (v-2) at (0,-2) {};
  \node [every  neuron ] (v-3) at (4,-2) {};
  \node [every  neuron ] (v-4) at (-8,-4) {};
  \node [every  neuron ] (v-5) at (-6,-4) {};  
  \node [every  neuron ] (v-6) at (-4,-4) {};
  \node [every  neuron ] (v-7) at (-2,-4) {};
  \node [every  neuron ] (v-8) at (0,-4) {};
  \node [every  neuron ] (v-9) at (2,-4) {};
  \node [every  neuron ] (v-10) at (4,-4) {};  
  \node [every  neuron ] (v-11) at (6,-4) {};
  \node [every  neuron ] (v-12) at (8,-4) {};
  \node [every  neuron ] (v-13) at (-8,-8) {};
  \node [every  neuron ] (v-14) at (-10,-8) {};
  \node [every  neuron ] (v-15) at (-8,-10) {};  
  \node [every  neuron ] (v-16) at (6,-8) {};
  \node [every  neuron ] (v-17) at (6,-10) {};
  \node [every  neuron ] (v-18) at (4,-8) {};
  \draw [->] (v-0) -- (v-1);
  \draw [->] (v-0) -- (v-2);
  \draw [->] (v-0) -- (v-3);
  \draw [->] (v-1) -- (v-4);
  \draw [->] (v-1) -- (v-5);
  \draw [->] (v-1) -- (v-6);
  \draw [->] (v-2) -- (v-7);
  \draw [->] (v-2) -- (v-8);
  \draw [->] (v-2) -- (v-9);
  \draw [->] (v-3) -- (v-10);
  \draw [->] (v-3) -- (v-11);
  \draw [->] (v-3) -- (v-12);
\draw[-] (v-4) -- (-10, -8) -- (-6, -8) -- (v-4) ;
\draw[-] (v-11) -- (4, -8) -- (8, -8) -- (v-11) ;
\draw[->] (v-14) -- (v-17);
\draw[->] (v-18) -- (v-15);
\draw[->] (v-13) -- (v-15);
\draw[->] (v-16) -- (v-17);
\draw[<->] (-12, 0) -- (-12, -3.8); 
\draw[<->] (-12, -4.2) -- (-12, -10);
\node (L-l) at (-13, -1.9) {$l$};
\node (L-g) at (-13, -7) {$\frac{1}{2}g$};
\end{tikzpicture}
\end{document}

\caption{This is a tiger.}
\end{figure*}
\paragraph{Lemma}. Denote by $U$ the layer at with the maximal numbe of vertices. \ctt{Here}  height $l$ and denote by $ \omega = \frac{1}{2}\delta_{0}\Delta$. The number of vertices in $U$ is at least: 
\begin{equation*}
  \begin{split}
    \left( \frac{1}{2}\delta_0\Delta \right)^{l}- \frac{\left( \frac{\Delta^{2}}{\omega} \right)\Delta^{2l-g} - \omega^{l-g/2} }{\frac{\Delta^{2}}{\omega}-1}
  \end{split}
\end{equation*}

\paragraph{Proof:} We will think of $T$ as the tree obtained by purning a tree, with degree greater than $\frac{1}{2}\delta_{0}\Delta$, at height $l$, namely any vertex at height $i \le l $ with two parents can be counted as subtree with height $l-i$ that has pruned. Then the problem is equivalence to bounding from above the number of pruned vertices at height $l$. Define $X_{i}$ to be the number of pruned subtrees whose root is at height $i$. Then consider the next maximization problem: 
\begin{equation*}
  \begin{split}
    X_{i} & \le \Delta^{2\max \left\{ 0, l-g/2 \right\}} \\ 
    f & \le \sum_{g/2}^{l}{\omega^{l-i}X_{i} } \le \omega^{l-g/2}\sum_{0}^{l-g/2}{\omega^{-i}\Delta^{2i} } = \\ & \omega^{l-g/2}\frac{\left( \frac{\Delta^{2}}{\omega} \right)^{l-g/2+1}-1}{\frac{\Delta^{2}}{\omega}-1} =  
      \frac{\left( \frac{\Delta^{2}}{\omega} \right)\Delta^{2l-g} - \omega^{l-g/2} }{\frac{\Delta^{2}}{\omega}-1}
  \end{split}
\end{equation*}
By using the inequality $ |U| \ge \omega^{l} - f $ we get the bound $\square$ 
\paragraph{Claim.} there exists a constant $\Delta^{\prime}$ that depends only on $\delta_{0}$, such that if $\Delta \ge \Delta^{\prime}$, then the number number of vertices at height $\frac{3}{4}g$ is at least $ \frac{1}{2}\omega^{\frac{3}{4}g} $.

\paragraph{Proof:} The above lower bound yields: 
\begin{equation*}
  \begin{split}
    |U| \ge \omega^{\frac{3}{4}g} -  \frac{\left( \frac{\Delta^{2}}{\omega} \right)\Delta^{\frac{1}{2}g}  }{\frac{\Delta^{2}}{\omega}-1} \ge \omega^{\frac{3}{4}g} -  2 \Delta^{\frac{1}{2}g}  
  \end{split}
\end{equation*}
So it is enough to require that: $\left( \frac{1}{2}\delta_{0}\Delta  \right)^{\frac{3}{4}g} \ge 4 \Delta^{\frac{1}{2}g}$ :  
\begin{equation*}
  \begin{split}
    \Delta^{\frac{1}{4}g}\ge 4\cdot \left( \frac{2}{\delta_{0}} \right)^{\frac{3}{4}g} \Rightarrow\Delta \ge  4^{\frac{4}{g}} \left( \frac{2}{\delta_{0}} \right)^{3} 
  \end{split}
\end{equation*} So any $\Delta$ satisfies $\Delta> \left( \frac{2}{\delta_0} \right)^{3}$ is also satisfying the above inequality, and therefore for such $\Delta$-regular graphs, the size of $U$ will be at least $\frac{1}{2}\omega^{\frac{3}{4}g}$ $\square$   

\paragraph{Claim.} Assume that $ g \ge \frac{4}{3} \log_{\Delta -1 }\left( n \right) $, Then we have that for every $\varepsilon >0 $,  there exist $\Delta_{\varepsilon}$  such for every $\Delta \ge \Delta_{\varepsilon}$  it holds that $ \omega^{\frac{3}{4}g} \ge n^{1-\varepsilon}$.
\paragraph{Proof:}

\begin{equation*}
  \begin{split}
    \omega^{\frac{3}{4}g}&\ge \omega^{\log_{\Delta-1}\left( n \right)} = n^{\frac{1}{\log_{\frac{1}{2}\delta_{0}\Delta}\left( \Delta -1 \right)}} \\ 
    & \log_{\frac{1}{2}\delta_{0}\Delta}\left( \Delta -1 \right) = \frac{\ln\left( \Delta-1 \right)}{\ln\frac{1}{2}\delta_{0}+ \ln \Delta} \rightarrow_{\Delta\rightarrow \infty} 1
  \end{split}
\end{equation*} So for every $\varepsilon >0 $ there exist $\Delta_{\varepsilon}$ such that for $\Delta > \Delta_{\varepsilon}$ we get: $\omega^{\frac{3}{4}g} \ge n^{\frac{1}{1-\varepsilon}}$, and for any $\varepsilon < 1$ it holds that: $ n^{\frac{1}{1-\varepsilon}}\ge n^{1-\varepsilon}$ $\square$

\subsection{Tanner Code Testability For Small ($ \Delta^{\frac{g}{2}} $) Errors.} 
  \ctt{Consider $T$ at a height less than the graph diameter.  } 
  \subsection{Tanner Code Testability When With strongly-Honest Vertices. }
  The result obtained by the proof of Theorem 1 is that one cannot hide errors in a sublinear size tree. Namely, any codeword of the disagreement code with a weight less than $ \Omega \left( n^{1-\varepsilon} \right) $ could be reduced, regardless of the parameters of the small code. 
  \paragraph{Theorem 1-.} For every $\delta_{0}, \varepsilon > 0$, there exist $\Delta_{\varepsilon}\in \mathbb{N}$ such that if $\Delta \ge \Delta_{\varepsilon}$, then for every $x \in C_{\oplus}$ at Hamming weight at most $ \Theta \left( n^{1-\varepsilon} \right) $, there exists a vertex $v \in V$ and a small codeword $c_{v} \in C_{0} $ such that adding the assignment of it over the $v$'s edges to $x$ define the codeword $x + c_{v}$  such that $|x + c_{v}| < |x|$.  
  \paragraph{Proof:} Proof, Define $T$ again to be the DAG constructed as in the proof of Theorem 1 and denote by $U_{\max}$ the layer at the maximal size. The analyses of the flux projected by $x$ over the vertices suggested trivial code words remain the same. But now, as $U_{\max} \subset \text{supp}(x)$ follows that $|U_{\max}| \le 2|x| \le O(n^{1-\varepsilon})$. Therefore, $ \lim_{n \rightarrow \infty} \frac{|U_{\max}|}{n} \rightarrow 0 $ and we have that for sufficiently large $n$ it holds that: 

  \begin{equation*}
    \begin{split}
      w_{E/E^{\prime}}\left( x \right) & \ge \left( \delta_{0} - \frac{|U_{\max}|}{n} - \frac{\lambda}{\Delta} \right) \Delta|T|  \\ 
      \rightarrow_{n \rightarrow \infty} &  \ge  \left( \delta_{0} - o(1)  - \frac{\lambda}{\Delta} \right) \Delta|T| 
    \end{split}
  \end{equation*}
  Assuming Girth is greater than $\frac{4}{3}\log_{\Delta-1}\left( n \right)$  gives us a decoder against errors at weight at most $n^{\frac{3}{2}-\varepsilon}$ $\square$. 


  \subsection{ Almost LTC With Positive Rate \ctt{ Wrong Section.  } } 
  To overcome the vanishing rate issue, let us restrict ourselves to using only even expander graphs. Now, Let $C_{0}^{-}$ and $C_{0}^{+}$ be two small codes with parameters $C_{0}^{\pm} = [\Delta, \rho_{\pm}\Delta, \delta_{\mp}\Delta]$. We will refer to $C_{0}^{+}$ as the positive and $C_{0}^{-}$ as the negative codes. Similarly, we will call the positive vertices to the left side of $G$ and the negative vertices to the right side of $G$ and use the notation $G = \left( V_{-}, V_{+}, E \right)$ when $V = V_{-} \cup V_{+}$. 
  Consider the code $C$, which consists of all the words $x\in C$ such that $x|_{v} \in C_{0}^{\pm}$ for any $v \in V_{\pm}$.
  \paragraph{Claim.} If $\rho_{-} + \rho_{+} > 1$, $\delta_{\pm}\Delta > \lambda$ and $|V_{-}| = |V_{+}| = \frac{1}{2}n$ then the $C$ has positive rate, and relative distance. 
  \paragraph{Proof.} First, let us bound the rate, 
  \begin{equation*}
    \begin{split}
      \dim C & \ge \frac{1}{2}\Delta n - \left( 1 - \rho_{+} \right)\Delta|V_{+}| - \left( 1 - \rho_{-} \right)\Delta|V_{-}| \\
      & \frac{1}{2}\Delta n \left( \rho_{-} + \rho_{+}  - 1  \right) 
    \end{split}
  \end{equation*}
  For proving a linear distance, fix a codeword $x \in C$ and denote By $S_{\pm}$ the support of $x$ over the edges. Namely, a vertex $v\in V_{\pm}$ belongs to $S_{\pm}$ if it connects to nonzero edges regarding the assignment by $x$, Assume towards contradiction that $|x| = o\left( n \right)$ . By The Expander Mixining Lemma, we have that: 
  \begin{equation*}
    \begin{split}
      \frac{E\left( S_{\pm}, S_{\mp} \right)}{|S_{\pm}|}\le\frac{\Delta}{n}|S_{\mp}| + \lambda\sqrt{\frac{|S_{\mp}|}{|S_{\pm}|}}
    \end{split}
  \end{equation*}
  Notice $|x| < |S_{-}| + |S_{+}| < 2|x|$ and $|S_{\pm}| < \Delta |S_{\mp}|$, so if $S_{-}$ is sublinear, then $S_{+}$ is also sublinear. In addition, we can assume, without loss of generality, that $|S_{-}| < |S_{+}|$. Hence, we have that the average of nontrivial incoming edges to a vertex in $|S_{+}|$ is less than $ \Delta \frac{|S_{-}|}{n} + \lambda $. That implies that for any $x = o\left( n \right)$ there must be at least a single vertex $ v \in S_{+} $ with less than $ \lambda $ nontrivial incoming edges; in other words, the weight of the word seen by $v$ is less than $\lambda$, which contradicts the assumption that $x\in C$ $\square$

  \paragraph{Theorem 1+.} For every $\varepsilon > 0$, there exist $\Delta_{\varepsilon}\in \mathbb{N}$ and $\alpha>0$ such that if $\Delta \ge \Delta_{\varepsilon}$, then for every $x \in C_{\oplus}$ at Hamming weight at most $\alpha |E|^{1-\varepsilon}$, there exists a vertex $v \in V_{\pm}$ and a small codeword $c_{v} \in C_{0}^{\pm} $ such that adding the assignment of it over the $v$'s edges to $x$ define the codeword $x + c_{v}$  such that $|x + c_{v}| < |x|$.  

  \paragraph{Proof:} Define the $T$ precisely as is in the proof of Theorem 1, And assume, without loss of generality $\delta_{-} < \delta_{+}$. Hence it is clear that for any $v \in V^\prime$ it holds that $w_{E^\prime}\left( c_{v} \right) > \frac{1}{2}\delta_{-}\Delta$, so we could repeat exactly the above steps to get that the $T$ has a linear size and the inequality $|x| > \left( \delta_{-} - \frac{|U_{\max}|}{n} - \frac{\lambda}{\Delta} \right) $ still holds, So we can assume that $|U_{\max}|$ is also linear at $n$.

  Furthermore, if $U_{\max} \subset V_{-}$ then the inequality $\left( \delta_{+} - \frac{2}{3} - \frac{2}{\Delta}\lambda \right) \Delta |U_{\max}|$ also holds and the exact arguments as before proving the theorem.     


  Suppose that each negative layer $U_{i}^{-} \subset V_{-}$ is at the size at most $\sqrt{\Delta}n$, then we have that for any positive layer $U_{i}^{+}$ it holds that:

%\begin{equation*}
 % \begin{split}
  %  w_{E/E^{\prime}\left( x|_{U_{i}^{+}} \right)  \le \frac{1}{2}\delta_{-}\Delta
 % \end{split}
%\end{equation*}



  Therefore consider the case when $U_{\max} \subset V_{+}$. Denote by the $U_{-1}$ and $U_{+1}$ the previews and the next layers to $U_{\max}$.  
  \paragraph{Claim.} If $ |U_{\max}| = \Theta\left(n  \right)$ Then there exists at least $3$ layers in $V_{-}$ at linear size. And at least one of them is not adjusted to $U_{\max}$.    


