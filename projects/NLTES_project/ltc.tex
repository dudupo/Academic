\documentclass{article}
\usepackage[utf8]{inputenc}
\usepackage[a4paper, total={7in, 10in}]{geometry}
\usepackage{braket}
\usepackage{xcolor}
\usepackage{amsmath}
\usepackage{amssymb}
\usepackage{amsfonts}
\usepackage{graphicx}
\usepackage{svg}
\usepackage{float}
\usepackage{tikz}
\usepackage[ruled,vlined]{algorithm2e}
\usepackage{multicol}
\usepackage[backend=biber,style=alphabetic,sorting=ynt]{biblatex}

%\addbibresource{sample.bib} %Import the bibliography file

\newcommand{\commentt}[1]{\textcolor{blue}{ \textbf{[COMMENT]} #1}}
\newcommand{\ctt}[1]{\commentt{#1}}
\newcommand{\prb}[1]{ \mathbf{Pr} \left[ {#1} \right]}
\newcommand{\onotation}[1]{\(\mathcal{O} \left( {#1}  \right) \)}
\newcommand{\ona}[1]{\onotation{#1}}
\newcommand{\PSI}{{\ket{\psi}}}
\newcommand{\LESn}{\ket{\psi_n}}
\newcommand{\LESa}{\ket{\phi_n}}
\newcommand{\LESs}{\frac{1}{\sqrt{n}}\sum_{i}{\ket{\left(0^{i}10^{n-i}\right)^{n}}}}
\newcommand{\Hn}{\mathcal{H}_{n}}
\newcommand{\Ep}{\frac{1}{\sqrt{2^n}}\sum^{2^n}_{x}{ \ket{xx}}}
\newcommand{\HON}{\ket{\psi_{\text{honest}}}}
\newcommand{\Lemma}{\paragraph{Lemma.}}


\setlength{\columnsep}{0.6cm}

\newcommand{\Gz}{ G_{z}^{\delta} } 


\begin{document}

\title{Quantum LTC With Positive Rate}
\author{David Ponarovsky}
\maketitle
\begin{multicols*}{2}


  \paragraph{preamble.} preamble.  
  \paragraph{The Construction.} Fix primes $q,p_1,p_2,p_3$ such that each of them has $1 $ residue mode $4$. Let $A_{1},A_{2},A_{3}$ be a different generators sets of $ \mathbf{PGL}(2 , \mathbb{Z} / q\mathbb{Z} )  $ 
  obtained by taking the solutions for $a_{0}^{2} + a_{1}^{2} +a_{2}^{2} +a_{3}^{2} = p_i $ such that each pair $A_i,A_j$ satisfy the 
  TNC constraint. Then consider the graphs: ($G$ is the \textbf{PGL}$\times \mathbb{Z}_2$ group).  
  \begin{equation*}
    \begin{split}
      \Gamma_{1}  &= Cay_{2}\left(  G, A_{1} \right)\times_{G} Cay_{2}\left(  G, A_{2} \right) \\
      \Gamma_{2}  &= Cay_{2}\left(  G, A_{1} \right)\times_{G} Cay_{2}\left(  G, A_{3} \right) \\
      \Gamma_{\square_{1}} &= \left( G, \left\{ \left( g, agb \right) : a \in A_{1}, b \in A_{2} \right\}  \right) \\
      \Gamma_{\square_{2}} &= \left( G, \left\{ \left( g, agc \right) : a \in A_{1}, c \in A_{3} \right\}  \right) \\
      \Gamma_{\square \square} &= \left( G, \left\{ \left( g, gb, agc \right), \left( g, gc, agb \right) :
      a \in A_{1}, b \in A_{2}, c \in A_{3} \right\}  \right) 
    \end{split}
  \end{equation*}
   Then define the codes:
	\begin{equation*}
	  \begin{split}
	    C_{z}^{\perp} & = \mathcal{T}\left( \Gamma_{\square_{1}}, C_{A_1}^{\perp} \otimes  C_{A_2}^{\perp}  \right) \\
	    & \ \ | \ \mathcal{T}\left(  \Gamma_{\square_{2}}, C_{A_1}^{\perp} \otimes C_{A_3}^{\perp}  \right) \\
	    C_{x} &=  \mathcal{T}\left(  \Gamma_{\square_{1}}, \left(  C_{A_1} \otimes C_{A_2} \right)^{\perp}  \right) \\
	    & \ \ | \ \mathcal{T}\left( \Gamma_{\square_{2}}, \left(  C_{A_1} \otimes C_{A_3} \right)^{\perp}  \right) \\
	    C_{w} &=  \mathcal{T}\left( \Gamma_{\square \square}, \left(  C_{A_1} \otimes C_{A_2} \otimes C_{A_3} \right)^{\perp}  \right)   
	  \end{split}
	\end{equation*}
	Notice that the faces of $\Gamma_{\square_{1}},\Gamma_{\square_{2}}$ are disjointed and here the symbol $|$ means just joint them together. 
	The main focus here is to prove local test-ability for computation base (i.e $C_{x}$) and for completeness one also must to define the code 
	\begin{equation*}
	  \begin{split}
	    C_{w_{z}} &=  \mathcal{T}\left( \Gamma_{\square \square}, \left(  C_{A_1}^\perp \otimes C_{A_2}^\perp \otimes C_{A_3}^\perp \right)^{\perp}  \right)   
	  \end{split}
	\end{equation*}
	\paragraph{What We Currently Have.} Given a candidate for a codeword $c$ we could check efficiently if $c\in C_{z}^\perp$.  
	Additionally summing up the local correction of each vertex in $C_{x}$ yields a codeword in $C_{w}$. Now we would want to show 
	something similar to property 1 in Levarier and Zemor which imply that any codeword of $C_{w}$ with weigh beneath 
	a linear threshold $\eta n $ must to be also in $C_{X}$. (And therefore we can reject candidates with high weight). 
	%(And maybe we would also say something about the gap between the this minimal weight of $C_{w}/C_{x}$. 

	Assume that we have succeed to do so, Then the testing protocol will be looked as follow, 
	first we check that the candidate is not in $C_{z}^\perp$ and then we check that is indeed in $C_{x}$. And repeat again 
	in the phase base. Then there are constants $\kappa_1, \kappa_2$ 
	\begin{equation*}
	  \begin{split}
	    \text{accept} & \sim \kappa_1 \cdot  d\left( c,C_{z}^\perp \right)  \\ 
	    & +  \left[ 1 -  \kappa_1 \cdot  d\left( c,C_{z}^\perp \right)\right] \kappa_{2} d\left( c, C_{x} \right) \\
	    \text{reject} & \sim  \left[ 1 -  \kappa_1 \cdot  d\left( c,C_{z}^\perp \right)\right] \\ 
	    & +\kappa_1 \cdot  d\left( c,C_{z}^\perp \right) \cdot \left[ 1 - \kappa_{2} d\left( c, C_{x} \right) \right] 
	  \end{split}
	\end{equation*}
	\paragraph{Disclaimer.} The use of the $\sim$ was  made by purpose. The above should be formalize by inequalities. (And this also make another problem as 
	the term $ 1 - \kappa_{1} \cdot d\left(  \right) $ is in the opposite direction). 
	\paragraph{The Hard Part.} It seems (at least for now) that the hard part is to find an analog for Lemma 1 in Levrier-Zemor, Which can formalize 
	as follow: Consider a codeword $c \in C_{w}$ such that $|c| \le \eta n $ then we could always find a vertex in $\Gamma_{\square_{1}} $
	and a local codeword $\xi \in C_{A_1} \otimes c_{A_2} $ on his support such that $|c + \xi| < |c| $.     
      

	\paragraph{Tasks.}
	\begin{enumerate}
	  \item Prove that $\Gamma_{\square \square } $ is indeed an expander. Should be (relative) easy.
	  \item Prove a Lemma 1 analogy. And while do so, understand what are the properties we should require from the small code.
	    (i.e w-robustness and p-resistance for puncturing). 
	  \item Show that we could actually choose such $\left\{ A \right\}_{i}$ and the matched small codes.
	  \item Understand what it mean quantomlly test if a $c \in C_{w}/ C_{x}$. Namely, is weight counting can be consider as 
	    $X-$check which commute with the other $Z-$checks? 
	  \item Write a program which plot small complex in a small scale for getting more intuition. 
	\end{enumerate}
	% be a codeword with weight less then $\alpha n$.

	\newcommand{ \Aa } { C_{A_{1}}}  
	\newcommand{ \Ab } { C_{A_{2}}}
	\newcommand{ \Ac } { C_{A_{3}}}
	\newcommand{ \Aab } { \Aa \otimes \Ab } 
	\newcommand{ \Aac } { \Aa \otimes \Ac }
	\newcommand{ \Aabc } { \Aa \otimes \Ab \otimes \Ac }
	\newcommand{ \YY } {  y_{1}y_{2}^{\top} }
	\newcommand{ \ZZ } {  z_{1}z_{2}^{\top} } 
	\newcommand{ \TT } { \tilde{\tau} } 

	\paragraph{All The Verticis Are Normal } Define a noraml vertex in $ V_{1} $ to be a vertex such his local view (a codeword in a dual tensor code). 
	supported on less then $w = \Delta^\frac{3}{2}$ faces.
	Consider the code $C_{w}$ defined above, and assume in addition that the diststance and the rate of 
	the small codes $C_{A_{j}}$, $\delta \Delta$ satisfay the eqution $ r\left(1 - \delta\right) < \frac{1}{2}\delta^{3} $ and also the code $ C_{A_{1}}  $ 
      contains the word $ 1^{\Delta} $.


      Then for any $x \in C_{w}$ such that all the vertices in the induced graphs $\Gamma_{\square_{1}},\Gamma_{\square_{2}}$  by it are noramls. 
	Then there exists a vertex $ g \in V_{0} $ and a local codeword $ c \in \Aabc $ supported entirly on the neighberhood of $ g $ such that: 
	$ |x + c| \le |x| $.


	\paragraph{Proof.} Let $g$ be an aribtrary vertex in $V_{0}$ we know by Leverrir and Zemor that the local views of $g$ in  $\Gamma_{\square_{1}},\Gamma_{\square_{2}}$ are $\Delta^{3/2}$ close to 
      $\Aab $ and $ \Aac $ by the $w$-robustness propperty. 

	So we can repreasnt the locals views on $g$ as the following disjointess vectors, each lays on $\Gamma_{\square_1},\Gamma_{\square_2}$:
	\begin{equation*}
	  \begin{split}
	    y &= y_{1}y_{2}^{\top} + \xi_{y} \\ 
	    z &= z_{1}z_{2}^{\top} + \xi_{z}
	  \end{split}
	\end{equation*}
	such that $ y_{1}y_{2}^{\top} \in \Aab $,  $z_{1}z_{2}^{\top}\in \Aac $ and the $\xi_{y}, \xi_{z} $ are the coressponded errors of the local views from the tensor codes. 
 
	Let $ \{ y_{1}^{j}y_{2}^{i \ \top} \} ,\{ z_{1}^{j}z_{2}^{i \ \top} \}  $ be the bases for $ \Aab $ and $ \Aac $ such that $ y_{1}^{j}, z_{1}^{j} \in \Aa $ and $ y_{2}^{i} \in \Ab, z_{2}^{i} \in \Ac $.  
	And denote by $\alpha_{ij},\beta_{ij} \in \mathbb{F}_2 $ the coffeicents of $\YY$ and $\ZZ$. 

	By the fact that $1^\Delta \in \Aa$ we have that for any $i,j$ the vecotr: 
	\begin{equation*}
	  \begin{split}
	  \bar{y_{1}}^{j}y_{2}^{i \ \top} & =  1^\Delta y_{2}^{i \ \top} \\ 
	    & + y_{1}^{j}y_{2}^{i \ \top} = \left(1^\Delta +  y_{1}^{j} \right)y_{2}^{i \ \top} \\
	    & \in \Aab
	  \end{split}
	\end{equation*}
      And by the same calculation we get also that $ \bar{z_{1}}^{j}z_{2}^{i \ \top} \in \Aac$.

      \paragraph{Claim.} Assume that $ \YY $ and $ \ZZ $ are in the  bases defined above. Let $\tau \in \mathbb{F}_{2}^{A\times B\times C} $ such that $ \tau_{abc} = \left( \YY \right)_{ab}\left( \ZZ \right)_{ac} $ then: 
      \begin{equation*}	
      d\left( \tau, \Aabc \right) \le \left( 1-\delta \right)\Delta^{3}  
      \end{equation*}
      \paragraph{Proof.} First noitce that $y_{1a}y_{2b}z_{2c} $ is a valid codeword of $\Aabc$. That beacuse that the projection obtained bt 
      fixing any two coordinates yields ethier a zero or a codeword of one the codes.
      
      Therefore we could consider the following codeword $ \TT_{abc} = \left( y_{1a} + \bar{z}_{1a} \right)y_{2b}y_{2c} $ and bounding the distance of $\tau$ by 
      \begin{equation*}
	\begin{split}
      d \left( \tau, \Aabc \right) & \le d \left( \tau, \TT \right) \\ 
      = \sum_{abc}{   \left( y_{1a} + \bar{z}_{1a} \right)y_{2b}y_{2c} \oplus  \left( y_{1a}z_{1a} \right)y_{2b}y_{2c}  }
	\end{split}
      \end{equation*}



      Consider the local
      \begin{equation*}
	\begin{split}
	  x_{abc} &= y_{ab}z_{ac} = \left( \YY + \xi_{y} \right)_{ab}   \left( \ZZ + \xi_{z} \right)_{ac} \\ & \left( y_{1}y_{2}^{\top}\right)_{ab}\left(z_{1}z_{2}^{\top} \right)_{ac} = 
	  \left( \sum_{ij} { \alpha_{ij} y_{1}^{i}y_{2}^{j  \top}}  \right)_{ab}\left(   \sum_{ij} { \beta_{ij} z_{1}^{i}z_{2}^{j  \top}} \right)_{ac}\\ 
	  &= \sum_{ijlk} { \alpha_{ij}\beta_{lk} y_{1a}^{i}y_{2b}^{j \top}} z_{1a}^{l}z_{2c}^{k \top} 
	\end{split}
      \end{equation*}
	\paragraph{Lemma} There exists $u \in C_{A_{1}}, v \in C_{A_{2}}, w \in C_{A_{3}}$ such that   
\end{multicols*}
  % \printbibliography 
\end{document}


