\documentclass{article}
\usepackage[utf8]{inputenc}
\usepackage[a4paper, total={7in, 10in}]{geometry}
\usepackage{braket}
\usepackage{xcolor}
\usepackage{amsmath}
\usepackage{amssymb}
\usepackage{amsfonts}
\usepackage{graphicx}
\usepackage{svg}
\usepackage{float}
\usepackage{tikz}
\usepackage[ruled,vlined]{algorithm2e}
\usepackage{multicol}
\usepackage[backend=biber,style=alphabetic,sorting=ynt]{biblatex}
\usepackage{xcolor}
%\addbibresource{sample.bib} %Import the bibliography file

\newcommand{\commentt}[1]{\textcolor{blue}{ \textbf{[COMMENT]} #1}}
\newcommand{\ctt}[1]{\commentt{#1}}
\newcommand{\prb}[1]{ \mathbf{Pr} \left[ {#1} \right]}
\newcommand{\onotation}[1]{\(\mathcal{O} \left( {#1}  \right) \)}
\newcommand{\ona}[1]{\onotation{#1}}
\newcommand{\PSI}{{\ket{\psi}}}
\newcommand{\LESn}{\ket{\psi_n}}
\newcommand{\LESa}{\ket{\phi_n}}
\newcommand{\LESs}{\frac{1}{\sqrt{n}}\sum_{i}{\ket{\left(0^{i}10^{n-i}\right)^{n}}}}
\newcommand{\Hn}{\mathcal{H}_{n}}
\newcommand{\Ep}{\frac{1}{\sqrt{2^n}}\sum^{2^n}_{x}{ \ket{xx}}}
\newcommand{\HON}{\ket{\psi_{\text{honest}}}}
\newcommand{\Lemma}{\paragraph{Lemma.}}


\setlength{\columnsep}{0.6cm}

\newcommand{\Gz}{ G_{z}^{\delta} } 

\begin{document}

\title{Quantum LTC With Positive Rate}
\author{David Ponarovsky}
\maketitle
\begin{multicols*}{2}
\newcommand{ \Hw }{ \delta\Delta -\Delta^{\frac{1}{2}-\varepsilon}/\delta  }
	\newcommand{ \Nw }{ \Delta^{\frac{3}{2}-\varepsilon}} 
	  \newcommand{ \Gu } { \Gamma^{\cup} }
	  \newcommand{ \Guq } { \Gamma^{\cup, \square} }

    	\newcommand{ \Gsa } {\Gamma_{\square_{1}} }
	\newcommand{ \Gsb } {\Gamma_{\square_{2}} }
        \newcommand{ \Aa } { C_{A_{1}}}  
	\newcommand{ \Ab } { C_{A_{2}}}
	\newcommand{ \Ac } { C_{A_{3}}}
	\newcommand{ \Aab } { \Aa^{\perp} \otimes \Ab^{\perp} } 
	\newcommand{ \Aac } { \Aa^{\perp} \otimes \Ac^{\perp} }
	\newcommand{ \Aabc } { \Aa^{\perp} \otimes \Ab^{\perp} \otimes \Ac^{\perp} }
	\newcommand{ \YY } {  y_{1}y_{2}^{\top} }
	\newcommand{ \ZZ } {  z_{1}z_{2}^{\top} } 
	\newcommand{ \TT } { \tilde{\tau} } 


  \paragraph{preamble.} preamble.  
  \paragraph{The Construction.} Fix primes $q,p_1,p_2,p_3$ such that each of them has $1 $ residue mode $4$. Let $A_{1},A_{2},A_{3}$ be a different generators sets of $ \mathbf{PGL}(2 , \mathbb{Z} / q\mathbb{Z} )  $ 
  obtained by taking the solutions for $a_{0}^{2} + a_{1}^{2} +a_{2}^{2} +a_{3}^{2} = p_i $ such that each pair $A_i,A_j$ satisfy the 
  TNC constraint. Then consider the graphs: ($G$ is the \textbf{PGL}$\times \mathbb{Z}_2$ group).  
  \begin{equation*}
    \begin{split}
      \Gamma_{1}  &= Cay_{2}\left(  G, A_{1} \right)\times_{G} Cay_{2}\left(  G, A_{2} \right) \\
      \Gamma_{2}  &= Cay_{2}\left(  G, A_{1} \right)\times_{G} Cay_{2}\left(  G, A_{3} \right) \\
      \Gamma_{\square_{1}} &= \left( G, \left\{ \left( g, agb \right) : a \in A_{1}, b \in A_{2} \right\}  \right) \\
      \Gamma_{\square_{2}} &= \left( G, \left\{ \left( g, agc \right) : a \in A_{1}, c \in A_{3} \right\}  \right) \\
      \Gamma_{\square \square} &= \left( G, \left\{ \left( g, gb, agc \right), \left( g, gc, agb \right) :
      a \in A_{1}, b \in A_{2}, c \in A_{3} \right\}  \right) 
    \end{split}
  \end{equation*}
   Then define the codes:
	\begin{equation*}
	  \begin{split}
	    C_{z}^{\perp} & = \mathcal{T}\left( \Gamma_{\square_{1}}, C_{A_1} \otimes  C_{A_2}  \right) \\
	    & \ \ | \ \mathcal{T}\left(  \Gamma_{\square_{2}}, C_{A_1} \otimes C_{A_3}  \right) \\
	    C_{x} &=  \mathcal{T}\left(  \Gamma_{\square_{1}}, \left(  C_{A_1}^{\perp} \otimes C_{A_2}^{\perp} \right)^{\perp}  \right) \\
	    & \ \ | \ \mathcal{T}\left( \Gamma_{\square_{2}}, \left(  C_{A_1}^{\perp} \otimes C_{A_3}^{\perp} \right)^{\perp}  \right) \\
	    C_{w} &=  \mathcal{T}\left( \Gamma_{\square \square}, \left(  C_{A_1}^{\perp} \otimes C_{A_2}^{\perp} \otimes C_{A_3}^{\perp} \right)^{\perp}  \right)   
	  \end{split}
	\end{equation*}
	Notice that the faces of $\Gamma_{\square_{1}},\Gamma_{\square_{2}}$ are disjointed and here the symbol $|$ means just joint them together. 
	The main focus here is to prove local test-ability for computation base (i.e $C_{x}$) and for completeness one also must to define the code 
	\begin{equation*}
	  \begin{split}
	    C_{w_{z}} &=  \mathcal{T}\left( \Gamma_{\square \square}, \left(  C_{A_1} \otimes C_{A_2} \otimes C_{A_3} \right)^{\perp}  \right)   
	  \end{split}
	\end{equation*}
	%\newcommand{}
	\paragraph{Definition} Define the mapping (not linear)  
	\begin{equation*}	
	\phi : \mathcal{T}\left(\Gsa \cup \Gsb , \mathbb{F}_2  \right) \rightarrow  \mathcal{T}\left( \Gamma_{\square \square}, \mathbb{F}_{2} \right)
      \end{equation*} 
      as the summtion over the fowlloing local maps $\phi_{g}$. which for given vertex $g \in V\left( \Gamma_{\square \square}\right)$ with local view $c_1$ on $\Gsa$ and local view $c_2$ on $\Gsb$ compute the tensor $c_{abc} = c_1_{ab}c_2_{ac}$ and set result bit on the plaquette defined by the vertices $ g, ag, gb, gc, agb, agc$.    

      We will abuse the notation by defining for every subset of vertices $S \subset V $  the map $\phi_S = \sum_{g \in S} \phi_{g}$. 

      \paragraph{Lemma 1} Fix a vertex $g$ and assume that the local views $c_1,c_2$ that lay over the graphs $\Gsa, \Gsb$ belongs to the dual tensors $\Cab, \Cac$. And inaddtion $ 

	\paragraph{What We Currently Have.} Given a candidate for a codeword $c$ we could check efficiently if $c\in C_{z}^\perp$.  
	Additionally summing up the local correction of each vertex in $C_{x}$ yields a codeword in $C_{w}$. Now we would want to show 
	something similar to property 1 in Levarier and Zemor which imply that any codeword of $C_{w}$ with weigh beneath 
	a linear threshold $\eta n $ must to be also in $C_{X}$. (And therefore we can reject candidates with high weight). 
	%(And maybe we would also say something about the gap between the this minimal weight of $C_{w}/C_{x}$. 

	Assume that we have succeed to do so, Then the testing protocol will be looked as follow, 
	first we check that the candidate is not in $C_{z}^\perp$ and then we check that is indeed in $C_{x}$. And repeat again 
	in the phase base. Then there are constants $\kappa_1, \kappa_2$ 
	\begin{equation*}
	  \begin{split}
	    \text{accept} & \sim \kappa_1 \cdot  d\left( c,C_{z}^\perp \right)  \\ 
	    & +  \left[ 1 -  \kappa_1 \cdot  d\left( c,C_{z}^\perp \right)\right] \kappa_{2} d\left( c, C_{x} \right) \\
	    \text{reject} & \sim  \left[ 1 -  \kappa_1 \cdot  d\left( c,C_{z}^\perp \right)\right] \\ 
	    & +\kappa_1 \cdot  d\left( c,C_{z}^\perp \right) \cdot \left[ 1 - \kappa_{2} d\left( c, C_{x} \right) \right] 
	  \end{split}
	\end{equation*}
	\paragraph{Disclaimer.} The use of the $\sim$ was  made by purpose. The above should be formalize by inequalities. (And this also make another problem as 
	the term $ 1 - \kappa_{1} \cdot d\left(  \right) $ is in the opposite direction). 
	\paragraph{The Hard Part.} It seems (at least for now) that the hard part is to find an analog for Lemma 1 in Levrier-Zemor, Which can formalize 
	as follow: Consider a codeword $c \in C_{w}$ such that $|c| \le \eta n $ then we could always find a vertex in $\Gamma_{\square_{1}} $
	and a local codeword $\xi \in C_{A_1} \otimes c_{A_2} $ on his support such that $|c + \xi| < |c| $.     
      

	\paragraph{Tasks.}
	\begin{enumerate}
	  \item Prove that $\Gamma_{\square \square } $ is indeed an expander. Should be (relative) easy.
	  \item Prove a Lemma 1 analogy. And while do so, understand what are the properties we should require from the small code.
	    (i.e w-robustness and p-resistance for puncturing). 
	  \item Show that we could actually choose such $\left\{ A \right\}_{i}$ and the matched small codes.
	  \item Understand what it mean quantomlly test if a $c \in C_{w}/ C_{x}$. Namely, is weight counting can be consider as 
	    $X-$check which commute with the other $Z-$checks? 
	  \item Write a program which plot small complex in a small scale for getting more intuition. 
	\end{enumerate}
	% be a codeword with weight less then $\alpha n$.

	
	\paragraph{All The Vertices Are Normal } Define a normal vertex in $ V_{1} $ to be a vertex such his local view (a codeword in a dual tensor code). 
	supported on less then $w = \Delta^\frac{3}{2}$ faces.
	Consider the code $C_{w}$ defined above, and assume in addition that the distance and the rate of 
	the small codes $C_{A_{j}}$, $\delta \Delta$ satisfy the equation $ \left(\Delta r\right)^{4}\left(1 - \color{red}2\color{black}\delta\right) < \frac{1}{2}\delta^{3} $ and also the code $ C_{A_{1}}  $ 
      contains the word $ 1^{\Delta} $.


      Then for any $x \in C_{w}$ such that all the vertices in the induced graphs $\Gamma_{\square_{1}},\Gamma_{\square_{2}}$  by it are normal. 
	Then there exists a vertex $ g \in V_{0} $ and a local codeword $ c \in \Aabc $ supported entirely on the neighborhood of $ g $ such that: 
	$ |x + c| \le |x| $.


	\paragraph{Proof.} Let $g$ be an arbitrary vertex in $V_{0}$ we know by Leverrir and Zemor that the local views of $g$ in  $\Gamma_{\square_{1}},\Gamma_{\square_{2}}$ are $\Delta^{3/2}$ close to 
      $\Aab $ and $ \Aac $ by the $w$-robustness property. 

	So we can represent the locals views on $g$ as the following disjointed vectors, each lays on $\Gamma_{\square_1},\Gamma_{\square_2}$:
	\begin{equation*}
	  \begin{split}
	    y &= y_{1}y_{2}^{\top} + \xi_{y} \\ 
	    z &= z_{1}z_{2}^{\top} + \xi_{z}
	  \end{split}
	\end{equation*}
	such that $ y_{1}y_{2}^{\top} \in \Aab $,  $z_{1}z_{2}^{\top}\in \Aac $ and the $\xi_{y}, \xi_{z} $ are the corresponded errors of the local views from the tensor codes. 
 
	Let $ \{ y_{1}^{j}y_{2}^{i \ \top} \} ,\{ z_{1}^{j}z_{2}^{i \ \top} \}  $ be the bases for $ \Aab $ and $ \Aac $ such that $ y_{1}^{j}, z_{1}^{j} \in \Aa $ and $ y_{2}^{i} \in \Ab, z_{2}^{i} \in \Ac $.  
	And denote by $\alpha_{ij},\beta_{ij} \in \mathbb{F}_2 $ the coefficients of $\YY$ and $\ZZ$. 

	By the fact that $1^\Delta \in \Aa$ we have that for any $i,j$ the vector: 
	\begin{equation*}
	  \begin{split}
	  \bar{y_{1}}^{j}y_{2}^{i \ \top} & =  1^\Delta y_{2}^{i \ \top} \\ 
	    & + y_{1}^{j}y_{2}^{i \ \top} = \left(1^\Delta +  y_{1}^{j} \right)y_{2}^{i \ \top} \\
	    & \in \Aab
	  \end{split}
	\end{equation*}
      And by the same calculation we get also that $ \bar{z_{1}}^{j}z_{2}^{i \ \top} \in \Aac$.

      \paragraph{Claim.} Assume that $ \YY $ and $ \ZZ $ are in the  bases defined above. Let $\tau \in \mathbb{F}_{2}^{A\times B\times C} $ such that $ \tau_{abc} = \left( \YY \right)_{ab}\left( \ZZ \right)_{ac} $ then: 
      \begin{equation*}	
      d\left( \tau, \Aabc \right) \le \left( 1-\delta \right)\Delta^{3}  
      \end{equation*}
      \paragraph{Proof.} First notice that $y_{1a}y_{2b}z_{2c} $ is a valid codeword of $\Aabc$. That because that the projection obtained by 
      fixing any two coordinates yields either a zero or a codeword of one the codes.
      
      Therefore we could consider the following codeword $ \TT_{abc} = \left( y_{1a} + \bar{z}_{1a} \right)y_{2b}y_{2c} $ and bounding the distance of $\tau$ by 
      \begin{equation*}
	\begin{split}
      & d \left( \tau, \Aabc \right)  \le d \left( \tau, \TT \right) \\ 
      & = \sum_{abc}{   \left( y_{1a} + \bar{z}_{1a} \right)y_{2b}y_{2c} \oplus  \left( y_{1a}z_{1a} \right)y_{2b}y_{2c}  } \\
      & =  \sum_{abc}{   \left( y_{1a} + \bar{z}_{1a} \oplus y_{1a}z_{1a}  \right)y_{2b}y_{2c}  } \\
      & \le | \left\{ y_{1a} = 0 \text{ and }  z_{1a} = 0   \right\} | \cdot \Delta^{2} \le \left( 1-\color{red}2\color{black}\delta \right)\Delta^{3} 
	\end{split}
      \end{equation*}

      \paragraph{Claim.} Let $\YY, \ZZ$ be codewords in $\Aab, \Aac$. And let $w$ be the vector define by $w_{abc} = \left( \YY \right)_{ab} \left( \ZZ \right)_{ac} $. Then  
      \begin{equation*}
	d\left(w, \Aabc \right) \le \left( r\Delta \right)^{4}\left( 1 - \delta \right)\Delta^{3} + \Theta \left( \Delta^{2\frac{1}{2}}  \right)
      \end{equation*}

      Consider again the representation of the local view $w$ on the vertex $g$. 
      \begin{equation*}
	\begin{split}
	  & w_{abc} = y_{ab}z_{ac} = \left( \YY + \xi_{y} \right)_{ab}   \left( \ZZ + \xi_{z} \right)_{ac} \\ 
	  & \left( y_{1}y_{2}^{\top}\right)_{ab}\left(z_{1}z_{2}^{\top} \right)_{ac} = 
	  \left( \sum_{ij} { \alpha_{ij} y_{1}^{i}y_{2}^{j  \top}}  \right)_{ab}\left(   \sum_{ij} { \beta_{ij} z_{1}^{i}z_{2}^{j  \top}} \right)_{ac}\\ 
	  &= \sum_{ijlk} { \alpha_{ij}\beta_{lk} y_{1a}^{i}y_{2b}^{j \top}} z_{1a}^{l}z_{2c}^{k \top} \\
	  & \Rightarrow d\left( \sum_{abc} \left( y_{1}y_{2}^{\top}\right)_{ab}\left(z_{1}z_{2}^{\top} \right)_{ac} , \Aabc \right) \\ 
	  & \ \ \ \ \le \left( \Delta r \right)^{4} \left( 1-\delta \right)\Delta^{3} \\ & 
	  %+  \left( \xi_{y} z_{1}z_{2}^\top + \xi_{y}\xi_{z}, \Aabc   \right)  
	\end{split}
      \end{equation*}
      In addition its clear that $ | \sum_{abc}{\xi_{ab}\left( \ZZ + \xi\right)_{ac}} | \le  \sum_{c}\sum_{ab}{|\xi_{ab}|} \le \Delta^{ 2\frac{1}{2}} $. Hence, we have that 
  \begin{equation*}
    \begin{split}
      d\left( w, \Aabc  \right) \le  \left( r\Delta \right)^{4}\left( 1 - \delta \right)\Delta^{3} + \Theta \left( \Delta^{2\frac{1}{2}}  \right)
    \end{split}
  \end{equation*}
  		\paragraph{Dense Normal Net Counting} Let us call the normal vertices the vertices with degree less then $\xi$ in $\Guq = \Gamma_{\square, 1}^{x} \cup \Gamma_{\square, 2}^{x}$. 
    And Let us say that that an edge of $\Gu$ is heavy if it is incident to at least $\eta$ squares in $\Gsa$ and $\Gsb$. Let $ T$ be set of vertices in $V_0$ that are connected to (at least) one normal vertex through a heavy edge. 

    First notice that the number of vertices in the induced graph by $x$ is bounded by it's weight:
    $ |S| \le \frac{2|x|}{ \delta\Delta }$ 

    By the mixing Lemma we get:  
    \begin{equation*}
      \begin{split}
	|E\left( S,T \right)| & \ge \eta |T| \\ 
	|E\left( S,T \right)| & = |E\left( S,T \right)_{\Gamma_{1}} \cup E\left( S,T \right)_{\Gamma_{2}} | \\ 
      \le & \frac{|S||T|}{n}\left( 2 \cdot 2\Delta - \Delta  \right) \\ & + \sqrt{|S||T|}\left( 2\cdot \lambda_{ \text{double cover} } + \lambda_{ \text{ramnujan} }    \right)
      \end{split}
    \end{equation*}
    Hence we have that:
    \begin{equation*}
      \begin{split}
	& |T|\left( \eta - \frac{2|x|}{\delta\Delta}\cdot\frac{3\Delta}{n}  \right)\le \sqrt{|S||T|}\lambda^{\star} \\ 
	& |T| \le \left(\frac{\lambda^{\star}}{\eta - \frac{6|x|}{n \delta}}\right)^{2}|S|
      \end{split}
    \end{equation*}
    Denote by $S_{e}$ the set of vertices in $\Guq$ with degree greater then $\xi$. Then by repeating on the above calculation, while substituting $\Gamma_{i}$ by $\Gamma_{i, \square}$, We obtain that there is $\lambda^{\star}_{2}$ such that:
     \begin{equation*}
      \begin{split}
	& |S_{e}| \le \left(\frac{\lambda^{\star}_{2}}{\xi - \left(2\Delta^{2} - \Delta  \right)\frac{|x|}{n \delta\Delta}}\right)^{2}|S|
      \end{split}
    \end{equation*}
    Define $\bar{d}_{T}$ to be the average (over $T$) of heavy edges incident to a vertex of $T$. So 
    \begin{equation*}
      \begin{split}
	\bar{d}_{T} &= \frac{|E\left( T, S / S_{e} \right)|}{T} \ge \frac{|S| - |S_{e}|}{|T|} \\
	& \ge \left( 1 -   \left(\frac{\lambda^{\star}_{2}}{\xi - \left(2\Delta^{2} - \Delta  \right)\frac{|x|}{n \delta\Delta}}\right)^{2}\right) /  
	  \left(\frac{\lambda^{\star}}{\eta - \frac{6|x|}{n \delta}}\right)^{2}
      \end{split}
    \end{equation*}
    Let us call to the quantity above $\Delta\rho$ and denote by $1 - \tau$ the fraction of vertices of $ T $ with degree less then $\frac{1}{2}\Delta\rho$. 
 Then $ \Delta\rho \le \bar{d}_{T} \le 3\Delta\tau + \left( 1 - \tau \right) \Delta\rho  $ 
 $ \Rightarrow \tau \ge \frac{\rho}{2\left( 3 - \rho \right)}\ge \rho/3 $. Namely, at least $\rho/3$ of vertices of $T$ are incident to at least $\frac{1}{2}\Delta\rho$ heavy edges. 

 Since $\Gu$ is $3\Delta$ regular we get that $|S| - |S_{e}| \le 3\Delta |T| $. In the other-hand we have shown that 
 \begin{equation*}
   \begin{split}
     |S_{e}| & \le \left(\frac{\lambda^{\star}_{2}}{\xi - \left(2\Delta^{2} - \Delta  \right)\frac{|x|}{n \delta\Delta}}\right)^{2}|S| \\ &
     \Rightarrow |S| \le \left( 1 - \left(\frac{\lambda^{\star}_{2}}{\xi - \left(2\Delta^{2} - \Delta  \right)\frac{|x|}{n \delta\Delta}}\right)^{2}\right)^{-1}3\Delta|T| \\
     & = \left( 1 - \theta^2 \right) 3 \Delta |T|
   \end{split}
 \end{equation*}
 And by using again the mixing Lemma we have that: 
 \begin{equation*}
   \begin{split}
     E\left( S_{e},T \right) &\le \frac{\theta^2}{1- \theta^2}3\Delta|T|^2 \frac{3\Delta}{n} + \lambda^{\star}\sqrt{\frac{\theta^2}{1- \theta^2}}|T| \\ 
     & \le  \left( \frac{\theta^2}{1- \theta^2}9\Delta^{2} + \lambda^{\star}\sqrt{\frac{\theta^2}{1- \theta^2}}\right)|T| \\ 
     & \le \left(9\Delta^{2} + \lambda^{\star} \right) |T|
  \end{split}
 \end{equation*}
 Hence at most an $\frac{1}{6}\rho $ proportion of vertices of $T$ are adjacent to more than $\frac{6}{\rho} \left( 9\Delta^{2} + \lambda^{\star}  \right) $ vertices of $S_{e}$, And at least $\frac{5}{6}\rho$ proportion of $T$ are adjacent to less then $\frac{6}{\rho} \left( 9\Delta^{2} + \lambda^{\star}  \right) $ . 
 And therefore we have that at least $\frac{1}{6}\rho$ vertices are:
 \begin{enumerate}
   \item Incident to at least $\frac{1}{2}\Delta\rho$ heavy edges. 
   \item Adjacent to at most $ \frac{6}{\rho} \left( 9\Delta^{2} + \lambda^{\star}  \right)$ vertices of $S_{e}$.  
 \end{enumerate}
 \paragraph{Proof Of Theorem 1} Let us call to the set of vertices satisfy the constraints above \textbf{good vertices}. Pick any good vertex $g \in T$.
 Remember that each heavy edge between a normal vertex of $S$ and a vertex of $T$ corresponds to either a row or a column shared by the two local views.
 

 By $w$-robustness, for any small enough $\xi \le w $, the local view of any normal vertex is supported on at most $\frac{\xi}{\delta\Delta}$ rows and columns. 
 Hence, the row (or column) shared between the normal vertex and $v$ is at distance at most $\frac{\xi}{\delta\Delta}$ from a nonzero codeword of $\Aa$ (or $\Ab$, $\Ac$).


 Let us denote by $x_{v^{\prime}}$ the the local view obtained by taking only the rows and columns that shared between $v$ and normal vertices. The $\gamma$-resistance to puncturing property implies that if we could find $ \eta, \xi  $ such that for any $ |x| \le d $ we have:
 \begin{equation*}
   \begin{split}
     &  \frac{6}{\rho} \left( 9\Delta^{2} + \lambda^{\star}  \right) \le \gamma \ \ \ \ \ \left( \Theta\left(  \Delta^{\frac{1}{2}} \right) \right)
   \end{split}
 \end{equation*}
 Then the local view of $v$ is at distance at most:
 \begin{equation*}
   \begin{split}
     & d\left(x_{v}, \Aabc\right) \\ 
     & \ \ \ \ \le d\left(x_{v^{\prime} }, \cdot \right) + |\text{ ignored bits }|\\
     & \ \ \ \ \le  d\left(x_{v^{\prime}}, \cdot \right) +  \color{red}\frac{3}{2} \color{black} \Delta^{\color{red} 2 \color{black} } \cdot \frac{6}{\rho} \left( 9\Delta^{2} + \lambda^{\star}  \right) 
   \end{split}
 \end{equation*}
 Choosing $ \eta, \xi, \delta, \gamma, w, |x| < d $ such that the above is lower than $\frac{1}{2}\left( \delta\Delta \right)^{3}$ finishes the proof. 
 \paragraph{Theorem 2.} \textit{ The code $C_{w} / \mathcal{T}\left( \Gamma_{\square \square}, \left(  C_{A_1} \otimes C_{A_2} \otimes C_{A_3} \right)  \right)  $ has positive rate and linear distance.}
 \paragraph{Theorem 3.} \textit{ The code defined by $C_{x}$ has an efficient test for rejecting candidate with high error weigh. } 
\end{multicols*}
  % \printbibliography 
\end{document}


