% Drawing a graph using the PG 3.0 graphdrawing library
% Author: Mark Wibrow
\documentclass[tikz,border=10pt]{article}
\usepackage{tikz}
\usetikzlibrary{positioning}
\begin{document}

\tikzset{%
  every neuron/.style={
    circle (2pt),
    draw,
    fill=black
  },
  neuron missing/.style={
    draw=none,
    scale=2,
    text height=0.25cm,
    execute at begin node=\color{black}$\vdots$
  },
}

\begin{tikzpicture}[x=0.3cm, y=0.5cm, >=stealth]

  \node [every  neuron ] (v-0) at (0,0) {};  
  \node [every  neuron ] (v-1) at (-4,-2) {};
  \node [every  neuron ] (v-2) at (0,-2) {};
  \node [every  neuron ] (v-3) at (4,-2) {};
  \node [every  neuron ] (v-4) at (-8,-4) {};
  \node [every  neuron ] (v-5) at (-6,-4) {};  
  \node [every  neuron ] (v-6) at (-4,-4) {};
  \node [every  neuron ] (v-7) at (-2,-4) {};
  \node [every  neuron ] (v-8) at (0,-4) {};
  \node [every  neuron ] (v-9) at (2,-4) {};
  \node [every  neuron ] (v-10) at (4,-4) {};  
  \node [every  neuron ] (v-11) at (6,-4) {};
  \node [every  neuron ] (v-12) at (8,-4) {};
  \node [every  neuron ] (v-13) at (-8,-8) {};
  \node [every  neuron ] (v-14) at (-10,-8) {};
  \node [every  neuron ] (v-15) at (-8,-10) {};  
  \node [every  neuron ] (v-16) at (6,-8) {};
  \node [every  neuron ] (v-17) at (6,-10) {};
  \node [every  neuron ] (v-18) at (4,-8) {};
  \draw [->] (v-0) -- (v-1);
  \draw [->] (v-0) -- (v-2);
  \draw [->] (v-0) -- (v-3);
  \draw [->] (v-1) -- (v-4);
  \draw [->] (v-1) -- (v-5);
  \draw [->] (v-1) -- (v-6);
  \draw [->] (v-2) -- (v-7);
  \draw [->] (v-2) -- (v-8);
  \draw [->] (v-2) -- (v-9);
  \draw [->] (v-3) -- (v-10);
  \draw [->] (v-3) -- (v-11);
  \draw [->] (v-3) -- (v-12);
\draw[-] (v-4) -- (-10, -8) -- (-6, -8) -- (v-4) ;
\draw[-] (v-11) -- (4, -8) -- (8, -8) -- (v-11) ;
\draw[->] (v-14) -- (v-17);
\draw[->] (v-18) -- (v-15);
\draw[->] (v-13) -- (v-15);
\draw[->] (v-16) -- (v-17);
\draw[<->] (-12, 0) -- (-12, -3.8); 
\draw[<->] (-12, -4.2) -- (-12, -10);
\node (L-l) at (-13, -1.9) {$l$};
\node (L-g) at (-13, -7) {$\frac{1}{2}g$};
\end{tikzpicture}
\end{document}
