% Drawing a graph using the PG 3.0 graphdrawing library
% Author: Mark Wibrow
\documentclass[tikz,border=10pt]{article}
\usepackage{tikz}
\usetikzlibrary{positioning}
\begin{document}

\tikzset{%
  every neuron/.style={
    circle,
    draw,
    minimum size=0.8cm
  },
  neuron missing/.style={
    draw=none,
    scale=2,
    text height=0.25cm,
    execute at begin node=\color{black}$\vdots$
  },
}

\begin{tikzpicture}[x=0.5cm, y=1cm, >=stealth]

  \node [every neuron/.try, neuron 7.try] (v-0) at (0,0) {};  
  \node [every neuron/.try, neuron 7.try] (v-1) at (-2,-2) {};
  \node [every neuron/.try, neuron 7.try] (v-2) at (0,-2) {};
  \node [every neuron/.try, neuron 7.try] (v-3) at (2,-2) {};
  \node [every neuron/.try, neuron 7.try] (v-4) at (-6,-4) {};
  \node [every neuron/.try, neuron 7.try] (v-5) at (-4,-4) {};  
  \node [every neuron/.try, neuron 7.try] (v-6) at (-2,-4) {};
  \node [every neuron/.try, neuron 7.try] (v-7) at (0,-4) {};
  \node [every neuron/.try, neuron 7.try] (v-8) at (2,-4) {};
  \node [every neuron/.try, neuron 7.try] (v-9) at (4,-4) {};
  \node [every neuron/.try, neuron 7.try] (v-10) at (6,-4) {};  
  \node [every neuron/.try, neuron 7.try] (v-11) at (8,-4) {};
  \node [every neuron/.try, neuron 7.try] (v-12) at (10,-4) {};
  \draw [->] (v-0) -- (v-1);
  \draw [->] (v-0) -- (v-2);
  \draw [->] (v-0) -- (v-3);
  \draw [->] (v-1) -- (v-4);
  \draw [->] (v-1) -- (v-5);
  \draw [->] (v-1) -- (v-6);
  \draw [->] (v-2) -- (v-7);
  \draw [->] (v-2) -- (v-8);
  \draw [->] (v-2) -- (v-9);
  \draw [->] (v-3) -- (v-10);
  \draw [->] (v-3) -- (v-11);
  \draw [->] (v-3) -- (v-12);


\end{tikzpicture}
\end{document}
