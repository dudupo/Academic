% Drawing a graph using the PG 3.0 graphdrawing library
% Author: Mark Wibrow
\documentclass[tikz,border=10pt]{article}
\usepackage{tikz}
\begin{document}

\tikzset{%
  every neuron/.style={
    circle,
    draw,
    minimum size=0.8cm
  },
  neuron missing/.style={
    draw=none,
    scale=2,
    text height=0.25cm,
    execute at begin node=\color{black}$\vdots$
  },
}

\begin{tikzpicture}[x=1.5cm, y=1cm, >=stealth]

\foreach \m/\l [count=\y] in {1,2,3,4,missing,5}
  \node [every neuron/.try, neuron \m/.try] (input-\m) at (1 ,1-\y) {};

% \node [every neuron/.try, neuron 6.try] (middle-1) at (3 ,-1) {};
\node [every neuron/.try, neuron 7.try] (middle-2) at (3 ,-3) {};
\node [every neuron/.try, neuron 7.try] (out-1) at (5 ,-2) {};

\foreach \l [count=\i] in {1,2,3,4, 5}
  \draw [<-] (input-\i) -- ++(-1,0)
    node [above, midway] {$I_\i$};

\draw [->] (out-1) -- ++(1,0)
    node [above, midway] {$O$};

\foreach \y [count=\yi] in {1,2,3,4,5}
%   \foreach \j in {1,...,2}
       \pgfmathtruncatemacro{\cur}{\y}
       \pgfmathtruncatemacro{\next}{\y + 1}
      \draw [->](input-\cur) -- (middle-2);
    %   \draw [->] (input-\cur) to [out=50,in=160] (input-5);


% \foreach \y [count=\yi] in {1,2,3}
% %   \foreach \j in {1,...,2}
%       \pgfmathtruncatemacro{\cur}{\y}
%       \pgfmathtruncatemacro{\next}{\y + 1}
%     %   \draw [->](input-\cur) -- (input-\next)
%       \draw [->] (input-1) to [out=50,in=130] (input-\next);
% \draw [->](input-1) -- (input-5);
\draw [->] (input-1) to (out-1);
% \draw [->] (middle-1) to (out-1);
\draw [->] (middle-2) to (out-1);
    % \draw [->] (input-\y) -- (input-(1+\y));



\foreach \l [count=\x from 0] in {Input, Ouput}
  \node [align=center, above] at (1 + \x*4,0.5) {\l \\ layer};

\end{tikzpicture}
\end{document}
