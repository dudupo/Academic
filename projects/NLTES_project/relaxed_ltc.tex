\documentclass{article}

\usepackage[utf8]{inputenc}
\usepackage[a4paper, total={7.5in, 10in} ]{geometry}
\usepackage{braket}
\usepackage{xcolor}
\usepackage{amsmath}
\usepackage{amssymb}
\usepackage{amsfonts}
\usepackage{graphicx}
\usepackage{svg}
\usepackage{float}
\usepackage{tikz}
\usetikzlibrary{patterns,shapes.arrows}
\usepackage{adjustbox}
\usepackage{tikz-network}
\usepackage[ruled,vlined,linesnumbered]{algorithm2e}
\usepackage{multicol}
\usepackage[backend=biber,style=alphabetic,sorting=ynt]{biblatex}
\usepackage{xcolor}
\usepackage{pgfplots}
\DeclareUnicodeCharacter{2212}{−}
\usepgfplotslibrary{groupplots,dateplot}
\pgfplotsset{compat=newest}

\addbibresource{sample.bib} %Import the bibliography file

\newcommand{\commentt}[1]{\textcolor{blue}{ \textbf{[COMMENT]} #1}}
\newcommand{\ctt}[1]{\commentt{#1}}
\newcommand{\prb}[1]{ \mathbf{Pr} \left[ {#1} \right]}
\newcommand{\onotation}[1]{\(\mathcal{O} \left( {#1}  \right) \)}
\newcommand{\ona}[1]{\onotation{#1}}
\newcommand{\PSI}{{\ket{\psi}}}
\newcommand{\LESn}{\ket{\psi_n}}
\newcommand{\LESa}{\ket{\phi_n}}
\newcommand{\LESs}{\frac{1}{\sqrt{n}}\sum_{i}{\ket{\left(0^{i}10^{n-i}\right)^{n}}}}
\newcommand{\Hn}{\mathcal{H}_{n}}
\newcommand{\Ep}{\frac{1}{\sqrt{2^n}}\sum^{2^n}_{x}{ \ket{xx}}}
\newcommand{\HON}{\ket{\psi_{\text{honest}}}}
\newcommand{\Lemma}{\paragraph{Lemma.}}
\newcommand{\PonB}{ \rho + \frac{5}{16}\delta\le \frac{3}{4} + \frac{1}{16} } 
\newcommand{\Cpa}{[n, \rho n, \delta n]}
%\setlength{\columnsep}{0.6cm}

\newcommand{\Gz}{ G_{z}^{\delta} } 
\newcommand{ \Tann } {  \mathcal{T}\left( G, C_0 \right) }
\begin{document}




\title{ The Recover Problem. \ctt{Consider other name.}  } 
\author{David Ponarovsky}
\maketitle
\abstract{We answer what is the complexitiy of recovering a system after absorpting a sagnificent damge. Formally we show that if it gurnted that a amlicus/virus program took the control on at most $\Theta\left( n \right) $ ( \ctt{ $\frac{1}{4}n$ } parties than one can ethier recover the system at a cost linear at dame that made by the malicious. } 
\begin{multicols*}{2}

  \section{Introduction.} Consider the case that an orginzation detects a malicious progaram in his net, knowing that the virous toched at most quarter of the computers he needs to take a decision wheter is going to reformat the whole system or paying the cost of recover the damge. Even the problem might sounds artifical it actually modolates situations in which the damge that made for the companies is not well understood. In this work we suggest a simple soultion for that problem and also review how efficently use other constructions from the error correction codes field.
  Formally we consdier an $n$ computers that store a state at diminsion $k$. The design of the network is fexed at intialliztion time such that any computer is restricted to communicate with at most $\Delta$ other commponents. At time $t$ a maliouses detected and we have to.  system    

\end{multicols*}
\end{document}
