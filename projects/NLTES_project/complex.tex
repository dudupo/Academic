 \documentclass{article}
\usepackage[utf8]{inputenc}
\usepackage[a4paper, total={7in, 10in}]{geometry}
\usepackage{braket}
\usepackage{xcolor}
\usepackage{amsmath}
\usepackage{amssymb}
\usepackage{amsfonts}
\usepackage{graphicx}
\usepackage{svg}
\usepackage{float}
\usepackage{tikz}
\usepackage[ruled,vlined]{algorithm2e}
\usepackage{multicol}
\usepackage[backend=biber,style=alphabetic,sorting=ynt]{biblatex}
\usepackage{xcolor}
%\addbibresource{sample.bib} %Import the bibliography file

\newcommand{\commentt}[1]{\textcolor{blue}{ \textbf{[COMMENT]} #1}}
\newcommand{\ctt}[1]{\commentt{#1}}
\newcommand{\prb}[1]{ \mathbf{Pr} \left[ {#1} \right]}
\newcommand{\onotation}[1]{\(\mathcal{O} \left( {#1}  \right) \)}
\newcommand{\ona}[1]{\onotation{#1}}
\newcommand{\PSI}{{\ket{\psi}}}
\newcommand{\LESn}{\ket{\psi_n}}
\newcommand{\LESa}{\ket{\phi_n}}
\newcommand{\LESs}{\frac{1}{\sqrt{n}}\sum_{i}{\ket{\left(0^{i}10^{n-i}\right)^{n}}}}
\newcommand{\Hn}{\mathcal{H}_{n}}
\newcommand{\Ep}{\frac{1}{\sqrt{2^n}}\sum^{2^n}_{x}{ \ket{xx}}}
\newcommand{\HON}{\ket{\psi_{\text{honest}}}}
\newcommand{\Lemma}{\paragraph{Lemma.}}


\setlength{\columnsep}{0.6cm}

\newcommand{\Gz}{ G_{z}^{\delta} } 

\begin{document}

\title{Quantum LTC With Positive Rate}
\author{David Ponarovsky}
\maketitle
%\begin{multicols*}{2}
\newcommand{ \Hw }{ \delta\Delta -\Delta^{\frac{1}{2}-\varepsilon}/\delta  }
	\newcommand{ \Nw }{ \Delta^{\frac{3}{2}-\varepsilon}} 
	  \newcommand{ \Gu } { \Gamma^{\cup} }
	  \newcommand{ \Guq } { \Gamma^{\cup, \square} }

    	\newcommand{ \Gsa } {\Gamma_{\square_{1}} }
	\newcommand{ \Gsb } {\Gamma_{\square_{2}} }
        \newcommand{ \Aa } { C_{A_{1}}}  
	\newcommand{ \Ab } { C_{A_{2}}}
	\newcommand{ \Ac } { C_{A_{3}}}
	\newcommand{ \Aab } { \Aa \otimes \Ab } 
	\newcommand{ \Aac } { \Aa \otimes \Ac }
	\newcommand{ \Aabc } { \Aa \otimes \Ab \otimes \Ac }
	\newcommand{ \Aabp } { \Aa^{\perp} \otimes \Ab^{\perp} } 
	\newcommand{ \Aacp } { \Aa^{\perp} \otimes \Ac^{\perp} }
	\newcommand{ \Aabcp } { \Aa^{\perp} \otimes \Ab^{\perp} \otimes \Ac^{\perp} }
	\newcommand{ \Aabpp } { \left( \Aabp \right)^\perp } 
	\newcommand{ \Aacpp } { \left( \Aacp \right)^\perp }
	\newcommand{ \Aabcpp } { \left( \Aabcp \right)^\perp }
	\newcommand{ \YY } {  y_{1}y_{2}^{\top} }
	\newcommand{ \ZZ } {  z_{1}z_{2}^{\top} } 
	\newcommand{ \TT } { \tilde{\tau} } 


  \paragraph{preamble.} preamble.  
  \begin{figure}[H]
            %\label{fig:square}
            \begin{center}
            \begin{tikzpicture}
            \draw[thick](0,0);
\node at (-0.1,0) {$ g $};
\node at (4.505185115588901,0.5924060293985914) {$ a_{ 0 }g $};
\node at (5.989044389062352,0.6550992459070106) {$ a_{ 1 }g $};

            \end{tikzpicture}
            \end{center}
            \caption{Square of the complex, with edges $(g,ag), (agb, gb) \in E_A,
            (g,gb), (agb, ag) \in E_B.$ \label{fig:square}
            }
            \end{figure}
 \begin{figure}[H]
            %\label{fig:square}
            \begin{center}
            \begin{tikzpicture}
            \draw[thick](0,0);
\node at (-0.1,0) {$ g $};
\node at (6.087446704494186,1.2210676115167616) {$ a_{ 0 }g $};
\node at (4.563356672041692,0.8442656636397119) {$ a_{ 1 }g $};

            \end{tikzpicture}
            \end{center}
            \caption{Square of the complex, with edges $(g,ag), (agb, gb) \in E_A,
            (g,gb), (agb, ag) \in E_B.$ \label{fig:square}
            }
            \end{figure}
 \begin{figure}[H]
            %\label{fig:square}
            \begin{center}
            \begin{tikzpicture}
            \draw[thick](0,0);
\node at (-0.1,0) {$ g $};
\node at (5.1381325992855125,0.18520277420738068) {$ a_{ 0 }g $};
\node at (5.1452049386218786,0.2218305861202555) {$ a_{ 1 }g $};

            \end{tikzpicture}
            \end{center}
            \caption{Square of the complex, with edges $(g,ag), (agb, gb) \in E_A,
            (g,gb), (agb, ag) \in E_B.$ \label{fig:square}
            }
            \end{figure}
 \begin{figure}[H]
            %\label{fig:square}
            \begin{center}
            \begin{tikzpicture}
            \draw[thick](0,0);
\node at (-0.1,0) {$ g $};
\node at (5.406132496971162,0.8404883863427538) {$ a_{ 0 }g $};
\node at (4.699625643799616,1.1316513634191563) {$ a_{ 1 }g $};
\node at (4.728055059251398,0.3168919990758161) {$ a_{ 2 }g $};
\node at (4.139297870090619,1.269073273755287) {$ a_{ 3 }g $};

            \end{tikzpicture}
            \end{center}
            \caption{Square of the complex, with edges $(g,ag), (agb, gb) \in E_A,
            (g,gb), (agb, ag) \in E_B.$ \label{fig:square}
            }
            \end{figure}
 
%\end{multicols*}
  % \printbibliography 
\end{document}

 