\documentclass{article}
\usepackage[utf8]{inputenc}
\usepackage[a4paper, total={7in, 10in}]{geometry}
\usepackage{braket}
\usepackage{xcolor}
\usepackage{amsmath}
\usepackage{amssymb}
\usepackage{amsfonts}
\usepackage{graphicx}
\usepackage{svg}
\usepackage{float}
\usepackage{tikz}
\usepackage[ruled,vlined]{algorithm2e}
\usepackage{multicol}
\usepackage[backend=biber,style=alphabetic,sorting=ynt]{biblatex}

\addbibresource{sample.bib} %Import the bibliography file

\newcommand{\commentt}[1]{\textcolor{blue}{ \textbf{[COMMENT]} #1}}
\newcommand{\ctt}[1]{\commentt{#1}}
\newcommand{\prb}[1]{ \mathbf{Pr} \left[ {#1} \right]}
\newcommand{\onotation}[1]{\(\mathcal{O} \left( {#1}  \right) \)}
\newcommand{\ona}[1]{\onotation{#1}}
\newcommand{\PSI}{{\ket{\psi}}}
\newcommand{\LESn}{\ket{\psi_n}}
\newcommand{\LESa}{\ket{\phi_n}}
\newcommand{\LESs}{\frac{1}{\sqrt{n}}\sum_{i}{\ket{\left(0^{i}10^{n-i}\right)^{n}}}}
\newcommand{\Hn}{\mathcal{H}_{n}}
\newcommand{\Ep}{\frac{1}{\sqrt{2^n}}\sum^{2^n}_{x}{ \ket{xx}}}
\newcommand{\HON}{\ket{\psi_{\text{honest}}}}
\newcommand{\Lemma}{\paragraph{Lemma.}}


\setlength{\columnsep}{0.6cm}

\newcommand{\Gz}{ G_{z}^{\delta} } 


\begin{document}

\title{NLTES - property 1}
\author{David Ponarovsky}
\maketitle
\begin{multicols*}{2}


  \paragraph{preamble.} Even those good LDPC codes have used as critical gredint to prove the NLTES conjecture, it is hard not to noitce the fact that the linarity of the code distance was used only in indirectly fassion. This point puts the necessity of the code in question. In this work we have showing an anlogy of propert (1) whcih holds for a larger family of CSS code.     
  
  \paragraph{Claim} \textit{ for any ?  $ \left[ \left[ n,k,d \right] \right] $ \textbf{CSS} code property 1 holds }. 
\textbf{Proof.} let $y \in \{0,1\}^{n}$ be a vector such $ y \in \Gz $, let assume that $|y|_{c^{x^{\perp}}} \le C_{2} d$ then for any $ c \in C_{x}^{\perp}$: 

\begin{equation*}
  \begin{split}
    \delta r_z \ge | H_{z} y | = | H_{z} \left( y + c \right) |  
  \end{split}
\end{equation*}

  \paragraph{Definition.} \textit{ Let \(H_{i}\) be a single term of the Hamiltonian, we will define the \( \text{support}(H_{i}) \) to be a vector \( v \in \mathbb{F}^{n}_2\) such that \(v_j = 1\) if \(H_{i}\) act non-triviality on the \(j\)th qubit and \(0\) else.}
\end{multicos*}
  % \printbibliography 
\end{document}


