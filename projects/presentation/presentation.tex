\documentclass[usenames, aspectratio=169]{beamer}
\newcommand{\Beme} 

\usepackage[utf8]{inputenc}
%\usepackage[a4paper, total={6in, 9in}]{geometry}
%\usepackage{braket}
\usepackage{xcolor}
\usepackage{amsmath}
\usepackage{amsfonts}
\usepackage{amsthm}
\usepackage{amssymb}
%\usepackage[ocgcolorlinks]{hyperref}
\usepackage{hyperref}
%\usepackage{hyperref,xcolor}
%\usepackage[ocgcolorlinks]{ocgx2}
\usepackage{cleveref}
\usepackage{graphicx}
\usepackage{svg}
\usepackage{float}
\usepackage{tikz}
\usetikzlibrary{patterns, shapes.arrows}
\usepackage{adjustbox}
%\usepackage{tikz-network}
\usepackage{tkz-graph}
\usepackage{tkz-berge}
\usepackage[linesnumbered]{algorithm2e}
\usepackage{multicol}
\usepackage[backend=biber,style=alphabetic,sorting=ynt]{biblatex}
%\usepackage{xcolor}
%\usepackage{tkz-berge}
%\usepackage{tkz-graph}
\usepackage{pgfplots}
\usepackage{sagetex}
\usepackage{setspace}
\usepackage{etoc}
%\usepackage{wrapfig}
\usepackage{pgfgantt}
\DeclareUnicodeCharacter{2212}{−}
\usepgfplotslibrary{groupplots,dateplot}
\pgfplotsset{compat=newest}

\ifdefined\Beme
\else
\newtheorem{theorem}{Theorem}
\newtheorem{definition}{Definition}
\newtheorem{example}{Example}
\newtheorem{claim}{Claim}
\newtheorem{fact}{Fact}
\newtheorem{remark}{Remark}
\newtheorem*{theorem*}{Theorem}
\newtheorem{lemma}{Lemma}
\crefname{lemma}{Lemma}{Lemmas}
\hypersetup{colorlinks=true}
% , allcolors=blue,allbordercolors=blue,pdfborderstyle={0 0 1}}
%\hypersetup{pdfborder={2 2 2}}
% pdfpagemode=FullScreen,
% backref 

\newtheorem{problem}{Problem}
\crefname{problem}{Problem}{Problems}
\fi 


%\input{newcommands}
\usetheme[progressbar=frametitle]{metropolis}
\addbibresource{sample.bib} 
\begin{document}

\title[ Introduction to Quantum Error Correction Codes. ] % (optional, only for long titles)
{Introduction to Quantum Error Correction Codes.}

\subtitle{  }
\author[D.~Ponarovsky] % (optional, for multiple authors)
	{D.~Ponarovsky\inst{1}}

\institute[HUJI] % (optional)
{  Faculty of Computer Science\newline
  Hebrew University of Jerusalem
}
\date[2022-23] % (optional)
{Huji}
\subject{QECC}
\begin{frame}
  \maketitle
\end{frame}

\begin{frame}
  \frametitle{ Today }
  
  \begin{itemize}[<+->]
    \item Motivation - \uncover<->{ Preseting classical fault tolerance }
    \item Error Correction Codes (ECC), Good codes, LDPC codes.
    \item Quantum Noise and Quantum Codes.  
    \item 
  \end{itemize}
\end{frame}

\begin{frame}
  \uncover<1->{today \uncover<2->{motivation \uncover<3->{classical Error Correction Codes \uncover<4->{quantum noise \uncover<0->{finding good classical codes is easy.}}}}}
\end{frame}

\printbibliography
\end{document}





