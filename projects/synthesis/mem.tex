

%\newcommand*{\ACM}{}%

\ifdefined\ACM

%\documentclass[sigplan,screen]{acmart}
\documentclass[manuscript,screen,review]{acmart}

\else
 \documentclass[14pt]{article}
%\usepackage{libertine}
\usepackage{cuted}%\usepackage{widetext}
\usepackage[utf8]{inputenc}
\usepackage[a4paper, total={6in, 9in}]{geometry}
\usepackage{braket}
\usepackage{xcolor}
\usepackage{amsmath}
\usepackage{amsfonts}
\usepackage{amsthm}
\usepackage{amssymb}
%\usepackage[ocgcolorlinks]{hyperref}
\usepackage{hyperref}
%\usepackage{hyperref,xcolor}
%\usepackage[ocgcolorlinks]{ocgx2}
\usepackage{cleveref}
\usepackage{graphicx}
\usepackage{svg}
\usepackage{float}
\usepackage{tikz}
\usetikzlibrary{patterns, shapes.arrows}
\usepackage{adjustbox}
%\usepackage{tikz-network}
\usepackage{tkz-graph}
\usepackage{tkz-berge}
\usepackage[linesnumbered]{algorithm2e}
\usepackage{multicol}
\usepackage[backend=biber,style=alphabetic,sorting=ynt]{biblatex}
%\usepackage{xcolor}
%\usepackage{tkz-berge}
%\usepackage{tkz-graph}
\usepackage{pgfplots}
\usepackage{sagetex}
\usepackage{setspace}
\usepackage{etoc}
%\usepackage{wrapfig}
\usepackage{pgfgantt}
\DeclareUnicodeCharacter{2212}{−}
\usepgfplotslibrary{groupplots,dateplot}
\pgfplotsset{compat=newest}

\newtheorem{theorem}{Theorem}
\newtheorem{definition}{Definition}
\newtheorem{example}{Example}
\newtheorem{claim}{Claim}
\newtheorem{fact}{Fact}
\newtheorem{remark}{Remark}
\newtheorem*{theorem*}{Theorem}
\newtheorem{lemma}{Lemma}
\crefname{lemma}{Lemma}{Lemmas}
\hypersetup{colorlinks=true}
% , allcolors=blue,allbordercolors=blue,pdfborderstyle={0 0 1}}
%\hypersetup{pdfborder={2 2 2}}
% pdfpagemode=FullScreen,
% backref 

\newtheorem{problem}{Problem}
\crefname{problem}{Problem}{Problems}

\DeclareMathOperator{\Ima}{Im}


\usepackage{cancel}
\usepackage{subcaption}
\addbibresource{./sample.bib}

\fi

\begin{document}
\newcommand{\commentt}[1]{\textcolor{blue}{ \textbf{[COMMENT]} #1}}
\newcommand{\ctt}[1]{\commentt{#1}}
\newcommand{\prb}[1]{ \mathbf{Pr} \left[ #1 \right]}
\newcommand{\prbm}[2]{ \mathbf{Pr}_{ #2 }\left[ #1 \right]}
\newcommand{\prbc}[3]{ \mathbf{Pr}_{ #2 }\left[ #1 \right | #3]}
\newcommand{\prbcprb}[3]{ \prbc{#2}{#1}{#3} \cdot \prb{#3} } 
\newcommand{\expp}[1]{ \mathbf{E} \left[ {#1} \right]}
\newcommand{\onotation}[1]{\(\mathcal{O} \left( {#1}  \right) \)}
\newcommand{\ona}[1]{\onotation{#1}}
\newcommand{\PSI}{{\ket{\psi}}}
\newcommand{\xij} { X_{ij} } 
\DeclareMathOperator{\Ima}{Im}
%\newcommand{\LESn}{\ket{\psi_n}}
%\newcommand{\LESa}{\ket{\phi_n}}
%\newcommand{\LESs}{\frac{1}{\sqrt{n}}\sum_{i}{\ket{\left(0^{i}10^{n-i}\right)^{n}}}}
%\newcommand{\Hn}{\mathcal{H}_{n}}
%\newcommand{\Ep}{\frac{1}{\sqrt{2^n}}\sum^{2^n}_{x}{ \ket{xx}}}
%\newcommand{\HON}{\ket{\psi_{\text{honest}}}}
%\newcommand{\Lemma}{\paragraph{Lemma.}}
\newcommand{\Cpa}{[n, \rho n, \delta n]}
%\setlength{\columnsep}{0.6cm}
\newcommand{\Jvv}{ \bar{J_{v}} } 
\newcommand{\Cvv}{ \tilde{C_{v}} } 

\newcommand{\Gz}{ G_{z}^{\delta} } 
\newcommand{ \Tann } {  \mathcal{T}\left( G, C_0 \right) }
\newcommand{\ireducable}{ireducable \hyperref[ire]{[\ref{ire}]} }
\newcommand{\cutUU}{E(U_{-1} \bigcup U_{+1} ,U)} 
\newcommand{\wcutUU}{w\left( E(U_{-1} \bigcup U_{+1} ,U)  \right)}
\newcommand{\testgo}{  \mathcal{T}\left(J, q , C_{0}\right) } 

\newcommand{\duC}{\left( C_{A}^{\perp}\otimes C_{B}^{\perp} \right)^{\perp}}
\newcommand{\duduC}{\left( C_{A}\otimes C_{B}\right)^{\perp}}
  




%\title{ $\textbf{QNC}_{1} \subset $ noisy-\textbf{BQP}}
\title{ Memory. }
\author{Michael Ben-Or \ \ David Ponarovsky}
\maketitle

\newcommand*{\Mbas}{\mathcal{X}^\prime}
\newcommand*{\bas}{\mathcal{X}}
\newcommand*{\sMbas}{\Mbas}
\newcommand*{\QQ}{C_{X}/C_{Z}^\perp }
\newcommand*{\trig}{ Triorthogonal }
\newcommand*{\Hyp}{ Hyperproduct }
\newcommand*{\Cin}{ C_{\text{initial}} }
\newcommand*{\Ctan}{ C_{\text{Tan}} }



\newcommand*{\QACze}{\mathbf{QAC}_{0}}
\newcommand*{\QNCzef}{\mathbf{QNC}_{0,f}}
\newcommand*{\QNCze}{\mathbf{QNC}_{0}}
\newcommand*{\QNCon}{\mathbf{QNC}_{1}}
\newcommand*{\NCon}{\mathbf{NC}_{1}}
\newcommand*{\noiseQNCon}{noisy-$\QNCon$}
\newcommand*{\QNC}{\mathbf{QNC}}
\newcommand*{\QNCG}{\mathbf{QNC_G}}
\newcommand*{\NC}{\mathbf{NC}}
\newcommand*{\QNCiG}{\mathbf{QNC_{G,i}}}


% Constant depth fault tolerance construction.   
\newcommand*{\CDO} {CDFT} 



  \section{ Strategies to get \CDO. }  \label{sec:opt}
 
The second gadget is Memory, a particular type of code which allows restraining the error rate by exhibiting a constant depth procedure that, when promising that the error rate is below a threshold, suppresses the error by at least a constant factor. Using memory, we will be able to promise with high probability that the error rate is lower than some fraction. 

 \subsection{Memory.} 
 \newcommand*{\DD}{\mathbf{D} }
Informal memory code is a code that stores a logical state for a long time while keeping the noise below a certain amount. We define it formally by saying that memory codes will reduce an error that affects at most $\beta$ portion of the qubits into an error that affects at most $\gamma$ portion of the qubits.

 \begin{definition}[Ideal $(\beta,\gamma)$-Memory]
   We say that a (quantum) error correction code $C$ is an Ideal $(\beta,\gamma)$-Memory code if there is a constant depth procedure $\DD$ such that for any $I$ of size $|I| \ge (1 - \beta) n$ and a mixed states $\sigma$ and $\rho$ such $\sigma$ distributed over the $C$'s codewords $\sigma \in C$ and $\Tr_{I}\left(\rho\right) = \Tr_{I}\left(\sigma\right)$, we have that there is subset of qubits $J$ at size at least $(1-\gamma)n$:
   \begin{equation*}
     \begin{split}
        \Tr_{J} \DD \left(\rho\right) = \Tr_{J}\left(\sigma\right) 
     \end{split}
   \end{equation*}

 \end{definition}
We would like to extend the memory gadgets to work with high probability, which motivates us to define the following:
\newcommand*{\Po}{\mathcal{P}_{1}}
\newcommand*{\Pt}{\mathcal{P}_{2}}
\newcommand*{\Nn}{\mathcal{N}}
\begin{definition}[ $\left(\Po,\Pt \right)$- thermal couple. ]
Let $\Po,\Pt$ be sets of density matrices induced over the $n$-qubit Hilbert space, and let $\Nn$ be a $p$-stochastic local noise channel for some constant $p \in (0,1)$. We say that the couple $\left(\Po,\Pt \right)$ is a thermal couple if for any $\rho \in \Pt$, we have $\Nn(\rho) \in \Po$ with high probability.
\end{definition}

 \begin{definition}[$(\Po,\Pt)$-Memory]
   Consider a $\left(\Po,\Pt \right)$- thermal couple, We say that C is a $(\Po, \Pt)$-Memory if there is a constant depth procedure $\DD$, such that for any $\rho \in \Po$ we have $\DD\left( \rho \right) \in \Pt$, with high probability. 
 \end{definition}
 For example, consider a code $C$ with a $\Delta$-regular Tanner graph. Let $\Po$ be all the noisy states derived from codewords in $C$ such that the syndrome graph induced by them can be decomposed into disjoint $\Delta/2$-connected components $A_{1},A_{2},..A_{l}$, each of size at most $|A_{i}| < \beta \sqrt{n}$, and the $\Delta/2$-distance between any two of them $A_{i}, A_{j}$, namely the number of edges needed to add to merge them into one single $\Delta/2$-connected component, is at least $\theta \min \left( |A_{i}|, |A_{j}| \right)$. We call such decomposition characterization $(\beta \sqrt{n}, \theta )$ error decomposition. 

 Now let $\Pt$ be all the deviations from $C$, such that the syndrome graph induced by them can be decomposed into $(\gamma \sqrt{n}, \frac{\beta}{\gamma}  \theta )$ error decomposition. The couple $\left( \Po, \Pt \right)$ is thermal couple, And combining the quantum expander code and the parallel small set-flip decoder \cite{grospellier:tel-03364419} they defines a $\left( \Po, \Pt \right)$-memory. 


 \newcommand{\Px}[2]{P^{(v)}_{#1}(#2)}

\begin{claim}
  The probability to have $\Px{\alpha\Delta}{x} \le $ 
\end{claim}

\begin{claim}
  Any $\alpha\Delta$-connected component $E$ can be decompized to $\alpha\Delta-1$ connected component and more $\Theta( E/\Delta^{3})$ edges. 
\end{claim}

\begin{proof}
  $E$ is connected. Let $T$ be its spanning tree. Now consider $Y$, a subset of edges obtained by colorizing from any vertex at an odd level of $T$ a single forward edge. And let $E^{\prime} = E/Y$. First, observes that $E$ is an $\alpha\Delta - 1$ connecnted componnent. On the otherhand: 
  \begin{equation*}
    \begin{split}
      |Y| &= \frac{1}{\Delta-1}\sum_{i}^{h/2} E\left( T^{2i + 1} \right) =  \frac{1}{\Delta-1}\sum_{i}^{h/2} \frac{1}{2} \left( E\left( T^{2i + 1} \right)+ E\left( T^{2i + 1} \right) \right) \\  
      & \ge \frac{1}{\Delta-1}\sum_{i}^{h/2} \frac{1}{2} \frac{1}{\Delta} \left( E\left( T^{2i + 1} \right)+ E\left( T^{2i} \right) \right) = \frac{1}{2 \left( \Delta - 1 \right) \Delta }|T| \\ 
      & \ge \frac{1}{2 \left( \Delta - 1 \right) \Delta } \frac{1}{\Delta}|E| \ge \frac{1}{2\Delta^{3}}|E|  \\
      & \left( \ge \frac{1}{2 (\Delta - 1) \Delta } \left( V(T) - 1 \right)     \right) 
    \end{split}
  \end{equation*}
\end{proof}


\begin{proof}
  Assume that $J$ is vertices subset that support an $\alpha\Delta$ connected $E$ in $G$, then it's also the support of $\alpha\Delta -1$ connected, denote by $E^{\prime}$ that sub componnent. So we can construct $E$ by first sample $E^{\prime}$ and then find a mathcing between the left veritcis. Thus:  
  \begin{equation*}
    \begin{split}
      \Px{\alpha\Delta}{x} \le \Px{\alpha\Delta - 1}{x} \cdot (\Delta p)^{\frac{x}{2\Delta^{2}}} \le (\Delta p)^{\frac{x}{2 \Delta^{2} } \alpha \Delta } =  (\Delta p)^{\frac{\alpha \Delta }{2 } x}
    \end{split}
  \end{equation*}
\end{proof}

\begin{claim}
  The ptobability to have $n^{\varepsilon}$ connencted component is: 
\end{claim}
\begin{proof}
  \begin{equation*}
    \begin{split}
      \le &  n \sum_{n^{\varepsilon}}^{n} \sum_{v \in V}\Px{\alpha\Delta}{x}   
      \le     n  \frac{  (\Delta p)^{\frac{n^{\varepsilon}}{2} \alpha \Delta }}{ 1  - (\Delta p)^{ \frac{1}{2}\alpha \Delta }} \rightarrow 0 
    \end{split}
  \end{equation*}<++>
\end{proof}








  \newcommand{\sliceb}[1]{ \slice[style=blue, label style={inner sep=1pt,anchor=south west,rotate=40}]{#1}}

%\slice{ $\gamma\left( c \alpha + p  \right) + p < \alpha $ error rate. }
%\slice{ $c\alpha + p$ error rate.  } 
  %row sep=0.3cm, column sep=0.7cm,
%slice style=blue,slice label style={inner sep=1pt,anchor=south west,rotate=40}
  % \slice{ $\rho$ at $\alpha$ error rate. }
  \begin{figure}[h]
    \centering
    \begin{quantikz}[slice style=blue,slice label style={inner sep=1pt,anchor=south west,rotate=40}]
      \lstick{$q_1$} & \qw & \qw & \qw &  \sliceb{  $ \rho$ at $\alpha$ error rate.  }  & \gate[wires=9][1.7cm]{U}  \sliceb{ $c\alpha + p$ error rate.  }  & \gate[wires=9][1.7cm]{ F }  & \gate{\mathcal{N}}  \sliceb{ $\gamma\left( c \alpha + p  \right) + p < \alpha $ error rate. }& \qw & \\
  \lstick{$q_2$} & \qw & \qw & \qw &  &                  &    & \gate{\mathcal{N}} & \qw & \\
  \lstick{$q_3$} & \qw & \qw & \qw &  &                  &    & \gate{\mathcal{N}} & \qw & \\
  \lstick{$q_4$} & \qw & \qw & \qw &  &                  &    & \gate{\mathcal{N}} & \qw & \\
  \lstick{$q_5$} & \qw & \qw & \qw &  &                  &    & \gate{\mathcal{N}} & \qw & \\
  \lstick{$q_6$} & \qw & \qw & \qw &  &                  &    & \gate{\mathcal{N}} & \qw & \\
  \lstick{$q_7$} & \qw & \qw & \qw &  &                  &    & \gate{\mathcal{N}} & \qw & \\
  \lstick{$q_8$} & \qw & \qw & \qw &  &                  &    & \gate{\mathcal{N}} & \qw & \\
  \lstick{$q_9$} & \qw & \qw & \qw &  &                  &    & \gate{\mathcal{N}} & \qw & 
\end{quantikz}
\caption{ Usage of  Ideal $(\beta,\gamma)$-Memory to obtain fault tolerance computation. }
    \label{fig:mem}
  \end{figure}

\end{document}


