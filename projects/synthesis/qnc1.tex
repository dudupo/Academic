

%\newcommand*{\ACM}{}%

\ifdefined\ACM

%\documentclass[sigplan,screen]{acmart}
\documentclass[manuscript,screen,review]{acmart}

\else
  \documentclass[18pt]{article}
\usepackage{libertine}
\usepackage{cuted}
%\usepackage{widetext}
\usepackage[utf8]{inputenc}
\usepackage[a4paper, total={6.5in, 10in} ]{geometry}
\usepackage{braket}
\usepackage{xcolor}
\usepackage{amsmath}
\usepackage{amssymb}
\usepackage{amsfonts}
\usepackage{graphicx}
\usepackage{svg}
\usepackage{float}
\usepackage{tikz}
\usetikzlibrary{patterns, shapes.arrows}
\usepackage{adjustbox}
\usepackage{tikz-network}
\usepackage[ruled,lined,linesnumbered]{algorithm2e}
\usepackage{multicol}
\usepackage[backend=biber,style=alphabetic,sorting=ynt]{biblatex}
\usepackage{xcolor}
\usepackage{pgfplots}
\DeclareUnicodeCharacter{2212}{−}
\usepgfplotslibrary{groupplots,dateplot}
\pgfplotsset{compat=newest}



\usepackage{cancel}
\usepackage{subcaption}
\addbibresource{./sample.bib}

\fi

\begin{document}
\input{newcommands}
\title{ $\textbf{QNC}_{1} \subset $ noisy-\textbf{BQP}}
\author{Michael Benor \ \ David Ponarovsky}
\maketitle

\newcommand*{\Mbas}{\mathcal{X}^\prime}
\newcommand*{\bas}{\mathcal{X}}
\newcommand*{\sMbas}{\Mbas}
\newcommand*{\QQ}{C_{X}/C_{Z}^\perp }
\newcommand*{\trig}{ Triorthogonal }
\newcommand*{\Hyp}{ Hyperproduct }
\newcommand*{\Cin}{ C_{\text{initial}} }
\newcommand*{\Ctan}{ C_{\text{Tan}} }



%\section{Introduction} blablabla


\section{ Notations. }
We denote by $C_{g}$ the good qLDPC code \cite{Dinur} \cite{Pavel} \cite{leverrier2022quantum}, and by $C_{ft}$ the concatenation code presented at \cite{aharonov1999faulttolerant} ($ft$ stands for fault tolerance). For a code $C_{y}$, we use $\Phi_{y}, E_{y}, D_{y}$ to denote the channel maps circuits into the circuits computed in the code space, the encoder, and the decoder, respectively. We use $\Phi_{U}$ to denote the 'Bell'-state storing the gate $U$. We say that a state $\ket{\psi}$ is at a distance $d$ from a quantum code $C$ if there exists an operator $U$ that sends $\ket{\psi}$ into $C$ such that $U$ is spanned on Paulis with a degree of at most $d$. Sometimes, when the code being used is clear from the context, we will say that a block $B$ of qubits has absorbed at most $d$ noise if the state encoded on $B$ is at a distance of at most $d$ from that code.


\section{ The Noise Model }


\section{ Fault Tolerance (With Resets gates) at Linear Depth. } 

\begin{claim}
There exists a value $p_{th} \in (0,1)$ such that if $p < p_{th}$, then any quantum circuit $C$ with a depth of $D$ and a width of $W$ can be computed by a $p$-noisy circuit $C^{\prime}$, which allows for resets. The depth of $C^{\prime}$ is at most $\max{ \{O(D), O(\log(WD)) \} }$.
\end{claim}


\subsection{Initializing Magic for Teleportation gates and encodes ancillaries.}
The Protocol: \begin{enumerate}
  \item Initialization of zeros: The qubits are divided into blocks of size $|B|$. Each block is encoded in $C_{g}$ using $D_{ft} \Phi_{ft}[E_{g}] \ket{0^{|B|}}$.
  \item Initialization of Magic for Teleportation gates: The gates in the original circuit are encoded in $C_{g}$ using $D_{ft} \Phi_{ft}[E_{g}] \ket{\Phi_{U}}$.
  \item Gate teleportation: Each gate in the original circuit is replaced by a gate teleportation.
  \item Error reduction: After the initialization step, at each time tick, each block runs a single round of error reduction.
\end{enumerate}

\begin{claim}
  \label{claim:error} 
  Assume that an error $|e| = \gamma n $, i.e $e$ is supported on less than $\gamma n$ bits, then a single correction round reduce $e$ into an error $e^\prime$ such $|e^{\prime}| < \nu |e|$. 
\end{claim}


\begin{claim}
  The gate $ D_{ft} \Phi_{ft}[E_{g}]$ initializes states encoded in $C_{g}$ subject to $3p$-noise channel.  
\end{claim}
\begin{proof}
  Clearly $\Phi_{ft}[E_{g}]$ success, with high probability, let's say $1 - \frac{1}{poly(n)}$, to encode in to $ C_{ft} \circ C_{g}$. Denote by $E_{i}, D_{i}$ the encoder and the decoder at the $i$th level of the concatination construction. Recall that by definition $D_{i}E_{i} = I$, or in other words $D_{i}= E_{i}^{\dagger}$. Consider the decoder under $\mathcal{N}$ action. $P_{2}D_{1}P_{2}D_{2},..,P_{i-1}D_{i}P_{i}$, by the fault-tolerance construction  a logical error happens at the $i$th stage occurs with probability $p^{2^{i}}$, therefore by the union bound the probability that in one of the steps the circuit absorbs an error that is not corrected is less than $p + p^{2} + p^{4} + .. < 2p$. Hence any decoded qubit absorbs a noise with probability less than $2p$.


  Thus in overall we can bound the porobability a single qubit to be faulty by: 
  \begin{equation*}
    \begin{split}
      \prb{\text{fault} } &=  \prb{\text{fault} |  \Phi_{ft}[E_{g}] }\cdot \prb{\Phi_{ft}[E_{g}]} + \prb{\text{fault} | \overline{\Phi_{ft}[E_{g}]} }\cdot \prb{\overline{\Phi_{ft}[E_{g}]}}\\
      &\le  \prb{\text{fault} |  \Phi_{ft}[E_{g}] } + \prb{\overline{\Phi_{ft}[E_{g}]}} \le 2p + \frac{1}{poly(n)} \le 3p
    \end{split}
  \end{equation*}

  \begin{remark}
    In our construction we use the concatinate-code to encode $\log(n)$-length block, Thus any $poly(n)$ in the above should be replaced by $\log(n)$. Yet it doesn't effect anything since the inequality dosn't depend on $n$.
  \end{remark}

%
%
%  \begin{equation*}
%    \begin{split}
%      \mathcal{N}(D) &= \left((\mathcal{N}(D))^{\dagger}\right)^{\dagger} =  \left(\sum_{P_{1}, P_{2}, .., P_{i} \in \mathcal{P}}{ \prb{P_{1}, P_{2}, .., P_{i}}  \left(D_{1}P_{2}D_{2},..,P_{i-1}D_{i}P_{i}\right)^{\dagger}} \right)^{\dagger} \\ 
%      &= \left( \sum_{P_{1}, P_{2}, .., P_{i} \in \mathcal{P}}{ \prb{P_{1}, P_{2}, .., P_{i}}  P_{i}E_{i}P_{i-1}E_{i-1},..,P_{1}E_{1} } \right)^{\dagger}\\
%      &= \left( \left( 1 -\frac{1}{poly(n)} \right)\sum_{P_{i} \in \mathcal{P}}\prb{P_{i}}P_{i}E + \frac{1}{poly(n)} A  \right)^{\dagger} \\ 
%      &= \left( 1 -\frac{1}{poly(n)} \right)\sum_{P_{i} \in \mathcal{P}}\prb{P_{i}}DP_{i} + \frac{1}{poly(n)} A 
%\end{split}
%  \end{equation*}
%
%  %Since $D$ is semi-transversal gate, it preserves the 
%
%
%  And notice that $\star$ is with probability $1 - \frac{1}{poly(n)}$ equals to $E_{i}E_{i-1}..,E_{1}=E$. Hence $\mathcal{N}(D)$ equals to $\left( P E \right)^{\dagger} = PD$.
%
%  \begin{equation*}
%    \begin{split}
%      \braket{ \psi^{\prime} | P_{i}E_{i}P_{i-1}E_{i-1},..,P_{1}E_{1} \psi } = \braket{ \psi^{\prime} P_{i}D_{i}P_{i-1}D_{i-1},..,P_{1}D_{1} | \psi }
%    \end{split}
%  \end{equation*}
%  Thus for any pauli-channel $\mathcal{N} : L(H) \rightarrow L(H)$, and $\psi^{\prime}$ which is a codeword we get: 
%  \begin{equation*}
%    \begin{split}
%      \braket{ \psi^{\prime} \mathcal{N}(D) | \psi } &=  \sum_{P_{1}, P_{2}, .., P_{i} \in \mathcal{P}}{ \prb{P_{1}, P_{2}, .., P_{i}}  \braket{ \psi^{\prime} P_{i}D_{i}P_{i-1}D_{i-1},..,P_{1}D_{1} | \psi }} \\
%      &=  \sum_{P_{1}, P_{2}, .., P_{i} \in \mathcal{P}^{\star}}{  \prb{P_{1}, P_{2}, .., P_{i}}\braket{ \psi^{\prime} | P_{i}E_{i}P_{i-1}E_{i-1},..,P_{1}E_{1} \psi }} \pm O(  \frac{1}{poly(n)})\\
%      &=  \sum_{P_{1}, P_{2}, .., P_{i} \in \mathcal{P}^{\star}}{  \prb{P_{1}, P_{2}, .., P_{i}}\braket{ \psi^{\prime} | P_{i} E \psi }} \pm O(  \frac{1}{poly(n)})\\
%      &\le  \sum_{ P_{i} \in \mathcal{P}}{  \prb{ P_{i}}\braket{ \psi^{\prime} | P_{i} E \psi }} \pm O(  \frac{1}{poly(n)}) \\
%      &\le  \sum_{ P_{i} \in \mathcal{P}^{\le d}}{  \prb{ P_{i}}\braket{ \psi^{\prime} | P_{i} E \psi }} \pm O (e^{-d \cdot n} ) \pm O(  \frac{1}{poly(n)}) \\
%      & \le   \sum_{ P_{i} \in \mathcal{P}/\mathcal{P}^{\star}}{  \prb{ P_{j} \in B_{d}\left( P_{i} \right)}\braket{ \psi^{\prime} | P_{i} E \psi }}  \pm O (e^{-d \cdot n} ) \pm O(  \frac{1}{poly(n)}) 
%    \end{split}
%  \end{equation*}
%  Using the fact that the concatenation code is monotonic (\Cref{def:mono}) we get that the probability to have physical fault $P_{j}$.   
%%\end{widetext}
\end{proof}

\begin{claim}
  \label{claim:prob}
  With probability $ 1 - \frac{WD}{|B|} \cdot D 2e^{-2|B|(\beta - p)} $, the total amount of noise been absorb in a block, in any time $t$, is less than $\gamma n$. 
\end{claim}
\begin{proof}
  Consider the $i$th block,  denoted by $B_{i}$. Using the Hoeffding's inequality we have that the probability that more than $\beta |B|$ bits are flipped at time $t$ is less than $ \le 2e^{-2|B|(\beta - p)} $. Using the union bounds over all the blocks at all the different time locations we get that with probability $ 1 - \frac{WD}{|B|} \cdot D 2e^{-2|B|(\beta - p)} $ the noise been absorbed in a block is less than $|\beta|B$ for the whole computation.

  Denote by $X_{t}$ the support's size of the error over $B_{i}$ at time $t$. Now using \Cref{claim:error}, given that $X_{t-1} \le \gamma n$, it follows that the total amount of error absorbed by a block until time $t$ can be bounded by: 
  \begin{equation*}
    \begin{split}
  X_{t} \le \nu \cdot (X_{t-1} + \beta |B| ) \le  \nu(\gamma+\beta) |B| \le \gamma |B|
    \end{split}
  \end{equation*}

%  \begin{equation*}
%    \begin{split}
%      \prb{ \beta n \text{ bits were flipped in } B_{i} \text{ at time } t  } \le 2e^{-2|B|(\beta - p)} 
%    \end{split}
%  \end{equation*}
%  Taking the union bound over the 
\end{proof}


\begin{claim}
  The total depth of the circuit is $O\left( \log n  \right)$. 
\end{claim}
\begin{proof}
  The gate for encoding $|B|$-length blocks in $C_{g}$, is a Clifford and therefore can be computed in $O(\log|B|)$ depth. The encoding of the magic/bell states, done by first compute them in the logical space (un-encoded qubits) and then by using the encoder. Hence it's foult-tolerence version of both initializing ancillaries and magic states /bell states cost $O( (\log |B|) \cdot \log^{c}( |B| \log |B| ) )$ \footnote{The width of the original circuit is $|B|^{2}$ so the number of locations is $ |B|^{2} \cdot \log |B|$} depth \cite{aharonov1999faulttolerant}. Backing into $C_{g}$ from $C_{ft}$ by decoding the concatenation code takes exactly as the encoding namely. 

  Then using the bell measurements any of the logical gates takes $O(1)$ depth and since we use perform only a single round of error correction we get that the reaming computation till the last decoding stage is a at most constant time of the original depth. Finally we pay $O(\log |B|)$ for complete decoding. Summing all, we get: 
  \begin{equation*}
    \begin{split}
     &  O ( \log |B|\cdot  \log^{c}( |B| \log |B| ) )  + O ( \text{original depth} ) + O ( \log |B| ) \\ 
     = & O ( \text{original depth} ) + O ( \log^{c} |B| )
    \end{split}
  \end{equation*}
\end{proof}

Taking the block length to be $|B| = \log ( (W \cdot D)^{c} )$ gives, by \Cref{claim:prob}, a linear\footnote{Assuming $W$ is polynomial in $D$} fault tolerance construction that success with probability $ 1- \frac{1}{\log^{c_{2}}( W\cdot D)} $. Particularly, the fault tolerance version of  circuits in $\textbf{QNC}_{1}$ has logarithmic depth. Then using the construction in \cite{aharonov1996limitationsnoisyreversiblecomputation} yields a polynomial fault tolerance circuit, in the only reversible gates setting.




%\input{./tempmagic.tex}
%\cite{leverrier2022quantum}
%\cite{moore1998parallel}
%\cite{Tillich_2014}
%\cite{meier2012magicstate}
%\cite{bravyi2012magic}
\printbibliography

\end{document}

