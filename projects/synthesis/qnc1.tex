

%\newcommand*{\ACM}{}%

\ifdefined\ACM

%\documentclass[sigplan,screen]{acmart}
\documentclass[manuscript,screen,review]{acmart}

\else
  \documentclass[18pt]{article}
\usepackage{libertine}
\usepackage{cuted}
%\usepackage{widetext}
\usepackage[utf8]{inputenc}
\usepackage[a4paper, total={6.5in, 10in} ]{geometry}
\usepackage{braket}
\usepackage{xcolor}
\usepackage{amsmath}
\usepackage{amssymb}
\usepackage{amsfonts}
\usepackage{graphicx}
\usepackage{svg}
\usepackage{float}
\usepackage{tikz}
\usetikzlibrary{patterns, shapes.arrows}
\usepackage{adjustbox}
\usepackage{tikz-network}
\usepackage[ruled,lined,linesnumbered]{algorithm2e}
\usepackage{multicol}
\usepackage[backend=biber,style=alphabetic,sorting=ynt]{biblatex}
\usepackage{xcolor}
\usepackage{pgfplots}
\DeclareUnicodeCharacter{2212}{−}
\usepgfplotslibrary{groupplots,dateplot}
\pgfplotsset{compat=newest}



\usepackage{cancel}
\usepackage{subcaption}
\addbibresource{./sample.bib}

\fi

\begin{document}
\input{newcommands}
%\title{ $\textbf{QNC}_{1} \subset $ noisy-\textbf{BQP}}
\title{ On The Cost of Fault-Tolerazing. }
\author{Michael Benor \ \ David Ponarovsky}
\maketitle

\newcommand*{\Mbas}{\mathcal{X}^\prime}
\newcommand*{\bas}{\mathcal{X}}
\newcommand*{\sMbas}{\Mbas}
\newcommand*{\QQ}{C_{X}/C_{Z}^\perp }
\newcommand*{\trig}{ Triorthogonal }
\newcommand*{\Hyp}{ Hyperproduct }
\newcommand*{\Cin}{ C_{\text{initial}} }
\newcommand*{\Ctan}{ C_{\text{Tan}} }



%\section{Introduction} blablabla


\section{Todo:}
\begin{enumerate}
  \item Move to encoding each qubit by logarithmic width (instead of chanks) the reason is that the gate teleportation becomes complicated when it applied over higher dimension. 
  \item Then showing for 2-qubit gates set that is indeed works.
  \item Treating separately to noise observed in two qubits gates. 
\end{enumerate}


\section{ Fault tolerance Toffoli. } 

\ctt{In that section the $\cdot$ operation is the pair wise product (pair wise AND).}

Assume that $\bar{0}, \bar{1} \in C_{X}$ and that they belong to two different cosets of $\QQ$. Let $x,y \in \{ \bar{0},\bar{1}  \}$. 
\begin{equation}
  \label{equation:toff}
  \begin{split}
&    \sum_{z,z^{\prime},w \in \Czdu }{ \ket{z}\ket{z^{\prime}}\ket{w} } \\  
&    \sum_{z,z^{\prime},w \in \Czdu }{ \ket{z}\ket{z^{\prime}}\ket{w + z\cdot z^{\prime}} } \\  
&    \sum_{z,z^{\prime},w \in \Czdu }{ \ket{z + x }\ket{z^{\prime} + y}\ket{w + z\cdot z^{\prime}} } \\  
&    \sum_{z,z^{\prime},w \in \Czdu }{ \ket{z + x }\ket{z^{\prime} + y}\ket{ x\cdot y + x \cdot z^{\prime} + y \cdot z + z z^{\prime} + w + z\cdot z^{\prime}} } \\  
&    \sum_{z,z^{\prime},w \in \Czdu }{ \ket{z + x }\ket{z^{\prime} + y}\ket{ x\cdot y + x \cdot z^{\prime} + y \cdot z +  w } } 
  \end{split}
\end{equation}
Since $x,y \in \{ \bar{0},\bar{1}  \}$ we have that $ x\cdot z^{\prime}$ equals to either $z^{\prime}$ or $\bar{0}$. Hence $ \sum_{w \in \Czdu}{\ket { \xi +x \cdot z + w } } =  \sum_{w \in \Czdu}{\ket { \xi + w } } $. So the idea is the following, suppose that one has to compute Toffoli at time $t$ over the registers $R_{1},R_{2},R_{3}$. First, at time $0$, he initialize a logical zero $\ket{\Czdu}$ in each register, then he compute pairwise Toffoli $R_{1},R_{2}$ into $R_{3}$. That gives the ket $\sum_{z,z^{\prime},w \in \Czdu}{\ket{ z\cdot z^{\prime} + w}}$,  immediately afterwords encode $R_{3}$ again into a good quantum code. Denote by $\tau$ the time required for decoding $R_{3}$ back, at time $t-\tau$ start to decode $R_{3}$. Eventually at time time $t$ compute again the transversal Toffoli, by \Cref{equation:toff} we gets the desired.  


By similar arguments exhibited at \Cref{claim:noisepa} one can show that the errors behaves according to a Pauli noise channel. \ctt{That is not correct, since the concatenation construction assumes that all the registers initialized to physical zeros in the begging of the computation}.

\subsection{Another Idea, $z\cdot z^{\prime}$ cann't contribute too mach.}
Clearly we have that  $|z\cdot z^{\prime}| \le |z|,|z^{\prime}|$ therefore we have that $\prbm{ | z \cdot z^{\prime} | \ge t}{z,z^\prime \in \Czdu} \le \prbm{ | z | \ge t}{z\in \Czdu}$. Now assume that the tanner code by which the code defined is bipartite graph and denote by $z_{+},z_{-}$ the grouping of the $z$'s generators supported on the even and the odd vertices of the graph. By triangle inequality $|z| = |z_{+} + z_{-}| \le |z_{+}| + |z_{-}|$, So if $|z| > t$ then at least one of $|z_{-}|,|z_{+}|$ is greater than $t/2$. Hence via the union  bound: 
\begin{equation*}
  \begin{split}
    \prbm{ |z|  }{z\in \Czdu} \le \prbm{ \bigcup_{i \in \pm }{|z_{i}| \ge t/2}   }{z\in \Czdu} \le \sum_{i \in \pm }{\prbm{ |z_{i}| \ge t/2   }{z\in \Czdu}}
  \end{split}
\end{equation*}

Since any two positive (negative) generators are disjoint we have that  $|z_{+}|$ is a sum of the independent random variables each stands for the weight contributed by a positive vertex. Let us denote by $V^{+}, V^{-}$ the positive and the negative vertices and for each vertex $v \in V$ we will denote by $_{v}$ the bits of $z$ restricted to $v$ edges. So $|z_{\pm}| = \sum_{v \in V^{\pm}}{ |z_{v}| }$. For simplicity assume that $|V^{+}| = |V^{-}| = n/2$ and that $\exppm{|z|}{z \in C_{A}\otimes C_{B}} = \mu $. Then we can use concentration inequality to have: 

\begin{equation*}
  \begin{split}
    \prbm{ |z|  }{z\in \Czdu} \le \sum_{i \in \pm }{\prbm{ \sum_{v \in V^{i}}{ |z_{v}|} \ge t/2   }{z\in \Czdu}} \le 2e^{-(\mu - \frac{t}{2}) n }
  \end{split}
\end{equation*}
Thus if $\mu - \gamma  \ge O (1) $ (from \Cref{claim:error} ) then with high probability the Toffoli is computed up to reducible error.    

\section{ Notations. }
We denote by $C_{g}$ the good qLDPC code \cite{Dinur} \cite{Pavel} \cite{leverrier2022quantum}, and by $C_{ft}$ the concatenation code presented at \cite{aharonov1999faulttolerant} ($ft$ stands for fault tolerance). For a code $C_{y}$, we use $\Phi_{y}, E_{y}, D_{y}$ to denote the channel maps circuits into the their matched circuits compute in the code space, the encoder, and the decoder, respectively. We use $\Phi_{U}$ to denote the 'Bell'-state storing the gate $U$. We say that a state $\ket{\psi}$ is at a distance $d$ from a quantum code $C$ if there exists an operator $U$ that sends $\ket{\psi}$ into $C$ such that $U$ is spanned on Paulis with a degree of at most $d$. Sometimes, when the code being used is clear from the context, we will say that a block $B$ of qubits has absorbed at most $d$ noise if the state encoded on $B$ is at a distance of at most $d$ from that code.

\section{ The Noise Model }


\section{ Fault Tolerance (With Resets gates) at Linear Depth. } 

\begin{claim}
There exists a value $p_{th} \in (0,1)$ such that if $p < p_{th}$, then any quantum circuit $C$ with a depth of $D$ and a width of $W$ can be computed by a $p$-noisy circuit $C^{\prime}$, which allows for resets. The depth of $C^{\prime}$ is at most $\max{ \{O(D), O(\log(WD)) \} }$.
\end{claim}


\subsection{Initializing Magic for Teleportation gates and encodes ancillaries.}
The Protocol: \begin{enumerate}
  \item Initialization of zeros: The qubits are divided into blocks of size $|B|$. Each block is encoded in $C_{g}$ using $D_{ft} \Phi_{ft}[E_{g}] \ket{0^{|B|}}$.
  \item Initialization of Magic for Teleportation gates: The gates in the original circuit are encoded in $C_{g}$ using $D_{ft} \Phi_{ft}[E_{g}] \ket{\Phi_{U}}$.
  \item Gate teleportation: Each gate in the original circuit is replaced by a gate teleportation.
  \item Error reduction: After the initialization step, at each time tick, each block runs a single round of error reduction.
\end{enumerate}

\begin{claim}[From \cite{leverrier2022decodingquantumtannercodes}]
  \label{claim:error} 
  Assuming that an error $|e| \le \gamma n $, i.e $e$ is supported on less than $\gamma n$ bits, then a single correction round reduce $e$ to an error $e^\prime$ such that $|e^{\prime}| < \nu |e|$. 
\end{claim}
 %Recall that by definition, $D_{i}E_{i} = I$, or in other words, $D_{i}= E_{i}^{\dagger}$.  
\begin{claim}
  \label{claim:noisepa}
  The gate $ D_{ft} \Phi_{ft}[E_{g}]$ initializes states encoded in $C_{g}$ subject to a $3p$-noise channel.  
\end{claim}
\begin{proof}
  Clearly, with high probability, $\Phi_{ft}[E_{g}]$ successfully encodes into $C_{ft} \circ C_{g}$, let's say with probability $1 - \frac{1}{poly(n)}$. Denote by $E_{i}$ and $D_{i}$ the encoder and decoder at the $i$th level of the concatenation construction. Consider the decoder under $\mathcal{N}$ action: $P_{2}D_{1}P_{2}D_{2},..,P_{i-1}D_{i}P_{i}$, by the fault-tolerance construction, a logical error at the $i$th stage occurs with probability $p^{2^{i}}$. Therefore, by the union bound, the probability that in one of the steps the circuit absorbs an error that is not corrected is less than $p + p^{2} + p^{4} + .. < 2p$. Hence, any decoded qubit absorbs noise with probability less than $2p$.


  Thus, overall, we can bound the probability of a single qubit being faulty by:
  \begin{equation*}
    \begin{split}
      \prb{\text{fault} } &=  \prb{\text{fault} |  \Phi_{ft}[E_{g}] }\cdot \prb{\Phi_{ft}[E_{g}]} + \prb{\text{fault} | \overline{\Phi_{ft}[E_{g}]} }\cdot \prb{\overline{\Phi_{ft}[E_{g}]}}\\
      &\le  \prb{\text{fault} |  \Phi_{ft}[E_{g}] } + \prb{\overline{\Phi_{ft}[E_{g}]}} \le 2p + \frac{1}{poly(n)} \le 3p
    \end{split}
  \end{equation*}

  \begin{remark}
In our construction, we use the concatenation code to encode blocks of length $\log(n)$. Therefore, any $poly(n)$ in the above should be replaced by $\log(n)$. However, this does not affect anything since the inequality does not depend on $n$.
  \end{remark}

%
%
%  \begin{equation*}
%    \begin{split}
%      \mathcal{N}(D) &= \left((\mathcal{N}(D))^{\dagger}\right)^{\dagger} =  \left(\sum_{P_{1}, P_{2}, .., P_{i} \in \mathcal{P}}{ \prb{P_{1}, P_{2}, .., P_{i}}  \left(D_{1}P_{2}D_{2},..,P_{i-1}D_{i}P_{i}\right)^{\dagger}} \right)^{\dagger} \\ 
%      &= \left( \sum_{P_{1}, P_{2}, .., P_{i} \in \mathcal{P}}{ \prb{P_{1}, P_{2}, .., P_{i}}  P_{i}E_{i}P_{i-1}E_{i-1},..,P_{1}E_{1} } \right)^{\dagger}\\
%      &= \left( \left( 1 -\frac{1}{poly(n)} \right)\sum_{P_{i} \in \mathcal{P}}\prb{P_{i}}P_{i}E + \frac{1}{poly(n)} A  \right)^{\dagger} \\ 
%      &= \left( 1 -\frac{1}{poly(n)} \right)\sum_{P_{i} \in \mathcal{P}}\prb{P_{i}}DP_{i} + \frac{1}{poly(n)} A 
%\end{split}
%  \end{equation*}
%
%  %Since $D$ is semi-transversal gate, it preserves the 
%
%
%  And notice that $\star$ is with probability $1 - \frac{1}{poly(n)}$ equals to $E_{i}E_{i-1}..,E_{1}=E$. Hence $\mathcal{N}(D)$ equals to $\left( P E \right)^{\dagger} = PD$.
%
%  \begin{equation*}
%    \begin{split}
%      \braket{ \psi^{\prime} | P_{i}E_{i}P_{i-1}E_{i-1},..,P_{1}E_{1} \psi } = \braket{ \psi^{\prime} P_{i}D_{i}P_{i-1}D_{i-1},..,P_{1}D_{1} | \psi }
%    \end{split}
%  \end{equation*}
%  Thus for any pauli-channel $\mathcal{N} : L(H) \rightarrow L(H)$, and $\psi^{\prime}$ which is a codeword we get: 
%  \begin{equation*}
%    \begin{split}
%      \braket{ \psi^{\prime} \mathcal{N}(D) | \psi } &=  \sum_{P_{1}, P_{2}, .., P_{i} \in \mathcal{P}}{ \prb{P_{1}, P_{2}, .., P_{i}}  \braket{ \psi^{\prime} P_{i}D_{i}P_{i-1}D_{i-1},..,P_{1}D_{1} | \psi }} \\
%      &=  \sum_{P_{1}, P_{2}, .., P_{i} \in \mathcal{P}^{\star}}{  \prb{P_{1}, P_{2}, .., P_{i}}\braket{ \psi^{\prime} | P_{i}E_{i}P_{i-1}E_{i-1},..,P_{1}E_{1} \psi }} \pm O(  \frac{1}{poly(n)})\\
%      &=  \sum_{P_{1}, P_{2}, .., P_{i} \in \mathcal{P}^{\star}}{  \prb{P_{1}, P_{2}, .., P_{i}}\braket{ \psi^{\prime} | P_{i} E \psi }} \pm O(  \frac{1}{poly(n)})\\
%      &\le  \sum_{ P_{i} \in \mathcal{P}}{  \prb{ P_{i}}\braket{ \psi^{\prime} | P_{i} E \psi }} \pm O(  \frac{1}{poly(n)}) \\
%      &\le  \sum_{ P_{i} \in \mathcal{P}^{\le d}}{  \prb{ P_{i}}\braket{ \psi^{\prime} | P_{i} E \psi }} \pm O (e^{-d \cdot n} ) \pm O(  \frac{1}{poly(n)}) \\
%      & \le   \sum_{ P_{i} \in \mathcal{P}/\mathcal{P}^{\star}}{  \prb{ P_{j} \in B_{d}\left( P_{i} \right)}\braket{ \psi^{\prime} | P_{i} E \psi }}  \pm O (e^{-d \cdot n} ) \pm O(  \frac{1}{poly(n)}) 
%    \end{split}
%  \end{equation*}
%  Using the fact that the concatenation code is monotonic (\Cref{def:mono}) we get that the probability to have physical fault $P_{j}$.   
%%\end{widetext}
\end{proof}

\begin{claim}
  \label{claim:prob}
  With a probability $ 1 - \frac{WD}{|B|} \cdot D 2e^{-2|B|(\beta - p)} $, the total amount of noise absorbed in a block at any given time $t$, is less than $\gamma n$. 
\end{claim}
\begin{proof}
Consider the $i$th block, denoted by $B_{i}$. By applying Hoeffding's inequality, we have that the probability that more than $\beta |B|$ qubits are flipped at time $t$ is less than $2e^{-2|B|(\beta - p)}$. By using the union bound over all blocks at all time locations, we can conclude that with probability $1 - \frac{WD}{|B|} \cdot D 2e^{-2|B|(\beta - p)}$, the noise absorbed in a block is less than $|\beta|B$ for the entire computation.

Let $X_{t}$ denote the support size of the error over $B_{i}$ at time $t$. Using \Cref{claim:error}, we can bound the total amount of error absorbed by a block until time $t$ as follows:
\begin{equation*}
\begin{split}
X_{t} \le \nu \cdot (X_{t-1} + \beta |B| ) \le \nu(\gamma+\beta) |B| \le \gamma |B|
\end{split}
\end{equation*}
\end{proof}


\begin{claim}
  The total depth of the circuit is $O ( D  ) + O ( \log^{c} |B| )$. 
\end{claim}
\begin{proof}
  The gate for encoding $|B|$-length blocks in $C_{g}$ is a Clifford gate and can therefore be computed in $O(\log|B|)$ depth. The encoding of the magic/bell states is done by first computing them in the logical space (un-encoded qubits) and then encode them using the encoder. Hence, the fault-tolerant version of both initializing ancillaries and magic states/bell states costs $O( (\log |B|) \cdot \log^{c}( |B| \log |B| ) )$ \footnote{The width of the original circuit is $|B|^{2}$ so the number of locations is $ |B|^{2} \cdot \log |B|$} depth \cite{aharonov1999faulttolerant}. Backing into $C_{g}$ from $C_{ft}$ by decoding the concatenation code takes exactly as long as the encoding, namely $O( (\log |B|) \cdot \log^{c}( |B| \log |B| ) )$.

  Then, using the bell measurements, any of the logical gates takes $O(1)$ depth. Since we only perform a single round of error correction, the remaining computation until the last decoding stage takes at most constant time of the original depth. Finally, we pay $O(\log |B|)$ for complete decoding. Summing all, we get: 
  \begin{equation*}
    \begin{split}
     &  O ( \log |B|\cdot  \log^{c}( |B| \log |B| ) )  + O ( D  ) + O ( \log |B| ) \\ 
     = & O ( D  ) + O ( \log^{c} |B| )
    \end{split}
  \end{equation*}
\end{proof}

Assuming that $W$ is polynomial in $D$, taking the block length to be $|B| = \log((W \cdot D)^c)$, as shown in \Cref{claim:prob}, results in a linear fault tolerance construction with a success probability of $1 - \frac{1}{\log^{c_2}(W \cdot D)}$. This means that the fault tolerance version of circuits in $\textbf{QNC}_1$ has a logarithmic depth. Additionally, using the construction in \cite{aharonov1996limitationsnoisyreversiblecomputation} produces a polynomial fault tolerance circuit in the reversible gates setting. \ctt{ We missed the fact that it requires non trivial classical computation to compute what gate should be applied after the gate teleportation (i.e $UPU^{\dagger}$ )}.




%\input{./tempmagic.tex}
%\cite{leverrier2022quantum}
%\cite{moore1998parallel}
%\cite{Tillich_2014}
%\cite{meier2012magicstate}
%\cite{bravyi2012magic}
\printbibliography

\end{document}

