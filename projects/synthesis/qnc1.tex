

%\newcommand*{\ACM}{}%

\ifdefined\ACM

%\documentclass[sigplan,screen]{acmart}
\documentclass[manuscript,screen,review]{acmart}

\else
  \documentclass[14pt]{article}
\usepackage{libertine}
\usepackage[utf8]{inputenc}
\usepackage[a4paper, total={6.5in, 10in} ]{geometry}
\usepackage{braket}
\usepackage{xcolor}
\usepackage{amsmath}
\usepackage{amssymb}
\usepackage{amsfonts}
\usepackage{graphicx}
\usepackage{svg}
\usepackage{float}
\usepackage{tikz}
\usetikzlibrary{patterns, shapes.arrows}
\usepackage{adjustbox}
\usepackage{tikz-network}
\usepackage[ruled,lined,linesnumbered]{algorithm2e}
\usepackage{multicol}
\usepackage[backend=biber,style=alphabetic,sorting=ynt]{biblatex}
\usepackage{xcolor}
\usepackage{pgfplots}
\DeclareUnicodeCharacter{2212}{−}
\usepgfplotslibrary{groupplots,dateplot}
\pgfplotsset{compat=newest}



\usepackage{cancel}
\usepackage{subcaption}
\addbibresource{./sample.bib}

\fi

\begin{document}
\input{newcommands}
\title{ $\textbf{QNC}_{1} \subset $ noisy-\textbf{BQP}}
\author{Michael Benor \ \ David Ponarovsky}
\maketitle

\newcommand*{\Mbas}{\mathcal{X}^\prime}
\newcommand*{\bas}{\mathcal{X}}
\newcommand*{\sMbas}{\Mbas}
\newcommand*{\QQ}{C_{X}/C_{Z}^\perp }
\newcommand*{\trig}{ Triorthogonal }
\newcommand*{\Hyp}{ Hyperproduct }
\newcommand*{\Cin}{ C_{\text{initial}} }
\newcommand*{\Ctan}{ C_{\text{Tan}} }



%\section{Introduction} blablabla


\section{ Notations. }
$C_{g}$ - good qLDPC, $C_{ft}$ - concatenation code ($ft$ stands for fault tolerance). For a code $C_{y}$ we use $\Phi_{y}, E_{y}, D_{y}$ to denote the maps circuits into the circuits compute in the code space, the encoder, and the decoder.    


\section{ The Noise Model }


\section{ Fault Tolerance (With Resets gates) at Linear Depth. } 

\begin{claim}
  There is $p_{th} \in (0,1)$ such that if $p < p_{th}$ then any quantum circuit $C$ with depth $D$ and width $W$ can be computed by $p$-noisy, resets allowed, circuit $C^{\prime}$, with a depth at most $\max{ \{D, \log(WD) \} }$. 
\end{claim}


\subsection{Initializing Magic for Teleportation gates and encodes ancillaries.}


\begin{claim}
  There is prot
\end{claim}




\begin{enumerate}
  \item Initializing Magic for Teleportation gates and encodes ancillaries.
  \item Each gate is replaced by gate teleportation.  
  \item  
\end{enumerate}







%\input{./tempmagic.tex}
%\cite{leverrier2022quantum}
%\cite{moore1998parallel}
%\cite{Tillich_2014}
%\cite{meier2012magicstate}
%\cite{bravyi2012magic}
\printbibliography

\end{document}

