

%\newcommand*{\ACM}{}%

\ifdefined\ACM

%\documentclass[sigplan,screen]{acmart}
\documentclass[manuscript,screen,review]{acmart}

\else
  \documentclass[18pt]{article}
\usepackage{libertine}
\usepackage{cuted}
%\usepackage{widetext}
\usepackage[utf8]{inputenc}
\usepackage[a4paper, total={6.5in, 10in} ]{geometry}
\usepackage{braket}
\usepackage{xcolor}
\usepackage{amsmath}
\usepackage{amssymb}
\usepackage{amsfonts}
\usepackage{graphicx}
\usepackage{svg}
\usepackage{float}
\usepackage{tikz}
\usetikzlibrary{patterns, shapes.arrows}
\usepackage{adjustbox}
\usepackage{tikz-network}
\usepackage[ruled,lined,linesnumbered]{algorithm2e}
\usepackage{multicol}
\usepackage[backend=biber,style=alphabetic,sorting=ynt]{biblatex}
\usepackage{xcolor}
\usepackage{pgfplots}
\DeclareUnicodeCharacter{2212}{−}
\usepgfplotslibrary{groupplots,dateplot}
\pgfplotsset{compat=newest}



\usepackage{cancel}
\usepackage{subcaption}
\addbibresource{./sample.bib}

\fi

\begin{document}
\input{newcommands}
%\title{ $\textbf{QNC}_{1} \subset $ noisy-\textbf{BQP}}
\title{ On The Cost of Fault-Tolerainzing Shallow Circuits. }
\author{Michael Ben-Or \ \ David Ponarovsky}
\maketitle

\newcommand*{\Mbas}{\mathcal{X}^\prime}
\newcommand*{\bas}{\mathcal{X}}
\newcommand*{\sMbas}{\Mbas}
\newcommand*{\QQ}{C_{X}/C_{Z}^\perp }
\newcommand*{\trig}{ Triorthogonal }
\newcommand*{\Hyp}{ Hyperproduct }
\newcommand*{\Cin}{ C_{\text{initial}} }
\newcommand*{\Ctan}{ C_{\text{Tan}} }



\newcommand*{\QACze}{ \mathbf{QAC}_{0} }
\newcommand*{\QNCzef}{ \mathbf{QNC}_{0,f} }
\newcommand*{\QNCon}{ \mathbf{QNC}_{1} }
\newcommand*{\noiseQNCon}{ noisy-$\QNCon$ }


\abstract{In this work we study the overall depth overhead cost required for constructing fault tolerance circuits. We focus on shallow depth circuits classes, In particular, $\QACze, \QNCzef$ and $\QNCon$ and certain knowns problem candidates for demonstrating quantum advantage such as factoring \cite{Shor_1997} and Instantaneous Quantum Polynomial-time \cite{Bremner_2017}, \cite{Paletta_2024}. We only give a partial answers, Yet, clues that might pave the way towards a full understanding the complexity versus fault tolerance trade-off. 

  \section{Introduction.} The question about the feasibility of computation under noise is almost as ancient as the computer science field itself, initialized by Von Neumann \cite{Neumann+1956+43+98} at the time that classical computation putted in debuts. Time been pass and the followed works had pointed that not even a polynomial computation in the presence of noise is still reasonable but one can implement a fault tolerance version at a most constant times cost at the circuits depth \cite{Pippenger}. Or in asymptotic sense, classical computation in the presence of noise is as exactly hard as computation in ideal environment.  

  Once again, the feasibility question raised again, this time regarding quantum computing, and while an intensive work has been done, and also succeed to prove that polynomial quantum computation can be made fault tolerance, \cite{aharonov1999faulttolerant},\cite{gottesman2014faulttolerant} and even with only constant overhead at the original circuit width \cite{grospellier:tel-03364419}, the required depth over-head is till not well understood. We stress out that in all the familiar constructions, in construct to Pippenger \cite{Pippenger}, original constant-depth gates are mapped to asymptotically grow\footnote{Note, that here, classical computation is also counted in the overall depth cost} depth gates. 
  
  This work address the above, We ask whether a magnitude depth overhead is an unavoidable price that one has to pay. And, in particular, whether an ideal $\QNCon$ circuits can be computed in noisy-$\QNCon$ circuits. We show how using the ideas presented in \cite{grospellier:tel-03364419} and \cite{Pippenger} gives almost immediately $\QNCzef \subset$ \noiseQNCon and that sampling from IQP \cite{Bremner_2017} also can be done in logarithmic depth circuits.    
  \section{ Notations. } In the following we present the notations used in the paper, readers who familiar with the literature of quantum fault tolerance might skip this section.  


\subsection{Classical Coding Theory.} The notation used in this paper follows standard conventions for coding theory. We use $n$ to represent the length of the code, $k$ for the code's dimension, and $\rho$ for its rate. The minimum distance of the code will be denoted as $d$, and the relative distance, i.e., $d/n$, as $\delta$. In this paper, $n$ and $k$ will sometimes refer to the number of physical and logical bits. Codes will be denoted by a capital $C$ followed by either a subscript or superscript. When referring to multiple codes, we will use the above parameters as functions. For example, $\rho(C_{1})$ represents the rate of the code $C_{1}$.

Square brackets are used to present all these parameters compactly, and we use them as follows: $C=[n,k,d]$ to declare a code with the specified length, dimension, and distance. Any theorem, lemma, or claim that states a statement that is true in the asymptotic sense refers to a family of codes. The parity check matrix of the code will be denoted as $H$, with the rows of $H$ representing the parity check equations. The generator matrix of the code will be denoted as $G$, with the rows of $G$ representing the basis of codewords. The syndrome of a received word will be denoted as $s$, which is the result of multiplying $r$ by the transpose of $H$. We use $C^\perp$ to denote the dual code of $C$, which is defined such that any codeword of it $z\in C^\perp$ is orthogonal to any $x\in C$, meaning $z\cdot x = 0$, where the product is defined as $x\cdot z = \sum_{i}{x_{i}z_{i}}$. $C^{\top}$ stands for the code obtained by taking the parity check matrix of $C$ and transposing it.

In this paper, we define the triple product $\mathbb{F}_2^{n}\times \mathbb{F}_2^{n}\times\mathbb{F}_2^{n} \rightarrow \mathbb{Z}$ as $|x\cdot y \cdot z| = \sum_{i}^{n}{x_{i}y_{i}z_{i}}$. Similarly, we define the binary product $|x \cdot y|$, noting that this product differs from the standard product by mapping into $\mathbb{Z}$ rather than $\mathbb{F}_{2}$. For $w \in \mathbb{F}_{2}^{n}$, we use the super operator $ \cdot |_{w} $ to map an operator originally defined in an $n$-dimensional space to an operator that only acts on coordinates restricted to $w$. For example, $x|_{w}$ is the vector in $\mathbb{F}_{2}^{|w|}$ obtained by taking the values of $x$ on coordinates where $w$ is not zero. $|x\cdot y|_{w} = \sum_{i:w_{i}\neq 0}{x_{i}y_{i}}$ and $C|_{w}$ is the code obtained by taking the codewords of $C$ restricted to $w$.

%\begin{definition}
%  \label{def:trig}
%  Let $C$, $\tilde{C}$ be linear binary codes at the same length, We will say that $\tilde{C}$ is a \trig with respect to $C$ if: 
%  \begin{enumerate}
%    \item $\tilde{C} \subset C$
%    \item $|x\cdot y \cdot z|$ is even for $x,y,z \in C$ such that at least one of $x,y,z$  belongs to $\tilde{C}$. 
%    \item $|x\cdot y|$ is even for $x,y \in C$ such that at least one of $x,y$  belongs to $\tilde{C}$. 
%  \end{enumerate}
%  If a code $C$ is \trig with respect to itself then we will say that $C$ is a self \trig code. 
%\end{definition}
%For example, the empty code, that contains only the zero code word, i.e $C = \{ 0 \}$, is a \trig with respect to any code. In fact for proving \Cref{theorem:main} taking the empty code is sufficient. For other example, the \trig codes defined in \cite{bravyi2012magic} are \trig with respect to themself. 

\subsection{Quantum Codes.} A quantum code over $n$ qubits is an embedding of $\mathcal{H}_{2}^{\otimes k}$ as a subspace of $\mathcal{H}_{2}^{\otimes n}$. Similar to classical codes, we will call $n$ and $k$ the physical and logical qubits. The embeddings of states in $\mathcal{H}_{2}^{\otimes k}$ are called codewords or encoded states. In addition, we will use the term "logical operator" (i.e. logical $X_{i}$) to describe an operator that acts on the code space exactly as it would act on the logical space $\mathcal{H}_{2}^{\otimes k}$ (in our example, turning on and off the encoded state corresponds to the $i$th qubit exactly as $X_{i}$ acts as Pauli $X$ on the $i$th qubit in $\mathcal{H}_{2}^{\otimes k}$). 

We will denote by $X$ and $Z$ the single $X$ and $Z$ Pauli operators, by $X_{i}$ the application of $X$ on the $i$th qubit and nothing else (identity) on the rest of the qubits. By $X^{(v)}$ for some $v \in \mathbb{F}_{2}^{n}$, we mean the operator composed by applying $X$ on each of the qubits whose index is a non-trivial coordinate of $v$ and identity elsewhere. In a similar fashion, we define $Z^{(v)}$. When the context is clear, we will allow ourselves to omit the brackets, i.e. $Z^{v}$. The weight of a Pauli operator is the number of coordinates on which the operator acts non-trivially. Recall that the set of Pauli $+ I$ spans all the Hermitian matrices. We say that the Pauli weight of an operator is the maximal weight of a Pauli in its Pauli decomposition. For example, consider the operator $A = IXX + ZII$, the weight of $A$ is $2$. The distance of a quantum code is the minimal weight of an operator that takes one codeword to another. We use the standard bracket notation to describe quantum states and in addition, we define for a vector space $A \subset \mathbb{F}_{2}^{n}$ the notation $\ket{A}$ to represent the uniform superposition of all the vectors belonging to that space, namely: \begin{equation*}
  \begin{split}
\ket{A} = \frac{1}{\sqrt{|A|}}\sum_{x \in A}{\ket{x}}
  \end{split}
\end{equation*}
We define in the same way the notation to hold for affine spaces, $\ket{x +A}$. We will use $\propto$ to denote a quantum states up to normalization factor, for example $\ket{\psi} \propto \ket{0} + \ket{1}$ means that $\ket{\psi} = \frac{1}{\sqrt{2}}(\ket{0} + \ket{1})$.
A CSS code is a quantum code defined by a pair of classical codes $C_{X}$ and $C_{Z}$, satisfying $C_{Z}^{\perp} \subset C_{X}$, such that any codeword of it has the form $\ket{x + C_{Z}^\perp}$, where $x \in C_{X}$. We will use $Q$ to refer to a CSS code in general and use $\QQ$ to refer to the vectors associated with the $X$-generators or the encoded states in the computational basis. In the same way, $C_{Z}/C_{X}^{\perp}$ refers to the $Q$ in the phase basis. We will say that a CSS code $Q$ is a LDPC if $C_{X}$ and $C_{Z}$ are both LDPC codes. Our construction uses the classical Tanner code \cite{Tanner}, the expander codes \cite{ExpanderCodes}, and \Hyp  code (quantum expanders) \cite{Leverrier_2015}, \cite{Tillich_2014}, \cite{overheadofquantumerrorcorrection}. We will not describe these constructions and refer the reader to those papers for further information.

\subsection{Decoders.}
\label{sec:deoders.}

We denote by $C_{g}$ the good qLDPC code \cite{Dinur} \cite{Pavel} \cite{leverrier2022quantum}, and by $C_{ft}$ the concatenation code presented at \cite{aharonov1999faulttolerant} ($ft$ stands for fault tolerance). For a code $C_{y}$, we use $\Phi_{y}, E_{y}, D_{y}$ to denote the channel maps circuits into the their matched circuits compute in the code space, the encoder, and the decoder, respectively. We use $\Phi_{U}$ to denote the 'Bell'-state storing the gate $U$. We say that a state $\ket{\psi}$ is at a distance $d$ from a quantum code $C$ if there exists an operator $U$ that sends $\ket{\psi}$ into $C$ such that $U$ is spanned on Paulis with a degree of at most $d$. Sometimes, when the code being used is clear from the context, we will say that a block $B$ of qubits has absorbed at most $d$ noise if the state encoded on $B$ is at a distance of at most $d$ from that code.


\section{Todo:}
\begin{enumerate}
  \item Move to encoding each qubit by logarithmic width (instead of chanks) the reason is that the gate teleportation becomes complicated when it applied over higher dimension. 
  \item Then showing for 2-qubit gates set that is indeed works.
  \item Treating separately to noise observed in two qubits gates. 
\end{enumerate}


\section{ Fault tolerance Toffoli. } 

\ctt{In that section the $\cdot$ operation is the pair wise product (pair wise AND).}

Assume that $\bar{0}, \bar{1} \in C_{X}$ and that they belong to two different cosets of $\QQ$. Let $x,y \in \{ \bar{0},\bar{1}  \}$. 
\begin{equation}
  \label{equation:toff}
  \begin{split}
&    \sum_{z,z^{\prime},w \in \Czdu }{ \ket{z}\ket{z^{\prime}}\ket{w} } \\  
&    \sum_{z,z^{\prime},w \in \Czdu }{ \ket{z}\ket{z^{\prime}}\ket{w + z\cdot z^{\prime}} } \\  
&    \sum_{z,z^{\prime},w \in \Czdu }{ \ket{z + x }\ket{z^{\prime} + y}\ket{w + z\cdot z^{\prime}} } \\  
&    \sum_{z,z^{\prime},w \in \Czdu }{ \ket{z + x }\ket{z^{\prime} + y}\ket{ x\cdot y + x \cdot z^{\prime} + y \cdot z + z z^{\prime} + w + z\cdot z^{\prime}} } \\  
&    \sum_{z,z^{\prime},w \in \Czdu }{ \ket{z + x }\ket{z^{\prime} + y}\ket{ x\cdot y + x \cdot z^{\prime} + y \cdot z +  w } } 
  \end{split}
\end{equation}
Since $x,y \in \{ \bar{0},\bar{1}  \}$ we have that $ x\cdot z^{\prime}$ equals to either $z^{\prime}$ or $\bar{0}$. Hence $ \sum_{w \in \Czdu}{\ket { \xi +x \cdot z + w } } =  \sum_{w \in \Czdu}{\ket { \xi + w } } $. So the idea is the following, suppose that one has to compute Toffoli at time $t$ over the registers $R_{1},R_{2},R_{3}$. First, at time $0$, he initialize a logical zero $\ket{\Czdu}$ in each register, then he compute pairwise Toffoli $R_{1},R_{2}$ into $R_{3}$. That gives the ket $\sum_{z,z^{\prime},w \in \Czdu}{\ket{ z\cdot z^{\prime} + w}}$,  immediately afterwords encode $R_{3}$ again into a good quantum code. Denote by $\tau$ the time required for decoding $R_{3}$ back, at time $t-\tau$ start to decode $R_{3}$. Eventually at time time $t$ compute again the transversal Toffoli, by \Cref{equation:toff} we gets the desired.  


By similar arguments exhibited at \Cref{claim:noisepa} one can show that the errors behaves according to a Pauli noise channel. \ctt{That is not correct, since the concatenation construction assumes that all the registers initialized to physical zeros in the begging of the computation}.

\subsection{Another Idea, $z\cdot z^{\prime}$ cann't contribute too mach.}
Clearly we have that  $|z\cdot z^{\prime}| \le |z|,|z^{\prime}|$ therefore we have that $\prbm{ | z \cdot z^{\prime} | \ge t}{z,z^\prime \in \Czdu} \le \prbm{ | z | \ge t}{z\in \Czdu}$. Now assume that the tanner code by which the code defined is bipartite graph and denote by $z_{+},z_{-}$ the grouping of the $z$'s generators supported on the even and the odd vertices of the graph. By triangle inequality $|z| = |z_{+} + z_{-}| \le |z_{+}| + |z_{-}|$, So if $|z| > t$ then at least one of $|z_{-}|,|z_{+}|$ is greater than $t/2$. Hence via the union  bound: 
\begin{equation*}
  \begin{split}
    \prbm{ |z|  }{z\in \Czdu} \le \prbm{ \bigcup_{i \in \pm }{|z_{i}| \ge t/2}   }{z\in \Czdu} \le \sum_{i \in \pm }{\prbm{ |z_{i}| \ge t/2   }{z\in \Czdu}}
  \end{split}
\end{equation*}

Since any two positive (negative) generators are disjoint we have that  $|z_{+}|$ is a sum of the independent random variables each stands for the weight contributed by a positive vertex. Let us denote by $V^{+}, V^{-}$ the positive and the negative vertices and for each vertex $v \in V$ we will denote by $_{v}$ the bits of $z$ restricted to $v$ edges. So $|z_{\pm}| = \sum_{v \in V^{\pm}}{ |z_{v}| }$. For simplicity assume that $|V^{+}| = |V^{-}| = n/2$ and that $\exppm{|z|}{z \in C_{A}\otimes C_{B}} = \mu $. Then we can use concentration inequality to have: 

\begin{equation*}
  \begin{split}
    \prbm{ |z|  }{z\in \Czdu} \le \sum_{i \in \pm }{\prbm{ \sum_{v \in V^{i}}{ |z_{v}|} \ge t/2   }{z\in \Czdu}} \le 2e^{-(\mu - \frac{t}{2}) n }
  \end{split}
\end{equation*}
Thus if $\mu - \gamma  \ge O (1) $ (from \Cref{claim:error} ) then with high probability the Toffoli is computed up to reducible error.    

\section{ The Noise Model }


\section{ Fault Tolerance (With Resets gates) at Linear Depth. } 

\begin{claim}
There exists a value $p_{th} \in (0,1)$ such that if $p < p_{th}$, then any quantum circuit $C$ with a depth of $D$ and a width of $W$ can be computed by a $p$-noisy circuit $C^{\prime}$, which allows for resets. The depth of $C^{\prime}$ is at most $\max{ \{O(D), O(\log(WD)) \} }$.
\end{claim}


\subsection{Initializing Magic for Teleportation gates and encodes ancillaries.}
The Protocol: \begin{enumerate}
  \item Initialization of zeros: The qubits are divided into blocks of size $|B|$. Each block is encoded in $C_{g}$ using $D_{ft} \Phi_{ft}[E_{g}] \ket{0^{|B|}}$.
  \item Initialization of Magic for Teleportation gates: The gates in the original circuit are encoded in $C_{g}$ using $D_{ft} \Phi_{ft}[E_{g}] \ket{\Phi_{U}}$.
  \item Gate teleportation: Each gate in the original circuit is replaced by a gate teleportation.
  \item Error reduction: After the initialization step, at each time tick, each block runs a single round of error reduction.
\end{enumerate}

\begin{claim}[From \cite{leverrier2022decodingquantumtannercodes}]
  \label{claim:error} 
  Assuming that an error $|e| \le \gamma n $, i.e $e$ is supported on less than $\gamma n$ bits, then a single correction round reduce $e$ to an error $e^\prime$ such that $|e^{\prime}| < \nu |e|$. 
\end{claim}
 %Recall that by definition, $D_{i}E_{i} = I$, or in other words, $D_{i}= E_{i}^{\dagger}$.  
\begin{claim}
  \label{claim:noisepa}
  The gate $ D_{ft} \Phi_{ft}[E_{g}]$ initializes states encoded in $C_{g}$ subject to a $3p$-noise channel.  
\end{claim}
\begin{proof}
  Clearly, with high probability, $\Phi_{ft}[E_{g}]$ successfully encodes into $C_{ft} \circ C_{g}$, let's say with probability $1 - \frac{1}{poly(n)}$. Denote by $E_{i}$ and $D_{i}$ the encoder and decoder at the $i$th level of the concatenation construction. Consider the decoder under $\mathcal{N}$ action: $P_{2}D_{1}P_{2}D_{2},..,P_{i-1}D_{i}P_{i}$, by the fault-tolerance construction, a logical error at the $i$th stage occurs with probability $p^{2^{i}}$. Therefore, by the union bound, the probability that in one of the steps the circuit absorbs an error that is not corrected is less than $p + p^{2} + p^{4} + .. < 2p$. Hence, any decoded qubit absorbs noise with probability less than $2p$.


  Thus, overall, we can bound the probability of a single qubit being faulty by:
  \begin{equation*}
    \begin{split}
      \prb{\text{fault} } &=  \prb{\text{fault} |  \Phi_{ft}[E_{g}] }\cdot \prb{\Phi_{ft}[E_{g}]} + \prb{\text{fault} | \overline{\Phi_{ft}[E_{g}]} }\cdot \prb{\overline{\Phi_{ft}[E_{g}]}}\\
      &\le  \prb{\text{fault} |  \Phi_{ft}[E_{g}] } + \prb{\overline{\Phi_{ft}[E_{g}]}} \le 2p + \frac{1}{poly(n)} \le 3p
    \end{split}
  \end{equation*}

  \begin{remark}
In our construction, we use the concatenation code to encode blocks of length $\log(n)$. Therefore, any $poly(n)$ in the above should be replaced by $\log(n)$. However, this does not affect anything since the inequality does not depend on $n$.
  \end{remark}

%
%
%  \begin{equation*}
%    \begin{split}
%      \mathcal{N}(D) &= \left((\mathcal{N}(D))^{\dagger}\right)^{\dagger} =  \left(\sum_{P_{1}, P_{2}, .., P_{i} \in \mathcal{P}}{ \prb{P_{1}, P_{2}, .., P_{i}}  \left(D_{1}P_{2}D_{2},..,P_{i-1}D_{i}P_{i}\right)^{\dagger}} \right)^{\dagger} \\ 
%      &= \left( \sum_{P_{1}, P_{2}, .., P_{i} \in \mathcal{P}}{ \prb{P_{1}, P_{2}, .., P_{i}}  P_{i}E_{i}P_{i-1}E_{i-1},..,P_{1}E_{1} } \right)^{\dagger}\\
%      &= \left( \left( 1 -\frac{1}{poly(n)} \right)\sum_{P_{i} \in \mathcal{P}}\prb{P_{i}}P_{i}E + \frac{1}{poly(n)} A  \right)^{\dagger} \\ 
%      &= \left( 1 -\frac{1}{poly(n)} \right)\sum_{P_{i} \in \mathcal{P}}\prb{P_{i}}DP_{i} + \frac{1}{poly(n)} A 
%\end{split}
%  \end{equation*}
%
%  %Since $D$ is semi-transversal gate, it preserves the 
%
%
%  And notice that $\star$ is with probability $1 - \frac{1}{poly(n)}$ equals to $E_{i}E_{i-1}..,E_{1}=E$. Hence $\mathcal{N}(D)$ equals to $\left( P E \right)^{\dagger} = PD$.
%
%  \begin{equation*}
%    \begin{split}
%      \braket{ \psi^{\prime} | P_{i}E_{i}P_{i-1}E_{i-1},..,P_{1}E_{1} \psi } = \braket{ \psi^{\prime} P_{i}D_{i}P_{i-1}D_{i-1},..,P_{1}D_{1} | \psi }
%    \end{split}
%  \end{equation*}
%  Thus for any pauli-channel $\mathcal{N} : L(H) \rightarrow L(H)$, and $\psi^{\prime}$ which is a codeword we get: 
%  \begin{equation*}
%    \begin{split}
%      \braket{ \psi^{\prime} \mathcal{N}(D) | \psi } &=  \sum_{P_{1}, P_{2}, .., P_{i} \in \mathcal{P}}{ \prb{P_{1}, P_{2}, .., P_{i}}  \braket{ \psi^{\prime} P_{i}D_{i}P_{i-1}D_{i-1},..,P_{1}D_{1} | \psi }} \\
%      &=  \sum_{P_{1}, P_{2}, .., P_{i} \in \mathcal{P}^{\star}}{  \prb{P_{1}, P_{2}, .., P_{i}}\braket{ \psi^{\prime} | P_{i}E_{i}P_{i-1}E_{i-1},..,P_{1}E_{1} \psi }} \pm O(  \frac{1}{poly(n)})\\
%      &=  \sum_{P_{1}, P_{2}, .., P_{i} \in \mathcal{P}^{\star}}{  \prb{P_{1}, P_{2}, .., P_{i}}\braket{ \psi^{\prime} | P_{i} E \psi }} \pm O(  \frac{1}{poly(n)})\\
%      &\le  \sum_{ P_{i} \in \mathcal{P}}{  \prb{ P_{i}}\braket{ \psi^{\prime} | P_{i} E \psi }} \pm O(  \frac{1}{poly(n)}) \\
%      &\le  \sum_{ P_{i} \in \mathcal{P}^{\le d}}{  \prb{ P_{i}}\braket{ \psi^{\prime} | P_{i} E \psi }} \pm O (e^{-d \cdot n} ) \pm O(  \frac{1}{poly(n)}) \\
%      & \le   \sum_{ P_{i} \in \mathcal{P}/\mathcal{P}^{\star}}{  \prb{ P_{j} \in B_{d}\left( P_{i} \right)}\braket{ \psi^{\prime} | P_{i} E \psi }}  \pm O (e^{-d \cdot n} ) \pm O(  \frac{1}{poly(n)}) 
%    \end{split}
%  \end{equation*}
%  Using the fact that the concatenation code is monotonic (\Cref{def:mono}) we get that the probability to have physical fault $P_{j}$.   
%%\end{widetext}
\end{proof}

\begin{claim}
  \label{claim:prob}
  With a probability $ 1 - \frac{WD}{|B|} \cdot D 2e^{-2|B|(\beta - p)} $, the total amount of noise absorbed in a block at any given time $t$, is less than $\gamma n$. 
\end{claim}
\begin{proof}
Consider the $i$th block, denoted by $B_{i}$. By applying Hoeffding's inequality, we have that the probability that more than $\beta |B|$ qubits are flipped at time $t$ is less than $2e^{-2|B|(\beta - p)}$. By using the union bound over all blocks at all time locations, we can conclude that with probability $1 - \frac{WD}{|B|} \cdot D 2e^{-2|B|(\beta - p)}$, the noise absorbed in a block is less than $|\beta|B$ for the entire computation.

Let $X_{t}$ denote the support size of the error over $B_{i}$ at time $t$. Using \Cref{claim:error}, we can bound the total amount of error absorbed by a block until time $t$ as follows:
\begin{equation*}
\begin{split}
X_{t} \le \nu \cdot (X_{t-1} + \beta |B| ) \le \nu(\gamma+\beta) |B| \le \gamma |B|
\end{split}
\end{equation*}
\end{proof}


\begin{claim}
  The total depth of the circuit is $O ( D  ) + O ( \log^{c} |B| )$. 
\end{claim}
\begin{proof}
  The gate for encoding $|B|$-length blocks in $C_{g}$ is a Clifford gate and can therefore be computed in $O(\log|B|)$ depth. The encoding of the magic/bell states is done by first computing them in the logical space (un-encoded qubits) and then encode them using the encoder. Hence, the fault-tolerant version of both initializing ancillaries and magic states/bell states costs $O( (\log |B|) \cdot \log^{c}( |B| \log |B| ) )$ \footnote{The width of the original circuit is $|B|^{2}$ so the number of locations is $ |B|^{2} \cdot \log |B|$} depth \cite{aharonov1999faulttolerant}. Backing into $C_{g}$ from $C_{ft}$ by decoding the concatenation code takes exactly as long as the encoding, namely $O( (\log |B|) \cdot \log^{c}( |B| \log |B| ) )$.

  Then, using the bell measurements, any of the logical gates takes $O(1)$ depth. Since we only perform a single round of error correction, the remaining computation until the last decoding stage takes at most constant time of the original depth. Finally, we pay $O(\log |B|)$ for complete decoding. Summing all, we get: 
  \begin{equation*}
    \begin{split}
     &  O ( \log |B|\cdot  \log^{c}( |B| \log |B| ) )  + O ( D  ) + O ( \log |B| ) \\ 
     = & O ( D  ) + O ( \log^{c} |B| )
    \end{split}
  \end{equation*}
\end{proof}

Assuming that $W$ is polynomial in $D$, taking the block length to be $|B| = \log((W \cdot D)^c)$, as shown in \Cref{claim:prob}, results in a linear fault tolerance construction with a success probability of $1 - \frac{1}{\log^{c_2}(W \cdot D)}$. This means that the fault tolerance version of circuits in $\textbf{QNC}_1$ has a logarithmic depth. Additionally, using the construction in \cite{aharonov1996limitationsnoisyreversiblecomputation} produces a polynomial fault tolerance circuit in the reversible gates setting. \ctt{ We missed the fact that it requires non trivial classical computation to compute what gate should be applied after the gate teleportation (i.e $UPU^{\dagger}$ )}.




%\input{./tempmagic.tex}
%\cite{leverrier2022quantum}
%\cite{moore1998parallel}
%\cite{Tillich_2014}
%\cite{meier2012magicstate}
%\cite{bravyi2012magic}
\printbibliography

\end{document}

