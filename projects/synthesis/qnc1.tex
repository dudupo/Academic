

%\newcommand*{\ACM}{}%

\ifdefined\ACM

%\documentclass[sigplan,screen]{acmart}
\documentclass[manuscript,screen,review]{acmart}

\else
  \documentclass[14pt]{article}
\usepackage{libertine}
\usepackage[utf8]{inputenc}
\usepackage[a4paper, total={6.5in, 10in} ]{geometry}
\usepackage{braket}
\usepackage{xcolor}
\usepackage{amsmath}
\usepackage{amssymb}
\usepackage{amsfonts}
\usepackage{graphicx}
\usepackage{svg}
\usepackage{float}
\usepackage{tikz}
\usetikzlibrary{patterns, shapes.arrows}
\usepackage{adjustbox}
\usepackage{tikz-network}
\usepackage[ruled,lined,linesnumbered]{algorithm2e}
\usepackage{multicol}
\usepackage[backend=biber,style=alphabetic,sorting=ynt]{biblatex}
\usepackage{xcolor}
\usepackage{pgfplots}
\DeclareUnicodeCharacter{2212}{−}
\usepgfplotslibrary{groupplots,dateplot}
\pgfplotsset{compat=newest}



\usepackage{cancel}
\usepackage{subcaption}
\addbibresource{./sample.bib}

\fi

\begin{document}
\input{newcommands}
\title{ $\textbf{QNC}_{1} \subset $ noisy-\textbf{BQP}}
\author{Michael Benor \ \ David Ponarovsky}
\maketitle

\newcommand*{\Mbas}{\mathcal{X}^\prime}
\newcommand*{\bas}{\mathcal{X}}
\newcommand*{\sMbas}{\Mbas}
\newcommand*{\QQ}{C_{X}/C_{Z}^\perp }
\newcommand*{\trig}{ Triorthogonal }
\newcommand*{\Hyp}{ Hyperproduct }
\newcommand*{\Cin}{ C_{\text{initial}} }
\newcommand*{\Ctan}{ C_{\text{Tan}} }



%\section{Introduction} blablabla


\section{ Notations. }
$C_{g}$ - good qLDPC, $C_{ft}$ - concatenation code ($ft$ stands for fault tolerance). For a code $C_{y}$ we use $\Phi_{y}, E_{y}, D_{y}$ to denote the channel maps circuits into the circuits compute in the code space, the encoder, and the decoder. We use $\Phi_{U}$ to denote the 'Bell'-state storing the gate $U$. 


\section{ The Noise Model }


\section{ Fault Tolerance (With Resets gates) at Linear Depth. } 

\begin{claim}
  There is $p_{th} \in (0,1)$ such that if $p < p_{th}$ then any quantum circuit $C$ with depth $D$ and width $W$ can be computed by $p$-noisy, resets allowed, circuit $C^{\prime}$, with a depth at most $\max{ \{D, \log(WD) \} }$. 
\end{claim}


\subsection{Initializing Magic for Teleportation gates and encodes ancillaries.}

The Protocol:
\begin{enumerate}
  \item Initializing zeros. Divide the qubits into $|B|$-size blocks. Encodes each block in $C_{g}$ via $D_{ft} \Phi_{ft}[E_{g}] \ket{0^{|B|}}$.
  \item Initializing Magic for Teleportation gates encoded in $C_{g}$ via $D_{ft} \Phi_{ft}[E_{g}] \ket{\Phi_{U}}$ for each gate $U$ in the original circit .
  \item Each gate is replaced by gate teleportation.  
  \item At any time tick, any block runs a single round of error reduction.  
\end{enumerate}


\begin{claim}
  \label{claim:error} 
  Assume that an error $|e| = \gamma n $, i.e $e$ is supported on less than $\gamma n$ bits, then a single correction round reduce $e$ into an error $e^\prime$ such $|e^{\prime}| < \nu |e|$. 
\end{claim}


\begin{claim}
  The gate $ D_{ft} \Phi_{ft}[E_{g}]$ initializes states encoded in $C_{g}$ subject to $p$-noise channel.  
\end{claim}

\begin{claim}
  With probability almost surly, the total amount of noise been absorb in a block is less than $\alpha n$. 
\end{claim}
\begin{proof}
  Consider the $i$th block,  denoted by $B_{i}$. Using the Hoeffding's inequality we have that the probability that more than $\beta |B|$ bits are flipped at time $t$ is less than $ \le 2e^{-2|B|(\beta - p)} $. Using the union bounds over all the blocks at all the different time location we get that with probability $ 1 - \frac{WD}{|B|} \cdot D 2e^{-2|B|(\beta - p)} $. Now using the \Cref{claim:error} it follows that total amount of error absorbed by a block until time $t$, call it $X_{t}$ can be bounded by: $X_{t} \le \nu \cdot (X_{t-1} + \beta |B| ) \le  \nu(\gamma+\beta) |B| \le \gamma |B|$.
%  \begin{equation*}
%    \begin{split}
%      \prb{ \beta n \text{ bits were flipped in } B_{i} \text{ at time } t  } \le 2e^{-2|B|(\beta - p)} 
%    \end{split}
%  \end{equation*}
%  Taking the union bound over the 
\end{proof}






%\input{./tempmagic.tex}
%\cite{leverrier2022quantum}
%\cite{moore1998parallel}
%\cite{Tillich_2014}
%\cite{meier2012magicstate}
%\cite{bravyi2012magic}
\printbibliography

\end{document}

