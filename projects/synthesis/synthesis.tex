

%\newcommand*{\ACM}{}%

\ifdefined\ACM

%\documentclass[sigplan,screen]{acmart}
  \documentclass[manuscript,screen,review]{acmart}

\else
  \documentclass{article}
  \usepackage[utf8]{inputenc}
\usepackage[a4paper, total={6in, 9in}]{geometry}
\usepackage{braket}
\usepackage{xcolor}
\usepackage{amsmath}
\usepackage{amsfonts}
\usepackage{amsthm}
\usepackage{amssymb}
%\usepackage[ocgcolorlinks]{hyperref}
\usepackage{hyperref}
%\usepackage{hyperref,xcolor}
%\usepackage[ocgcolorlinks]{ocgx2}
\usepackage{cleveref}
\usepackage{graphicx}
\usepackage{svg}
\usepackage{float}
\usepackage{tikz}
\usetikzlibrary{patterns, shapes.arrows}
\usepackage{adjustbox}
%\usepackage{tikz-network}
\usepackage{tkz-graph}
\usepackage{tkz-berge}
\usepackage[linesnumbered]{algorithm2e}
\usepackage{multicol}
\usepackage[backend=biber,style=alphabetic,sorting=ynt]{biblatex}
%\usepackage{xcolor}
%\usepackage{tkz-berge}
%\usepackage{tkz-graph}
\usepackage{pgfplots}
\usepackage{sagetex}
\usepackage{setspace}
\usepackage{etoc}
%\usepackage{wrapfig}
\usepackage{pgfgantt}
\DeclareUnicodeCharacter{2212}{−}
\usepgfplotslibrary{groupplots,dateplot}
\pgfplotsset{compat=newest}

\newtheorem{theorem}{Theorem}
\newtheorem{definition}{Definition}
\newtheorem{example}{Example}
\newtheorem{claim}{Claim}
\newtheorem{fact}{Fact}
\newtheorem{remark}{Remark}
\newtheorem*{theorem*}{Theorem}
\newtheorem{lemma}{Lemma}
\crefname{lemma}{Lemma}{Lemmas}
\hypersetup{colorlinks=true}
% , allcolors=blue,allbordercolors=blue,pdfborderstyle={0 0 1}}
%\hypersetup{pdfborder={2 2 2}}
% pdfpagemode=FullScreen,
% backref 

\newtheorem{problem}{Problem}
\crefname{problem}{Problem}{Problems}

\DeclareMathOperator{\Ima}{Im}


  \addbibresource{./sample.bib} 

\fi

\begin{document}
\newcommand{\commentt}[1]{\textcolor{blue}{ \textbf{[COMMENT]} #1}}
\newcommand{\ctt}[1]{\commentt{#1}}
\newcommand{\prb}[1]{ \mathbf{Pr} \left[ #1 \right]}
\newcommand{\prbm}[2]{ \mathbf{Pr}_{ #2 }\left[ #1 \right]}
\newcommand{\prbc}[3]{ \mathbf{Pr}_{ #2 }\left[ #1 \right | #3]}
\newcommand{\prbcprb}[3]{ \prbc{#2}{#1}{#3} \cdot \prb{#3} } 
\newcommand{\expp}[1]{ \mathbf{E} \left[ {#1} \right]}
\newcommand{\onotation}[1]{\(\mathcal{O} \left( {#1}  \right) \)}
\newcommand{\ona}[1]{\onotation{#1}}
\newcommand{\PSI}{{\ket{\psi}}}
\newcommand{\xij} { X_{ij} } 
\DeclareMathOperator{\Ima}{Im}
%\newcommand{\LESn}{\ket{\psi_n}}
%\newcommand{\LESa}{\ket{\phi_n}}
%\newcommand{\LESs}{\frac{1}{\sqrt{n}}\sum_{i}{\ket{\left(0^{i}10^{n-i}\right)^{n}}}}
%\newcommand{\Hn}{\mathcal{H}_{n}}
%\newcommand{\Ep}{\frac{1}{\sqrt{2^n}}\sum^{2^n}_{x}{ \ket{xx}}}
%\newcommand{\HON}{\ket{\psi_{\text{honest}}}}
%\newcommand{\Lemma}{\paragraph{Lemma.}}
\newcommand{\Cpa}{[n, \rho n, \delta n]}
%\setlength{\columnsep}{0.6cm}
\newcommand{\Jvv}{ \bar{J_{v}} } 
\newcommand{\Cvv}{ \tilde{C_{v}} } 

\newcommand{\Gz}{ G_{z}^{\delta} } 
\newcommand{ \Tann } {  \mathcal{T}\left( G, C_0 \right) }
\newcommand{\ireducable}{ireducable \hyperref[ire]{[\ref{ire}]} }
\newcommand{\cutUU}{E(U_{-1} \bigcup U_{+1} ,U)} 
\newcommand{\wcutUU}{w\left( E(U_{-1} \bigcup U_{+1} ,U)  \right)}
\newcommand{\testgo}{  \mathcal{T}\left(J, q , C_{0}\right) } 

\newcommand{\duC}{\left( C_{A}^{\perp}\otimes C_{B}^{\perp} \right)^{\perp}}
\newcommand{\duduC}{\left( C_{A}\otimes C_{B}\right)^{\perp}}
  




\title{State Synthesis Using PRS.} 
\author{David Ponarovsky}
\maketitle

\begin{abstract}
  We studies the complexity of synthesis quantum states using PRS, our reasch continues the work by \cite{searchtodecision}, \cite{rosenthal2023efficient}, \cite{rosenthal2021interactive}, \cite{metger2023stateqip}, \cite{delavenne2023quantum}.
\end{abstract}

\section{Pseudorandomness.}

\begin{definition}[Pseudorandom Quantum states] Let $\mathcal{H}$, $\mathcal{K}$ be the Hilbert and the key spaces, their diminsions depeand on a security paramter $n$. A state famliy $\{\ket{\psi_{k}}\}_{k\in \mathcal{K}}$ is a pseudiorandom, if the following hold:
  \begin{enumerate}
    \item Efficient generation. There is a polynomial-time quantum algorithm $G$ that generates state $\ket{\psi_{k}}$ on input $k$. 
    \item Pseudorandomness. Any polynomially many copies of $\ket{\phi_{k}}$ with the same random $k\in K$ is computationaly indistinguishable from the same number of copies of the Haar random state. 
  \end{enumerate}
\end{definition}

\begin{definition}[Pseudorandom Unitary Operators] A famliy of unitary operators $\{ U_{k} \in U(\mathcal{H}) \}_{k \in \mathcal{K}}$ is pseudorandom, if two conditions hold: 
  \begin{enumerate}
    \item Efficient computation. There is an efficient quantum algorithm $Q$, such that for all $k$ and any $\ket{\psi} \in \mathcal{H}$ $Q(k,\ket{\psi}) = U_{k}\ket{\psi}$. 
    \item Pseudorandomness. The uniform random distribution on $U_{k}$ is computationally in distinguishable from a Haar random unitary operator. 
  \end{enumerate} 
\end{definition}

\begin{definition}[The keeping setting] Let $R^{A}\otimes R^{B}$ be a general two registers domain. We define the \textbf{keeping setting} to let one construct quntum/classical circuits\footnote{On which we think as a canidate for PRS/PRF/PRG generator.} $G : R^{A}\otimes R^{B} \rightarrow R^{A} \otimes R^{B}$ such that it is gurnted that the register $R^{B}$ cann't be acssed after the computation.   
\end{definition}

\begin{claim}
  Let $G$ be a PRS generator, than under the the keeping setting one can assume that $G$ takes as input two register, the first contains $n$ ancille qubits initiliazied to $\ket{0}$ and the seconed contain a classic string initiliezed to be the seed $k$.  
\end{claim}

\begin{proof}
  Given a PRS $G : R^{A} \rightarrow R^{A}$ define $\tilde{G} : R^{A} \otimes R^{B} \rightarrow R^{A} \otimes R^{B}$ as follow, first $\tilde{G}$ copy the calsical state in $R^{B}$ (the $k$-length seed) to $R^{A}$ and then appaly $G$ on $R^{A}$, Hence on sampled seed $k \in R^{B}$ results the output $ \ket{\psi_{k}}\otimes\ket{k}$. Under the keeping setting any polynomial distingushier-canidate $D$ has acsses only for $\ket{\psi_{k}}$, So if $D$ distinguish between the distrubition generated by $\tilde{G}$ and the Haar measure then it also distingush between $G$ and Haar measure. 
\end{proof}

\begin{claim}
  Let $G : \ket{0}^{n} \otimes \mathbb{F}_{2}^{k} \rightarrow \{ \ket{\psi_{k}} \}_{k \in \mathcal{K}}$ be a PRS generator uses $n$- ancilles and $k$ classicl bits. Then for any unitery $V: \mathcal{H}_{n} \rightarrow \mathcal{H}_{n}$ it holds that $(V \otimes I^{\otimes k} ) G$ is also a PRS.
\end{claim}
\begin{proof}
   
\end{proof}

\begin{claim}[Levis Lemma for PRS]
  Let $f : \mathcal{H} \rightarrow R$ be a \textbf{BQP}-computible fuction on the $n$-qubits hilbert space, and let $g: (0,1) \rightarrow \mathbb{R} $ a function such that:     
  \begin{equation*}
    \begin{split}
      \prbm{f\left(\ket{\psi} \right) > \varepsilon}{ \ket{\psi} \sim U} < g(\varepsilon)
    \end{split}
  \end{equation*}
  Then, a similar inequality also holds for states sampled by the PRS, when the probability for the measure $f$-value grater than $\varepsilon$ is bounded by $g(2\varepsilon)$. Namely,  
\begin{equation*}
    \begin{split}
      \prbm{f\left(\ket{\psi} \right) > \varepsilon}{ \ket{\psi} \sim \ket{\psi_{k}} } < g(2\varepsilon)
    \end{split}
  \end{equation*}
  In praticular, Levi's lemma has a version that capture consetration of states sampled by PRS generator, states the following: Assume there exsists $K$ such that for any $\ket{\psi},\ket{\phi} \in \mathcal{S}(\mathbb{C}^{d})$ $|f(\ket{\psi}) - f(\ket{\phi})| < K | \ket{\psi} - \ket{\phi}|$. Then there exsists a universal constant $C > 0$ such: 
\begin{equation*}
    \begin{split}
      \prbm{ |f\left(\ket{\psi} \right) - \exppm{f\left(\ket{\phi}\right)}{\ket{\phi} \sim U}| > \varepsilon}{ \ket{\psi} \sim \ket{\psi_{k}} } < exp\left(-\frac{Cd}{K^{2}}  4\varepsilon^{2} \right)
    \end{split}
  \end{equation*}
\end{claim}

\begin{proof}
  
\end{proof}

\begin{claim}
  Probablisitc counting argument and $\varepsilon$-net over PRS. 
\end{claim}

\begin{claim}
  exsistness of poly($n$) gates $G_{1}, G_{2} ..$ such that, any $G_{i}$ has a polynomial depth, $\braket{p(G_{i})|\tau} > a$ and $\braket{\tau^{\perp}|p(G_{j})}\braket{p(G_{i})|\tau^{\perp}} < b$ for any $i\neq j$. 
\end{claim}
\begin{proof}
  
\end{proof}
\begin{claim}
 bla bla bla 
\end{claim}

\begin{definition}
  $\varepsilon$-bised test $2$-degree for testing RPU/RPS. $f(\braket{x_{j}|G_{s}|\theta}) = 1$ For example ask if $\braket{\psi_{j^{\prime}}\tau^{\perp}}\braket{\tau^{\perp}|\psi_{j}}$ what I can say about that quantenty as polynomail?. 
\end{definition}

\section{What We Need for Synthesis.}

\begin{definition}[Pseudorandom Unitary for Synthesis] A famliy of unitary operators $\{ U_{k} \in U(\mathcal{H}) \}_{k \in \mathcal{K}}$ is pseudorandom for synthesis, if two conditions hold: 
  \begin{enumerate}
    \item Efficient computation. There is an efficient quantum algorithm $Q$, such that for all $k$ and any $\ket{\psi} \in \mathcal{H}$ $Q(k,\ket{\psi}) = U_{k}\ket{\psi}$. 
    \item Pseudorandomness for synthesis. Given a state $\ket{\tau}$ and polynomial number of samples $U_{1},U_{2}..U_{m}$. Then: 
      \begin{enumerate}
        \item $|\braket{\Phi(\tau,U_{k})|U_{k}\tau}|^{2} > a$
        \item $|\braket{\Phi(\tau,U_{k})|U_{k}\tau^{\perp}}\braket{\tau^{\perp}U^{\dagger}_{j}|\Phi(\tau,U_{j})}|^{2} < b$
      \end{enumerate}
      The uniform random distribution on $U_{k}$ is computationally in distinguishable from a Haar random unitary operator. 
  \end{enumerate} 
\end{definition}

What about, Assume that $U$ is a quantum circuit such that $\log n$ qubits are intilaized to some to some input and instead anciles, we have noisy ancilea, can we show that circuit is equavilanent to $\log n $ circuit? That will enable us to prove a quantum version for Nisan Wigerzdon PRG (BPP = P). 

\paragraph{Problem.} Let $U$ be a quntum circuit which get $\log n$ stable qubits and poly$(n)$ more random qubits obtained from the random Haar masure, can we simulate the circuit in $\log n$ time?  

approximate the absoulte value function, For example, you can consider the binomial expansion of $\sqrt{1-y}$ on $[0,1]$. Namely, setting $y=1-x^2$, we have $|x|=\sqrt{1-y}=\sum_{m=0}^{\infty}\binom{1/2}{m}(-y)^m, \quad x\in[-1,1]$. That will allow me to bound the $k$-design. 

Denote by $q_{d}(x)$ the $d$-order approximation of $|x|$, Namely 
\begin{equation*}
  \begin{split}
q_{d}(x) &= \sum_{m=0}^{d}\binom{1/2}{m}(-1)^m\left( 1 - x^{2} \right)^{m}
%&= \sum_{m=0}^{d}\sum_{n=0}^{m}\binom{1/2}{n}(-1)^n\left( 1 - x^{2} \right)^{m}\\
  \end{split}
\end{equation*}
and as the series is convergres to any $x \in (-1,1)$ we have that $|x| = q_{d}(x) + O( \binom{1/2}{d}(1-x^{2})^d )$ which by the fact that $1- x^{2} \in (-1,1)$ can be simpllified to $|x| = q_{d}(x) + O( \binom{1/2}{d}) = q_{d}(x) + O( 1/d^{1 +1/2})$.
 
% \exppm{   }{ U \sim D }
\begin{equation*}
  \begin{split}
  \exppm{ ( \braket{\Phi(\tau,U)|\Rea U\tau} )^{2}  }{ U \sim D } & =  \exppm{  \frac{1}{2^{n/2}}\sum_{x}(-1)^{\sign(\Rea \braket{x|U\tau})}\Rea \braket{x|x}\braket{x|U\tau}  }{ U \sim D } \\
& = \exppm{ \frac{1}{2^{n/2}}\sum_{x}|\Rea \braket{x|U\tau}|  }{ U \sim D } \\
& = \exppm{ \sum_{x}|\Rea \braket{x|U\tau} / 2^{n/2}|  }{ U \sim D } \\
& \ge \exppm{  \sum_{x} q_{d}\left( \Ima \braket{x|U\tau}| / 2^{n/2}  \right) - \binom{1/2}{d} \left(\frac{ | \Ima \braket{x|U\tau}| }{2^{n/2}}\right)^{d}}{ U \sim D } \\
& \ge \exppm{  \sum_{x} q_{d}\left( \Ima \braket{x|U\tau}| / 2^{n/2}  \right) - \binom{1/2}{d} \left(\frac{ | \Ima \braket{x|U\tau}| }{2^{n/2}}\right)^{d}}{ U \sim Haar } - \delta \cdot 2 ^{n} \\
& \ge \exppm{  \sum_{x} |\Rea \braket{x|U\tau} / 2^{n/2}|  - \mathbf{2} \cdot  \binom{1/2}{d} \left(\frac{ | \Ima \braket{x|U\tau}| }{2^{n/2}}\right)^{d}}{ U \sim Haar } - \delta \cdot 2 ^{n} \\
& \sim \exppm{  \sum_{x} |\Rea \braket{x|U\tau} / 2^{n/2}|}{ U \sim Haar } - \delta \cdot 2 ^{n} 
  \end{split}
\end{equation*}
\begin{equation*}
  \begin{split}
    \exppm{  \braket{\Phi(\tau,U)|U \tau^{\perp}}\braket{\tau^{\perp}U_{2}^{\dagger}|\Phi(\tau,U_{2})}  }{ U,U_{2} \sim D } & = 
  \end{split}
\end{equation*}


\begin{claim}
  fix a state $\ket{\tau}$. Let $U$ be a unitary sampled from $k$-design  distribution $D$ and denote by $\ket{s}$ the vector which $U$ sends $\ket{\tau}$ to. Now, observes that $U$ can be written as $U = \ket{s}\bra{\tau} + V$ when $V$ act on space ortogonal to $ \ket{\tau}$ denote it by $\ket{\tau^{\perp}}$. Then the distribuation over $V$ is also a $k$-design relative to the Haar mesure on $\ket{\tau^{\perp}}$. 
\end{claim}
\begin{proof}  
\end{proof}

\begin{definition}
  Denote by 
  \begin{equation*}
    \begin{split}
  M(\tau,U)(x) & = \max \left\{ |\Rea \braket{x|U\tau}|, |\Ima \braket{x|U\tau}| \right\} \\
  \bar{M}(\tau,U)(x) & = \min \left\{ |\Rea \braket{x|U\tau}|, |\Ima \braket{x|U\tau}| \right\} 
    \end{split}
  \end{equation*}
  When it will be clear form the context we omit $\tau,U$ and use only $M(x), \bar{M}(x)$.
\end{definition}

\begin{equation*}
  \begin{split}
    |\braket{\Phi(\tau,U)|U\phi}|^2 = | \braket{\Phi(\tau, U)|\Rea U\phi}|^2  + |\braket{\Phi(\tau,U)|\Ima U\phi}|^2 
  \end{split}
\end{equation*}
\begin{equation*}
  \begin{split}
    \braket{\Phi(\tau,U_{k})|MU_{k}\phi} & = \sum_{x}{ \left( -1 \right)^{\sign M\left( \braket{x|U\tau} \right) } \frac{1}{2^{n/2}} \braket{x|U\phi} }\\
    & = \sum_{ \tau,\phi \text{ agree on } x}{ | \frac{1}{2^{n/2}}M \left( \braket{x|U\phi} \right)| } - \sum_{ \tau,\phi \text{ disagree on } x}{  | \frac{1}{2^{n/2}}M \left( \braket{x|U\phi} \right)| }  \\
    & \approx \sum_{ \tau,\phi \text{ agree on } x}{ q_{d} \left(\frac{1}{2^{n/2}}\bar{M} \left( \braket{x|U\phi} \right) \right) } - \sum_{ \tau,\phi \text{ disagree on } x}{  q_{d}\left( \frac{1}{2^{n/2}}\bar{M} \left( \braket{x|U\phi} \right) \right) }  \pm 2^{n}\zeta_{d}\left(\frac{1}{2^{n/2}} \right) 
  \end{split}
\end{equation*}
noitce that we obtained a $d$-degree polinomial, denote it by $T_{\phi}$.
\begin{equation*}
  \begin{split}
    | \braket{\Phi(\tau,U)|MU\phi} |  \approx & q_{d^{\prime}}\left( \braket{\Phi(\tau,U)|U\phi} \right) + \zeta_{d^{\prime}}\left( \braket{\Phi(\tau,U)|U\phi} \right) \\
\approx & q_{d^{\prime}}\left( \braket{\Phi(\tau,U)|U\phi} \right) + \zeta_{d^{\prime}}\left( \braket{\Phi(\tau,U)|U\phi} \right)\\
\approx & q_{d^{\prime}}\left(  T_{\phi}\right) + \zeta_{d^{\prime}}\left( T_{\phi} \right)\\
\approx & q_{d^{\prime}}\left(  T_{\phi}\right) + \zeta_{d^{\prime}}\left( T_{\phi} \right)
  \end{split}
\end{equation*}

Assume that our $k$-design collection is defined such that for any $\ket{\varphi}$ it holds that: 
\begin{equation*}
  \begin{split}
    \prbm{  \sign(\Rea \braket{x|U_{1}\varphi} ) = \sign(\Rea \braket{x^{\prime}|U_{2}\varphi} )}{ U_{1},U_{2}\sim D } = \frac{1}{2}
  \end{split}
\end{equation*}

\begin{claim}
  left $f : N \rightarrow \{ \pm \}$ then the set $ (-1)^{f(x)}\ket{x}\bra{x}U$ is a $k$-design. 
\end{claim}
\begin{proof}
  \begin{equation*}
    \begin{split}
      tr\left( U^{\prime}V^{\prime,\dagger} \right)  = & tr\left( (-1)^{f(x)}\ket{x}\bra{x}UV^{\dagger} (-1)^{f(x)}\ket{x}\bra{x} \right) \\ 
  = & tr\left( (-1)^{f(y)}\ket{y}\bra{y}(-1)^{f(x)}\ket{x}\bra{x}UV^{\dagger} \right) = tr(UV^{\dagger})  
    \end{split}
  \end{equation*}
  So, we get that:
  \begin{equation*}
    \begin{split}
      \frac{1}{|X|^{\prime, 2}}\sum_{U,V\in X^{\prime}}{ |tr(UV^{\dagger} )|^{2t} } & =  \frac{1}{|X|^{2}}\sum_{U,V\in X}{ |tr(UV^{\dagger} )|^{2t} }\\
      & = \int{ |tr(U)|^{2t} dU} 
    \end{split}
  \end{equation*}
\end{proof}

Ok the tactics is going to be the follow, we need the $k$-design property only for the first stage.  When we want to show that $\ket{\Phi}$ has an overlap with $\ket{\tau}$ after that, we can give up on that assumption and by using $f,g$ universal we can ensure a small overllap betweem pair of differnet $U,V$. 

\begin{claim}
  Assume $f$ above sampled from a universal femily hash functions. Then we have that :  
  \begin{equation*}
    \begin{split}
    \exppm{ |\braket{\Phi(\tau,V) V^{\dagger}|\psi}\braket{\psi|U\Phi(\tau,U)}|^{2} }{U,V \sim X , f\sim \mathcal{H}} \approx_{\delta} \exppm{ |\braket{\Phi(\tau,V) V^{\dagger}|\psi}\braket{\psi|U\Phi(\tau,U)}|^{2} }{U,V \sim Haar }
    \end{split}
  \end{equation*}
\end{claim}

\begin{proof}
 
  \begin{equation*}
    \begin{split}
       \braket{\Phi(\tau,V) V^{\dagger}|\psi} = & \frac{1}{2^{n/2}}\sum_{x} (-1)^{\sign( \Rea \braket{x|V|\tau})}\braket{x |V \psi} \\
       = & \frac{1}{2^{n/2}}\sum_{x} (-1)^{ f(x) + \sign( \Rea \braket{x|V|\tau})}\braket{x |V^{\prime} \psi} \\  
       = & \frac{1}{2^{n/2}}\sum_{x} (-1)^{ f(x) +  f(x) \cdot \sign( \Rea \braket{x|V^{\prime}|\tau})}\braket{x |V^{\prime} \psi} \\  
       = & \frac{1}{2^{n/2}}\sum_{x} (-1)^{ f(x) ( 1 +  \sign( \Rea \braket{x|V^{\prime}|\tau}))}\braket{x |V^{\prime} \psi} \\  
       = & \frac{1}{2^{n/2}}\sum_{x} (-1)^{ f(x) ( 1 +  \sign( \Rea \braket{x|V^{\prime}|\tau}))}\braket{x |V^{\prime} \psi} \\  
     \end{split}
   \end{equation*}
   \begin{equation*}
     \begin{split}
       \Rightarrow  \braket{\Phi(\tau,V) V^{\dagger}|\psi} & \braket{\psi|U\Phi(\tau,U) }  \\
       = & \frac{1}{2^{n}}\sum_{x,x^{\prime}} (-1)^{ g(x^{\prime})+f(x) + \sign( \Rea \braket{x|V|\tau})+\sign( \Rea \braket{x^{\prime}|U|\tau})} \cdot \\
       & \ \ \ \ \ \ \cdot  \braket{x |V^{\prime} \psi}\braket{x^{\prime} |U^{\prime} \psi} 
    \end{split}
  \end{equation*}
\begin{equation*}
     \begin{split}
       \Rightarrow  |\braket{\Phi(\tau,V) V^{\dagger}|\psi} & \braket{\psi|U\Phi(\tau,U) }|^2  \\
       = & \frac{1}{2^{2n}}\sum_{y,y^{\prime},x,x^{\prime}} (-1)^{ g(x^{\prime})+f(x) + \sign( \Rea \braket{x|V|\tau})+\sign( \Rea \braket{x^{\prime}|U|\tau})} \cdot \\
       & \ \ \ \ \ \ \cdot  \braket{y |V^{\prime} \psi}\braket{y^{\prime} |U^{\prime} \psi} \cdot  \\
  & \cdot (-1)^{ g(y^{\prime})+f(y) + \sign( \Rea \braket{y|V|\tau})+\sign( \Rea \braket{y^{\prime}|U|\tau})} \cdot \\
  & \ \ \ \ \ \ \cdot  \braket{y |V^{\prime} \psi}^{*}\braket{y^{\prime} |U^{\prime} \psi}^{*}
    \end{split}
  \end{equation*}
  \begin{equation*}
    \begin{split}
& \exppm{ |\braket{\Phi(\tau,V) V^{\dagger}|\psi}\braket{\psi|U\Phi(\tau,U)}|^{2} }{U,V \sim X , f,g\sim \mathcal{H}^2} \\  
      & \exppm{ |\braket{\varphi V^{\prime, \dagger}|x}\braket{x|U^{\prime}\varphi}|^{2} }{U,V \sim X , f,g\sim \mathcal{H}^2}  \\ 
= & \exppm{ \braket{y|U^{\prime}|\phi}^{*} \braket{y^{\prime}|V^{\dagger, \prime}|\phi}^{*} \braket{x|U^{\prime}|\phi} \braket{x^{\prime}|V^{\dagger, \prime}|\phi}}{ U^{prime},V^{\dagger, \prime} \sim X , f,g\sim \mathcal{H}^2} \\
= & \exppm{ (-1)^{f(x)+g(x^{\prime})+f(y)+g(y^{\prime})} \braket{y|U|\phi}^{*} \braket{y^{\prime}|V|\phi}^{*} \braket{x|U|\phi} \braket{x^{\prime}|V|\phi}}{ U,V \sim X , f,g\sim \mathcal{H}^2} \\
= & \exppm{ \mathbf{1}_{x=x^{\prime=y=y^{\prime}}} \braket{y|U|\phi}^{*} \braket{y^{\prime}|V|\phi}^{*} \braket{x|U|\phi} \braket{x^{\prime}|V|\phi}}{ U,V \sim X , f,g\sim \mathcal{H}^2} \\
\le &\frac{2^{n}}{2^{2n}} = \frac{1}{2^n}
%= & \exppm{|\braket{x|U|\phi}|^{2}  |\braket{x|V^{\dagger}|\phi}|^{2} }{ U,V \sim X } = \exppm{|\braket{x|U|\phi}|^{2}}{ U \sim X }^{2}
    \end{split}
  \end{equation*}


\end{proof}

\begin{claim}
  $|\braket{\Phi(\tau,U_{k})|U_{k}\tau^{\perp}}\braket{\tau^{\perp}U^{\dagger}_{j}|\Phi(\tau,U_{j})}|^{2} < b$
\end{claim}
\begin{proof}
  
  \begin{equation*}
    \begin{split}
      & \exppm{|\braket{\Phi(\tau,U_{k})|U_{k}\tau^{\perp}}\braket{\tau^{\perp}U^{\dagger}_{j}|\Phi(\tau,U_{j})}|^{2} }{U \sim D  } \\
      \le &  \exppm{|\braket{\Phi(\tau,U_{k})|U_{k}\tau^{\perp}}|^{2}\cdot |\braket{\tau^{\perp}U^{\dagger}_{j}|\Phi(\tau,U_{j})}|^{2} }{U \sim D  } \\ 
      = &  \exppm{|\braket{\Phi(\tau,U_{k})|U_{k}\tau^{\perp}}|^{2} }{U \sim D  }^{2} \\ 
      %= & \exppm{\sum_{x,y}{| (-1)^{\sign(\Rea\braket{x|U_{k}\tau}) + }\braket{x|U_{k}\tau^{\perp}}|^{2}} }{U \sim D  }^{2} 
      = & \exppm{|\sum_{x}{\braket{xU_{k}\tau^{\perp}}} |^{2}}{U \sim D  }^{2} \\
      = & \exppm{\sum_{x}{|\braket{x|U_{k}\tau^{\perp}}|^{2}} }{U \sim D  }^{2} 
    \end{split}
  \end{equation*}
\end{proof}





\section{The Distillation.}
suppose that for any $\ket{\psi^{(k)}_{i}}$, $\ket{\psi^{(k)}_{j}}$ it holds that: $|\braket{\psi^{(k)}_{i}|\psi^{(k)}_{j}}| < exp(-n)$. After conditianl swap the normlaized state has the form:  
\begin{equation*}
  \begin{split}
    & \ket{\psi^{(k)}_{i}}\ket{\psi^{(k)}_{j}}+\ket{\psi^{(k)}_{j}}\ket{\psi^{(k)}_{i}} \mapsto \frac{ \ket{\psi^{(k)}_{i}}\ket{\psi^{(k)}_{j}}+\ket{\psi^{(k)}_{j}}\ket{\psi^{(k)}_{i}} }{ \sqrt{2 + 2 |\braket{\psi^{(k)}_{i}|\psi^{(k)}_{j}}}|^2 } 
  \end{split}
\end{equation*}
%\frac{ \ket{\psi^{(k)}_{t}}\ket{\psi^{(k)}_{s}}+\ket{\psi^{(k)}_{s}}\ket{\psi^{(k)}_{t}} }{ \sqrt{2 + 2 |\braket{\psi^{(k)}_{t}|\psi^{(k)}_{s}}}|^2 } 

%\frac{ \bra{\psi^{(k)}_{i}}\bra{\psi^{(k)}_{j}}+\bra{\psi^{(k)}_{j}}\bra{\psi^{(k)}_{i}} }{ \sqrt{2 + 2 |\braket{\psi^{(k)}_{i}|\psi^{(k)}_{j}}}|^2 } 
So 
\begin{equation*}
  \begin{split}
  |\braket{\psi^{(k+1)}_{i}|\psi^{(k+1)}_{t}}| =& | \frac{ \bra{\psi^{(k)}_{i}}\bra{\psi^{(k)}_{j}}+\bra{\psi^{(k)}_{j}}\bra{\psi^{(k)}_{i}} }{ \sqrt{2 + 2 |\braket{\psi^{(k)}_{i}|\psi^{(k)}_{j}}}|^2 } \cdot \frac{ \ket{\psi^{(k)}_{t}}\ket{\psi^{(k)}_{s}}+\ket{\psi^{(k)}_{s}}\ket{\psi^{(k)}_{t}} }{ \sqrt{2 + 2 |\braket{\psi^{(k)}_{t}|\psi^{(k)}_{s}}}|^2 } | \\
  \le & \frac{1}{2+2exp(-2n)}\cdot \sum_{\{v,u\} = \{i,j\},\{v^{\prime},u^{\prime}\} = \{t,s\}}{ |\braket{\psi^{(k)}_{v}|\psi^{(k)}_{v^{\prime}}} \braket{\psi^{(k)}_{u}|\psi^{(k)}_{u^{\prime}}}|} \\ 
  \le &\frac{1}{2+2exp(-2n)} \cdot 4 exp(-2n)
  \end{split}
\end{equation*}

$ \frac{ 1 + |\braket{\psi^{(k)}_{j}|\psi^{(k)}_{i}}|^2 }{ \sqrt{2 + 2 |\braket{\psi^{(k)}_{i}|\psi^{(k)}_{j}}}|^2 }\ket{\psi^{(k)}_{i}} $

$ \frac{ 1 + |\braket{\psi^{(k)}_{s}|\psi^{(k)}_{t}}|^2 }{ \sqrt{2 + 2 |\braket{\psi^{(k)}_{t}|\psi^{(k)}_{s}}}|^2 }\bra{\psi^{(k)}_{t}} $



\printbibliography

\end{document}







