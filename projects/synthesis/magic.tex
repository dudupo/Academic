

%\newcommand*{\ACM}{}%

\ifdefined\ACM

%\documentclass[sigplan,screen]{acmart}
\documentclass[manuscript,screen,review]{acmart}

\else
\documentclass{article}
\usepackage{libertine}
\usepackage[utf8]{inputenc}
\usepackage[a4paper, total={6.5in, 10in} ]{geometry}
\usepackage{braket}
\usepackage{xcolor}
\usepackage{amsmath}
\usepackage{amssymb}
\usepackage{amsfonts}
\usepackage{graphicx}
\usepackage{svg}
\usepackage{float}
\usepackage{tikz}
\usetikzlibrary{patterns, shapes.arrows}
\usepackage{adjustbox}
\usepackage{tikz-network}
\usepackage[ruled,lined,linesnumbered]{algorithm2e}
\usepackage{multicol}
\usepackage[backend=biber,style=alphabetic,sorting=ynt]{biblatex}
\usepackage{xcolor}
\usepackage{pgfplots}
\DeclareUnicodeCharacter{2212}{−}
\usepgfplotslibrary{groupplots,dateplot}
\pgfplotsset{compat=newest}



\usepackage{cancel}
\usepackage{subcaption}
\addbibresource{./sample.bib}

\fi

\begin{document}
\input{newcommands}
\title{ $\sqrt{n} \mapsto \Theta(n)$  Magic States 'Distillation' Using
Quantum LDPC Codes. }
\author{David Ponarovsky}
\maketitle

\newcommand*{\Mbas}{\mathcal{X}^\prime}

\section{The Construction.}

Let $\mathcal{X} = \{x_{0}, x_{1}, .. x_{k-1}\} \in \mathbb{F}_{2}^{n}$ be a
base for the code $C_{X}/C_{Z}^\perp$.  Denote by $w \in \mathbb{F}_{2}^{n}$
the binary string presents $Z$-generator that anti commute with the
$X$-generator corresponds to $x_{0}$, So $x_{0}\cdot w = 1$ and for any other
$x^\prime \in \mathcal{X}/x_{0}$ it holds that $x^{\prime}\cdot w = 0 $. Let
us denote by $\Mbas$ the base $\{ y_{1}, y_{2}, .., y_{k-1} \} \in
\mathbb{F}_{2}^{n}$ such $y_{i}= x_{i} + x_{0}$. Denote by $E$ the circuit
that encodes the logical $ith$ bit to $y_{i}$, by $T^{(w)}$ the application of
$T$ gates on the qubits for which $w$ act non trivial, means $T^{(w)}$ is a
tensor product of $T$'s and identity where on the $i$th qubit $T^{(w)}$ apply
$T$ if $w_{i}$ is $1$ and identity otherwise. And finally by $D$ denote the gate that decode binary strings in $\mathbb{F}_{2}^{n}$ back into the logical space, 


\section{Proof of Theorem 1.}

\begin{claim}
  Let $\ket{\Mbas} \propto \sum_{x \in \text{span } \Mbas }{\ket{x}}$.
\end{claim}





%\input{./tempmagic.tex}

\printbibliography

\end{document}

