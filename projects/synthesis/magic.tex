

%\newcommand*{\ACM}{}%

\ifdefined\ACM

%\documentclass[sigplan,screen]{acmart}
\documentclass[manuscript,screen,review]{acmart}

\else
  \documentclass[14pt]{article}
\usepackage{libertine}
\usepackage[utf8]{inputenc}
\usepackage[a4paper, total={6.5in, 10in} ]{geometry}
\usepackage{braket}
\usepackage{xcolor}
\usepackage{amsmath}
\usepackage{amssymb}
\usepackage{amsfonts}
\usepackage{graphicx}
\usepackage{svg}
\usepackage{float}
\usepackage{tikz}
\usetikzlibrary{patterns, shapes.arrows}
\usepackage{adjustbox}
\usepackage{tikz-network}
\usepackage[ruled,lined,linesnumbered]{algorithm2e}
\usepackage{multicol}
\usepackage[backend=biber,style=alphabetic,sorting=ynt]{biblatex}
\usepackage{xcolor}
\usepackage{pgfplots}
\DeclareUnicodeCharacter{2212}{−}
\usepgfplotslibrary{groupplots,dateplot}
\pgfplotsset{compat=newest}



\usepackage{cancel}
\usepackage{subcaption}
\addbibresource{./sample.bib}

\fi

\begin{document}
\input{newcommands}
\title{ $\sqrt{n} \mapsto \Theta(n)$  Magic States 'Distillation' Using
Quantum LDPC Codes. }
\author{David Ponarovsky}
\maketitle

\newcommand*{\Mbas}{\mathcal{X}^\prime}
\newcommand*{\sMbas}{\text{span }\Mbas}
\newcommand*{\QQ}{C_{X}/C_{Z}^\perp }
\newcommand*{\trig}{Triorthogonal }
\newcommand*{\Hyp}{Hyperproduct }
\newcommand*{\Cin}{ C_{\text{initial}} }
\newcommand*{\Ctan}{ C_{\text{Tanner}} }
\section{The Construction.}

Let $x_{0}$ be a codeword of $\QQ$,  Denote by $w \in \mathbb{F}_{2}^{n}$
the binary string presents the $Z$-generator that anti commute with the
$X$-generator corresponds to $x_{0}$. Let $\mathcal{X} = \{x_{0}, x_{1}, .. x_{k^\prime}\} \in \mathbb{F}_{2}^{n}$ be a
subset of a base for the code $\QQ$. Such $\left(\text{ span } \mathcal{X}/x_0 \ \right)|_{w}$ is \trig code.  
Let us denote by $\Mbas$ the base $\{ y_{1}, y_{2}, .., y_{k^\prime} \} \in
\mathbb{F}_{2}^{n}$ defined such: $ y_{i} = x_{j} + x_{0}$. 

Denote by $E$ the circuit that encodes the logical $ith$ bit to $y_{i}$, by $T^{(w)}$ the application of
$T$ gates on the qubits for which $w$ act non trivial, means $T^{(w)}$ is a
tensor product of $T$'s and identity where on the $i$th qubit $T^{(w)}$ apply
$T$ if $w_{i}$ is $1$ and identity otherwise. And finally by $D$ denote the gate that decode binary strings in $\mathbb{F}_{2}^{n}$ back into the logical space.


\section{Proof of Theorem 1.}

\begin{claim}
  There exists family of non-trivial distance quantum LDPC codes $Q$ such the codes $\sMbas$ chosen respect to them has a positive rate. Furthermore, the rate of $\sMbas$ is a asymptotically converges to $Q$ rate:
  \begin{equation*}
    \begin{split}
      \left| \rho\left(Q\right) - \rho\left(\sMbas\right) \right| = o(1)
    \end{split}
  \end{equation*}
\end{claim}
\begin{proof}
  Let $\Delta$ be a constant integer, $C_{0},\tilde{C}_{0}$ codes over $\Delta$ bits such $\tilde{C}_{0}$ is \trig and $C_{0}$ contains $\tilde{C}_{0}$, $C_{0}$ has parameters $\Delta[1,\delta_{0},\rho_{0}]$, and $C_{0}^\top$ has relative distance greater than $\delta_{0}$. Let $\Ctan$ be a Tanner code, defined by taking an expander graph with good expansion and $C_{0}$ as the small code. Let $\Cin$ be the dual-tensor code obtained by taking $(\Ctan^\perp \otimes \Ctan^\perp )^\perp$. Notes that first this code has positive rate and $\Theta(\sqrt{n})$ distance, second this code is an LDPC code as well. Notice also that $\Cin^{\top}$ obtained by transporting the parity check matrix, and therefore equals to  $(\Ctan^{\top, \perp} \otimes \Ctan^{\top, \perp} )^\perp$. Hence $\Cin^{\top}$ has a square root distance as well.

  Let $Q$ the CSS code, obtained by taking the \Hyp of $\Cin$ with itself. So $Q$ is an quantum qLDPC code with parameters $[n, \Theta(n^{\frac{1}{4}}), \Theta(n)]$. Pick $x_{0}$ and $w \in \mathbb{F}_{2}^{n}$ , which correspond to the supports of anti commute $X$ and $Z$ generators, such that $w$ can be obtains by setting a codeword of $\Ctan$ on the first $n^{\frac{1}{4}}$ bits and padding by zeros the rest. Clearly, $|w| = \Theta(n^{\frac{1}{4}})$.

  Now for defying $\text{span }\mathcal{X}$, we are going to consider the parity checks matrix obtained by adding restrictions to $C_{X}$ restrictions as follows: Divide the first $w$ bits into $\Delta$-size buckets, define by $w(i)$ the $i$th coordinate on which $w$ isn't trivial, for example if $w(1)=j$ then $j$ is the first nonzero coordinate of $w$, Denote by $B_{1},B_2,.., B_{|w|/\Delta}$ the partion of $w$'s bits: 
  \begin{equation*}
    \begin{split}
      B_{1} &= \left\{w(1), w(2), ..,w(\Delta)\right\}\\
      B_{2} &= \left\{w(\Delta + 1), w(\Delta + 2), ..,w(2\Delta) \right\}\\
      B_{i} &= \left\{w((i-1)\Delta + 1), w((i-1)\Delta + 2), ..,w(i\Delta) \right\}
    \end{split}
  \end{equation*}

  Then let $\text{span }\mathcal{X}$ be all the codewords of $\QQ$ satisfying $\tilde{C}_{0}$ restrictions for each bucket, Let us name the union of $\tilde{C}_{0}$ restrictions over the buckets by $B$. The dimension of the space satisfies both $C_{X}$ restrictions and $B$ is at least:
  \begin{equation*}
    \begin{split}
      \rho(C_{X})\cdot n - |B|\cdot (1 - \rho(\tilde{C}_{0}))\Delta \ge \rho(C_{X})\cdot n - n^{\frac{1}{4}}
    \end{split}
  \end{equation*}
  And by the fact that the dimension of $C_{Z}^\perp$'s codewords satisfying $B$ is strictly lower then $\dim C_{Z}^\perp$, we get the following lower bound:
  \begin{equation*}
    \begin{split}
      \dim \text{span } \mathcal{X} & \ge \rho(C_{X})\cdot n - n^{\frac{1}{4}} + \rho(C_{Z})\cdot n - n \\
     & \ge \rho (Q) - n^{\frac{1}{4}}
    \end{split}
  \end{equation*}
\end{proof}

\begin{remark}
  We emphasise that the above proof can be easily adapted to result the following for general CSS codes: 
\begin{equation*}
    \begin{split}
      \left| \rho\left(Q\right) - \rho\left(\sMbas\right) \right| = d(Q)(1 - \rho(\tilde{C}_{0}))
    \end{split}
  \end{equation*}
  For example lets consider the quantum Tanners code. Since the distance of the quantum Tanner codes is $\sim n/\Delta$, where $\Delta^{2}$ is the degree of the square complex graph, (obtained by taking a codeword for which each local view of it is supported only on rows correspond to a specific single left generator), we get that for any $\rho \in (0,\frac{1}{2})$ one there is a good qLDPC such that the dimension of $\sMbas$ obtained respecting to it $\ge (1-2\rho)^{2}n - n/\Delta \cdot (1 - \rho(\tilde{C}_{0}))$.  
\end{remark}

From now on 

\begin{claim}
  Let $\ket{\Mbas} \propto \sum_{x \in \text{span } \Mbas }{\ket{x}}$. Then $T^{(w)}\ket{\Mbas} \propto \sum_{x \in \text{span }i}{  x  }$
\end{claim}


% Notice , So $x_{0}\cdot w = 1$ and for any other
%$x^\prime \in \mathcal{X}/x_{0}$ it holds that $x^{\prime}\cdot w = 0 $.



%\input{./tempmagic.tex}

\printbibliography

\end{document}

