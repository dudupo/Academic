

%\newcommand*{\ACM}{}%

\ifdefined\ACM

%\documentclass[sigplan,screen]{acmart}
  \documentclass[manuscript,screen,review]{acmart}

\else
  \documentclass{article}
  \usepackage[utf8]{inputenc}
\usepackage[a4paper, total={6.5in, 10in} ]{geometry}
\usepackage{braket}
\usepackage{xcolor}
\usepackage{amsmath}
\usepackage{amssymb}
\usepackage{amsfonts}
\usepackage{graphicx}
\usepackage{svg}
\usepackage{float}
\usepackage{tikz}
\usetikzlibrary{patterns, shapes.arrows}
\usepackage{adjustbox}
\usepackage{tikz-network}
\usepackage[ruled,lined,linesnumbered]{algorithm2e}
\usepackage{multicol}
\usepackage[backend=biber,style=alphabetic,sorting=ynt]{biblatex}
\usepackage{xcolor}
\usepackage{pgfplots}
\DeclareUnicodeCharacter{2212}{−}
\usepgfplotslibrary{groupplots,dateplot}
\pgfplotsset{compat=newest}



  \addbibresource{./sample.bib} 

\fi

\begin{document}

\input{newcommands}

\title{Why The Following Doesn't Give Log-Local, Constant Gap Hamiltonian?} 
\author{David Ponarovsky}
%\author{Noa Viner, David Ponarovsky}

\ifdefined\ACM
  \affiliation{%
    \institution{The Th{\o}rv{\"a}ld Group}
    \streetaddress{1 Th{\o}rv{\"a}ld Circle}
    \city{Hekla}
  \country{Iceland}}
  \email{larst@affiliation.org}
\else
  \maketitle
\fi
%\begin{abstract}
  Quantum feasibility hinges on the assumption that the basic gate's noise rate is below a certain threshold. Here we study the behavior of computation models when the noise is slightly greater than that threshold. In particular, We ask if one can design a fault tolerance schema such that if the noise is above the threshold, it is still grunted that the final generated state would have a value. 
\end{abstract}

\ifdefined\ACM
  \maketitle
\fi


%\section{Introduction.}
%\begin{itemize}
%  \item Solovey-Kitaev shown to be true also for BQL. 
%  \item Even though postpting the measurements to the end seems to inflat the wide and the length of the circuit. It was provn that there is an effiucent way to elimanating intermidate measurements, So  BQL = BQuL. 
%  \item There is some error-rduction that I have to complete.  
%  \item Ta-shma shown that inverting a matrix is BQL complete. 
%  \item Also computing the eigen values of Hrmitan matrix is BQP complete. The algoritmes that have been used are has qauderic adavantage in space.
%\end{itemize}
%
%

\section{What we would like to have:}

\newcommand{\Hin}{H_{\text{init}}}
  \newcommand{\Hpr}{H_{\text{prop}}}
\newcommand{\Hen}{H_{\text{end}}}
%
%Consider the LPS expnader on $n$ veritces and denote $ t \sim l $ if $t$ is adjoint to $l$. Let $M_\Delta \in \mathbb{C}^{n \times n}$ be the matrix defined by the product:
%\begin{equation*}
%  \begin{split}
%    \bra{u}M_{\Delta}\ket{l}\bra{l+1}M_{\Delta}\ket{t-1} \bra{t} M_{\Delta}\ket{v} = \mathbf{1}_{ t \sim l } \mathbf{1}_{u = t} \mathbf{1}_{v = l } \ \  
%  \end{split}
%\end{equation*}
% Given the Hamiltonian $\Hin + m \cdot 2I - \Hpr + \Hen$, consider the Hamiltonian $\alpha \Hin + m \cdot 2I - \Hpr M_\Delta \Hpr + \beta \Hen$. Denote $\Hpr$ by $U_{1} \otimes \ket{2}\bra{1}  + U_{2}^{\dagger} \otimes \ket{1}\bra{2} + \cdots $.
% Now Let $\Lambda_{t,l}$ be defined such that:
% 
% \begin{equation*}
%   \begin{split}
%     \Lambda^{\dagger}_{l,t} U_{l}^{\dagger}U_{t} \Lambda_{t,l} = U_{l}U_{l-1}..U_{t+1}U_{t}
%   \end{split}
% \end{equation*}
% And codsider the diagonliztion $W^{\dagger}\Hpr M_{\Delta} \Hpr W$. Where:
%
%\begin{equation*}
%  \begin{split}
%    & W = \sum\Lambda_{t,l}U_{t-1}U_{t-2}..U_{1} \otimes \ket{t}\bra{t} M_{\Delta}\ket{l}\bra{t} \\ 
%    \Rightarrow & W^{\dagger} = \sum U_{1}^{\dagger}U_{2}^{\dagger} .. U_{t-1}^{\dagger} \Lambda_{t,l}^{\dagger} \otimes \ket{t}\bra{t} M_{\Delta}\ket{l}^{*}\bra{t} 
%  \end{split}
%\end{equation*}
%  Noitce that: 
%
%\begin{equation*}
%  \begin{split}
%    & W^{\dagger}U_{l}^{\dagger}U_{t}\ket{l}\bra{l+1}M_{\Delta}\ket{t+1}\bra{t}W = \\
%    & W^{\dagger}U_{l}U_{t}\ket{l+1}\bra{l}M_{\Delta}\ket{t}\bra{t}\ket{t}\bra{t}M_{\Delta}\ket{v}\bra{t}\Lambda_{t,v}U_{t-1}U_{t-2}..U_{1} = \\
%    & U_{1}^{\dagger}U_{2}^{\dagger}.. \Lambda_{l,u}^{\dagger}  U_{l-1}^{\dagger}U_{t}\Lambda_{t,l}U_{t-1}.. U_{1} \ket{l}\bra{l}M_{\Delta}\ket{u}^{*}\bra{l}\ket{l}\bra{l+1} M_{\Delta}\ket{t-1} \bra{t}\ket{t}\bra{t} M_{\Delta}\ket{v} \ket{l}\bra{t} \\
%    & U_{1}^{\dagger} .. U_{l}^{\dagger} \Lambda^{\dagger}_{l,t} U_{l}^{\dagger}U_{t} \Lambda_{t,l} U_{t-1} .. U_{1} \ket{l}\bra{t} = \ket{l}\bra{t} \\ 
%    &\Rightarrow  W^{\dagger}\Hpr W = \sum_{i\sim j}\ket{i}\bra{j}
%  \end{split}
%\end{equation*}
%
%And the history state will look like:
%
%\begin{equation*}
%  \begin{split}
%    \ket{\eta} =  \sum \Lambda_{t,l}U_{t-1}U_{t-2}..U_{1}  \ket{x_{0}} \otimes  \ket{t} 
%  \end{split}
%\end{equation*}

Consider the LPS expander on $n$ vertices and denote $t \sim l$ if $t$ is adjacent to $l$. Let $M_\Delta \in \mathbb{C}^{n \times n}$ be the matrix defined by the product: \ctt{Such $M_{\Delta}$ dosn't exists.} 
\begin{equation*}
  \begin{split}
    &\bra{u}M_{\Delta}\ket{l}^{*}\bra{l+1}M_{\Delta}\ket{t-1} \bra{t} M_{\Delta}\ket{v} = \mathbf{1}_{ t \sim l } \mathbf{1}_{u = t} \mathbf{1}_{v = l } 
  \end{split}
\end{equation*}
Given the Hamiltonian $\Hin + m \cdot 2I - \Hpr + \Hen$, consider the Hamiltonian $\alpha \Hin + m \cdot 22\Delta I - \Hpr M_\Delta \Hpr + \beta \Hen$. Denote $\Hpr$ by $U_{1} \otimes \ket{2}\bra{1}  + U_{2}^{\dagger} \otimes \ket{1}\bra{2} + \cdots $.
Now let $\Lambda_{t,l}$ be defined such that:
 
 \begin{equation*}
   \begin{split}
     \Lambda^{\dagger}_{l,t} U_{l}^{\dagger}U_{t} \Lambda_{t,l} = U_{l}U_{l-1}..U_{t+1}U_{t}
   \end{split}
 \end{equation*}
And consider the diagonalization $W^{\dagger}\Hpr M_{\Delta} \Hpr W$. Where:

\begin{equation*}
  \begin{split}
    & W = \sum\Lambda_{t,l}U_{t-1}U_{t-2}..U_{1} \otimes \ket{t}\bra{t} M_{\Delta}\ket{l}\bra{t} \\ 
    \Rightarrow & W^{\dagger} = \sum U_{1}^{\dagger}U_{2}^{\dagger} .. U_{t-1}^{\dagger} \Lambda_{t,l}^{\dagger} \otimes \ket{t}\bra{t} M_{\Delta}\ket{l}^{*}\bra{t} 
  \end{split}
\end{equation*}
Notice that: 

\begin{equation*}
  \begin{split}
    & W^{\dagger}U_{l}^{\dagger}U_{t}\ket{l}\bra{l+1}M_{\Delta}\ket{t-1}\bra{t}W = \\
    & W^{\dagger}U_{l}U_{t}\ket{l+1}\bra{l}M_{\Delta}\ket{t}\bra{t}\ket{t}\bra{t}M_{\Delta}\ket{v}\bra{t}\Lambda_{t,v}U_{t-1}U_{t-2}..U_{1} = \\
    & U_{1}^{\dagger}U_{2}^{\dagger}.. \Lambda_{l,u}^{\dagger}  U_{l-1}^{\dagger}U_{t}\Lambda_{t,l}U_{t-1}.. U_{1} \ket{l}\bra{l}M_{\Delta}\ket{u}^{*}\bra{l}\ket{l}\bra{l+1} M_{\Delta}\ket{t-1} \bra{t}\ket{t}\bra{t} M_{\Delta}\ket{v} \ket{l}\bra{t} \\
    & U_{1}^{\dagger} .. U_{l}^{\dagger} \Lambda^{\dagger}_{l,t} U_{l}^{\dagger}U_{t} \Lambda_{t,l} U_{t-1} .. U_{1} \ket{l}\bra{t} = \ket{l}\bra{t} \\ 
    &\Rightarrow  W^{\dagger}\Hpr M_{\Delta} \Hpr W = \sum_{i\sim j}\ket{i}\bra{j}
  \end{split}
\end{equation*}
And the history state will look like:

\begin{equation*}
  \begin{split}
    \ket{\eta} =  \sum \Lambda_{t,l}U_{t-1}U_{t-2}..U_{1}   \ket{x_{0}} \otimes  \ket{t} \bra{t} M_{\Delta} \ket{l}
  \end{split}
\end{equation*}


\section{Lets change it a little bit.}
Mabye we should define $\Lambda$ to be depands on a single paramter, namely $\Lambda_{t}$ and: 
\begin{equation*}
   \begin{split}
     \Lambda^{\dagger}_{l} U_{l}^{\dagger}U_{t} \Lambda_{t} = U_{l}U_{l-1}..U_{t+1}
   \end{split}
 \end{equation*}
 That wil allow us to group terms, and if 
\begin{equation*}
  \begin{split}
     \sum_{v,u}\bra{u}D\ket{l}^{*}\bra{l+1}M_{\Delta}\ket{t-1} \bra{t} D\ket{v} = \mathbf{1}_{ t \sim l } % \mathbf{1}_{u = t} \mathbf{1}_{v = l } 
  \end{split}
\end{equation*}
Then we win. So now we ask wheter there exsits such $D,M_{\Delta}$ and $\Lambda_{t}$'s. (Or approximation).   
\begin{claim}
  There are such $\Lambda$'s and they given by: 
  
  \begin{equation*}
    \begin{split}
      \Lambda_{l}^{\dagger} = U_{l}\Lambda_{l-1}^{\dagger}U_{l-1}^{\dagger}U_{l}
    \end{split}
  \end{equation*}
\end{claim}
\begin{proof}
  By induction, assume existness for any  $l,t \le l-1$, namely $\Lambda_{l-1}= U_{l-1}^{\dagger}U_{l-2}\Lambda_{l-2}U^{\dagger}_{l-1}$. Then:  
\begin{equation*}
    \begin{split} 
      \Lambda^{\dagger}_{l} U_{l}^{\dagger}U_{t} \Lambda_{t} = &   \Lambda^{\dagger}_{l} U_{l}^{\dagger}U_{l-1}U_{l-1}^{\dagger}U_{t} \Lambda_{t} \\ 
      & \Lambda^{\dagger}_{l} U_{l}^{\dagger}U_{l-1}\Lambda_{l-1}\Lambda^{\dagger}_{l-1}U_{l-1}^{\dagger}U_{t} \Lambda_{t} = \Lambda^{\dagger}_{l} U_{l}^{\dagger}U_{l-1}\Lambda_{l-1} \cdot \ U_{l-1}..U_{t+1}  = \\ 
      & U_{l}U_{l-1}..U_{t+1} = \\ 
      & \ \Rightarrow   \Lambda_{l}^{\dagger} = U_{l}\Lambda_{l-1}^{\dagger}U_{l-1}^{\dagger}U_{l}
    \end{split}
  \end{equation*}
\end{proof}

What about defining $\tilde{D} = \bra{t} \mathbf{1}_{t\sim l} \ket{l} $, $D = \tilde{D}/det(D)$ and $\bra{l+1} M_{\Delta} \ket{t-1}  =  \mathbf{1}_{t\sim l} 1/\Delta^{2}$ ?  

\section{Ideas.}
\begin{enumerate}
  \item $M_{\Delta}$ has to be unitar (and not just hermitan).  
  \item $\Hin$ and $\Hen$ are the critical terms and deserve more gentle treatment.
\end{enumerate}

\section{Constant Clock.}

We can encode the time by unarity encoding. namely $\ket{t} = \ket{1^{t}000..}$. Then the check $\ket{l}\bra{t}$ replaced by the check $\ket{1_{l}0}\bra{1_{t}0}$.  And we also add checks for the validity of the input $ \ket{10}\bra{10} $.  
% In addition we have to add panility for any non ilegal edge. So, if $t\not{\sim}l$ then we add panility at the form $ J \cdot \ket{1_{l}0}\bra{1_{t}0}$.


%\begin{multicols*}{2}
% \section{Preambles}
  In this work, we propose a new construction for good LDPC codes, which also have a good testability parameter. In the sense that verfining a constant number of random checks, would be enough to detect any error with probability proportional to the error size. In contrast to previews, constructions made by \cite{Dinur}, \cite{leverrier2022quantum} and \cite{Pavel}, our construction doesn't require spicel properties of the small codes, such as $w$-robustness and $p$-resistance for puncturing. 
  
  Our proof also indirectly answers the following question. Why most of the good LDPC codes are known to be bad in terms of detecting errors? In other words, It seems that for most of them, there exist strings that are very far from being in the code and, meanwhile, fail to satisfy only a small number of restrictions.
  While the previous LDPC constructions focused on ensuring that the yielded code would have a good rate and distance parameters, our construction enforces the restrictions collection to have a nontrivial fraction of degeneration. That is, removing a single restriction will not change the code, as any restriction is linearly dependent on the others.



% %Coding theory has emerged by the need to transfer information in noisy communication channels. By embedding a message in higher dimension space, one can guarantee robustness against possible faults. The ratio of the original content length to the passed message \emph{length} is the \emph{rate} of the code, and it measures how consuming our communication protocol is. Furthermore, the \emph{distance} of the code quantifies how many faults the scheme can absorb such that the receiver can recover the original message. We could consider the code as all the strings that satisfy a specified restrictions collection.
%  
%
%  Non-formally, code is good if its distance and rate are scaled linearly in the encoded message length. In practice, one is also interested in implementing those checks efficiently. We say that a code is an LDPC if any bit is involved in a constant number of restrictions, each of which is a linear equation, and if any restriction contains a fixed number of variables.
%
%  Furthermore, finally, another characteristic of the code is its testability, which is the complexity of the number of random checks one should do to negate that a given candidate is in the code. Besides good codes being considered efficient in terms of robustness and overhead, they are also vital components in establishing secure multiparty computation \cite{MultiParty} and have a deep connection to probabilistic proofs.
%
%  First, we state the notations, definitions, and formal theorem in section 2. Then in sections 3 and 4, we review past results and provide their proofs to make this paper self-contained. Readers familiar with the basic concepts of LDPC, Tanner, and Expanders codes construction should consider skipping directly to section 5, in which we provide our proof. 
%

Coding theory has emerged due to the need to transfer information in noisy communication channels. By embedding a message in a higher-dimensional space, one can guarantee robustness against possible faults. The ratio of the original content length to the transmitted message \emph{length} is the \emph{rate} of the code, and it measures how consuming our communication protocol is. Additionally, the \emph{distance} of the code quantifies how many faults the scheme can absorb such that the receiver can recover the original message. We can consider the code as a collection of all strings that satisfy specified restrictions.

Non-formally, a code is good if its distance and rate scale linearly with the encoded message length. In practice, one is also interested in implementing these checks efficiently. We say that a code is an LDPC if any bit is involved in a constant number of restrictions, each of which is a linear equation, and if any restriction contains a fixed number of variables.

Moreover, another characteristic of the code is its testability, which is the complexity of the number of random checks one must do to verify that a given candidate is in the code. Besides being considered efficient in terms of robustness and overhead, good codes are also vital components in establishing secure multiparty computation \cite{MultiParty} and in the proof of the PCP theorem~\cite{PCPoriginal}.

%In Section 2, we state the notations, definitions, and formal theorem. Then, in Sections 3 and 4, we review past results and provide their proofs to make this paper self-contained. Readers familiar with the basic concepts of LDPC, Tanner, and Expanders codes construction may consider skipping directly to Section 5, in which we provide our proof.
%Readers familiar with the basic concepts of LDPC, Tanner, and Expanders codes construction may skip Sections 2, 3, and 4 and proceed directly to Section 5, where we provide our proof.


\printbibliography
\end{document}





