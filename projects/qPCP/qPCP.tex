

%\newcommand*{\ACM}{}%

\ifdefined\ACM

%\documentclass[sigplan,screen]{acmart}
  \documentclass[manuscript,screen,review]{acmart}

\else
  \documentclass{article}
  \usepackage[utf8]{inputenc}
\usepackage[a4paper, total={6in, 9in}]{geometry}
\usepackage{braket}
\usepackage{xcolor}
\usepackage{amsmath}
\usepackage{amsfonts}
\usepackage{amsthm}
\usepackage{amssymb}
%\usepackage[ocgcolorlinks]{hyperref}
\usepackage{hyperref}
%\usepackage{hyperref,xcolor}
%\usepackage[ocgcolorlinks]{ocgx2}
\usepackage{cleveref}
\usepackage{graphicx}
\usepackage{svg}
\usepackage{float}
\usepackage{tikz}
\usetikzlibrary{patterns, shapes.arrows}
\usepackage{adjustbox}
%\usepackage{tikz-network}
\usepackage{tkz-graph}
\usepackage{tkz-berge}
\usepackage[linesnumbered]{algorithm2e}
\usepackage{multicol}
\usepackage[backend=biber,style=alphabetic,sorting=ynt]{biblatex}
%\usepackage{xcolor}
%\usepackage{tkz-berge}
%\usepackage{tkz-graph}
\usepackage{pgfplots}
\usepackage{sagetex}
\usepackage{setspace}
\usepackage{etoc}
%\usepackage{wrapfig}
\usepackage{pgfgantt}
\DeclareUnicodeCharacter{2212}{−}
\usepgfplotslibrary{groupplots,dateplot}
\pgfplotsset{compat=newest}

\newtheorem{theorem}{Theorem}
\newtheorem{definition}{Definition}
\newtheorem{example}{Example}
\newtheorem{claim}{Claim}
\newtheorem{fact}{Fact}
\newtheorem{remark}{Remark}
\newtheorem*{theorem*}{Theorem}
\newtheorem{lemma}{Lemma}
\crefname{lemma}{Lemma}{Lemmas}
\hypersetup{colorlinks=true}
% , allcolors=blue,allbordercolors=blue,pdfborderstyle={0 0 1}}
%\hypersetup{pdfborder={2 2 2}}
% pdfpagemode=FullScreen,
% backref 

\newtheorem{problem}{Problem}
\crefname{problem}{Problem}{Problems}

\DeclareMathOperator{\Ima}{Im}


  \addbibresource{./sample.bib} 

\fi

\begin{document}

\newcommand{\commentt}[1]{\textcolor{blue}{ \textbf{[COMMENT]} #1}}
\newcommand{\ctt}[1]{\commentt{#1}}
\newcommand{\prb}[1]{ \mathbf{Pr} \left[ #1 \right]}
\newcommand{\prbm}[2]{ \mathbf{Pr}_{ #2 }\left[ #1 \right]}
\newcommand{\prbc}[3]{ \mathbf{Pr}_{ #2 }\left[ #1 \right | #3]}
\newcommand{\prbcprb}[3]{ \prbc{#2}{#1}{#3} \cdot \prb{#3} } 
\newcommand{\expp}[1]{ \mathbf{E} \left[ {#1} \right]}
\newcommand{\onotation}[1]{\(\mathcal{O} \left( {#1}  \right) \)}
\newcommand{\ona}[1]{\onotation{#1}}
\newcommand{\PSI}{{\ket{\psi}}}
\newcommand{\xij} { X_{ij} } 
\DeclareMathOperator{\Ima}{Im}
%\newcommand{\LESn}{\ket{\psi_n}}
%\newcommand{\LESa}{\ket{\phi_n}}
%\newcommand{\LESs}{\frac{1}{\sqrt{n}}\sum_{i}{\ket{\left(0^{i}10^{n-i}\right)^{n}}}}
%\newcommand{\Hn}{\mathcal{H}_{n}}
%\newcommand{\Ep}{\frac{1}{\sqrt{2^n}}\sum^{2^n}_{x}{ \ket{xx}}}
%\newcommand{\HON}{\ket{\psi_{\text{honest}}}}
%\newcommand{\Lemma}{\paragraph{Lemma.}}
\newcommand{\Cpa}{[n, \rho n, \delta n]}
%\setlength{\columnsep}{0.6cm}
\newcommand{\Jvv}{ \bar{J_{v}} } 
\newcommand{\Cvv}{ \tilde{C_{v}} } 

\newcommand{\Gz}{ G_{z}^{\delta} } 
\newcommand{ \Tann } {  \mathcal{T}\left( G, C_0 \right) }
\newcommand{\ireducable}{ireducable \hyperref[ire]{[\ref{ire}]} }
\newcommand{\cutUU}{E(U_{-1} \bigcup U_{+1} ,U)} 
\newcommand{\wcutUU}{w\left( E(U_{-1} \bigcup U_{+1} ,U)  \right)}
\newcommand{\testgo}{  \mathcal{T}\left(J, q , C_{0}\right) } 

\newcommand{\duC}{\left( C_{A}^{\perp}\otimes C_{B}^{\perp} \right)^{\perp}}
\newcommand{\duduC}{\left( C_{A}\otimes C_{B}\right)^{\perp}}
  





\title{Why The Following Doesn't Give Log-Local, Constant Gap Hamiltonian?} 
\author{David Ponarovsky}
%\author{Noa Viner, David Ponarovsky}

\ifdefined\ACM
  \affiliation{%
    \institution{The Th{\o}rv{\"a}ld Group}
    \streetaddress{1 Th{\o}rv{\"a}ld Circle}
    \city{Hekla}
  \country{Iceland}}
  \email{larst@affiliation.org}
\else
  \maketitle
\fi
%
\abstract{We propose an alternative simple construction of good LTC codes. In contrast to previews, constructions made by \cite{Dinur}, \cite{leverrier2022quantum} and \cite{Pavel}, our construction doesn't require spicel properties of the small codes, such as $w$-robustness and $p$-resistance for puncturing.  
} 


\ifdefined\ACM
  \maketitle
\fi


%\section{Introduction.}
%\begin{itemize}
%  \item Solovey-Kitaev shown to be true also for BQL. 
%  \item Even though postpting the measurements to the end seems to inflat the wide and the length of the circuit. It was provn that there is an effiucent way to elimanating intermidate measurements, So  BQL = BQuL. 
%  \item There is some error-rduction that I have to complete.  
%  \item Ta-shma shown that inverting a matrix is BQL complete. 
%  \item Also computing the eigen values of Hrmitan matrix is BQP complete. The algoritmes that have been used are has qauderic adavantage in space.
%\end{itemize}
%
%

\section{What we would like to have:}

\newcommand{\Hin}{H_{\text{init}}}
  \newcommand{\Hpr}{H_{\text{prop}}}
\newcommand{\Hen}{H_{\text{end}}}
%
%Consider the LPS expnader on $n$ veritces and denote $ t \sim l $ if $t$ is adjoint to $l$. Let $M_\Delta \in \mathbb{C}^{n \times n}$ be the matrix defined by the product:
%\begin{equation*}
%  \begin{split}
%    \bra{u}M_{\Delta}\ket{l}\bra{l+1}M_{\Delta}\ket{t-1} \bra{t} M_{\Delta}\ket{v} = \mathbf{1}_{ t \sim l } \mathbf{1}_{u = t} \mathbf{1}_{v = l } \ \  
%  \end{split}
%\end{equation*}
% Given the Hamiltonian $\Hin + m \cdot 2I - \Hpr + \Hen$, consider the Hamiltonian $\alpha \Hin + m \cdot 2I - \Hpr M_\Delta \Hpr + \beta \Hen$. Denote $\Hpr$ by $U_{1} \otimes \ket{2}\bra{1}  + U_{2}^{\dagger} \otimes \ket{1}\bra{2} + \cdots $.
% Now Let $\Lambda_{t,l}$ be defined such that:
% 
% \begin{equation*}
%   \begin{split}
%     \Lambda^{\dagger}_{l,t} U_{l}^{\dagger}U_{t} \Lambda_{t,l} = U_{l}U_{l-1}..U_{t+1}U_{t}
%   \end{split}
% \end{equation*}
% And codsider the diagonliztion $W^{\dagger}\Hpr M_{\Delta} \Hpr W$. Where:
%
%\begin{equation*}
%  \begin{split}
%    & W = \sum\Lambda_{t,l}U_{t-1}U_{t-2}..U_{1} \otimes \ket{t}\bra{t} M_{\Delta}\ket{l}\bra{t} \\ 
%    \Rightarrow & W^{\dagger} = \sum U_{1}^{\dagger}U_{2}^{\dagger} .. U_{t-1}^{\dagger} \Lambda_{t,l}^{\dagger} \otimes \ket{t}\bra{t} M_{\Delta}\ket{l}^{*}\bra{t} 
%  \end{split}
%\end{equation*}
%  Noitce that: 
%
%\begin{equation*}
%  \begin{split}
%    & W^{\dagger}U_{l}^{\dagger}U_{t}\ket{l}\bra{l+1}M_{\Delta}\ket{t+1}\bra{t}W = \\
%    & W^{\dagger}U_{l}U_{t}\ket{l+1}\bra{l}M_{\Delta}\ket{t}\bra{t}\ket{t}\bra{t}M_{\Delta}\ket{v}\bra{t}\Lambda_{t,v}U_{t-1}U_{t-2}..U_{1} = \\
%    & U_{1}^{\dagger}U_{2}^{\dagger}.. \Lambda_{l,u}^{\dagger}  U_{l-1}^{\dagger}U_{t}\Lambda_{t,l}U_{t-1}.. U_{1} \ket{l}\bra{l}M_{\Delta}\ket{u}^{*}\bra{l}\ket{l}\bra{l+1} M_{\Delta}\ket{t-1} \bra{t}\ket{t}\bra{t} M_{\Delta}\ket{v} \ket{l}\bra{t} \\
%    & U_{1}^{\dagger} .. U_{l}^{\dagger} \Lambda^{\dagger}_{l,t} U_{l}^{\dagger}U_{t} \Lambda_{t,l} U_{t-1} .. U_{1} \ket{l}\bra{t} = \ket{l}\bra{t} \\ 
%    &\Rightarrow  W^{\dagger}\Hpr W = \sum_{i\sim j}\ket{i}\bra{j}
%  \end{split}
%\end{equation*}
%
%And the history state will look like:
%
%\begin{equation*}
%  \begin{split}
%    \ket{\eta} =  \sum \Lambda_{t,l}U_{t-1}U_{t-2}..U_{1}  \ket{x_{0}} \otimes  \ket{t} 
%  \end{split}
%\end{equation*}

Consider the LPS expander on $n$ vertices and denote $t \sim l$ if $t$ is adjacent to $l$. Let $M_\Delta \in \mathbb{C}^{n \times n}$ be the matrix defined by the product: \ctt{Such $M_{\Delta}$ dosn't exists.} 
\begin{equation*}
  \begin{split}
    &\bra{u}M_{\Delta}\ket{l}^{*}\bra{l+1}M_{\Delta}\ket{t-1} \bra{t} M_{\Delta}\ket{v} = \mathbf{1}_{ t \sim l } \mathbf{1}_{u = t} \mathbf{1}_{v = l } 
  \end{split}
\end{equation*}
Given the Hamiltonian $\Hin + m \cdot 2I - \Hpr + \Hen$, consider the Hamiltonian $\alpha \Hin + m \cdot 22\Delta I - \Hpr M_\Delta \Hpr + \beta \Hen$. Denote $\Hpr$ by $U_{1} \otimes \ket{2}\bra{1}  + U_{2}^{\dagger} \otimes \ket{1}\bra{2} + \cdots $.
Now let $\Lambda_{t,l}$ be defined such that:
 
 \begin{equation*}
   \begin{split}
     \Lambda^{\dagger}_{l,t} U_{l}^{\dagger}U_{t} \Lambda_{t,l} = U_{l}U_{l-1}..U_{t+1}U_{t}
   \end{split}
 \end{equation*}
And consider the diagonalization $W^{\dagger}\Hpr M_{\Delta} \Hpr W$. Where:

\begin{equation*}
  \begin{split}
    & W = \sum\Lambda_{t,l}U_{t-1}U_{t-2}..U_{1} \otimes \ket{t}\bra{t} M_{\Delta}\ket{l}\bra{t} \\ 
    \Rightarrow & W^{\dagger} = \sum U_{1}^{\dagger}U_{2}^{\dagger} .. U_{t-1}^{\dagger} \Lambda_{t,l}^{\dagger} \otimes \ket{t}\bra{t} M_{\Delta}\ket{l}^{*}\bra{t} 
  \end{split}
\end{equation*}
Notice that: 

\begin{equation*}
  \begin{split}
    & W^{\dagger}U_{l}^{\dagger}U_{t}\ket{l}\bra{l+1}M_{\Delta}\ket{t-1}\bra{t}W = \\
    & W^{\dagger}U_{l}U_{t}\ket{l+1}\bra{l}M_{\Delta}\ket{t}\bra{t}\ket{t}\bra{t}M_{\Delta}\ket{v}\bra{t}\Lambda_{t,v}U_{t-1}U_{t-2}..U_{1} = \\
    & U_{1}^{\dagger}U_{2}^{\dagger}.. \Lambda_{l,u}^{\dagger}  U_{l-1}^{\dagger}U_{t}\Lambda_{t,l}U_{t-1}.. U_{1} \ket{l}\bra{l}M_{\Delta}\ket{u}^{*}\bra{l}\ket{l}\bra{l+1} M_{\Delta}\ket{t-1} \bra{t}\ket{t}\bra{t} M_{\Delta}\ket{v} \ket{l}\bra{t} \\
    & U_{1}^{\dagger} .. U_{l}^{\dagger} \Lambda^{\dagger}_{l,t} U_{l}^{\dagger}U_{t} \Lambda_{t,l} U_{t-1} .. U_{1} \ket{l}\bra{t} = \ket{l}\bra{t} \\ 
    &\Rightarrow  W^{\dagger}\Hpr M_{\Delta} \Hpr W = \sum_{i\sim j}\ket{i}\bra{j}
  \end{split}
\end{equation*}
And the history state will look like:

\begin{equation*}
  \begin{split}
    \ket{\eta} =  \sum \Lambda_{t,l}U_{t-1}U_{t-2}..U_{1}   \ket{x_{0}} \otimes  \ket{t} \bra{t} M_{\Delta} \ket{l}
  \end{split}
\end{equation*}


\section{Lets change it a little bit.}
Mabye we should define $\Lambda$ to be depands on a single paramter, namely $\Lambda_{t}$ and: 
\begin{equation*}
   \begin{split}
     \Lambda^{\dagger}_{l} U_{l}^{\dagger}U_{t} \Lambda_{t} = U_{l}U_{l-1}..U_{t+1}
   \end{split}
 \end{equation*}
 That wil allow us to group terms, and if 
\begin{equation*}
  \begin{split}
     \sum_{v,u}\bra{u}D\ket{l}^{*}\bra{l+1}M_{\Delta}\ket{t-1} \bra{t} D\ket{v} = \mathbf{1}_{ t \sim l } % \mathbf{1}_{u = t} \mathbf{1}_{v = l } 
  \end{split}
\end{equation*}
Then we win. So now we ask wheter there exsits such $D,M_{\Delta}$ and $\Lambda_{t}$'s. (Or approximation).   
\begin{claim}
  There are such $\Lambda$'s and they given by: 
  
  \begin{equation*}
    \begin{split}
      \Lambda_{l}^{\dagger} = U_{l}\Lambda_{l-1}^{\dagger}U_{l-1}^{\dagger}U_{l}
    \end{split}
  \end{equation*}
\end{claim}
\begin{proof}
  By induction, assume existness for any  $l,t \le l-1$, namely $\Lambda_{l-1}= U_{l-1}^{\dagger}U_{l-2}\Lambda_{l-2}U^{\dagger}_{l-1}$. Then:  
\begin{equation*}
    \begin{split} 
      \Lambda^{\dagger}_{l} U_{l}^{\dagger}U_{t} \Lambda_{t} = &   \Lambda^{\dagger}_{l} U_{l}^{\dagger}U_{l-1}U_{l-1}^{\dagger}U_{t} \Lambda_{t} \\ 
      & \Lambda^{\dagger}_{l} U_{l}^{\dagger}U_{l-1}\Lambda_{l-1}\Lambda^{\dagger}_{l-1}U_{l-1}^{\dagger}U_{t} \Lambda_{t} = \Lambda^{\dagger}_{l} U_{l}^{\dagger}U_{l-1}\Lambda_{l-1} \cdot \ U_{l-1}..U_{t+1}  = \\ 
      & U_{l}U_{l-1}..U_{t+1} = \\ 
      & \ \Rightarrow   \Lambda_{l}^{\dagger} = U_{l}\Lambda_{l-1}^{\dagger}U_{l-1}^{\dagger}U_{l}
    \end{split}
  \end{equation*}
\end{proof}

What about defining $\tilde{D} = \bra{t} \mathbf{1}_{t\sim l} \ket{l} $, $D = \tilde{D}/det(D)$ and $\bra{l+1} M_{\Delta} \ket{t-1}  =  \mathbf{1}_{t\sim l} 1/\Delta^{2}$ ?  

\section{Ideas.}
\begin{enumerate}
  \item $M_{\Delta}$ has to be unitar (and not just hermitan).  
  \item $\Hin$ and $\Hen$ are the critical terms and deserve more gentle treatment.
\end{enumerate}

\section{Constant Clock.}

We can encode the time by unarity encoding. namely $\ket{t} = \ket{1^{t}000..}$. Then the check $\ket{l}\bra{t}$ replaced by the check $\ket{1_{l}0}\bra{1_{t}0}$.  And we also add checks for the validity of the input $ \ket{*10*1}\bra{*10*1} $ that add a quaderic number of checks.  
% In addition we have to add panility for any non ilegal edge. So, if $t\not{\sim}l$ then we add panility at the form $ J \cdot \ket{1_{l}0}\bra{1_{t}0}$.

\section{Using the classical LTC as hmiltonian} 
The idea of looking for a quantum LTC code through a construction of CSS code just committed to failure as approximating the ground state of local commute Hamiltonian sets on the expanders is in NP. Yet that fact also gives hope that using the classical LTC codes, as non-commute Hamiltonian on expanders, as they are as quantum Hamiltonian might yield a Hamiltonian which approximates it is in $QMA$. Let $H_{X} = J_{0}I - \mathcal{T} ( V^{+}, C_{A}\otimes C_{B}) $$H_{Z} = J_{0}I -   \mathcal{T} ( V^{+}, C_{A}^{\perp}\otimes C_{B}^{\perp}) $. Here the notation $H_{X}$ is used to describe Hamiltona and not a parity check matrix.  Denote $H = H_{X} + H_{Z}$.

\begin{definition}
  Consider the Hamitonain above, over $\frac{1}{4}\Delta^{2}n$ qubits, the decion problem q-c-LTC$[a,b]$ is to answer wheter there exsits state $\ket{\psi}$ such that $\bra{\psi}H\ket{\psi} \le a $ or that for any state the $\bra{\psi}H\ket{\psi} \ge  b $. 
\end{definition}


\begin{claim}
 q-c-LTC$[a,b]$ in QMA. 
\end{claim}
\begin{proof}
  By definition the problem is Local Hamiltonain with polynomail gap.%  We have to show that the gap is $\sim 1/poly$
\end{proof}

\begin{claim} 
 q-c-LTC$[a,b]$ in quantum PCP.  
\end{claim}


   


%\frac{1}{\sqrt{2}}\left( \bra{ \varphi} + \bra{\psi} \right)
\begin{equation*}
  \begin{split}
    \bra{\psi} H_{X} + H_{Z} \ket{\psi} \ge \kappa d\left( \psi, C_{X} \right) + \kappa d\left( \psi, C_{Z} \right)
  \end{split}
\end{equation*}
%\frac{1}{\sqrt{2}}\left( \ket{ \varphi} + \ket{\psi} \right)
\begin{equation*}
  \begin{split}
    & \frac{1}{\sqrt{2}}\left( \bra{ \varphi} + \bra{\psi} \right) H \frac{1}{\sqrt{2}}\left( \ket{ \varphi} + \ket{\psi} \right) \\ 
    & \frac{1}{2} \bra{ \varphi}H_{X} \ket{\varphi} + \frac{1}{2}\bra{\psi}H_{Z} \ket{\psi} -   \frac{1}{2}\bra{\varphi}H_{X}\ket{\psi} - \frac{1}{2}\bra{\varphi}H_{Z}\ket{\psi} + \\
    & +  \frac{1}{2} \bra{ \psi}H_{X} \ket{\psi} + \frac{1}{2}\bra{\varphi}H_{Z} \ket{\varphi} \\
& = a +  \frac{1}{2}\bra{\varphi}H_{X}\ket{\psi} - \frac{1}{2}\bra{\varphi}H_{Z}\ket{\psi} \\
    & +  \frac{1}{2} \bra{ \psi}H_{X} \ket{\psi} + \frac{1}{2}\bra{\varphi}H_{Z} \ket{\varphi} \\
& \ge a +  \frac{1}{2}\bra{\varphi}H_{X}\ket{\psi} - \frac{1}{2}\bra{\varphi}H_{Z}\ket{\psi} \\
& +  \frac{1}{2}\kappa d\left( C_{X}, \psi \right) + \frac{1}{2}\kappa d\left( C_{Z}, \varphi \right) \\
& \ge a +  \frac{1}{2}\bra{\varphi}H_{X}\ket{\psi} - \frac{1}{2}\bra{\varphi}H_{Z}\ket{\psi} \\
& +  \frac{1}{2}\kappa d\left( C_{X}, \psi \right) + \frac{1}{2}\kappa d\left( C_{Z}, \varphi \right) 
  \end{split}
\end{equation*}


\begin{equation*}
  \begin{split}
    \prb{\braket{\psi|H|\psi} \ge b } \le \delta  
  \end{split}
\end{equation*}

Suppose that $\prb{\braket{\psi|H|\psi} \ge b } \le \delta $, So at most $\delta$ of the vertices has energy greater than $b$ and at least $1 - \delta$ ofthe vertices has energy less than $a$. We will say that good vertex is a negative vertex that sibiling only to one positive vertex which doesn't pass the test. We will say that a normal vertex is a  positive non-passing vertex that adjoint only to good vertices. What can we say about the normal vertices?  


\begin{claim}
  Let $x \in \mathbb{F}^{\Delta}_{2}$ and denote by $H_{x}$ the Hamiltonain which on the $i$th coordinate applay $X$ if $x_{i} = 1$ and identty otherwise.And let $c(x) \in [\Delta,\rho\Delta,\delta\Delta]$ be the codeword obtained by encoding $x$. Then $H_{x} \le H_{c(x)}$.  
\end{claim}

\begin{equation*}
  \begin{split}
    & \sum{ H_{x_{i}}} \rightarrow \sum_{|I|=m}{ \prod_{x^{i} \in I } H_{x_{i}}}\rightarrow\sum_{|I|=m}{  H_{ \sum_{x_{i}\in I} x_{i}}} \\
    & \rightarrow \sum_{|I|=m}{  H_{ c\left(\sum_{x_{i}\in I} x_{i}\right)} } \rightarrow \sum_{|I|=m}{  H_{ \sum_{z_{i}} z_{i}}}  
  \end{split}
\end{equation*}


%\begin{multicols*}{2}
% \section{Preambles}

Localy Testable Codes, or LTC, are error correction codes such that verfining a uinformly random cchoosen check, would be enough to detect any error with probability proportional to it's size. Bisdes the clear computional adventage they offer, they took roles at the eriler PCP proofs.  
  In this work, we propose a new construction for good LTC codes, which also have a good testability parameter. In the sense   Our proof also indirectly answers the following question. Why most of the good LDPC codes are known to be bad in terms of detecting errors? In other words, It seems that for most of them, there exist strings that are very far from being in the code and, meanwhile, fail to satisfy only a small number of restrictions.
  While the previous LDPC constructions focused on ensuring that the yielded code would have a good rate and distance parameters, our construction enforces the restrictions collection to have a nontrivial fraction of degeneration. That is, removing a single restriction will not change the code, as any restriction is linearly dependent on the others.



% \section{Introduction.}
.. bla bla bla.. bla bla ..     
\definition[General Entanglement State]{ We say that $\PSI$ is general entanglement \label{def:gEnt} if .. }

\definition[Local-Measure-Circuit] { We say that a quantum circuit $C$ is a local measure circuit \label{def:lmc} if it's can be described as a decomposition of poly classical circuit and a constant depth quantum circuit which contains only 1-qubit gates and measurements. 

We would think about local measure circuits as chip circuits. }

\definition[$p_{0}-\Delta$ Fault Tolerance Circuit]{ We say that $\mathcal{C}$ is $p_{0}-\Delta$ fault tolerance \label{def:gft} presentation of abstract circuit $C$ if there exists a local measure circuit $C_{0}$ \ref{lmc} such it's grunted that for noise $p < p_{0}$ $\mathcal{C}$ compute $C$ w.h.p,
And in addition, if $p \in \left( p_{0}, p_{0} + \varepsilon \right)$ then by applying a $C_{0}$ on $\mathcal{C}$ output yields a general entanglement state \label{def:gEnt}}       

\ctt{We would like to add a complexity parameter for the above definition, for example, ``a general entanglement state over more than $\frac{1}{5}$ of the qubits.}  


\section{Exersices.}
\begin{exercise}[Beasd on Free Games] Consider the following protocol, First we measure $k$ arbitrary qubits in the Fourier base, then we take only the bits measrued zero.. \ctt{somthing here is wrong.} 
\end{exercise}

\begin{definition}
  BellQMA protocol is a QMA varition when the Arthur is restricted to perform only non adaptive and untangled measurments and classical computation. 
\end{definition}
\begin{claim}
  There is a BellQMA protocol which, given a 3-SAT instance with $m$ clauses, uses $\Theta\left( \sqrt{m} \right)$ Merlines, each of them sends $~Q(\log m)$ quabits. The protocol has a completeness $ 1 - exp(-\Omega(\sqrt{m})) $ and soundness $ 1 - \Omega(1)$. 
\end{claim}
Bottom line, They shown that the entangled measurment is not necessary.  

\newpage

\begin{definition}
  Given state $\ket{\psi}$ over $n$ qubits. Let $\ket{\psi^{(i)}}$ be one qubit state defined as $\ket{\psi^{(i)}} = \left( \braket{0|\psi} \right)  \ket{0} + \left( \braket{1|\psi} \right)  \ket{1}$. In addition, define the state $\ket{\psi}^{-i}$ to be the state over $n-1$ qubits, obtained by tracing out the $i$th qubit. We will abuse the notation and denote by $\ket{\psi^{-i}} \otimes \ket{\psi^{(i)}}$ the results by stacking in the qubit of $\ket{\psi^{(i)}}$ in the $i$th position. 
\end{definition}

\begin{claim} \label{claim:noisyproof}
  Denote by $H_{f}$ the $k$-local Hamiltonian obtained by applying Kitaev reduction on a fault tolerant circuit, with gap $b -a \ge 1/poly(n)$. And suppose there is a state $\ket{\psi}$ over $n$ qubits with energy lower than $a$. Than for any $i \in [n]$ it holds that 
  \begin{equation*}
    \begin{split}    
  \braket{\ket{\psi^{-i}} \otimes \ket{\psi^{(i)}}|H_{f}|\ket{\psi^{-i}} \otimes \ket{\psi^{(i)}}} < a
    \end{split}
  \end{equation*}
\end{claim}

\begin{definition} \label{definition:denseHam}
  Given $H_{f}$ Consider the Hamiltonian $H^{\prime}_{f}$ over $2n$ qubits defined by summing local terms $H_{j}$ such that either $H_{j}$ is a local term of $H_{f}$ or that there exist $H_{i}$ in $H_{f}$ such that $H_{i}$ equavilance to $H_{j}$ on $k-1$ nontrival coordinates, and in addition, let $l\in [n]$ be the $k$th nontrivial qubit been act by $H_{i}$ and denote the by $U$ the corresponding opereation applplied by it, namely $H_{i}^{(l)} = U$. Then $H_{j}^{(l + n)} = U$. 
\end{definition}

\begin{claim}
  If $H_{f}$ has $a,b$-gap, So is $H^{\prime}_{f}$. Furthmore $H^{\prime}_{f}$ has the same locality. 
\end{claim}
\begin{definition}
  Let $H$ be a local Hamiltonian and consider the a qunbit $q$. Denote by $H(q)$ the set of the local terms in $H$ act nontrivially on $q$. Now we will define the $q,\zeta$-majority-term relative to $H$, $M[H,q,\zeta]$, to be the $k^{2}$ Hamiltonian defined by: 
  \begin{equation*}
    \braket{\psi|M[H,q,\zeta]|\psi} = \begin{cases}
      \begin{split}
        & 1 & \prbm{ \braket{\psi|H_{i}|\psi} \ge 1 }{H_{i} \sim H(q)} \ge \zeta \\
        & 0 & \prbm{ \braket{\psi|H_{i}|\psi} \ge 1 }{H_{i} \sim H(q)} < \zeta 
    \end{split}
  \end{cases}
  \end{equation*}
\end{definition}

\begin{claim}
  There is a $f(k)$-time algorithm that compute $M[H,q,\zeta]$ where $f(k)$ is a function of $k$, namely doesn't depeand on $n$. 
\end{claim}

\begin{definition} \label{definition}  
  Let $H$ be a $k$-local, Denote by $M[H, \zeta]$ the $\zeta$-majority Hamiltonian relative to $H$ to be: 
  \begin{equation*}
    \begin{split}
      M[H,\zeta] = \frac{1}{n}\sum_{q\in[n]}{ M[H,q,\zeta]} 
    \end{split}
  \end{equation*}
\end{definition}

\begin{claim}
  There exist $\zeta$ such that $M[H,\zeta]$ is $k^{2}$ local Hamiltonian with $1\frac{1}{2}$ gap.   
\end{claim}

\begin{proof}
  Suppose that $H$ has a state $\ket{\psi}$ with energy below $a$. Then:  
  \begin{equation*}
    \begin{split}
      \braket{\psi|M[H,q,\zeta]|\psi} & =  \exppm{ \braket{\psi|M[H,q,\zeta]|\psi} }{\sim q} \\ 
      & \\
      & =\exppm{\prbm{ \braket{\psi|H_{i}|\psi} \ge 1 }{H_{i}\sim H(q)} \ge \zeta }{\sim q} \\
      & \\ 
      & \le  \frac{ \exppm{ \exppm{ \braket{\psi|H_{i}|\psi} | q }{ H_{i} \sim H(q)} }{ \sim q }  }{\zeta } \\
      & \\ 
      & \le  \frac{ \exppm{ \braket{\psi|H_{i}|\psi} }{ H_{i} \sim H }  }{\zeta } \le \frac{a}{\zeta}
    \end{split}
  \end{equation*}
  Frathmore consider the case in which for every state it holds that $\braket{\psi|H|\psi}\ge b$, and denote by $\alpha$ the portion of the qubits which see lass than $\zeta k$ energy around them, then: 

  \begin{equation*}
    \begin{split}
    \braket{\psi|H|\psi}  & =  \frac{1}{m}\sum_{H_{i} \in H} {\braket{\psi|H_{i}|\psi} }  \\
    & = \frac{1}{m\cdot k}\sum_{q \in [n]}\sum_{H_{i} \in H(q)} {\braket{\psi|H_{i}|\psi} }  \\
    & = \frac{1}{n \cdot k_{2}}\sum_{q \in [n]}\sum_{H_{i} \in H(q)} {\braket{\psi|H_{i}|\psi} }  \\
    & \le \frac{1}{n \cdot k_{2}} n \cdot k_{2}\left( \alpha \zeta + (1-\alpha) \right) \\
    & \Rightarrow \alpha  \le \frac{1  - b} { 1 -\zeta}
    \end{split}
  \end{equation*}

 \begin{equation*}
    \begin{split}
      \braket{\psi|M[H,q,\zeta]|\psi} & =  \exppm{ \braket{\psi|M[H,q,\zeta]|\psi} }{\sim q} \\ 
      & y
     \end{split}
  \end{equation*}
 

\end{proof}

\newpage
\begin{definition}
  Let $C$ a qaumtum circuit, and let $P : \mathcal{C} \times s  \rightarrow \mathcal{C}$ be a function which given a quantum circuit $C$ and a seed $s$ maps the circuit into another circuits. We will think about $P$ as a mapping ideal circuits to those which might run in the end. We will say that $C_{f}$ is a $P$-fault tolerece verision of $C$ if for any state such that $C\ket{\psi}$ measure $1$ with high probability, it holds that $P(C_{f},s)\ket{\psi}$ measure string $\bar{1}$ (on which we think as logical $1$) with high probability.  
\end{definition}

\begin{claim}
  Let $C_{f} = U_{T}U_{T-1}..U_{0}$ be a $P$-fault tolereance circuit verision of circuit $C$. Than for any $t < T$ it holds that:    
  \begin{equation*}
    \begin{split}
      || P\left(U_{1}^{\dagger} U_{2}^{\dagger}.. U_{t}^{\dagger} ,s \right) P\left(U_{t}.. U_{2} U_{1}, s^{\prime}\right) - I  ||_{op} < 1/poly(T)
    \end{split}
  \end{equation*}
\end{claim}


\begin{definition} 
Let $C = U_{T}U_{T-1} \dots U_{0}$ be a quantum circuit. Denote by $Z(C)$ the random variable circuit that is obtained by the following random process: $Z(C)$ is the chain of $X_{S}X_{S-1} \dots X_{0}$ such that $X_{0}=U_{0}$, $X_{S}=U_{T}$ and $X_{i}$ is defined recursively. If $X_{i-1} = U_{j}$ for $j>0$, then $X_{i}$ is chosen uniformly from  $\{ U_{j+1}, U_{j}^{\dagger} \}$. We will call any circuit that can be a result of such a process a $C$-Zigzag.
\end{definition}


\begin{definition} 
Let $C = U_{T}U_{T-1}..U_{0}$ be a quantum circuit. We will name any Hamiltonian that can be obtained by the following random process as a $C$-hashed. First, we chain $C$ with itself $\Theta(T)$ times as follows:  
  \begin{equation*}
    \begin{split}
      \rightarrow CC^{\dagger}CC^{\dagger}CC^{\dagger}C .. C^{\dagger}CC^{\dagger}CC^{\dagger}C  
    \end{split}
  \end{equation*}
  Now, any local terms in the propagation Hamiltonian will be at the form of: 
  \begin{equation*}
    \begin{split}
      I - \frac{1}{\Delta}  \left(  U_{i}\ket{t + 1 + 2T\cdot m^{\prime}}\bra{ t + 2T\cdot m} + U^{\dagger}_{i+1}\ket{t + 1 + 2T\cdot m}\bra{ t +1 + 2T\cdot m^{\prime}} \right)
    \end{split}
  \end{equation*}
  where $\Delta$ is the degree of the vertex associated with $\ket{ t + 2T \cdot m }$ in the adjacency graph (where each time coordinate is associated with a vertex and two vertices are connected only if there is a check that verifies their consistency with each other). We choose the local terms such that the adjacency graph has uniform degree and $\Delta$ is a constant number.
\end{definition}

\begin{definition}
  The $C$-hashed-Zigzag Hamiltonian will be the mixed of the two technics above. Given circuit $C$ we first generate a $C$-Zigzag circuit from it, denote it by $C^{\prime}$ with length $T^{\prime}$ and then we applay the hash reduction, but this time we also allow connection at the form:
  \begin{equation*}
    \begin{split}
      U_{i}\ket{t + 1 + 2T^{\prime}\cdot m^{\prime}}\bra{ t^{\prime} + 2T^{\prime}\cdot m}
    \end{split}
  \end{equation*}
  where the $U_{t}..U_{0} = U_{t^\prime}..U_{0}$ in $C^{\prime}$ (basically, in time $t^{\prime}$ we have returned to the state at time $t$). 
\end{definition}

\begin{definition}
  We will say that the following Hamiltonian is a $C$-multilayers if it can be obtained by the follow process. For a given graph $G$, any local term in $\Hpr$ will have the following form:  
  \begin{equation*}
    \begin{split}
      I -  \frac{1}{\Delta} \left( U_{i}\ket{u,t + 1}\bra{v,t} +   U_{i+1}^{\dagger} \ket{v,t}\bra{u,t+1}\right)
    \end{split}
  \end{equation*}
  Where $v,u$ are connected vertices in $G$. In some sanse, any pair of adjoint layers (belongs to consective time step) are perform 'expnder graph' (Yet, the total expension of the obtained graph is still low).
\end{definition}


\printbibliography


\end{document}





