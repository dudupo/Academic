There is some magic hides in the You right that if the amplitudes depends on the number of the qubits, then there is  

Equivalence to Fusion-controlled gates problem.
====
Stage (1) - Fusion-controlled circuit.
----
For a circuit, $U$ denotes by $U^{c}$, the controlled version of it. Define the fusion-controlled gate to be the circuit $$(U\otimes V)^{c} :0^{\Delta}\otimes 0^{U}\otimes 0^{V}   \rightarrow \begin{cases} 0 0^{\Delta -1} U0^{U} 0^{V} & x_{0} = 0\\ 1 0^{\Delta -1} 0^{U} V0^{V} & else \end{cases} $$  We first show that given $U^{c},V^{c}$ one can implement at the same depth cost the circuit $(U\otimes V)^{c}$

stage (2) - Induction 
----
Assume that for any set of given $2^{n-1}$ amplitudes encoding a state $\psi$ there is a $\log(n-1)$ circuit $U_{\psi}^{(n-1)}$ that generate $\psi$. Recall that any state in a $n$-dimisional space could be write as $\psi = \alpha_{0} 0 \psi_{0} + \alpha_{1} 1 \psi_{1} $. By the assumption there are $U_{\psi_{0}}^{(n-1)},U_{\psi_{1}}^{(n-1)}$ circuits each at depth at most $\log n$ generate $\psi_{0}$ and $\psi_{1}$ corespondly. We are going to construct a circuit that computes $\psi$ by the following: 

 1. Prepare $2 \times S_{n-1} + 1$ anciles. And arrange them by $S_{n-1} \ |0| \ S_{n-1}$. 
 2. Rotate the middle qubit as follow: $ 0 \mapsto \alpha_{0} 0 + \alpha_{1} 1 $.   
 3. Apply $\left(U_{\psi_{0}}^{(n-1)} \otimes U_{\psi_{1}}^{(n-1)}\right)^{c} $ to have $$\alpha_{0} 0 0^{\Delta -1} U_{\psi_{0}}^{(n-1)} 0^{(n-1)} 0^{(n-1)} + \alpha_{1} 1 0^{(\Delta - 1)}0^{(n-1)}U_{\psi_{1}}^{(n-1)}0^{(n-1)}$$
4. Now apply control swap, use the $\Delta$th qubit as the control wire and swap between $S_{0}$,$S_{1}$. That yields the state: 
$$ 0^{\Delta -1} \left(\alpha_{0} 0 U_{\psi_{0}}^{(n-1)} 0^{(n-1)} + \alpha_{1} 1 U_{\psi_{1}}^{(n-1)}0^{(n-1)} \right)0^{(n-1)} $$
5. By induction, the above state expanse to $\psi \otimes 0^{*}$.

So if we denote by $d(n), S(n)$ the depth and the space needed to compute a general state correspond to a given amplitude, It follows by the recursion that: 
$$ \begin{split} & S(n) = T_{S}[S(n-1)]  &  \\ & d(n) = T_{d}[d(n-1)] + \overbrace{1}^{ \text{rotation} } + \overbrace{n-1}^{\text{swap}} =  T_{d}[d(n-1)] + n   \end{split} $$     



First Soultion $\times 4$ Space.
==== 

1. Prapere $+2$ qubits.
2. Apply $CX$ from the first qubit to the second.
3. Apply $U^{c}$ negative-controlled by the first qubit over the first $S_{u}$ qubits, and in parallel apply $V^{c}$ controlled by the second qubit over the $S_{v}$ quibtis.   
4. Apply $CX$ from the first qubit to the second. (reverse step 2).

Clearly $T_{S}[S(n)] = 2 \cdot S(n) + 2$ and $T_{d}[d(n)] = 1 + d(n) + 1$ And that sumup to:

 $$ \begin{split} S(n) & = T_{S}[S(n-1)] = 2T_{S}[S(n-2)] + 2  & \\ & = 
2\cdot 2^{n - 1}...+2\cdot 2^{2} +2\cdot 2 + 2 & \\ & = 2\cdot 2^{n} & \\ & d(n) = T_{d}[d(n-1)] + \overbrace{1}^{ \text{rotation} } + \overbrace{n-1}^{\text{swap}} =  T_{d}[d(n-1)] + n   \end{split} $$

Second Solution $\times 2$ Space. 
====
Stage (1) - Fusion-controlled circuit.
----
For a circuit, $U$ denotes by $U^{c}$, the controlled version of it. We first show that given $U^{c},V^{c}$ one can implement at the same depth cost the circuit $(U\otimes V)^{c}$. It's well known that $U^{c}$ could be obtained by $U$ by adding single qubits gates on $U$ wires and connecting Cnot gates from the control wire to $U$ wires. Notice that for running $(V\otimes U)^{c}$ it's sufficient to handle the Conts as each of the single qubits gates operate independently in parallel. Consider the following recipe: 

On the $i$th iteration of the circuits,
  
1. If there is no conflict between $U^{c}$ and $V^{c}$, meaning that either only one of them uses the control wire at that step or that neither of them, then $(U \otimes V)^{c}_{t} \leftarrow U^{c}_{t} \otimes V^{c}_{t}$
2. Else, at the $i$ step the controlled wire flow for both of them, So denote by $ x_{c},x_{v},x_{u}$ the tree bits such at time $t$   


  
 

Third Solution T.C.S Approach.  
=====
What for God's Sake, Happens Here?
=====  
