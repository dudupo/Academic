

%\newcommand*{\ACM}{}%

\ifdefined\ACM

%\documentclass[sigplan,screen]{acmart}
  \documentclass[manuscript,screen,review]{acmart}

\else
  \documentclass{article}
  \usepackage[utf8]{inputenc}
\usepackage[a4paper, total={6in, 9in}]{geometry}
\usepackage{braket}
\usepackage{xcolor}
\usepackage{amsmath}
\usepackage{amsfonts}
\usepackage{amsthm}
\usepackage{amssymb}
%\usepackage[ocgcolorlinks]{hyperref}
\usepackage{hyperref}
%\usepackage{hyperref,xcolor}
%\usepackage[ocgcolorlinks]{ocgx2}
\usepackage{cleveref}
\usepackage{graphicx}
\usepackage{svg}
\usepackage{float}
\usepackage{tikz}
\usetikzlibrary{patterns, shapes.arrows}
\usepackage{adjustbox}
%\usepackage{tikz-network}
\usepackage{tkz-graph}
\usepackage{tkz-berge}
\usepackage[linesnumbered]{algorithm2e}
\usepackage{multicol}
\usepackage[backend=biber,style=alphabetic,sorting=ynt]{biblatex}
%\usepackage{xcolor}
%\usepackage{tkz-berge}
%\usepackage{tkz-graph}
\usepackage{pgfplots}
\usepackage{sagetex}
\usepackage{setspace}
\usepackage{etoc}
%\usepackage{wrapfig}
\usepackage{pgfgantt}
\DeclareUnicodeCharacter{2212}{−}
\usepgfplotslibrary{groupplots,dateplot}
\pgfplotsset{compat=newest}

\newtheorem{theorem}{Theorem}
\newtheorem{definition}{Definition}
\newtheorem{example}{Example}
\newtheorem{claim}{Claim}
\newtheorem{fact}{Fact}
\newtheorem{remark}{Remark}
\newtheorem*{theorem*}{Theorem}
\newtheorem{lemma}{Lemma}
\crefname{lemma}{Lemma}{Lemmas}
\hypersetup{colorlinks=true}
% , allcolors=blue,allbordercolors=blue,pdfborderstyle={0 0 1}}
%\hypersetup{pdfborder={2 2 2}}
% pdfpagemode=FullScreen,
% backref 

\newtheorem{problem}{Problem}
\crefname{problem}{Problem}{Problems}

\DeclareMathOperator{\Ima}{Im}


  \addbibresource{./sample.bib} 

\fi

\begin{document}

\newcommand{\commentt}[1]{\textcolor{blue}{ \textbf{[COMMENT]} #1}}
\newcommand{\ctt}[1]{\commentt{#1}}
\newcommand{\prb}[1]{ \mathbf{Pr} \left[ #1 \right]}
\newcommand{\prbm}[2]{ \mathbf{Pr}_{ #2 }\left[ #1 \right]}
\newcommand{\prbc}[3]{ \mathbf{Pr}_{ #2 }\left[ #1 \right | #3]}
\newcommand{\prbcprb}[3]{ \prbc{#2}{#1}{#3} \cdot \prb{#3} } 
\newcommand{\expp}[1]{ \mathbf{E} \left[ {#1} \right]}
\newcommand{\onotation}[1]{\(\mathcal{O} \left( {#1}  \right) \)}
\newcommand{\ona}[1]{\onotation{#1}}
\newcommand{\PSI}{{\ket{\psi}}}
\newcommand{\xij} { X_{ij} } 
\DeclareMathOperator{\Ima}{Im}
%\newcommand{\LESn}{\ket{\psi_n}}
%\newcommand{\LESa}{\ket{\phi_n}}
%\newcommand{\LESs}{\frac{1}{\sqrt{n}}\sum_{i}{\ket{\left(0^{i}10^{n-i}\right)^{n}}}}
%\newcommand{\Hn}{\mathcal{H}_{n}}
%\newcommand{\Ep}{\frac{1}{\sqrt{2^n}}\sum^{2^n}_{x}{ \ket{xx}}}
%\newcommand{\HON}{\ket{\psi_{\text{honest}}}}
%\newcommand{\Lemma}{\paragraph{Lemma.}}
\newcommand{\Cpa}{[n, \rho n, \delta n]}
%\setlength{\columnsep}{0.6cm}
\newcommand{\Jvv}{ \bar{J_{v}} } 
\newcommand{\Cvv}{ \tilde{C_{v}} } 

\newcommand{\Gz}{ G_{z}^{\delta} } 
\newcommand{ \Tann } {  \mathcal{T}\left( G, C_0 \right) }
\newcommand{\ireducable}{ireducable \hyperref[ire]{[\ref{ire}]} }
\newcommand{\cutUU}{E(U_{-1} \bigcup U_{+1} ,U)} 
\newcommand{\wcutUU}{w\left( E(U_{-1} \bigcup U_{+1} ,U)  \right)}
\newcommand{\testgo}{  \mathcal{T}\left(J, q , C_{0}\right) } 

\newcommand{\duC}{\left( C_{A}^{\perp}\otimes C_{B}^{\perp} \right)^{\perp}}
\newcommand{\duduC}{\left( C_{A}\otimes C_{B}\right)^{\perp}}
  




\title{Idea For FT computing.} 
\author{David Ponarovsky}

\begin{enumerate}
  \item encode $k$ qubit in $[n,10k, d]$ good qLDPC code. With a reducing Lemma for threshold $l$.  
  \item implement $X,Z,H,T$ in the straightforward way.  
  \item The $CX$, need more attention. Denote by $g_{i}$ a generator of $C$ and notice that we took only $1/10$-fraction of the generator in the encoding process. Now, any $CX$ will be followed by correction step. The idea we stretch a wire according  predetermined match between the qubits in the support of $g_{i}$ and the qubits in the support of $g_{j}$.    
  \item  As we took only a fraction of the code space, we can require that any codeword spanned by the $g_{j}$'s has an overlap with $g_{i}$ which less than $l/3$. Or in other words, the decoder can correct a non desire $CX$ in single step $\sim$.
  \item key point of the bullet above is that by overlap we mean to the bitwise AND, otherwise , saying that we have such $C$ is equivalence to say that we found a code with positive rate and distance greater than $\frac{1}{2}$. What we are actually want is that any code word will has a small weight, and there fore can't inferred as other codeword when we apply the $CX$ gate.     
\end{enumerate}
\printbibliography
\end{document}





