

%\newcommand*{\ACM}{}%

\ifdefined\ACM

%\documentclass[sigplan,screen]{acmart}
  \documentclass[manuscript,screen,review]{acmart}

\else
  \documentclass{article}
  \usepackage[utf8]{inputenc}
\usepackage[a4paper, total={6.5in, 10in} ]{geometry}
\usepackage{braket}
\usepackage{xcolor}
\usepackage{amsmath}
\usepackage{amssymb}
\usepackage{amsfonts}
\usepackage{graphicx}
\usepackage{svg}
\usepackage{float}
\usepackage{tikz}
\usetikzlibrary{patterns, shapes.arrows}
\usepackage{adjustbox}
\usepackage{tikz-network}
\usepackage[ruled,lined,linesnumbered]{algorithm2e}
\usepackage{multicol}
\usepackage[backend=biber,style=alphabetic,sorting=ynt]{biblatex}
\usepackage{xcolor}
\usepackage{pgfplots}
\DeclareUnicodeCharacter{2212}{−}
\usepgfplotslibrary{groupplots,dateplot}
\pgfplotsset{compat=newest}



  \addbibresource{./sample.bib} 

\fi

\begin{document}

\input{newcommands}
\title{Idea For FT computing.} 
\author{David Ponarovsky}

\begin{enumerate}
  \item encode $k$ qubit in $[n,10k, d]$ good qLDPC code. With a reducing Lemma for threshold $l$.  
  \item implement $X,Z,H,T$ in the straightforward way.  
  \item The $CX$, need more attention. Denote by $g_{i}$ a generator of $C$ and notice that we took only $1/10$-fraction of the generator in the encoding process. Now, any $CX$ will be followed by correction step. The idea we stretch a wire according  predetermined match between the qubits in the support of $g_{i}$ and the qubits in the support of $g_{j}$.    
  \item  As we took only a fraction of the code space, we can require that any codeword spanned by the $g_{j}$'s has an overlap with $g_{i}$ which less than $l/3$. Or in other words, the decoder can correct a non desire $CX$ in single step $\sim$.       
\end{enumerate}<++>
\printbibliography
\end{document}





