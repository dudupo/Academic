%\usepackage{courier}
\usepackage[utf8]{inputenc}
\ifdefined \bookflag
  \usepackage[a4paper, marginparwidth=100pt, total={5in, 9in}]{geometry}
\else
\usepackage[a4paper, total={6in, 9in}]{geometry}
\fi
\usepackage{braket}
\usepackage{xcolor}
\usepackage{amsmath}
\usepackage{amsfonts}
\usepackage{amsthm}
\usepackage{amssymb}
%\usepackage[ocgcolorlinks]{hyperref}
\usepackage{titling}
\usepackage{hyperref}
%\usepackage{hyperref,xcolor}
%\usepackage[ocgcolorlinks]{ocgx2}
\usepackage{cleveref}
\usepackage{graphicx}
\usepackage{svg}
\usepackage{float}
\usepackage{wrapfig}
\usepackage{tikz}
\usetikzlibrary{patterns, shapes.arrows}
\usepackage{adjustbox}
%\usepackage{tikz-network}
\usepackage{tkz-graph}
\usetikzlibrary{decorations.pathmorphing}
\usepackage{tkz-berge}
\usepackage[linesnumbered]{algorithm2e}
\usepackage{multicol}
\usepackage[backend=biber,style=alphabetic,sorting=ynt]{biblatex}
%\usepackage{xcolor}
%\usepackage{tkz-berge}
%\usepackage{tkz-graph}
\usepackage{pgfplots}
%\usepackage{sagetex}
\usepackage{setspace}
\usepackage{etoc}
%\usepackage{wrapfig}
\usepackage{pgfgantt}
\usepackage{pdfpages}
\DeclareUnicodeCharacter{2212}{−}
\usepgfplotslibrary{groupplots,dateplot}
\pgfplotsset{compat=newest}


\newcommand{\mynewtheorem}[2]{  
  \newtheorem{#1}{#2}
  \numberwithin{#1}{section}

}
\newcommand{\mynewtheoremstar}[2]{  
  \newtheorem*{#1}{#2}

}
\mynewtheorem{corollary}{Corollary}
\mynewtheorem{theorem}{Theorem}
\mynewtheorem{definition}{Definition}
\mynewtheorem{example}{Example}
\mynewtheorem{claim}{Claim}
\mynewtheorem{fact}{Fact}
\mynewtheorem{remark}{Remark}
\mynewtheoremstar{theorem*}{Theorem}
\mynewtheorem{lemma}{Lemma}
\mynewtheorem{conjecture}{Conjecture}
\mynewtheorem{exercise}{Exercise}
\mynewtheorem{solution}{Solution}
\mynewtheorem{openproblem}{Open problem}
\crefname{lemma}{Lemma}{Lemmas}
\Crefname{algocf}{Algorithm}{Algorithms}
\hypersetup{colorlinks=true}
% , allcolors=blue,allbordercolors=blue,pdfborderstyle={0 0 1}}
%\hypersetup{pdfborder={2 2 2}}
% pdfpagemode=FullScreen,
% backref 

\mynewtheorem{problem}{Problem}
\crefname{problem}{Problem}{Problems}

\DeclareMathOperator{\Ima}{Im}
\DeclareMathOperator{\Rea}{Re}
\DeclareMathOperator{\rank}{rank}
\DeclareMathOperator{\sign}{sign}
