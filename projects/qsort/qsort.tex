

%\newcommand*{\ACM}{}%

\ifdefined\ACM

%\documentclass[sigplan,screen]{acmart}
  \documentclass[manuscript,screen,review]{acmart}

\else
  \documentclass{article}
  \usepackage[utf8]{inputenc}
\usepackage[a4paper, total={6in, 9in}]{geometry}
\usepackage{braket}
\usepackage{xcolor}
\usepackage{amsmath}
\usepackage{amsfonts}
\usepackage{amsthm}
\usepackage{amssymb}
%\usepackage[ocgcolorlinks]{hyperref}
\usepackage{hyperref}
%\usepackage{hyperref,xcolor}
%\usepackage[ocgcolorlinks]{ocgx2}
\usepackage{cleveref}
\usepackage{graphicx}
\usepackage{svg}
\usepackage{float}
\usepackage{tikz}
\usetikzlibrary{patterns, shapes.arrows}
\usepackage{adjustbox}
%\usepackage{tikz-network}
\usepackage{tkz-graph}
\usepackage{tkz-berge}
\usepackage[linesnumbered]{algorithm2e}
\usepackage{multicol}
\usepackage[backend=biber,style=alphabetic,sorting=ynt]{biblatex}
%\usepackage{xcolor}
%\usepackage{tkz-berge}
%\usepackage{tkz-graph}
\usepackage{pgfplots}
\usepackage{sagetex}
\usepackage{setspace}
\usepackage{etoc}
%\usepackage{wrapfig}
\usepackage{pgfgantt}
\DeclareUnicodeCharacter{2212}{−}
\usepgfplotslibrary{groupplots,dateplot}
\pgfplotsset{compat=newest}

\newtheorem{theorem}{Theorem}
\newtheorem{definition}{Definition}
\newtheorem{example}{Example}
\newtheorem{claim}{Claim}
\newtheorem{fact}{Fact}
\newtheorem{remark}{Remark}
\newtheorem*{theorem*}{Theorem}
\newtheorem{lemma}{Lemma}
\crefname{lemma}{Lemma}{Lemmas}
\hypersetup{colorlinks=true}
% , allcolors=blue,allbordercolors=blue,pdfborderstyle={0 0 1}}
%\hypersetup{pdfborder={2 2 2}}
% pdfpagemode=FullScreen,
% backref 

\newtheorem{problem}{Problem}
\crefname{problem}{Problem}{Problems}

\DeclareMathOperator{\Ima}{Im}


  \addbibresource{./sample.bib} 

\fi

\begin{document}

\newcommand{\commentt}[1]{\textcolor{blue}{ \textbf{[COMMENT]} #1}}
\newcommand{\ctt}[1]{\commentt{#1}}
\newcommand{\prb}[1]{ \mathbf{Pr} \left[ #1 \right]}
\newcommand{\prbm}[2]{ \mathbf{Pr}_{ #2 }\left[ #1 \right]}
\newcommand{\prbc}[3]{ \mathbf{Pr}_{ #2 }\left[ #1 \right | #3]}
\newcommand{\prbcprb}[3]{ \prbc{#2}{#1}{#3} \cdot \prb{#3} } 
\newcommand{\expp}[1]{ \mathbf{E} \left[ {#1} \right]}
\newcommand{\onotation}[1]{\(\mathcal{O} \left( {#1}  \right) \)}
\newcommand{\ona}[1]{\onotation{#1}}
\newcommand{\PSI}{{\ket{\psi}}}
\newcommand{\xij} { X_{ij} } 
\DeclareMathOperator{\Ima}{Im}
%\newcommand{\LESn}{\ket{\psi_n}}
%\newcommand{\LESa}{\ket{\phi_n}}
%\newcommand{\LESs}{\frac{1}{\sqrt{n}}\sum_{i}{\ket{\left(0^{i}10^{n-i}\right)^{n}}}}
%\newcommand{\Hn}{\mathcal{H}_{n}}
%\newcommand{\Ep}{\frac{1}{\sqrt{2^n}}\sum^{2^n}_{x}{ \ket{xx}}}
%\newcommand{\HON}{\ket{\psi_{\text{honest}}}}
%\newcommand{\Lemma}{\paragraph{Lemma.}}
\newcommand{\Cpa}{[n, \rho n, \delta n]}
%\setlength{\columnsep}{0.6cm}
\newcommand{\Jvv}{ \bar{J_{v}} } 
\newcommand{\Cvv}{ \tilde{C_{v}} } 

\newcommand{\Gz}{ G_{z}^{\delta} } 
\newcommand{ \Tann } {  \mathcal{T}\left( G, C_0 \right) }
\newcommand{\ireducable}{ireducable \hyperref[ire]{[\ref{ire}]} }
\newcommand{\cutUU}{E(U_{-1} \bigcup U_{+1} ,U)} 
\newcommand{\wcutUU}{w\left( E(U_{-1} \bigcup U_{+1} ,U)  \right)}
\newcommand{\testgo}{  \mathcal{T}\left(J, q , C_{0}\right) } 

\newcommand{\duC}{\left( C_{A}^{\perp}\otimes C_{B}^{\perp} \right)^{\perp}}
\newcommand{\duduC}{\left( C_{A}\otimes C_{B}\right)^{\perp}}
  





\title{$\log n$ - Space, $n^{3/2}$ Time Quantum Sort.} 
\author{David Ponarovsky}
%\author{Noa Viner, David Ponarovsky}
\maketitle
%It were proven that any quantum algoirthm in the quantum circuits which sorts at time $T$ and storage space $S$ has to satisfty the restrication $TS = \Omega(n^{3/2})$ \cite{klauck2003quantum} where in the regime of $S \ge \log^{3}(n) $ they shown that the bound is tight up to logratmic factors. Yet in the regime where $S$ is strictly $\Theta(\log(n))$ no much advences as been reached byeoned the $T = n^{1 \frac{1}{2}} \log n$. Here we preasent a quantum algorithm that sort at $\log(n)$ storage memory and $n^{3/2}$ time. We achieved that by quntifiy the sort algorithm invented by Stanley P. Y. Fung \cite{Simplesort} who coined its name - "ICan'tBelieveItCanSort" - due to the surprise of having such a simple sorting algorithm. 

It has been proven that any quantum algorithm in the quantum circuits which sorts at time $T$ and storage space $S$ has to satisfy the restriction $TS = \Omega(n^{3/2})$ \cite{klauck2003quantum}. In the regime of $S \ge \log^{3}(n)$, it has been shown that the bound is tight up to logarithmic factors. However, in the regime where $S$ is strictly $\Theta(\log(n))$, not much advancement has been reached beyond $T = \Theta( n^{1 \frac{1}{2}} \log n )$. Here, we present a quantum algorithm that sorts with $\log(n)$ storage memory and $ \Theta(n^{3/2})$ time. We achieved this by quantifying the sorting algorithm invented by Stanley P. Y. Fung \cite{Simplesort}, who coined its name - "ICan'tBelieveItCanSort" - due to the surprise of having such a simple sorting algorithm.

The insight that allows getting rid of the logarithmic factor is the fact that in any iteration of the "ICan'tBelieveItCanSort" algorithm, it looks for the first position $k < i$ such that $A_{k} > A_{i}$, assuming $A_{1}\le A_{2} .. A_{i-1}$. Under this assumption, this task can be done using Grover in time $\sqrt{i}$, while in the previous attempts, the subroutines that were being used were extracting the maximum, which requires $\Omega(\sqrt{n} \log n)$ when done accurately. 

The paper is organized as follows: first, we introduce "ICan'tBelieveItCanSort" presented in \Cref{alg:alg1} and prove its correctness. The correctness proof will imply the equivalence of \Cref{alg:alg2}. Then, we present the quantum space bound version, \Cref{alg:alg3}, and analyze its complexity.

\section{ Classical Starting Point - "ICan'tBelieveItCanSort".}

\begin{algorithm}
\SetAlgoLined
\KwResult{Sorting $A_{1},A_{2},..A_{n}$ }
\caption{ "ICan'tBelieveItCanSort"  alg.} \label{alg:alg1}
\For{ $ i \in [n]$} {
  \For{ $ j \in [n]$} {
    \If { $A_{i} < A_{j} $} {
      swap $A_{i} \leftrightarrow A_{j}$
    }
  }
}
\end{algorithm}
\begin{claim}
  \label{claim:first}
  After the $i$th iteration, $A_{1} \le A_{2} \le A_{3} .. \le A_{i}$ and $A_{i}$ is the maximum of the whole array. 
\end{claim}
\begin{proof}
  By induction on the iteration number $i$. 
  \begin{enumerate}
    \item Base. For $i=1$, it is clear that when $j$ reaches the position of the maximum element, an exchange will occur and $A_{1}$ will be set to be the maximum element. Thus, the condition on line (3) will not be satisfied for the remaining $j$-iterations of the inner loop. Therefore, at the end of the first iteration, $A_{1}$ is indeed the maximum.
    \item Assumption. Assume the correctness of the claim for any $i^{\prime} < i$. 
    \item Step. Consider the $i$th iteration. And observe that if $A_{i} = A_{i-1}$ then $A_{i}$ is also the maximal element in $A$, namely no exchange will be made in the $i$th iteration, yet $A_{1} \le A_{2} \le .. \le A_{i-1}$ by the induction assumption, thus  $A_{1} \le A_{2} \le .. \le A_{i-1} \le A_{i}$ and $A_{i}$ is the maximal element, so the claim holds in the end of the iteration. 
      If $A_{i} < A_{i-1}$ then there exists $k \in [1,i-1]$ such that $A_{k} > A_{i}$. Set $k$ to be the minimal position for which the inequality holds. For convenience, denote by $A^{(j)}$ the array in the beginning of the $j$th iteration of the inner loop. And let's split into cases according to $j$ value. 
      \begin{enumerate}
        \item $j < k$ By definition of $k$, for any $j < k, A^{(1)}_{j} < A^{(1)}_{i}$, Hence in the first $k-1$ iterations no exchange will be made and we can conclude that $A^{(j)}_{l} = A^{(1)}_{l}$ for any $l \in [n]$ and $j \le k$. 
        \item $j \ge k$ and $j\le i$, We claim that for each such $j$ an exchange will always occur. (The proof is given below.)
          \begin{claim}
            \label{claim:second}
            For any $j \in [k,i]$ we have that in the end of the $j$th iteration:  
            \begin{itemize}
              \item $A^{(j+1)}_{j} = A^{(j)}_{i}$.
              \item $A^{(j+1)}_{i} = A^{(j)}_{j} = A^{(1)}_{j}$.
              \item For any $l > j$ and $l \neq i$ we have $A^{(j+1)}_{l} = A^{(1)}_{l}$.
            \end{itemize}
          \end{claim}

        \item $j > i$, so we know that $A^{(i+1)}_{i}$ is the maximal element in $A$. Therefore, for any $j$, it holds that $A^{(i+1)}_{i}\ge A^{(i)}_{j}$. It follows that no exchange would be made and $A^{(j)}_{i}$ will remain the maximum til the end of the inner loop. Thus for any $j >i$:   
          \begin{equation*}
            \begin{split}
              A^{(j)}_{i}=A^{(j-1)}_{i}=.. =A^{(i+2)}_{i}=A^{(i+1)}_{i}=A^{(i)}_{i-1}=A^{(0)}_{i-1}=\max A
            \end{split}
          \end{equation*}
          And 
          \begin{equation*}
            \begin{split}
             & A^{(j)}_{1}, A^{(j)}_{2}, .. A^{(j)}_{k-1}, A^{(j)}_{k}, A^{(j)}_{k+1}, .. A^{(j)}_{i-1}, A^{(j)}_{i}, A^{(j)}_{i+1} , A^{(j)}_{i+2} , A^{(j)}_{i+3} .. \\
            = & A^{(0)}_{1}, A^{(0)}_{2}, .. A^{(0)}_{k-1}, A^{(0)}_{i}, A^{(0)}_{k}, .. A^{(0)}_{i-2}, A^{(0)}_{i-1}, A^{(0)}_{i+1} , A^{(0)}_{i+2} , A^{(0)}_{i+3} .. 
            \end{split}
          \end{equation*}
          In particular, for $j = n+1$ (Note that there is no $n+1$th iteration). Clearly, the inequalities are satisfied and we are done.
\end{enumerate}
\end{enumerate}
  \end{proof}
  \begin{proof}[Proof of \Cref{claim:second}.]
  Observe that the third section holds trivially by the definition of the algorithm. It doesn't touch any position greater than $j$ in the first $j$ iterations (inner loop) except the $i$th position. So we have to prove only the first two bullets. Again, we are going to prove them by induction on $j$.  
  \begin{enumerate}
    \item Base. $A^{(1)}_{k}$ is greater than $A_{i}$, and by the above case, we have that at the beginning of the $k$th iteration $A^{(k)}_{k}=A^{(1)}_{k}, A^{(k)}_{i}=A^{(1)}_{k}$. Therefore, the condition on line (3) is satisfied, an exchange is made, and $A^{(k+1)}_{k} =A^{(k)}_{i} = A^{(1)}_{i}$ and $A^{(k+1)}_{i} = A^{(k)}_{k}$. %Now, $A^{(k+1)}_{k+1} = A^{(k)}_{k+1} = A^{(0)}_{k+1}$.
    \item Assumption. Assume the correctness of the claim for any $k \le j^{\prime} < j \le i$. 
    \item Step. Consider the $j \in (k,i]$ iteration. By the induction assumption, we have that $A^{(j)}_{j-1} = A^{(j-1)}_{i}$ and $A^{(j)}_{i} = A^{(j-1)}_{j-1} = A^{(1)}_{j-1}$. On the other hand, by the induction assumption of \Cref{claim:first}, $j-1 < i \Rightarrow A^{(1)}_{j-1} \le A^{(1)}_{j}$. Combining the third bullet, we obtain that:                
      \begin{equation*}
        \begin{split}
          A^{(j)}_{j} = A^{(1)}_{j} \ge A^{(1)}_{j-1} = A^{(j)}_{i}
        \end{split}
      \end{equation*}
      And therefore, either there is an inequality and an exchange is made or there is equality. In both cases, after the $j$th iteration, we have $A^{(j+1)}_{j} = A^{(j)}_{i}$ and $A^{(j+1)}_{i} = A^{(j)}_{j} = A^{(1)}_{j}$.
  \end{enumerate}
\end{proof}


\begin{algorithm}

\SetAlgoLined
\KwResult{Sorting $A_{1},A_{2},..A_{n}$ }
\caption{ "ICan'tBelieveItCanSort"  alg.}\label{alg:alg2}
swap $A_{1} \leftrightarrow \max A$ \\
\For{ $ i \in [n-1]$} {
    Find the first $k$ such $A_{k} > A_{i}$ \\
    Set $A \leftarrow A_{1},A_{2}..A_{k-1},A_{i},A_{k},A_{k+1},..,A_{i-1},A_{i+1}..,A_{n}$
}
\end{algorithm}


\section{Sorting Quantumly in Space-Bounded Storage.}

\begin{algorithm}
\SetAlgoLined
\KwResult{Sorting $A_{1},A_{2},..A_{n}$ }
\caption{ "Quantum ICan'tBelieveItCanSort"  alg.}\label{alg:alg3}
swap $A_{1} \leftrightarrow \max A$ \\
\For{ $ i \in [n-1]$} {
    Set current $\leftarrow$ head.next \\

    $k$-pointer $\leftarrow$ Find the first '$k< i$' node such '$A_{k} > A_{i}$' using Grover querying the follow \\
    \ \ Ask if ( node.color $=$ red and node.value $ > $ current.value \\ 
    \ \ \ \ and node.back.value $\le$ current.value ) \\ 

    Set head.next $\leftarrow$ head.next.next \\
    Set head.next.back $\leftarrow$ head\\
    Set current.next $\leftarrow$ $k$-pointer \\
    Set current.back $\leftarrow$ $k$-pointer.back \\
    Set current.back.next $\leftarrow$ current \\ 
    Set current.color $\leftarrow$ red
}
\end{algorithm}
\printbibliography
\end{document}





