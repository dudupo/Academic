

%\newcommand*{\ACM}{}%

\ifdefined\ACM

%\documentclass[sigplan,screen]{acmart}
  \documentclass[manuscript,screen,review]{acmart}

\else
  \documentclass{article}
  \usepackage[utf8]{inputenc}
\usepackage[a4paper, total={6in, 9in}]{geometry}
\usepackage{braket}
\usepackage{xcolor}
\usepackage{amsmath}
\usepackage{amsfonts}
\usepackage{amsthm}
\usepackage{amssymb}
%\usepackage[ocgcolorlinks]{hyperref}
\usepackage{hyperref}
%\usepackage{hyperref,xcolor}
%\usepackage[ocgcolorlinks]{ocgx2}
\usepackage{cleveref}
\usepackage{graphicx}
\usepackage{svg}
\usepackage{float}
\usepackage{tikz}
\usetikzlibrary{patterns, shapes.arrows}
\usepackage{adjustbox}
%\usepackage{tikz-network}
\usepackage{tkz-graph}
\usepackage{tkz-berge}
\usepackage[linesnumbered]{algorithm2e}
\usepackage{multicol}
\usepackage[backend=biber,style=alphabetic,sorting=ynt]{biblatex}
%\usepackage{xcolor}
%\usepackage{tkz-berge}
%\usepackage{tkz-graph}
\usepackage{pgfplots}
\usepackage{sagetex}
\usepackage{setspace}
\usepackage{etoc}
%\usepackage{wrapfig}
\usepackage{pgfgantt}
\DeclareUnicodeCharacter{2212}{−}
\usepgfplotslibrary{groupplots,dateplot}
\pgfplotsset{compat=newest}

\newtheorem{theorem}{Theorem}
\newtheorem{definition}{Definition}
\newtheorem{example}{Example}
\newtheorem{claim}{Claim}
\newtheorem{fact}{Fact}
\newtheorem{remark}{Remark}
\newtheorem*{theorem*}{Theorem}
\newtheorem{lemma}{Lemma}
\crefname{lemma}{Lemma}{Lemmas}
\hypersetup{colorlinks=true}
% , allcolors=blue,allbordercolors=blue,pdfborderstyle={0 0 1}}
%\hypersetup{pdfborder={2 2 2}}
% pdfpagemode=FullScreen,
% backref 

\newtheorem{problem}{Problem}
\crefname{problem}{Problem}{Problems}

\DeclareMathOperator{\Ima}{Im}


  \addbibresource{./sample.bib} 

\fi

\begin{document}

\newcommand{\commentt}[1]{\textcolor{blue}{ \textbf{[COMMENT]} #1}}
\newcommand{\ctt}[1]{\commentt{#1}}
\newcommand{\prb}[1]{ \mathbf{Pr} \left[ #1 \right]}
\newcommand{\prbm}[2]{ \mathbf{Pr}_{ #2 }\left[ #1 \right]}
\newcommand{\prbc}[3]{ \mathbf{Pr}_{ #2 }\left[ #1 \right | #3]}
\newcommand{\prbcprb}[3]{ \prbc{#2}{#1}{#3} \cdot \prb{#3} } 
\newcommand{\expp}[1]{ \mathbf{E} \left[ {#1} \right]}
\newcommand{\onotation}[1]{\(\mathcal{O} \left( {#1}  \right) \)}
\newcommand{\ona}[1]{\onotation{#1}}
\newcommand{\PSI}{{\ket{\psi}}}
\newcommand{\xij} { X_{ij} } 
\DeclareMathOperator{\Ima}{Im}
%\newcommand{\LESn}{\ket{\psi_n}}
%\newcommand{\LESa}{\ket{\phi_n}}
%\newcommand{\LESs}{\frac{1}{\sqrt{n}}\sum_{i}{\ket{\left(0^{i}10^{n-i}\right)^{n}}}}
%\newcommand{\Hn}{\mathcal{H}_{n}}
%\newcommand{\Ep}{\frac{1}{\sqrt{2^n}}\sum^{2^n}_{x}{ \ket{xx}}}
%\newcommand{\HON}{\ket{\psi_{\text{honest}}}}
%\newcommand{\Lemma}{\paragraph{Lemma.}}
\newcommand{\Cpa}{[n, \rho n, \delta n]}
%\setlength{\columnsep}{0.6cm}
\newcommand{\Jvv}{ \bar{J_{v}} } 
\newcommand{\Cvv}{ \tilde{C_{v}} } 

\newcommand{\Gz}{ G_{z}^{\delta} } 
\newcommand{ \Tann } {  \mathcal{T}\left( G, C_0 \right) }
\newcommand{\ireducable}{ireducable \hyperref[ire]{[\ref{ire}]} }
\newcommand{\cutUU}{E(U_{-1} \bigcup U_{+1} ,U)} 
\newcommand{\wcutUU}{w\left( E(U_{-1} \bigcup U_{+1} ,U)  \right)}
\newcommand{\testgo}{  \mathcal{T}\left(J, q , C_{0}\right) } 

\newcommand{\duC}{\left( C_{A}^{\perp}\otimes C_{B}^{\perp} \right)^{\perp}}
\newcommand{\duduC}{\left( C_{A}\otimes C_{B}\right)^{\perp}}
  





\title{$\log n$ - Space, $n^{3/2}$ Time Quantum Sort.} 
\author{David Ponarovsky}
%\author{Noa Viner, David Ponarovsky}
\maketitle
%It were proven that any quantum algoirthm in the quantum circuits which sorts at time $T$ and storage space $S$ has to satisfty the restrication $TS = \Omega(n^{3/2})$ \cite{klauck2003quantum} where in the regime of $S \ge \log^{3}(n) $ they shown that the bound is tight up to logratmic factors. Yet in the regime where $S$ is strictly $\Theta(\log(n))$ no much advences as been reached byeoned the $T = n^{1 \frac{1}{2}} \log n$. Here we preasent a quantum algorithm that sort at $\log(n)$ storage memory and $n^{3/2}$ time. We achieved that by quntifiy the sort algorithm invented by Stanley P. Y. Fung \cite{Simplesort} who coined its name - "ICan'tBelieveItCanSort" - due to the surprise of having such a simple sorting algorithm. 

It has been proven that any quantum algorithm in the quantum circuits which sorts at time $T$ and storage space $S$ has to satisfy the restriction $TS = \Omega(n^{3/2})$ \cite{klauck2003quantum}. In the regime of $S \ge \log^{3}(n)$, it has been shown that the bound is tight up to logarithmic factors. However, in the regime where $S$ is strictly $\Theta(\log(n))$, not much advancement has been reached beyond $T = \Theta( n^{1 \frac{1}{2}} \log n )$. Here, we present a quantum algorithm that sorts with $\log(n)$ storage memory and $n^{3/2}$ time. We achieved this by quantifying the sorting algorithm invented by Stanley P. Y. Fung \cite{Simplesort}, who coined its name - "ICan'tBelieveItCanSort" - due to the surprise of having such a simple sorting algorithm.

\begin{algorithm}
\SetAlgoLined
\KwResult{Sorting $A_{1},A_{2},..A_{n}$ }
\caption{ "ICan'tBelieveItCanSort"  alg.}
\For{ $ i \in [n]$} {
  \For{ $ j \in [n]$} {
    \If { $A_{i} < A_{j} $} {
      swap $A_{i} \leftrightarrow A_{j}$
    }
  }
}
\end{algorithm}
\begin{algorithm}

\SetAlgoLined
\KwResult{Sorting $A_{1},A_{2},..A_{n}$ }
\caption{ "ICan'tBelieveItCanSort"  alg.}
swap $A_{1} \leftrightarrow \max A$ \\
\For{ $ i \in [n-1]$} {
    Find the first $k$ such $A_{k} > A_{i}$ \\
    Set $A \leftarrow A_{1},A_{2}..A_{k-1},A_{i},A_{k},A_{k+1},..,A_{i-1},A_{i+1}..,A_{n}$
}
\end{algorithm}

\begin{algorithm}
\SetAlgoLined
\KwResult{Sorting $A_{1},A_{2},..A_{n}$ }
\caption{ "Quantum ICan'tBelieveItCanSort"  alg.}
swap $A_{1} \leftrightarrow \max A$ \\
\For{ $ i \in [n-1]$} {
    Set current $\leftarrow$ head.next \\

    $k$-pointer $\leftarrow$ Find the first '$k< i$' node such '$A_{k} > A_{i}$' using Grover querying the follow \\
    \ \ Ask if ( node.color $=$ red and node.value $ > $ current.value \\ 
    \ \ \ \ and node.back.value $\le$ current.value ) \\ 

    Set head.next $\leftarrow$ head.next.next \\
    Set head.next.back $\leftarrow$ head\\
    Set current.next $\leftarrow$ $k$-pointer \\
    Set current.back $\leftarrow$ $k$-pointer.back \\
    Set current.back.next $\leftarrow$ current \\ 
    Set current.color $\leftarrow$ red
}
\end{algorithm}
\printbibliography
\end{document}





